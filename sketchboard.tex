\documentclass{article}
\usepackage[utf8]{inputenc}
\usepackage{indentfirst}
\usepackage{graphicx}
\usepackage{amssymb}
\usepackage{amsmath,amsthm,amsfonts}
\usepackage{esint}
\usepackage{upgreek}
\usepackage{tikz}
\usepackage{mathtools}
\usepackage{mathrsfs}
\usepackage{slashed}
\usepackage{hyperref}
\usepackage{bbold}
\usepackage[margin=2cm]{geometry}
\usepackage{titlesec}

\newcommand{\R}{\mathbb{R}}
\newcommand{\diff}{\mathrm{d}}
\newcommand{\inp}{\mathrm{i}}
\newcommand{\e}{\mathrm{e}}
\newcommand{\Lagr}{\mathscr{L}}
\newcommand{\PS}{\mathcal{P}}



\newcommand{\bite}{\begin{itemize}}
	\newcommand{\eat}{\end{itemize}}
\newcommand{\beq}{\begin{equation}}
	\newcommand{\eeq}{\end{equation}}
\newcommand{\rarrow}{\rightarrow}
\newcommand{\beqa}{\begin{align}}
	\newcommand{\eeqa}{\end{align}}
\newcommand{\barr}{\begin{array}}
	\newcommand{\earr}{\end{array}}
\newcommand{\del}{\partial}
\newcommand{\de}{\mathrm{d}}
%\newcommand{\mu\nu}{{\mu\nu}}
\renewcommand{\th}{\mathrm{th}}
\newcommand{\com}[1]{\begin{itemize}\color{RED}{{#1}}\end{itemize}}
\newcommand{\C}{\mathbb{C}}
\newcommand{\R}{\mathbb{R}}
\newcommand{\btw}[1]{\color{PURPLE}{{#1}}\color{BLACK}}
\newcommand{\cut}[1]{\color{RED}{{#1}}\color{BLACK}}

%text
\newcommand{\ie}{\textit{i.e.}~}
\newcommand{\eg}{\textit{e.g.}~}
\newcommand{\wrt}{\textit{w.r.t.}~}
\newcommand{\etc}{\textit{etc.}~}


\newcommand{\M}{\mathcal{M}}
\newcommand{\N}{\mathcal{N}}
% \newcommand{\H}{\mathcal{H}}

\newcommand{\bz}{\mathbf{z}}

\newcommand{\mb}[1]{\mathbf{#1}}
\newcommand{\mc}[1]{\mathcal{#1}}
\newcommand{\mbb}[1]{\mathbb{#1}}
\newcommand{\mf}[1]{\mathfrak{#1}}

\newcommand{\unit}[1]{\mathbf{\hat{#1}}}

%% Package bbold for bold identity symbol
\usepackage{bbold}

\newcommand{\id}{\mathbb{1}}

\newcommand{\utilde}[1]{\underaccent{\tilde}{#1}}

\newcommand{\vect}[1]{\boldsymbol{#1}}
\newcommand{\bvec}[1]{\boldsymbol{\vec #1}}
\newcommand{\expect}[1]{\langle #1\rangle}
\newcommand{\innerp}[2]{\langle #1 \vert #2 \rangle}
\newcommand{\expectop}[3]{\langle #1 \vert #2 \vert #3 \rangle}
\newcommand{\bra}[1]{\langle #1 \vert}
\newcommand{\ket}[1]{\vert #1 \rangle}
\newcommand{\supersc}[1]{$^{\textrm{#1}}$}
\newcommand{\subsc}[1]{$_{\textrm{#1}}$}
\newcommand{\sltwoc}{\mathfrak{sl}(2,\mathbb{C})}

\newcommand{\norm}[1]{\lVert #1 \rVert}

\newcommand{\rket}[1]{\vert #1 ]}
\newcommand{\rbra}[1]{[ #1 \vert}

\newcommand{\rinnerp}[2]{[ #1 \vert #2 ]}

\newcommand{\bket}[1]{\vert #1 )}
\newcommand{\bbra}[1]{( #1 \vert}

\newcommand{\binnerp}[2]{( #1 \vert #2 )}

\newcommand{\innerpA}[2]{\langle #1 \vert #2 ]}
\newcommand{\innerpB}[2]{[ #1 \vert #2 \rangle}

\newcommand{\onehalf}{\frac{1}{2}}

\newcommand{\Tr}{\mathrm{Tr}}

\newcommand{\Cyl}{\mathrm{Cyl}}

\newtheorem{theorem}{Theorem}[paragraph]
\newtheorem{proposition}{Proposition}[paragraph]
\newtheorem{corollary}{Corollary}[theorem]

\setcounter{secnumdepth}{4}

\titleformat{\paragraph}
{\normalfont\normalsize\bfseries}{\theparagraph}{1em}{}
\titlespacing*{\paragraph}
{0pt}{3.25ex plus 1ex minus .2ex}{1.5ex plus .2ex}

\title{Sketchboard}

\begin{document}

\maketitle

\newpage

\section{Basic concepts}

\subsection{With extra capital internal index on embbeding fields}

Promote embedding fields $X^{\mu}$ to have an internal group index with $D$ values
\begin{equation*}
    X^{\mu} \rightarrow X^{\mu I} \ , \ \ I = 0 , ... , d
\end{equation*}
\begin{equation*}
    g_{a b} = 2 f \partial _{a} X^{\mu} \partial _{b} X^{\nu} G_{\mu \nu} \rightarrow g_{a b} ^{I J} \sim f D_{a} X^{\mu I} D_{b} X^{\nu J} G_{\mu \nu}
\end{equation*}
\begin{equation*}
    D_{a} X^{\mu I} = \partial _{a} X^{\mu I} + \omega ^{I} _{a J} X^{\mu J}
\end{equation*}
\begin{equation*}
    e^{i} _{a} e^{j} _{b} \eta _{i j} = g_{a b} \rightarrow e^{i} _{a} e^{j} _{b} \eta _{i j} = \Tr (g_{a b} ^{I J} T_{I} T_{J}) = g_{a b} ^{I J} \eta _{I J} \ , \ \ i , j = 0 , 1
\end{equation*}
\begin{equation*}
    g = \det (g_{a b}) \rightarrow g = \det (\Tr (g_{a b} ^{I J} T_{I} T_{J})) = \det (g_{a b} ^{I J} \eta _{I J}) = \det (e^{i} _{a} e^{j} _{b} \eta _{i j}) = - \det (e)^{2} \implies \sqrt{- g} = \det (e)
\end{equation*}

\subsection{With 1 extra small internal index on embedding fields}

Promote embbeding fields to have a internal group index with 2 values
\begin{equation*}
    X^{\mu} \rightarrow X^{\mu i} \ , \ \ i = 0 , 1
\end{equation*}
\begin{equation*}
    g_{a b} = 2 f \partial _{a} X^{\mu} \partial _{b} X^{\nu} G_{\mu \nu} \rightarrow g_{a b} ^{i j} \sim f D_{a} X^{\mu i} D_{b} X^{\nu j} G_{\mu \nu}
\end{equation*}
\begin{equation*}
    D_{a} X^{\mu i} = \partial _{a} X^{\mu i} + \omega ^{i} _{a j} X^{\mu j}
\end{equation*}
\begin{equation*}
    e^{i} _{a} e^{j} _{b} \eta _{i j} = g_{a b} \rightarrow e^{i} _{a} e^{j} _{b} \eta _{i j} = \Tr (g_{a b} ^{i j} T_{i} T_{j}) = g_{a b} ^{i j} \eta _{i j} \implies e^{i} _{a} e^{j} _{b} = g_{a b} ^{i j}
\end{equation*}
\begin{equation*}
    g = \det (g_{a b}) \rightarrow g = \det (\Tr (g_{a b} ^{i j} T_{i} T_{j})) = \det (g_{a b} ^{i j} \eta _{i j}) = \det (e^{i} _{a} e^{j} _{b} \eta _{i j}) = - \det (e)^{2} \implies \sqrt{- g} = \det (e)
\end{equation*}

\subsection{With 2 extra small indices}

Promote embbeding fields $X^{\mu}$ to have two internal indices with 2 values
\begin{equation*}
    X^{\mu} \rightarrow X^{\mu i j} \ , \ \ i , j = 0 , 1
\end{equation*}

\subsection{Without extra index on embbeding fields}

Change from WS metric to WS zweibein and connection (which vanishes since in 2d metric is conformally flat)
\begin{equation*}
    e_{a} ^{i} e_{b} ^{j} \eta _{i j} = g_{a b} = 2 f \partial _{a} X^{\mu} \partial _{b} X^{\nu} G_{\mu \nu}
\end{equation*}
\begin{equation*}
    g = \det (g_{a b}) = \det (e_{a} ^{i} e_{b} ^{j} \eta _{i j}) = - \det (e)^{2} \implies \sqrt{- g} = \det (e)
\end{equation*}

\section{Building an Action}

Start with Polyakov action in curved space-time
\begin{equation*}
    S_{P} = - \frac{T_{0}}{2} \int \diff \tau \wedge \diff \sigma \sqrt{- g} g^{a b} \partial _{a} X^{\mu} \partial _{b} X^{\nu} G_{\mu \nu} \ ,
\end{equation*}

\subsection{With capital internal index}

... and promote partial derivative $\partial _{a}$ to covariant derivative $D_{a}$, giving us our first attempt at modified Polyakov action
\begin{equation*}
    S_{M P 1} = - \frac{T_{0}}{2} \int \diff \tau \wedge \diff \sigma \det (e) \eta ^{i j} e^{a} _{i} e^{b} _{j} D_{a} X^{\mu I} D_{b} X^{\nu J} E_{\mu} ^{K} E_{\nu K} \eta _{I J}
\end{equation*}
\begin{equation*}
    \updownarrow
\end{equation*}
\begin{equation*}
    \Lagr _{M P 1} = - \frac{T_{0}}{2} \det (e) \eta ^{i j} e^{a} _{i} e^{b} _{j} D_{a} X^{\mu I} D_{b} X^{\nu J} E_{\mu} ^{K} E_{\nu K} \eta _{I J}
\end{equation*}

\subsection{With small internal index}

\subsubsection{Without extra field, 1 internal index}

... and promote derivatives to covariant $D_{a}$, giving us another modified Polyakov action
\begin{equation*}
    S_{M P 2} = - \frac{T_{0}}{2} \int \diff \tau \wedge \diff \sigma \det (e) e^{a} _{i} e^{b} _{j} D_{a} X^{\mu i} D_{b} X^{\mu j} E_{\mu} ^{I} (X) E_{\nu I} (X)
\end{equation*}
\begin{equation*}
    \updownarrow
\end{equation*}
\begin{equation*}
    \Lagr _{M P 2} = - \frac{T_{0}}{2} \det (e) e^{a} _{i} e^{b} _{j} D_{a} X^{\mu i} D_{b} X^{\mu j} E_{\mu} ^{I} (X) E_{\nu I} (X)
\end{equation*}

\subsubsection{With extra field}

..., promote partial derivative to covariant and add extra internal WS field $v^{i}$ leading us to
\begin{equation*}
    S_{M P 3} = - \frac{T}{2} \int \diff \tau \wedge \diff \sigma \det (e) \eta ^{i j} e^{a} _{i} e^{b} _{j} D_{a} X^{\mu} _{k} v^{k} D_{b} X^{\nu} _{l} v^{l} E_{\mu} ^{I} E_{\nu} ^{J} \eta _{I J}
\end{equation*}
\begin{equation*}
    \updownarrow
\end{equation*}
\begin{equation*}
    \Lagr _{M P 3} = - \frac{T}{2} \det (e) \eta ^{i j} e^{a} _{i} e^{b} _{j} D_{a} (X^{\mu} _{k} v^{k}) D_{b} (X^{\nu} _{l} v^{l}) E_{\mu} ^{I} E_{\nu} ^{J} \eta _{I J}
\end{equation*}

\subsubsection{Without extra field, 2 internal indices}

... and promote derivatives to covariant $D_{a}$,
\begin{equation*}
    S_{M P 4} = - \frac{T}{2} \int \diff \tau \wedge \diff \sigma e e^{a} _{i} e^{b i} D_{a} X^{\mu j j'} \eta _{j j'} \eta _{k k'} D_{b} X^{\nu k k'} E^{I} _{\mu} E_{\nu I}
\end{equation*}
\begin{equation*}
    \Lagr _{M P 4} = - \frac{T}{2} e e^{a} _{i} e^{b} _{j} D_{a} X^{\mu i k} \eta _{k l} D_{b} X^{\nu l j} E^{I} _{\mu} E_{\nu I}
\end{equation*}

\subsection{Without extra index}

... and swap to new set of variables giving us the dyad-Polyakov action
\begin{equation*}
    S_{D P} = - \frac{T_{0}}{2} \int \diff \tau \wedge \diff \sigma \det (e) e^{a} _{i} e^{b i} \partial _{a} X^{\mu} \partial _{b} X^{\nu} E_{\mu} ^{I} (X) E_{\nu I} (X) \ .
\end{equation*}
\begin{equation*}
    \updownarrow
\end{equation*}
\begin{equation*}
    \Lagr _{D P} = - \frac{T_{0}}{2} \det (e) e^{a} _{i} e^{b i} \partial _{a} X^{\mu} \partial _{b} X^{\nu} E_{\mu} ^{I} (X) E_{\nu I} (X) \ .
\end{equation*}

\subsection{Linear Polyakov action}

..., promote partial derivative to covariant derivative and build a linear action with inclusion of $D$-dimensional gamma matrices and bulk spinors
\begin{equation*}
    S_{L P} = - \frac{T}{2} \int \diff \tau \wedge \diff \sigma \overline{\psi} e e^{a} _{i} D_{a} X^{\mu i} E^{I} _{\mu} \gamma _{I} \psi
\end{equation*}
\begin{equation*}
    \updownarrow
\end{equation*}
\begin{equation*}
    \Lagr _{L P} = - \frac{T}{2} \overline{\psi} e e^{a} _{i} D_{a} X^{\mu i} E^{I} _{\mu} \gamma _{I} \psi
\end{equation*}

\section{EoMs}

Start by writing $\det (e) = \frac{1}{2} \varepsilon ^{c d} \varepsilon _{m n} e_{c} ^{m} e_{d} ^{n}$ and $g_{a b} = g_{a b} ^{i j} \eta _{i j}$

\subsection{With small internal index}

\subsubsection{Without extra field, w.r.t $e$}

\begin{align*}
    \frac{\delta S_{M P 2}}{\delta e^{e} _{l}} & = \frac{D \Lagr _{M P 2}}{D e^{e} _{l}} = \left( \frac{\partial}{\partial e^{e} _{l}} - \partial _{f} \frac{\partial}{\partial (\partial _{f} e^{e} _{l})} \right) \left( - \frac{T}{2} \frac{1}{2} \varepsilon ^{c d} \varepsilon _{m n} e^{m} _{c} e^{n} _{d} e^{a} _{i} e^{b} _{j} D_{a} X^{\mu i} D_{b} X^{\nu j} E_{\mu} ^{I} E_{\nu I} \right) = \\
                                               & = - \frac{T}{4} \varepsilon ^{c d} \varepsilon _{m n} ( \eta ^{m l} g_{c e} e^{n} _{d} e^{a} _{i} e^{b} _{j} + e^{m} _{c} \eta ^{n l} g_{d e} e^{a} _{i} e^{b} _{j} +                                                                                                                                                                                  \\
                                               & + e^{m} _{c} e^{n} _{d} \delta ^{a} _{e} \delta ^{l} _{i} e^{b} _{j} + e^{m} _{c} e^{n} _{d} e^{a} _{i} \delta ^{b} _{e} \delta ^{l} _{j} ) D_{a} X^{\mu i} D_{b} X^{\nu j} E_{\mu} ^{I} E_{\nu I} =                                                                                                                                                   \\
                                               & = - \frac{T}{2} \left( \varepsilon ^{c d} \varepsilon _{m n} \eta ^{m l} g_{c e} e^{n} _{d} e^{a} _{i} e^{b} _{j} + 2 \det (e) e^{a} _{i} \delta ^{b} _{e} \delta ^{l} _{j} \right) D_{a} X^{\mu i} D_{b} X^{\nu j} E_{\mu} ^{I} E_{\nu I} =                                                                                                           \\
                                               & = - \frac{T}{2} (\varepsilon ^{c d} \varepsilon _{m n} \eta ^{m l} e^{k} _{c} e_{e k} e^{n} _{d} e^{a} _{i} e^{b} _{j} + 2 \det (e) e^{a} _{i} \delta ^{b} _{e} \delta ^{l} _{j}) D_{a} X^{\mu i} D_{b} X^{\nu j} E_{\mu} ^{I} E_{\nu I} =                                                                                                             \\
                                               & = - \frac{T}{2} (\varepsilon ^{c d} \varepsilon _{m n} e^{m} _{c} e^{n} _{d} e^{l} _{e} e^{a} _{i} e^{b} _{j} + 2 \det (e) e^{a} _{i} \delta ^{b} _{e} \delta ^{l} _{j}) D_{a} X^{\mu i} D_{b} X^{\nu j} E_{\mu} ^{I} E_{\nu I} =                                                                                                                      \\
                                               & = -T (- \det (e) e^{l} _{e} e^{a} _{i} e^{b} _{j} + \det (e) e^{a} _{i} \delta ^{b} _{e} \delta ^{l} _{j}) D_{a} X^{\mu i} D_{b} X^{\nu j} E_{\mu} ^{I} E_{\nu I} \overset{!}{=} 0
\end{align*}
\begin{equation*}
    \downarrow
\end{equation*}
\begin{equation*}
    T^{l} _{e} := E_{\mu} ^{I} E_{\nu I} (e^{a} _{i} D_{a} X^{\mu i} D_{e} X^{\nu l} - e^{l} _{e} e^{a} _{i} e^{b} _{j} D_{a} X^{\mu i} D_{b} X^{\nu j}) = 0
\end{equation*}
\begin{equation*}
    \Downarrow
\end{equation*}
\begin{equation*}
    e^{l} _{e} = f e^{a} _{i} D_{a} X^{\mu i} D_{e} X^{\nu l} E_{\mu} ^{I} E_{\nu I} \ ,
\end{equation*}
\begin{equation*}
    \frac{1}{f} = e^{a} _{i} e^{b} _{j} D_{a} X^{\mu i} D_{b} X^{\nu j} E_{\mu} ^{I} E_{\nu I}
\end{equation*}
\begin{equation*}
    \downarrow
\end{equation*}
\begin{align*}
    e^{l} _{e} e_{f l} & = (f e^{a} _{i} D_{a} X^{\mu i} D_{e} X^{\nu l} E^{I} _{\mu} E_{\nu I}) (f e^{a'} _{i'} D_{a'} X^{\mu ' i'} D_{f} X^{\nu '} _{l} E^{I'} _{\mu '} E_{\nu ' I'}) = \\
                       & = f^{2} D_{e} X^{\nu l} D_{f} X^{\nu '} _{l} E^{I'} _{\nu} E_{\nu ' I'} (e^{a} _{i} e^{a'} _{i'} D_{a} X^{\mu i} D_{a'} X^{\mu ' i'} E^{I} _{\mu} E_{\mu ' I}) = \\
                       & = f D_{e} X^{\mu i} D_{f} X^{\nu} _{i} E^{I} _{\mu} E_{\nu I}
\end{align*}
\begin{equation*}
    \Downarrow
\end{equation*}
\begin{equation*}
    D_{e} X^{\mu i} D_{f} X^{\nu} _{i} E^{I} _{\mu} E_{\nu I} = 2 \partial _{e} X^{\mu} \partial _{f} X^{\nu} G_{\mu \nu}
\end{equation*}
\begin{equation*}
    e^{a} _{i} e^{b} _{j} D_{a} X^{\mu i} D_{b} X^{\nu j} E^{I} _{\mu} E_{\nu I} = g^{a b} \partial _{a} X^{\mu} \partial _{b} X^{\nu} G_{\mu \nu}
\end{equation*}
\begin{equation*}
    \Downarrow
\end{equation*}
\begin{equation*}
    D_{a} X^{\mu 0} = - i \partial _{a} X^{\mu}
\end{equation*}
\begin{equation*}
    D_{a} X^{\mu 1} = \partial _{a} X^{\mu}
\end{equation*}
\begin{equation*}
    \Downarrow
\end{equation*}
\begin{equation*}
    X^{\mu 0} = - i X^{\mu 1}
\end{equation*}

\subsubsection{With extra field, w.r.t $e$}

\begin{align*}
    \frac{\delta S_{M P 3}}{\delta e^{e} _{o}} & = \frac{D \Lagr _{M P 3}}{D e^{e} _{o}} = \left( \frac{\partial}{\partial e^{e} _{o}} - \partial _{f} \frac{\partial}{\partial (\partial _{f} e^{e} _{o})} \right) \left( - \frac{T}{2} \frac{1}{2} \varepsilon ^{c d} \varepsilon _{m n} e^{m} _{c} e^{n} _{d} \eta ^{i j} e^{a} _{i} e^{b} _{j} D_{a} (X^{\mu} _{k} v^{k}) D_{b} (X^{\nu} _{l} v^{l}) E_{\mu} ^{I} E_{\nu} ^{J} \eta _{I J} \right) = \\
                                               & = - \frac{T}{4} \varepsilon ^{c d} \varepsilon _{m n} (g_{c e} \eta ^{m o} e^{n} _{d} \eta ^{i j} e^{a} _{i} e^{b} _{j} + e^{m} _{c} g_{d e} \eta ^{n o} \eta ^{i j} e^{a} _{i} e^{b} _{j} +                                                                                                                                                                                                            \\
                                               & + e^{m} _{c} e^{n} _{d} \eta ^{i j} \delta ^{a} _{e} \delta ^{o} _{i} e^{b} _{j} + e^{m} _{c} e^{n} _{d} \eta ^{i j} e^{a} _{i} \delta ^{b} _{e} \delta ^{o} _{j}) D_{a} (X^{\mu} _{k} v^{k}) D_{b} (X^{\nu} _{l} v^{l}) E^{I} _{\mu} E^{J} _{\nu} \eta _{I J} =                                                                                                                                        \\
                                               & = - \frac{T}{2} \varepsilon ^{c d} \varepsilon _{m n} (e_{c} ^{p} e_{e p} e^{n} _{d} \eta ^{m o} \eta ^{i j} e^{a} _{i} e^{b} _{j} + 2 \det (e) e^{a o} \delta ^{b} _{e})  D_{a} (X^{\mu} _{k} v^{k}) D_{b} (X^{\nu} _{l} v^{l}) E^{I} _{\mu} E^{J} _{\nu} \eta _{I J} =                                                                                                                                \\
                                               & = - T (- \det (e) e^{o} _{e} \eta ^{i j} e^{a} _{i} e^{b} _{j} + \det (e) e^{a o} \delta ^{b} _{e}) D_{a} (X^{\mu} _{k} v^{k}) D_{b} (X^{\nu} _{l} v^{l}) E^{I} _{\mu} E^{J} _{\nu} \eta _{I J} \overset{!}{=} 0
\end{align*}
\begin{equation*}
    \downarrow
\end{equation*}
\begin{equation*}
    \boxed{T^{o} _{e} := (e^{a o} D_{a} (X^{\mu} _{k} v^{k}) D_{e} (X^{\nu} _{l} v^{l}) - e^{o} _{e} \eta ^{i j} e^{a} _{i} e^{b} _{j} D_{a} (X^{\mu} _{k} v^{k}) D_{b} (X^{\nu} _{l} v^{l})) E^{I} _{\mu} E^{J} _{\nu} \eta _{I J} = 0}
\end{equation*}
\begin{equation*}
    \Downarrow
\end{equation*}
\begin{equation*}
    e^{o} _{e} = f \, e^{a o} D_{a} (X^{\mu} _{k} v^{k}) D_{e} (X^{\nu} _{l} v^{l}) E^{I} _{\mu} E^{J} _{\nu} \eta _{I J} \ ,
\end{equation*}
\begin{equation*}
    \frac{1}{f} = \eta ^{i j} e^{a} _{i} e^{b} _{j} D_{a} (X^{\mu} _{k} v^{k}) D_{b} (X^{\nu} _{l} v^{l}) E^{I} _{\mu} E^{J} _{\nu} \eta _{I J}
\end{equation*}

\subsubsection{With extra field, w.r.t $\omega$}

\begin{align*}
    \frac{\delta S_{M P 3}}{\delta \omega ^{m n} _{c}} & = \frac{D \Lagr _{M P 3}}{D \omega ^{m n} _{c}} = \left( \frac{\partial}{\partial \omega ^{m n} _{c}} - \partial _{d} \frac{\partial}{\partial (\partial _{d} \omega ^{m n} _{c})} \right) \left( - \frac{T}{2} \det (e) \eta ^{i j} e^{a} _{i} e^{b} _{j} D_{a} X^{\mu} _{k} v^{k} D_{b} X^{\nu} _{l} v^{l} E_{\mu} ^{I} E_{\nu} ^{J} \eta _{I J} \right) = \\
                                                       & = - \frac{T}{4} e g^{a b} ((- \delta ^{c} _{a} \delta ^{k'} _{[m} \eta _{n] k} X^{\mu} _{k'} v^{k}) D_{b} X^{\nu} _{l} v^{l} + D_{a} X^{\mu} _{k} v^{k} (- \delta ^{c} _{b} \delta ^{l'} _{[m} \eta _{n] l} X^{\nu} _{l'} v^{l})) G_{\mu \nu} =                                                                                                              \\
                                                       & = \frac{T}{4} e (g^{c a} X^{\mu} _{[m} v_{n]} D_{a} X^{\nu} _{k} v^{k} + g^{a c} D_{a} X^{\mu} _{k} v^{k} X^{\nu} _{[m} v_{n]}) G_{\mu \nu} =                                                                                                                                                                                                                \\
                                                       & = \frac{T}{2} e g^{a c} D_{a} X^{\mu} _{k} v^{k} X^{\nu} _{[m} v_{n]} G_{\mu \nu} \overset{!}{=} 0
\end{align*}
\begin{equation*}
    \downarrow
\end{equation*}
\begin{equation*}
    \boxed{\mathcal{T}^{i} _{a b} := e^{c i} D_{c} X^{\mu} _{k} v^{k} X^{\nu} _{[m} v_{n]} e^{m} _{a} e^{n} _{b} G_{\mu \nu} = 0}
\end{equation*}

\subsubsection{With extra field, w.r.t $v$}

\begin{align*}
    \frac{\delta S_{M P 3}}{\delta v^{l}} & = \frac{D \Lagr _{M P 3}}{D v^{l}} = \left( \frac{\partial}{\partial v^{l}} - \partial _{c} \frac{\partial}{\partial (\partial _{c} v^{l})} \right) \left( - \frac{T}{2} e e^{a} _{i} e^{b i} D_{a} X^{\mu} _{j} v^{j} D_{b} X^{\nu} _{k} v^{k} E^{I} _{\mu} E_{\nu I} \right) = \\
                                          & = - \frac{T}{2} g^{a b} (D_{a} X^{\mu} _{j} \delta ^{j} _{l} D_{b} X^{\nu} _{k} v^{k} + D_{a} X^{\mu} _{j} v^{j} D_{b} X^{\nu} _{k} \delta ^{k} _{l}) G_{\mu \nu} =                                                                                                              \\
                                          & = - T g^{a b} D_{a} X^{\mu} _{l} D_{b} X^{\nu} _{j} v^{j} G_{\mu \nu} \overset{!}{=} 0
\end{align*}
\begin{equation*}
    \downarrow
\end{equation*}
\begin{equation*}
    \boxed{g^{a b} D_{a} X^{\mu} _{l} D_{b} X^{\nu} _{j} v^{j} G_{\mu \nu} = 0}
\end{equation*}

\subsubsection{With extra field, w.r.t $X$}

\begin{align*}
    \frac{\delta S_{M P 3}}{\delta X^{\lambda} _{l}} & = \frac{D \Lagr _{M P 3}}{D X^{\lambda} _{l}} = \left( \frac{\partial}{\partial X^{\lambda} _{l}} - \partial _{c} \frac{\partial}{\partial (\partial _{c} X^{\lambda} _{l})} \right) \left( - \frac{T}{2} e e^{a} _{i} e^{b i} D_{a} X^{\mu} _{j} v^{j} D_{b} X^{\nu} _{k} v^{k} E^{I} _{\mu} E_{\nu I}  \right) = \\
                                                     & = - \frac{T}{2} \Bigg( e g^{a b} \Bigg( (- \omega ^{j'} _{a j} \delta ^{\mu} _{\lambda} \delta ^{l} _{j'} v^{j} D_{b} X^{\nu} _{k} v^{k} - D_{a} X^{\mu} _{j} v^{j} \omega ^{k'} _{b k} \delta ^{\nu} _{\lambda} \delta ^{l} _{k'} v^{k}) E^{I} _{\mu} E_{\nu I} +                                                 \\
                                                     & + e g^{a b} D_{a} X^{\mu} _{j} v^{j} D_{b} X^{\nu} _{k} v^{k} \left( \frac{\partial E^{I} _{\mu}}{\partial X^{\lambda} _{l}} E_{\nu I} + E^{I} _{\mu} \frac{\partial E_{\nu I}}{\partial X^{\lambda} _{l}} \right) \Bigg) -                                                                                        \\
                                                     & - \partial _{c} (e g^{a b} (\delta ^{c} _{a} \delta ^{\mu} _{\lambda} \delta ^{l} _{j} v^{j} D_{b} X^{\nu} _{k} v^{k} + D_{a} X^{\mu} _{j} v^{j} \delta ^{c} _{b} \delta ^{\nu} _{\lambda} \delta ^{l} _{k} v^{k}) G_{\mu \nu}) \Bigg) =                                                                           \\
                                                     & = - T \Bigg( - e g^{a b} D_{a} X^{\mu} _{j} v^{j} \omega ^{l} _{b k} v^{k} G_{\mu \lambda} + e g^{a b} D_{a} X^{\mu} _{j} v^{j} D_{b} X^{\nu} _{k} v^{k} \frac{\partial E^{I} _{\mu}}{\partial X^{\lambda} _{l}} E_{\nu I} - \partial _{c} (e g^{a c} v^{l} D_{a} X^{\mu} _{j} v^{j} G_{\mu \lambda}) \Bigg)       \\
                                                     & = - T \Bigg( e g^{a b} D_{a} X^{\mu} _{j} v^{j} D_{b} X^{\nu} _{k} v^{k} \frac{\partial E^{I} _{\mu}}{\partial X^{\lambda} _{l}} E_{\nu I} - D_{b} (e g^{a b} v^{l} D_{a} X^{\mu} _{j} v^{j} E^{I} _{\mu} E_{\lambda I}) \Bigg) \overset{!}{=} 0
\end{align*}
\begin{equation*}
    \downarrow
\end{equation*}
\begin{equation*}
    \boxed{D_{b} (e e^{a} _{i} e^{b i} v^{l} D_{a} X^{\mu} _{j} v^{j} E^{I} _{\mu} E_{\lambda I}) = e e^{a} _{i} e^{b i} D_{a} X^{\mu} _{j} v^{j} D_{b} X^{\nu} _{k} v^{k} \frac{\partial E^{I} _{\mu}}{\partial X^{\lambda} _{l}} E_{\nu I}}
\end{equation*}

\subsection{Without extra index}

\subsubsection{w.r.t $e$}

\begin{align*}
    \frac{\delta S_{D P}}{\delta e^{e} _{l}} & = \frac{D \Lagr _{D P}}{D e^{e} _{l}} = \left( \frac{\partial}{\partial e^{e} _{l}} - \partial _{f} \frac{\partial}{\partial (\partial _{f} e^{e} _{l})} \right) \left( - \frac{T_{0}}{2} \frac{1}{2} \varepsilon ^{c d} \varepsilon _{m n} e^{m} _{c} e^{n} _{d} \eta ^{i j} e^{a} _{i} e^{b} _{j} \partial _{a} X^{\mu} \partial _{b} X^{\nu} E^{I} _{\mu} E^{J} _{\nu} \eta _{I J} \right) = \\
                                             & = - \frac{T_{0}}{4}
\end{align*}

\subsection{Linear action}

\subsubsection{w.r.t $e$}

\begin{align*}
    \frac{\delta S_{L P}}{\delta \delta e^{b} _{j}} & = \frac{D \Lagr _{L P}}{D e^{b} _{j}} = \left( \frac{\partial}{\partial e^{b} _{j}} - \partial _{e} \frac{\partial}{\partial (\partial _{e} e^{b} _{j})} \right) \left( - \frac{T}{2} \overline{\psi} e e^{a} _{i} D_{a} X^{\mu i} E^{I} _{\mu} \gamma _{I} \psi \right) = \\
                                                    & = - \frac{T}{4} \overline{\psi} \varepsilon ^{c d} \varepsilon _{m n} ((g_{c b} \eta ^{m j} e^{n} _{d} + e^{m} _{c} g_{d b} \eta ^{n j}) e^{a} _{i} + e^{m} _{c} e^{n} _{d} \delta ^{a} _{b} \delta ^{j} _{i}) D_{a} X^{\mu i} E^{I} _{\mu} \gamma _{I} \psi =             \\
                                                    & = - \frac{T}{2} \overline{\psi} (- e e^{j} _{b} e^{a} _{i} D_{a} X^{\mu i} E^{I} _{\mu} \gamma _{I} + e D_{b} X^{\mu j} E^{I} _{\mu} \gamma _{I}) \psi \overset{!}{=} 0
\end{align*}
\begin{equation*}
    \downarrow
\end{equation*}
\begin{equation*}
    T^{j} _{b} := \overline{\psi} (D_{b} X^{\mu j} E^{I} _{\mu} \gamma _{I} - e^{j} _{b} e^{a} _{i} D_{a} X^{\mu i} E^{I} _{\mu} \gamma _{I}) \psi = 0
\end{equation*}
\begin{equation*}
    \downarrow
\end{equation*}
\begin{equation*}
    e^{j} _{b} = F \overline{\psi} D_{b} X^{\mu j} E^{I} _{\mu} \gamma _{I} \psi ,
\end{equation*}
\begin{equation*}
    \frac{1}{F} = \overline{\psi} e^{a} _{i} D_{a} X^{\mu i} E^{I} _{\mu} \gamma _{I} \psi
\end{equation*}
\begin{equation*}
    \Downarrow
\end{equation*}
\begin{equation*}
    \overline{\psi} \psi = 1 , \ \psi \overline{\psi} = \mathbb{1}
\end{equation*}
\begin{equation*}
    e^{a} _{i} e^{b} _{j} D_{a} X^{\mu i} D_{b} X^{\nu j} E^{I} _{\mu} E^{J} _{\nu} \eta _{I J} = g^{a b} \partial _{a} X^{\mu} \partial _{b} X^{\nu} G_{\mu \nu}
\end{equation*}

\section{Inverse Area Action}

\subsection{Nambu-Goto}

Start from Nambu-Goto action
\begin{equation*}
    S_{N G} = - T \int \diff ^2 x \sqrt{- h} ,
\end{equation*}
and make quantum geometry correction
\begin{equation*}
    \sqrt{- h} \rightarrow \sqrt{- (h + g \Delta)} \approx \sqrt{- h} \left( 1 - \frac{g \Delta}{2 (- h)} + \mathcal{O} \left( \frac{g^2}{h^2} \right) \right) ,
\end{equation*}
leading to modified NG action
\begin{equation*}
    S_{M N G} = - T \int \diff ^2 x \left( \sqrt{- h} - \frac{g \Delta}{2 \sqrt{- h}} \right) = S_{N G} + S_{I A}
\end{equation*}
\begin{equation*}
    \downarrow
\end{equation*}
\begin{equation*}
    \Lagr _{M N G} = - T \left( \sqrt{- h} - \frac{g \Delta}{2 \sqrt{- h}} \right)
\end{equation*}

\subsubsection{EoMs}

Let $\PS ^{\tau} _{\mu} \equiv \partial \Lagr _{M N G} / \partial \Dot{X}^{\mu}$ and $\PS ^{\sigma} _{\mu} \equiv \partial \Lagr _{M N G} / \partial X^{\prime \mu}$

\begin{align*}
    \delta S_{M N G} & = \int \diff ^2 x \left( \frac{\partial \Lagr _{M N G}}{\partial \Dot{X}^{\mu}} \delta \Dot{X}^{\mu} + \frac{\partial \Lagr _{M N G}}{\partial X^{\prime \mu}} \delta X^{\prime \mu} \right) =                                                   \\
                     & = - \int \diff ^2 x \left( \partial _{\tau} \PS ^{\tau} _{\mu} + \partial _{\sigma} \PS ^{\sigma} _{\mu} \right) \delta X^{\mu} + \int \diff \tau \PS ^{\sigma} _{\mu} \delta X^{\mu} \big| _{\sigma = 0} ^{\sigma = \sigma _1} \overset{!}{=} 0
\end{align*}
\begin{equation*}
    \downarrow
\end{equation*}
\begin{equation*}
    \mathrm{EoM:} \partial _{\tau} \PS ^{\tau} _{\mu} + \partial _{\sigma} \PS ^{\sigma} _{\mu} = 0
\end{equation*}
\begin{equation*}
    \mathrm{B.C.:} \PS ^{\sigma} _{\mu} \delta X^{\mu} \big| _{\sigma = 0} ^{\sigma = \sigma _1} = 0 ,
\end{equation*}
where
\begin{equation*}
    \PS ^{\tau} _{\mu} = \frac{\partial \Lagr _{M N G}}{\partial \Dot{X}^{\mu}} = - T \left( \frac{(\Dot{X} \cdot X') X' _{\mu} - (X')^2 \Dot{X}_{\mu}}{\sqrt{- h}} + \frac{g \Delta}{2} \frac{(\Dot{X} \cdot X') X' _{\mu} - (X')^2 \Dot{X}_{\mu}}{(- h)^{3/2}} \right) =
\end{equation*}
\begin{equation*}
    = - T \frac{(\Dot{X} \cdot X') X' _{\mu} - (X')^2 \Dot{X}_{\mu}}{\sqrt{- h}} \left( 1 + \frac{g \Delta}{2 (- h)} \right) = \PS ^{\tau} _{\mu (N G)} \left( 1 + \frac{g \Delta}{2 (- h)} \right)
\end{equation*}
\begin{equation*}
    \PS ^{\sigma} _{\mu} = \frac{\partial \Lagr _{M N G}}{\partial X^{\prime \mu}} = - T \left( \frac{(\Dot{X} \cdot X') \Dot{X}_{\mu} - (\Dot{X})^2 X' _{\mu}}{\sqrt{- h}} + \frac{g \Delta}{2} \frac{(\Dot{X} \cdot X') \Dot{X}_{\mu} - (\Dot{X})^2 X' _{\mu}}{(- h)^{3/2}} \right) =
\end{equation*}
\begin{equation*}
    = - T \frac{(\Dot{X} \cdot X') \Dot{X}_{\mu} - (\Dot{X})^2 X' _{\mu}}{\sqrt{- h}} \left( 1 + \frac{g \Delta}{2 (- h)} \right) = \PS ^{\sigma} _{\mu (N G)} \left( 1 + \frac{g \Delta}{2 (- h)} \right)
\end{equation*}
gauge fixing static gauge $\tau = t$ and transverse gauge $\frac{\partial X}{\partial \tau} \cdot \frac{\partial X}{\partial s} \frac{\diff s}{\diff \sigma} = 0$ ($s = \mathrm{length \ along \ string}$)
\begin{equation*}
    \downarrow
\end{equation*}
\begin{equation*}
    \PS ^{\tau \mu} _{(N G)} = T \frac{\diff s}{\diff \sigma} \gamma _{v_{\perp}} \frac{\partial X^{\mu}}{\partial t}
\end{equation*}
\begin{equation*}
    \PS ^{\sigma \mu} _{(N G)} = \frac{T}{\gamma _{v_{\perp}}} \frac{\partial X^{\mu}}{\partial s}
\end{equation*}
\begin{equation*}
    - h = \frac{1}{\gamma _{v_{\perp}} ^2} \left( \frac{\diff s}{\diff \sigma} \right) ^2
\end{equation*}
string energy get's redefined to
\begin{equation*}
    \frac{\partial}{\partial t} \left( T \frac{\diff s}{\diff \sigma} \gamma _{v_{\perp}} \left( 1 + \frac{g \Delta}{2 (- h)} \right) \right) = 0,
\end{equation*}
and for the spatial part we have
\begin{equation*}
    \partial _{\tau} \Vec{\PS}^{\tau} + \partial _{\sigma} \Vec{\PS}^{\sigma} = 0
\end{equation*}
\begin{equation*}
    \downarrow
\end{equation*}
\begin{equation*}
    \frac{\partial}{\partial t} \left[ T \frac{\diff s}{\diff \sigma} \gamma _{v_{\perp}} \frac{\partial \Vec{X}}{\partial t} \left( 1 + \frac{g \Delta}{2 (- h)} \right) \right] + \frac{\diff s}{\diff \sigma} \frac{\partial}{\partial s} \left[ - \frac{T}{\gamma _{v_{\perp}}} \frac{\partial \Vec{X}}{\partial s} \left( 1 + \frac{g \Delta}{2 (- h)} \right) \right] = 0
\end{equation*}
\begin{equation*}
    \downarrow
\end{equation*}
\begin{equation*}
    \mu \gamma _{v_{\perp}} \left( 1 + \frac{g \Delta}{2 (- h)} \right) \frac{\partial ^2 \Vec{X}}{\partial t^2} - \frac{\partial}{\partial s} \left[ \frac{T}{\gamma _{v_{\perp}}} \left( 1 + \frac{g \Delta}{2 (- h)} \right) \frac{\partial \Vec{X}}{\partial s} \right] = 0
\end{equation*}
$\implies$ effective mass density becomes $\mu _{eff} = \mu \gamma _{v_{\perp}} \left( 1 - \frac{g \Delta}{2 (- h)} \right)$ and effective tension becomes $T_{eff} = \frac{T}{\gamma _{v_{\perp}}} \left( 1 - \frac{g \Delta}{2 (- h)} \right)$
\begin{equation*}
    \downarrow
\end{equation*}
\begin{equation*}
    \mu _{eff} \frac{\partial ^2 \Vec{X}}{\partial t^2} - \frac{\partial}{\partial s} \left[ T_{eff} \frac{\partial \Vec{X}}{\partial s} \right] = 0
\end{equation*}
\begin{align*}
    \mathscr{H} & = \Dot{\Vec{X}} \cdot \Vec{\pi} - \Lagr =                                                                                                                                                                                                                                           \\
                & = \Vec{v}_{\perp} \cdot \left( T \frac{\diff s}{\diff \sigma} \gamma _{v_{\perp}} \left( 1 + \frac{g \Delta}{2 (- h)} \right) \Vec{v}_{\perp} \right) - \left( - T \frac{\diff s}{\diff \sigma} \frac{1}{\gamma _{v_{\perp}}} \left( 1 + \frac{g \Delta}{2 (- h)} \right) \right) = \\
                & = T \frac{\diff s}{\diff \sigma} \left( 1 + \frac{g \Delta}{2 (- h)} \right) \left( \gamma _{v_{\perp}} v_{\perp} ^2 + \frac{1}{\gamma _{v_{\perp}}} \right) =                                                                                                                      \\
                & = T \frac{\diff s}{\diff \sigma} \gamma _{v_{\perp}} \left( 1 + \frac{g \Delta}{2 (- h)} \right)
\end{align*}
let $\left( 1 + \frac{g \Delta}{2 (- h)} \right) = F$
\begin{equation*}
    \frac{\partial ^2 \Vec{X}}{\partial t^2} - \frac{1}{F \gamma _{v_{\perp}}} \frac{\diff \sigma}{\diff s} \frac{\partial}{\partial \sigma} \left[ \frac{1}{\gamma _{v_{\perp}}} F \frac{\diff \sigma}{\diff s} \frac{\partial \Vec{X}}{\partial \sigma} \right] = 0
\end{equation*}
\begin{equation*}
    \downarrow
\end{equation*}
\begin{equation*}
    A (\sigma) = \frac{\gamma _{v_{\perp}}}{F} \frac{\diff s}{\diff \sigma} \overset{!}{=} 1
\end{equation*}
\begin{equation*}
    \downarrow
\end{equation*}
\begin{equation*}
    \diff \sigma = \frac{\gamma _{v_{\perp}}}{F} \diff s = \frac{1}{T F^2} \diff E \implies \sigma (q) = \frac{1}{T} \int _{0} ^{q} \frac{1}{F^2} \diff E
\end{equation*}
\begin{equation*}
    \downarrow
\end{equation*}
\begin{equation*}
    F^2 \frac{\partial ^2 \Vec{X}}{\partial t^2} - \frac{\partial ^2 \Vec{X}}{\partial \sigma ^2} = 0
\end{equation*}
$\implies$ speed of wave on the string gets modified by correction factor $v = c / F$.
\begin{equation*}
    - h = \frac{1}{\gamma _{v_{\perp}} ^2} \left( \frac{\diff s}{\diff \sigma} \right) ^2 = \frac{F^2}{\gamma _{v_{\perp}} ^4} = F^2 \left( 1 - v_{\perp} ^2 \right) ^2
\end{equation*}
\begin{equation*}
    \downarrow
\end{equation*}
\begin{equation*}
    - h = \left( 1 + \frac{g \Delta}{2 (- h)} \right) ^2 \left( 1 - v_{\perp} ^2 \right) ^2
\end{equation*}
\begin{equation*}
    \downarrow
\end{equation*}
\begin{equation*}
    \frac{- h}{\left( 1 + \frac{g \Delta}{2 (- h)} \right) ^2} = \left( 1 - v_{\perp} ^2 \right) ^2
\end{equation*}
\begin{equation*}
    \ \ \ \ \ \ \ \ \ \ \ \ \ \ \ \ \ \ \ \ \ \ \ \downarrow (\mathrm{WolframAlpha})
\end{equation*}
\begin{equation*}
    - h = \frac{(1 - v_{\perp}^2)^2}{3} - f_1 - f_2 ,
\end{equation*}
where (let $a = \frac{g \Delta}{2}$ and $b = (1 - v_{\perp}^2)^2$)
\begin{equation*}
    f_1 = \frac{\sqrt[3]{- 27 a^2 b + 3 \sqrt{3} \sqrt{27 a^4 b^2 + 4 a^3 b^3} - 18 a b^2 - 2 b^3}}{3 \sqrt[3]{2}}
\end{equation*}
\begin{equation*}
    f_2 = \frac{(6 a b + b^2)}{6 f_1}
\end{equation*}
\begin{equation*}
    \downarrow
\end{equation*}
\begin{equation*}
    F = 1 + \frac{g \Delta}{2 (- h)} = 1 + \frac{g \Delta}{2 \left( \frac{b}{3} - f_1 - f_2 \right)}
\end{equation*}
\begin{equation*}
    \downarrow
\end{equation*}
\begin{equation*}
    v = \frac{1}{F} = \frac{1}{1 + \dfrac{g \Delta}{2 \left( \dfrac{b}{3} - f_1 - f_2 \right)}}
\end{equation*}

\section{Bimetric Polyakov}

Start with Polyakov action
\begin{equation*}
    S_{P} = - \frac{T}{2} \int \diff ^2 x \sqrt{- g} g^{a b} \partial _{a} X^{\mu} \partial _{b} X^{\nu} G_{\mu \nu},
\end{equation*}
and promote to bimetric action
\begin{equation*}
    \downarrow
\end{equation*}
\begin{equation*}
    S_{B P} = - \frac{T}{2} \int \diff ^2 x \sqrt{- g} g^{a b} \partial _{a} X^{\mu} \partial _{b} X^{\nu} G_{\mu \nu} - \frac{T'}{2} \int \diff ^2 x \sqrt{- h} h^{a b} \partial _{a} X^{\mu} \partial _{b} X^{\nu} H_{\mu \nu}
\end{equation*}

\subsection{EoMs}

\begin{align*}
    \frac{\delta S_{B P}}{\delta g^{c d}} & = - \frac{T}{2} \left( \frac{\partial \sqrt{- g}}{\partial g^{c d}} g^{a b} + \sqrt{- g} \frac{\partial g^{a b}}{\partial g^{c d}} \right) \partial _{a} X^{\mu} \partial _{b} X^{\nu} G_{\mu \nu} + 0 = \\
                                          & = - \frac{T}{2} \left( - \frac{1}{2} \sqrt{- g} g_{c d} g^{a b} + \sqrt{- g} \delta ^{a} _{(c} \delta ^{b} _{d)} \right) \partial _{a} X^{\mu} \partial _{b} X^{\nu} G_{\mu \nu} \overset{!}{=} 0
\end{align*}
\begin{equation*}
    T^{(G)} _{c d} := \left( \partial _{c} X^{\mu} \partial _{d} X^{\nu} - \frac{1}{2} g_{c d} g^{a b} \partial _{a} X^{\mu} \partial _{b} X^{\nu} \right) G_{\mu \nu} = 0
\end{equation*}
\begin{equation*}
    \Downarrow
\end{equation*}
\begin{equation*}
    g_{c d} = 2 f^{(G)} \partial _{c} X^{\mu} \partial _{d} X^{\nu} G_{\mu \nu} ,
\end{equation*}
\begin{equation*}
    \frac{1}{f^{(G)}} = g^{a b} \partial _{a} X^{\mu} \partial _{b} X^{\nu} G_{\mu \nu}
\end{equation*}

\begin{align*}
    \frac{\delta S_{B P}}{\delta h^{c d}} & = ... =                                                                                                                                                                                            \\
                                          & = - \frac{T'}{2} \left( - \frac{1}{2} \sqrt{- h} h_{c d} h^{a b} + \sqrt{- h} \delta ^{a} _{(c} \delta ^{b} _{d)} \right) \partial _{a} X^{\mu} \partial _{b} X^{\nu} H_{\mu \nu} \overset{!}{=} 0
\end{align*}
\begin{equation*}
    \downarrow
\end{equation*}
\begin{equation*}
    T^{(H)} _{c d} := \left( \partial _{c} X^{\mu} \partial _{d} X^{\mu} - \frac{1}{2} h_{c d} h^{a b} \partial _{a} X^{\mu} \partial _{b} X^{\nu} \right) H_{\mu \nu} = 0
\end{equation*}
\begin{equation*}
    \Downarrow
\end{equation*}
\begin{equation*}
    h_{c d} = 2 f^{(H)} \partial _{c} X^{\mu} \partial _{d} X^{\nu} H_{\mu \nu} ,
\end{equation*}
\begin{equation*}
    \frac{1}{f^{(H)}} = h^{a b} \partial _{a} X^{\mu} \partial _{b} X^{\nu} H_{\mu \nu}
\end{equation*}

\begin{align*}
    \frac{\delta S_{B P}}{\delta X^{\lambda}} & = \left( - \frac{T}{2} \sqrt{- g} g^{a b} \partial _{\lambda} G_{\mu \nu} - \frac{T'}{2} \sqrt{- h} h^{a b} \partial _{\lambda} H_{\mu \nu} \right) \partial _{a} X^{\mu} \partial _{b} X^{\nu} -                                                               \\
                                              & - \partial _{c} \left( - \frac{T}{2} \sqrt{- g} g^{a b} G_{\mu \nu} - \frac{T'}{2} \sqrt{- h} h^{a b} H_{\mu \nu} \right) (\delta ^{c} _{a} \delta ^{\mu} _{\lambda} \partial _{b} X^{\nu} + \partial _{a} X^{\mu} \delta ^{c} _{b} \delta ^{\nu} _{\lambda}) = \\
                                              & = \partial _{a} \left( T \sqrt{- g} g^{a b} G_{\mu \lambda} + T' \sqrt{- h} h^{a b} H_{\mu \lambda} \right) \partial _{b} X^{\mu} -                                                                                                                             \\
                                              & - \left( \frac{T}{2} \sqrt{- g} g^{a b} \partial _{\lambda} G_{\mu \nu} + \frac{T'}{2} \sqrt{- h} h^{a b} \partial _{\lambda} H_{\mu \nu} \right) \partial _{a} X^{\mu} \partial _{b} X^{\nu} \overset{!}{=} 0
\end{align*}
\begin{equation*}
    \downarrow
\end{equation*}
\begin{equation*}
    \partial _{a} \left( F^{a b} _{\mu \lambda} \partial _{b} X^{\mu} \right) = \frac{1}{2} \partial _{\lambda} F^{a b} _{\mu \nu} \partial _{a} X^{\mu} \partial _{b} X^{\nu} ,
\end{equation*}
\begin{equation*}
    F^{a b} _{\mu \nu} = T \sqrt{- g} g^{a b} G_{\mu \nu} + T' \sqrt{- h} h^{a b} H_{\mu \nu}
\end{equation*}
let $T' = T k \Delta / 2$ and $h = g^{-1} (H_{\mu \nu} = G_{\mu \nu})$
\begin{equation*}
    F^{a b} _{\mu \nu} = T \left( \sqrt{- g} g^{a b} G_{\mu \nu} + \frac{k \Delta}{2} \sqrt{- g^{-1}} U^{a} _{\ a'} (g^{-1})^{a' b'} U^{\ b} _{b'} (G^{-1})_{\mu \nu} \right) ,
\end{equation*}
add new field coupled to derivative $A_{a}$ transforming as
\begin{equation*}
    A_{a} g^{b c} \rightarrow \left( A_{a} g^{b c} \right) ' = A_{a} U^{b} _{\ b'} g^{b' c'} U^{\ c} _{c'} - \frac{1}{\alpha} g^{b' c'} (\partial _{a} U^{b} _{\ b'} U^{\ c} _{c'} + U^{b} _{\ b'} \partial _{a} U^{\ c} _{c'}) ,
\end{equation*}
defining new WS covariant derivative
\begin{equation*}
    D_{a} g^{b c} = \partial _{a} g^{b c} + \alpha A_{a} g^{b c} ,
\end{equation*}
such that derivative ignores $\mathrm{SO}(1,1)$ gauge freedom on inverse metric. Upgrade EoM to incorporate this derivative:
\begin{equation*}
    D_{a} (F^{a b} _{\mu \lambda} \partial _{b} X^{\mu}) = \frac{1}{2} \partial _{\lambda} F^{a b} _{\mu \nu} \partial _{a} X^{\mu} \partial _{b} X^{\nu}
\end{equation*}
\begin{equation*}
    \downarrow
\end{equation*}
\begin{equation*}
    F^{a b} _{\mu \lambda} \partial _{a} \partial _{b} X^{\mu} + D_{a} F^{a b} _{\mu \lambda} \partial _{b} X^{\mu} = \frac{1}{2} \partial _{\lambda} F^{a b} _{\mu \nu} \partial _{a} X^{\mu} \partial _{b} X^{\nu} ,
\end{equation*}
imposing conformal symmetry $g^{a b} = (\phi (x))^{-1} \eta ^{a b} \implies D_{a} F^{a b} _{\mu \lambda} = 0$
\begin{equation*}
    F^{a b} _{\mu \lambda} \partial _{a} \partial _{b} X^{\mu} = \frac{1}{2} \partial _{\lambda} F^{a b} _{\mu \nu} \partial _{a} X^{\mu} \partial _{b} X^{\nu} ,
\end{equation*}
\begin{equation*}
    F^{a b} _{\mu \lambda} = T \left( G_{\mu \lambda} + \frac{k \Delta}{2} (G^{-1})_{\mu \lambda} \right) \eta ^{a b}
\end{equation*}
\begin{equation*}
    (G^{-1})_{\mu \nu} \overset{?}{=} \frac{1}{- g} G_{\mu \nu}
\end{equation*}

\newpage

\section{Inverse Area Polyakov}

Start with Polyakov action
\begin{equation*}
    S_{P} = - \frac{T}{2} \int \diff ^2 x \sqrt{- g} g^{a b} \partial _{a} X^{\mu} \partial _{b} X^{\nu} G_{\mu \nu}
\end{equation*}
and make quantum geometry correction
\begin{equation*}
    \sqrt{- g} \rightarrow \sqrt{- (g + k \Delta)} \approx \sqrt{- g} \left( 1 + \frac{k \Delta}{2 g} + \mathcal{O} \left( \frac{k^2}{g^2} \right) \right)
\end{equation*}
leading to
\begin{equation*}
    S_{I A P} = - \frac{T}{2} \int \diff ^{2} x \left( \sqrt{- g} - \frac{k \Delta}{2 \sqrt{- g}} \right) g^{a b} \partial _{a} X^{\mu} \partial _{b} X^{\nu} G_{\mu \nu}
\end{equation*}

\subsection{EoMs}

\begin{align*}
    \frac{\delta S_{I A P}}{\delta g^{c d}} & = \frac{\partial}{\partial g^{c d}} \left( \sqrt{- g} - \frac{k \Delta}{2 \sqrt{- g}} \right) g^{a b} \partial _{a} X^{\mu} \partial _{b} X^{\nu} G_{\mu \nu} + \left( \sqrt{- g} - \frac{k \Delta}{2 \sqrt{- g}} \right) \delta ^{a} _{c} \delta ^{b} _{d} \partial _{a} X^{\mu} \partial _{b} X^{\nu} G_{\mu \nu} = \\
                                            & = \left( - \frac{1}{2} \sqrt{- g} g_{c d} + \frac{k \Delta}{4 \sqrt{- g}} g_{c d} \right) g^{a b} \partial _{a} X^{\mu} \partial _{b} X^{\nu} G_{\mu \nu} + \left( \sqrt{- g} - \frac{k \Delta}{2 \sqrt{- g}} \right) \partial _{c} X^{\mu} \partial _{d} X^{\nu} G_{\mu \nu} =                                       \\
                                            & = - \frac{1}{2} \left( \sqrt{- g} - \frac{k \Delta}{2 \sqrt{- g}} \right) g_{c d} g^{a b} \partial _{a} X^{\mu} \partial _{b} X^{\nu} G_{\mu \nu} + \left( \sqrt{- g} - \frac{k \Delta}{2 \sqrt{- g}} \right) \partial _{c} X^{\mu} \partial _{d} X^{\nu} G_{\mu \nu} \overset{!}{=} 0
\end{align*}
\begin{equation*}
    \downarrow
\end{equation*}
\begin{equation*}
    T_{c d} := \partial _{c} X^{\mu} \partial _{d} X^{\nu} G_{\mu \nu} - \frac{1}{2} g_{c d} g^{a b} \partial _{a} X^{\mu} \partial _{b} X^{\nu} G_{\mu \nu} = 0
\end{equation*}
\begin{equation*}
    \downarrow
\end{equation*}
\begin{equation*}
    g_{c d} = 2 f \partial _{c} X^{\mu} \partial _{d} X^{\nu} G_{\mu \nu} ,
\end{equation*}
\begin{equation*}
    \frac{1}{f} = g^{a b} \partial _{a} X^{\mu} \partial _{b} X^{\nu} G_{\mu \nu}
\end{equation*}

\begin{align*}
    \frac{\delta S_{I A P}}{\delta X^{\lambda}} & = \left( \sqrt{- g} - \frac{k \Delta}{2 \sqrt{- g}} \right) g^{a b} \partial _{a} X^{\mu} \partial _{b} X^{\nu} \partial _{\lambda} G_{\mu \nu} - \partial _{c} \left( \left( \sqrt{- g} - \frac{k \Delta}{2 \sqrt{- g}} \right) 2 g^{a b} \delta ^{c} _{a} \delta ^{\mu} _{\lambda} \partial _{b} X^{\nu} G_{\mu \nu} \right) = \\
                                                & = \left( \sqrt{- g} - \frac{k \Delta}{2 \sqrt{- g}} \right) g^{a b} \partial _{a} X^{\mu} \partial _{b} X^{\nu} \partial _{\lambda} G_{\mu \nu} - 2 \left( \sqrt{- g} - \frac{k \Delta}{2 \sqrt{- g}} \right) g^{a b} \partial _{a} \partial _{b} X^{\nu} G_{\lambda \nu} -                                                      \\
                                                & - 2 \partial _{a} \left( \left( \sqrt{- g} - \frac{k \Delta}{2 \sqrt{- g}} \right) g^{a b} \right) \partial _{b} X^{\nu} G_{\lambda \nu} =                                                                                                                                                                                       \\
                                                & = \left( 1 - \frac{k \Delta}{2 (- g)} \right) \sqrt{- g} g^{a b} \partial _{a} X^{\mu} \partial _{b} X^{\nu} \partial _{\lambda} G_{\mu \nu} - 2 \left( 1 - \frac{k \Delta}{2 (- g)} \right) \sqrt{- g} g^{a b} \partial _{a} \partial _{b} X^{\nu} G_{\lambda \nu} -                                                            \\
                                                & - 2 \partial _{a} \left( \left( 1 - \frac{k \Delta}{2 (- g)} \right) \sqrt{- g} g^{a b} \right) \partial _{b} X^{\nu} G_{\lambda \nu} \overset{!}{=} 0
\end{align*}
\begin{equation*}
    \downarrow
\end{equation*}
\begin{multline*}
    \left( 1 - \frac{k \Delta}{2 (- g)} \right) \sqrt{- g} g^{a b} \partial _{a} \partial _{b} X^{\nu} G_{\lambda \nu} + \partial _{a} \left( \left( 1 - \frac{k \Delta}{2 (- g)} \right) \sqrt{- g} g^{a b} \right) \partial _{b} X^{\nu} G_{\lambda \nu} = \\
    \frac{1}{2} \left( 1 - \frac{k \Delta}{2 (- g)} \right) \sqrt{- g} g^{a b} \partial _{a} X^{\mu} \partial _{b} X^{\nu} \partial _{\lambda} G_{\mu \nu} ,
\end{multline*}
imposing conformal symmetry $g^{a b} = (\phi (x))^{-1} \eta ^{a b}$ and let $F' = \left( 1 - \frac{k \Delta}{2 \phi ^2} \right)$ leads to
\begin{equation*}
    \eta ^{a b} \partial _{a} \partial _{b} X^{\nu} G_{\lambda \nu} + \frac{1}{F'} \eta ^{a b} \partial _{a} F' \partial _{b} X^{\nu} G_{\lambda \nu} = \eta ^{a b} \partial _{a} X^{\mu} \partial _{b} X^{\nu} \partial _{\lambda} G_{\mu \nu} ,
\end{equation*}
assuming flat background $G_{\mu \nu} = \eta _{\mu \nu}$
\begin{equation*}
    \eta ^{a b} \partial _{a} \partial _{b} X^{\nu} \eta _{\lambda \nu} + \frac{1}{F'} \eta ^{a b} \partial _{a} F' \partial _{b} X^{\nu} \eta _{\lambda \nu} = 0
\end{equation*}
\begin{equation*}
    \downarrow
\end{equation*}
\begin{equation*}
    \eta ^{a b} \partial _{a} \partial _{b} X^{\mu} + \eta ^{a b} \partial _{a} \ln (F') \partial _{b} X^{\mu} = 0 ,
\end{equation*}
use plane-wave ansatz $X^{\mu} = X^{\mu} _{0} \e ^{- i (E \tau - p \sigma)}$
\begin{equation*}
    (E^{2} - p^{2}) X^{\mu} + \partial _{\tau} \ln (F') i E X^{\mu} - \partial _{\sigma} \ln (F') i p X^{\mu} = 0
\end{equation*}
\begin{equation*}
    \downarrow
\end{equation*}
\begin{equation*}
    (E^{2} - p^{2}) X^{\mu} + i (E \partial _{\tau} \ln (F') - p \partial _{\sigma} \ln (F')) X^{\mu} = 0
\end{equation*}
\begin{equation*}
    \downarrow
\end{equation*}
\begin{equation*}
    i (E \partial _{\tau} \ln (F') - p \partial _{\sigma} \ln (F')) = - m^2 ,
\end{equation*}
\begin{equation*}
    - m^{2} = - E^{2} + p^{2}
\end{equation*}
\begin{equation*}
    \Downarrow
\end{equation*}
\begin{equation*}
    \partial _{\tau} \ln (F') = i E , \ \partial _{\sigma} \ln (F') = i p
\end{equation*}
\begin{equation*}
    \Downarrow
\end{equation*}
\begin{equation*}
    \ln (F') = i (E \tau + p \sigma) + a , \ a \in \mathbb{C}
\end{equation*}
\begin{equation*}
    \Downarrow
\end{equation*}
\begin{equation*}
    F' = A \e ^{i (E \tau + p \sigma + \phi)} , \ A > 0
\end{equation*}
\begin{equation*}
    \Downarrow
\end{equation*}
\begin{equation*}
    \phi (\tau , \sigma) = \sqrt{\frac{k \Delta}{2 \left( 1 - A \e ^{i (E \tau + p \sigma + \phi)} \right)}}
\end{equation*}
\begin{equation*}
    \downarrow
\end{equation*}
\begin{equation*}
    (\partial _{\sigma} ^{2} - \partial _{\tau} ^{2}) X^{\mu} - m^{2} X^{\mu} = 0
\end{equation*}

\newpage

\section{Inverse Area Polyakov v2}

Quantum geometry correction
\begin{equation*}
    \sqrt{- g} \rightarrow \sqrt{- \Tilde{g}} = \sqrt{- (g + k \Delta)} \approx \sqrt{- g} \left( 1 + \frac{k \Delta}{2 g} + \mathcal{O} \left( \frac{k^2}{g^2} \right) \right) ,
\end{equation*}
which implies
\begin{equation*}
    g_{a b} \rightarrow \Tilde{g}_{a b} = g_{a b} + \sqrt{k \Delta} \delta _{a b} ,
\end{equation*}
\begin{equation*}
    g^{a b} \rightarrow \Tilde{g}^{a b} = \frac{1}{2 (g + k \Delta)} \varepsilon ^{a c} \varepsilon ^{b d} \Tilde{g}_{c d} =
\end{equation*}
\begin{equation*}
    = \frac{1}{2 (g + k \Delta)} \varepsilon ^{a c} \varepsilon ^{b d} \left( g_{c d} + \sqrt{k \Delta} \delta _{c d} \right) =
\end{equation*}
\begin{equation*}
    = \frac{1}{2 (g + k \Delta)} \left( \varepsilon ^{a c} \varepsilon ^{b d} g_{c d} + \sqrt{k \Delta} \delta ^{a b} \right) \approx
\end{equation*}
\begin{equation*}
    \approx \frac{1}{2 g} \left( 1 - \frac{k \Delta}{g} \right) \left( \varepsilon ^{a c} \varepsilon ^{b d} g_{c d} + \sqrt{k \Delta} \delta ^{a b} \right) =
\end{equation*}
\begin{equation*}
    = \left( 1 - \frac{k \Delta}{g} \right) \left( g^{a b} + \frac{\sqrt{k \Delta}}{2 g} \delta ^{a b} \right)
\end{equation*}
leading to
\begin{equation*}
    S = - \frac{T}{2} \int \diff ^2 x \sqrt{- \Tilde{g}} \Tilde{g}^{a b} \partial _{a} X^{\mu} \partial _{b} X^{\nu} G_{\mu \nu} \approx
\end{equation*}
\begin{equation*}
    \approx - \frac{T}{2} \int \diff ^2 x \sqrt{- g} \left( 1 + \frac{k \Delta}{2 g} \right) \left( 1 - \frac{k \Delta}{g} \right) \left( g^{a b} + \frac{\sqrt{k \Delta}}{2 g} \delta ^{a b} \right) \partial _{a} X^{\mu} \partial _{b} X^{\nu} G_{\mu \nu} =
\end{equation*}
\begin{equation*}
    = - \frac{T}{2} \int \diff ^2 x \sqrt{- g} \left( g^{a b} + \frac{\sqrt{k \Delta}}{2 g} \delta ^{a b} + \frac{k \Delta}{2 g} g^{a b} - \frac{k \Delta}{g} g^{a b} + \mc{O} \left( \frac{(k \Delta)^{3/2}}{g^2} ; \frac{(k \Delta)^2}{g^2} \right) \right) \partial _{a} X^{\mu} \partial _{b} X^{\nu} G_{\mu \nu} \approx
\end{equation*}
\begin{equation*}
    \approx - \frac{T}{2} \int \diff ^2 x \left( \sqrt{- g} g^{a b} - \frac{\sqrt{k \Delta}}{2 \sqrt{- g}} \delta ^{a b} \right) \partial _{a} X^{\mu} \partial _{b} X^{\nu} G_{\mu \nu} = S_{P} + S_{c} ,
\end{equation*}
\begin{equation*}
    S_{c} = \frac{T \sqrt{k \Delta}}{4} \int \diff ^2 x \sqrt{- g}^{-1} \delta ^{a b} \partial _{a} X^{\mu} \partial _{b} X^{\nu} G_{\mu \nu}
\end{equation*}

\subsection{EoMs}

Since the total action is the Polyakov action plus a correction term, let's calculate just the EoMs for the correction action and later sum them with the known eqns of the regular Polyakov action.
\subsubsection{W.r.t $g$}

\begin{align*}
    \frac{\delta S_{c}}{\delta g^{c d}} & = \frac{\partial \sqrt{- g}^{- 1}}{\partial g^{c d}} \delta ^{a b} \partial _{a} X^{\mu} \partial _{b} X^{\nu} G_{\mu \nu} = \\
                                        & = - \frac{1}{2 \sqrt{- g}^{-1}} g_{c d} \delta ^{a b} \partial _{a} X^{\mu} \partial _{b} X^{\nu} G_{\mu \nu} =              \\
                                        & = - \frac{1}{2} \sqrt{- g} g_{c d} \delta ^{a b} \partial _{a} X^{\mu} \partial _{b} X^{\nu} G_{\mu \nu}
\end{align*}
\begin{equation*}
    \Downarrow
\end{equation*}
\begin{equation*}
    T_{c d} := \partial _{c} X^{\mu} \partial _{d} X^{\nu} G_{\mu \nu} - g_{c d} \left( \frac{1}{2} g^{a b} - \frac{\sqrt{k \Delta}}{4} \delta ^{a b} \right) \partial _{a} X^{\mu} \partial _{b} X^{\nu} G_{\mu \nu} = 0
\end{equation*}
\begin{equation*}
    \Downarrow
\end{equation*}
\begin{equation*}
    g_{c d} = 2 f' \partial _{c} X^{\mu} \partial _{d} X^{\nu} G_{\mu \nu} ,
\end{equation*}
\begin{equation*}
    \frac{1}{f'} = \left( g^{a b} - \frac{\sqrt{k \Delta}}{2} \delta ^{a b} \right) \partial _{a} X^{\mu} \partial _{b} X^{\nu} G_{\mu \nu}
\end{equation*}

\newpage

\section{Relating NG and Polyakov analysis}

From NG analysis
\begin{equation*}
    (F^{+})^2 \frac{\partial ^2 X^{\mu}}{\partial t^2} - \frac{\partial ^2 X^{\mu}}{\partial x^2} = 0 ,
\end{equation*}
make change of variables $\tau = \tau (t , x)$ and $\sigma = \sigma (t , x)$
\begin{equation*}
    (F^{+})^2 \left( \frac{\partial \tau}{\partial t} \frac{\partial}{\partial \tau} + \frac{\partial \sigma}{\partial t} \frac{\partial}{\partial \sigma} \right) \left( \frac{\partial \tau}{\partial t} \frac{\partial X^{\mu}}{\partial \tau} + \frac{\partial \sigma}{\partial t} \frac{\partial X^{\mu}}{\partial \sigma} \right) - \left( \frac{\partial \tau}{\partial x} \frac{\partial}{\partial \tau} + \frac{\partial \sigma}{\partial x} \frac{\partial}{\partial \sigma} \right) \left( \frac{\partial \tau}{\partial x} \frac{\partial X^{\mu}}{\partial \tau} + \frac{\partial \sigma}{\partial x} \frac{\partial X^{\mu}}{\partial \sigma} \right) = 0
\end{equation*}
\begin{multline*}
    (F^{+})^2 \left( \left( \frac{\partial \tau}{\partial t} \right) ^2 \frac{\partial ^2 X^{\mu}}{\partial \tau ^2} + 2 \frac{\partial \tau}{\partial t} \frac{\partial \sigma}{\partial t} \frac{\partial ^2 X^{\mu}}{\partial \tau \partial \sigma} + \left( \frac{\partial \sigma}{\partial t} \right) ^2 \frac{\partial ^2 X^{\mu}}{\partial \sigma ^2} \right) - \\
    - \left( \left( \frac{\partial \tau}{\partial x} \right) ^2 \frac{\partial ^2 X^{\mu}}{\partial \tau ^2} + 2 \frac{\partial \tau}{\partial x} \frac{\partial \sigma}{\partial x} \frac{\partial ^2 X^{\mu}}{\partial \tau \partial \sigma} + \left( \frac{\partial \sigma}{\partial x} \right) ^2 \frac{\partial ^2 X^{\mu}}{\partial \sigma ^2} \right) + \\
    + \left( (F^{+})^2 \frac{\partial ^2 \tau}{\partial t^2} - \frac{\partial ^2 \tau}{\partial x^2} \right) \frac{\partial X^{\mu}}{\partial \tau} + \left( (F^{+})^2 \frac{\partial ^2 \sigma}{\partial t^2} - \frac{\partial ^2 \sigma}{\partial x^2} \right) \frac{\partial X^{\mu}}{\partial \sigma} = 0
\end{multline*}
\begin{multline*}
    \left( (F^{+})^2 (\partial _{t} \tau)^2 - (\partial _{x} \tau)^2 \right) \partial _{\tau} ^2 X^{\mu} + \left( (F^{+})^2 (\partial _{t} \sigma)^2 - (\partial _{x} \sigma)^2 \right) \partial _{\sigma} ^2 X^{\mu} + \\
    + 2 \left( (F^{+})^2 \partial _{t} \tau \partial _{t} \sigma - \partial _{x} \tau \partial _{x} \sigma \right) \partial _{\tau} \partial _{\sigma} X^{\mu} + \left( (F^{+})^2 \partial _{t} ^2 \tau - \partial _{x} ^2 \tau \right) \partial _{\tau} X^{\mu} + \left( (F^{+})^2 \partial _{t} ^2 \sigma - \partial _{x} ^2 \sigma \right) \partial _{\sigma} X^{\mu} = 0
\end{multline*}
Without loss of generality, we can set (just rescale/rotate the coordinates)
\begin{align*}
    (F^{+})^2 (\partial _t \sigma)^2 - (\partial _x \sigma)^2 & = (\partial _x \tau)^2 - (F^{+})^2 (\partial _t \tau)^2 \\
    (F^{+})^2 \partial _t \tau \partial _t \sigma             & = \partial _x \tau \partial _x \sigma ,
\end{align*}
or more compactly
\begin{align*}
    \partial _{t} \tau   & = v \partial _{x} \sigma \\
    \partial _{t} \sigma & = v \partial _{x} \tau ,
\end{align*}
where $v = 1 / F^{+}$, which is equivalent to
\begin{align*}
    \partial _{t} ^2 \tau   & = \partial _{t} v \partial _{x} \sigma + v \partial _{x} v \partial _{x} \tau + v^2 \partial _{x} ^2 \tau      \\
    \partial _{t} ^2 \sigma & = \partial _{t} v \partial _{x} \tau + v \partial _{x} v \partial _{x} \sigma + v ^2 \partial _{x} ^2 \sigma ,
\end{align*}
reducing the big equation to
\begin{equation*}
    \partial _{\tau} ^2 X^{\mu} - \partial _{\sigma} ^2 X^{\mu} + \frac{\partial _{t} v \partial _{x} \sigma + v \partial _{x} v \partial _{x} \tau}{\left( (\partial _{t} \tau)^2 - v^2 (\partial _{x} \tau)^2 \right)} \partial _{\tau} X^{\mu} + \frac{\partial _{t} v \partial _{x} \tau + v \partial _{x} v \partial _{x} \sigma}{\left( (\partial _{t} \tau)^2 - v^2 (\partial _{x} \tau)^2 \right)} \partial _{\sigma} X^{\mu} = 0 .
\end{equation*}
Next introduce $X^{\mu} = \kappa (\tau , \sigma) Y^{\mu} (\tau , \sigma)$ s.t terms proportional to $\partial _{\tau} Y^{\mu}$ and $\partial _{\sigma} Y^{\mu}$ vanish:
\begin{align*}
    \partial _{\tau} X^{\mu}      & = \partial _{\tau} \kappa Y^{\mu} + \kappa \partial _{\tau} Y^{\mu}                                                                    \\
    \partial _{\sigma} X^{\mu}    & = \partial _{\sigma} \kappa Y^{\mu} + \kappa \partial _{\sigma} Y^{\mu}                                                                \\
    \partial _{\tau} ^2 X^{\mu}   & = \partial _{\tau} ^2 \kappa Y^{\mu} + 2 \partial _{\tau} \kappa \partial _{\tau} Y^{\mu} + \kappa \partial _{\tau} ^2 Y^{\mu}         \\
    \partial _{\sigma} ^2 X^{\mu} & = \partial _{\sigma} ^2 \kappa Y^{\mu} + 2 \partial _{\sigma} \kappa \partial _{\sigma} Y^{\mu} + \kappa \partial _{\sigma} ^2 Y^{\mu}
\end{align*}
\begin{align*}
    \kappa \frac{\partial _{t} v \partial _{x} \sigma + v \partial _{x} v \partial _{x} \tau}{\left( (\partial _{t} \tau)^2 - v^2 (\partial _{x} \tau)^2 \right)} + 2 \partial _{\tau} \kappa   & \overset{!}{=} 0 \\
    \kappa \frac{\partial _{t} v \partial _{x} \tau + v \partial _{x} v \partial _{x} \sigma}{\left( (\partial _{t} \tau)^2 - v^2 (\partial _{x} \tau)^2 \right)} - 2 \partial _{\sigma} \kappa & \overset{!}{=} 0
\end{align*}
\begin{align*}
    \partial _{\tau} \kappa   & = - \frac{\kappa (\partial _{t} v \partial _{x} \sigma + v \partial _{x} v \partial _{x} \tau)}{2 \left( (\partial _{t} \tau)^2 - v^2 (\partial _{x} \tau)^2 \right)} \\
    \partial _{\sigma} \kappa & = \frac{\kappa (\partial _{t} v \partial _{x} \tau + v \partial _{x} v \partial _{x} \sigma)}{2 \left( (\partial _{t} \tau)^2 - v^2 (\partial _{x} \tau)^2 \right)} ,
\end{align*}
undoing chain rule for $t$ by multiplying first eq by $\partial _{t} \tau$ and second by $\partial _{t} \sigma$ and summing gives
\begin{align*}
    \partial _{t} \kappa & = \partial _{t} \tau \partial _{\tau} \kappa + \partial _{t} \sigma \partial _{\sigma} \kappa =                                                                                                                                                                                                                                                                                                                        \\
                         & = \partial _{t} \tau \left( - \frac{\kappa (\partial _{t} v \partial _{x} \sigma + v \partial _{x} v \partial _{x} \tau)}{2 \left( (\partial _{t} \tau)^2 - v^2 (\partial _{x} \tau)^2 \right)} \right) + \partial _{t} \sigma \left( \frac{\kappa (\partial _{t} v \partial _{x} \tau + v \partial _{x} v \partial _{x} \sigma)}{2 \left( (\partial _{t} \tau)^2 - v^2 (\partial _{x} \tau)^2 \right)}  \right) =     \\
                         & = v \partial _{x} \sigma \left( - \frac{\kappa (\partial _{t} v \partial _{x} \sigma + v \partial _{x} v \partial _{x} \tau)}{2 \left( (\partial _{t} \tau)^2 - v^2 (\partial _{x} \tau)^2 \right)} \right) + v \partial _{x} \tau \left( \frac{\kappa (\partial _{t} v \partial _{x} \tau + v \partial _{x} v \partial _{x} \sigma)}{2 \left( (\partial _{t} \tau)^2 - v^2 (\partial _{x} \tau)^2 \right)}  \right) = \\
                         & = \frac{v \kappa \partial _{t} v \left( (\partial _{x} \tau)^2 - (\partial _{x} \sigma)^2 \right)}{2 \left( (\partial _{t} \tau)^2 - v^2 (\partial _{x} \tau)^2 \right)} =                                                                                                                                                                                                                                             \\
                         & = \frac{v \kappa \partial _{t} v \left( (\partial _{x} \tau)^2 - (\partial _{t} \tau / v)^2 \right)}{2 \left( (\partial _{t} \tau)^2 - v^2 (\partial _{x} \tau)^2 \right)} =                                                                                                                                                                                                                                           \\
                         & = - \frac{\kappa \partial _{t} v}{2 v} ,
\end{align*}
or simply
\begin{equation*}
    \frac{\partial _{t} \kappa}{\kappa} = - \frac{1}{2} \frac{\partial _{t} v}{v}
\end{equation*}
\begin{equation*}
    \partial _{t} \ln (\kappa) = - \frac{1}{2} \partial _{t} \ln (v)
\end{equation*}
\begin{equation*}
    \ln (\kappa) = - \frac{1}{2} \ln (v) + f(x) .
\end{equation*}
To determine $f(x)$, we undo the chain rule for $x$ now:
\begin{align*}
    \partial _{x} \kappa & = \partial _{x} \tau \partial _{\tau} \kappa + \partial _{x} \sigma \partial _{\sigma} \kappa =                                                                                                                                                                                                                                                                                                                    \\
                         & = \partial _{x} \tau \left( - \frac{\kappa (\partial _{t} v \partial _{x} \sigma + v \partial _{x} v \partial _{x} \tau)}{2 \left( (\partial _{t} \tau)^2 - v^2 (\partial _{x} \tau)^2 \right)} \right) + \partial _{x} \sigma \left( \frac{\kappa (\partial _{t} v \partial _{x} \tau + v \partial _{x} v \partial _{x} \sigma)}{2 \left( (\partial _{t} \tau)^2 - v^2 (\partial _{x} \tau)^2 \right)}  \right) = \\
                         & = \frac{\kappa v \partial _{x} v \left( (\partial _{x} \sigma)^2 - (\partial _{x} \tau)^2 \right)}{2 \left( (\partial _{t} \tau)^2 - v^2 (\partial _{x} \tau)^2 \right)} =                                                                                                                                                                                                                                         \\
                         & = \frac{\kappa v \partial _{x} v \left( (\partial _{t} \tau / v)^2 - (\partial _{x} \tau)^2 \right)}{2 \left( (\partial _{t} \tau)^2 - v^2 (\partial _{x} \tau)^2 \right)} =                                                                                                                                                                                                                                       \\
                         & = \frac{\kappa \partial _{x} v}{2 v}
\end{align*}
\begin{equation*}
    \frac{\partial _{x} \kappa}{ \kappa} = \frac{1}{2} \frac{\partial _{x} v}{v}
\end{equation*}
\begin{equation*}
    \partial _{x} \ln (\kappa) = \frac{1}{2} \partial _{x} \ln (v)
\end{equation*}
\begin{equation*}
    \ln (\kappa) = \frac{1}{2} \ln (v) ,
\end{equation*}
\begin{equation*}
    \partial _{x} \left( - \frac{1}{2} \ln (v) + f(x) \right) = \frac{1}{2} \partial _{x} \ln (v)
\end{equation*}
\begin{equation*}
    f' (x) = \partial _{x} \ln (v) \ \mathrm{(perhaps \ this \ means \ somewhere \ in \ the \ composition \ of \ functions \ time \ dependence \ is \ lost?)}
\end{equation*}
\begin{equation*}
    f (x) = \ln (v)
\end{equation*}
giving us
\begin{equation*}
    \kappa = v^{\frac{1}{2}} , \ v = v (x) .
\end{equation*}
Substituting this into the earlier equation yields
\begin{equation*}
    \partial _{\tau} ^2 Y^{\mu} - \partial _{\sigma} ^2 Y^{\mu} + m^2 Y^{\mu} = 0 ,
\end{equation*}
with
\begin{align*}
    m^2 & = \frac{\partial _{\tau} ^2 \kappa - \partial _{\sigma} ^2 \kappa}{\kappa} + \frac{(\partial _{t} v \partial _{x} \sigma + v \partial _{x} v \partial _{x} \tau) \partial _{\tau} \kappa + (\partial _{t} v \partial _{x} \tau + v \partial _{x} v \partial _{x} \sigma) \partial _{\sigma} \kappa}{\kappa \left( (\partial _{t} \tau)^2 - v^2 (\partial _{x} \tau)^2 \right)} = \\
        & = \frac{\partial _{\tau} ^2 \kappa - \partial _{\sigma} ^2 \kappa}{\kappa} + \frac{\partial _{t} v (\partial _{x} \sigma \partial _{\tau} \kappa + \partial _{x} \tau \partial _{\sigma} \kappa) + v \partial _{x} v \partial _{x} \kappa}{\kappa \left( (\partial _{t} \tau)^2 - v^2 (\partial _{x} \tau)^2 \right)} =                                                          \\
        & = \frac{\partial _{\tau} ^2 \kappa - \partial _{\sigma} ^2 \kappa}{\kappa} + \frac{(\partial _{x} v)^2}{\left( (\partial _{t} \tau)^2 - v^2 (\partial _{x} \tau)^2 \right)} ,
\end{align*}
where by use of chain rule we have that
\begin{equation*}
    \partial _{t} ^2 \kappa - v^2 \partial _{x} ^2 \kappa - v \partial _{x} v \partial _{x} \kappa = ((\partial _{t} \tau)^2 - v^2 (\partial _{x} \tau)^2) (\partial _{\tau} ^2 \kappa - \partial _{\sigma} ^2 \kappa) ,
\end{equation*}
where since $\kappa = v^{\frac{1}{2}}$, $m^2$ reduces to
\begin{equation*}
    m^2 = \frac{(\partial _{x} v)^2 - 2 v \partial _{x} ^2 v}{4 \left( (\partial _{t} \tau)^2 - v^2 (\partial _{x} \tau)^2 \right)} ,
\end{equation*}
with $\tau$ satisfying
\begin{align*}
    \partial _{t} \tau   & = v \partial _{x} \sigma \\
    v \partial _{x} \tau & = \partial _{t} \sigma
\end{align*}
such that $- m^2$ is indeed constant.

\newpage

\section{Solutions for the Polyakov KG eqn (WRONG ONE)}

From the inverse-area corrected Polyakov action we have the equation
\begin{equation*}
    (\partial _{\sigma} ^2 - \partial _{\tau} ^2) X^{\mu} - m^2 X^{\mu} = 0
\end{equation*}
subjected to the constraint
\begin{equation*}
    \partial _{c} X^{\mu} \partial _{d} X^{\nu} \eta _{\mu \nu} - \frac{1}{2} g_{c d} g^{a b} \partial _{a} X^{\mu} \partial _{b} X^{\nu} \eta _{\mu \nu} = 0 ,
\end{equation*}
which with conformal gauge $g_{a b} = \phi \eta _{a b}$ becomes
\begin{equation*}
    \partial _{c} X^{\mu} \partial _{d} X^{\nu} \eta_{\mu \nu} = \frac{1}{2} \eta _{c d} \eta ^{a b} \partial _{a} X^{\mu} \partial _{b} X^{\nu} \eta _{\mu \nu}
\end{equation*}
or more explicitly,
\begin{align*}
    \partial _{\tau} X^{\mu} \partial _{\tau} X^{\nu} \eta _{\mu \nu}     & = - \frac{1}{2} \eta ^{a b} \partial _{a} X^{\mu} \partial _{b} X^{\nu} \eta _{\mu \nu} \\
    \partial _{\sigma} X^{\mu} \partial _{\sigma} X^{\nu} \eta _{\mu \nu} & = \frac{1}{2} \eta ^{a b} \partial _{a} X^{\mu} \partial _{b} X^{\nu} \eta _{\mu \nu}   \\
    \partial _{\tau} X^{\mu} \partial _{\sigma} X^{\nu} \eta_{\mu \nu}    & = \partial _{\sigma} X^{\mu} \partial _{\tau} X^{\nu} \eta _{\mu \nu} = 0 .
\end{align*}
These simplify further by expanding the summation on $a,b$:
\begin{equation*}
    \partial _{\tau} X^{\mu} \partial _{\tau} X^{\nu} \eta _{\mu \nu} = - \frac{1}{2} \left( - \partial _{\tau} X^{\mu} \partial _{\tau} X^{\nu} \eta _{\mu \nu} + \partial _{\sigma} X^{\mu} \partial _{\sigma} X^{\nu} \eta _{\mu \nu} \right)
\end{equation*}
\begin{equation*}
    \partial _{\tau} X^{\mu} \partial _{\tau} X^{\nu} \eta _{\mu \nu} = - \partial _{\sigma} X^{\mu} \partial _{\sigma} X^{\nu} \eta _{\mu \nu} .
\end{equation*}
The EoM is just a one-dimensional Klein-Gordon equation, which has general solution given by Fourier transform
\begin{equation*}
    X^{\mu} (\tau , \sigma) = \frac{1}{2 \pi} \int \diff p \frac{1}{2 E (p)} \left( a^{\mu} (p) \e ^{- i (- E \tau + p \sigma)} + b^{\mu} (p) \e ^{i (- E \tau + p \sigma)} \right) ,
\end{equation*}
where $E (p) = \sqrt{p^2 + m^2}$. The derivatives are
\begin{align*}
    \partial _{\tau} X^{\mu}   & = \frac{i}{4 \pi} \int \diff p \left( a^{\mu} (p) \e^{- i (- E \tau + p \sigma)} - b^{\mu} (p) \e ^{i (- E \tau + p \sigma)} \right)                                \\
    \partial _{\sigma} X^{\mu} & = \frac{i}{4 \pi} \int \diff p \frac{p}{\sqrt{p^2 + m^2}} \left( - a^{\mu} (p) \e ^{-i (- E \tau + p \sigma)} + b^{\mu} (p) \e ^{i (- E \tau + p \sigma)} \right) .
\end{align*}
Starting with the mixed constraint,
\begin{multline*}
    \partial _{\tau} X^{\mu} \partial _{\sigma} X^{\nu} \eta _{\mu \nu} = - \frac{\eta _{\mu \nu}}{(4 \pi)^2} \int \diff p \left( a^{\mu} (p) \e^{- i (- E (p) \tau + p \sigma)} - b^{\mu} (p) \e ^{i (- E (p) \tau + p \sigma)} \right) \times \\
    \times \int \diff p' \frac{p'}{\sqrt{p^{\prime 2} + m^2}} \left( - a^{\nu} (p') \e ^{-i (- E (p') \tau + p' \sigma)} + b^{\nu} (p') \e ^{i (- E (p') \tau + p' \sigma)} \right) =
\end{multline*}
\begin{multline*}
    = - \frac{\eta _{\mu \nu}}{(4 \pi)^2} \int \diff p \diff p' \frac{p'}{\sqrt{p^{\prime 2} + m^2}} \left( a^{\mu} (p) \e^{- i (- E (p) \tau + p \sigma)} - b^{\mu} (p) \e ^{i (- E (p) \tau + p \sigma)} \right) \times \\
    \times \left( - a^{\nu} (p') \e ^{-i (- E (p') \tau + p' \sigma)} + b^{\nu} (p') \e ^{i (- E (p') \tau + p' \sigma)} \right) =
\end{multline*}
\begin{multline*}
    = - \frac{\eta _{\mu \nu}}{(4 \pi)^2} \int \diff p \diff p' \frac{p'}{\sqrt{p^{\prime 2} + m^2}} \Big( - a^{\mu} (p) a^{\nu} (p') \e ^{i (E (p) + E (p')) \tau} \e ^{- i (p + p') \sigma} + a^{\mu} (p) b^{\nu} (p') \e ^{i (E (p) - E (p')) \tau} \e ^{- i (p - p') \sigma} + \\
    + b^{\mu} (p) a^{\nu} (p') \e ^{- i (E (p) - E (p')) \tau} \e ^{i (p - p') \sigma} - b^{\mu} (p) b^{\nu} (p') \e ^{- i (E (p) + E (p')) \tau} \e ^{i (p + p') \sigma} \Big) = 0 .
\end{multline*}
If we integrate this expression over $\sigma$ we obtain
\begin{multline*}
    - \frac{\eta _{\mu \nu}}{(4 \pi)^2} \int \diff p \diff p' \frac{p'}{\sqrt{p^{\prime 2} + m^2}} \Big( - a^{\mu} (p) a^{\nu} (p') \e ^{i (E (p) + E (p')) \tau} \int \diff \sigma \e ^{-i (p + p') \sigma} + \\
    + a^{\mu} (p) b^{\nu} (p') \e ^{i (E (p) - E (p')) \tau} \int \diff \sigma \e ^{-i (p - p') \sigma} + b^{\mu} (p) a^{\nu} (p') \e ^{- i (E (p) - E (p')) \tau} \int \diff \sigma \e ^{i (p - p') \sigma} - \\
    - b^{\mu} (p) b^{\nu} (p') \e ^{- i (E (p) + E (p')) \tau} \int \diff \sigma \e ^{i (p + p') \sigma} \Big) =
\end{multline*}
\begin{multline*}
    = - \frac{\eta _{\mu \nu}}{8 \pi} \int \diff p \diff p' \frac{p'}{\sqrt{p^{\prime 2} + m^2}} \Big( - a^{\mu} (p) a^{\nu} (p') \e ^{i (E (p) + E (p')) \tau} \delta (p + p') + a^{\mu} (p) b^{\nu} (p') \e ^{i (E (p) - E (p')) \tau} \delta (p - p') + \\
    + b^{\mu} (p) a^{\nu} (p') \e ^{- i (E (p) - E (p')) \tau} \delta (p - p') - b^{\mu} (p) b^{\nu} (p') \e ^{- i (E (p) + E (p')) \tau} \delta (p + p') \Big) =
\end{multline*}
\begin{equation*}
    = - \frac{\eta _{\mu \nu}}{8 \pi} \int \diff p \frac{p}{\sqrt{p^2 + m^2}} \Big( a^{\mu} (p) a^{\nu} (- p) \e ^{2 i E (p) \tau} + a^{\mu} (p) b^{\nu} (p) + b^{\mu} (p) a^{\nu} (p) + b^{\mu} (p) b^{\nu} (- p) \e ^{- 2 i E (p) \tau} \Big) = 0
\end{equation*}
which implies the whole expression inside the integral must vanish, thus we have (leaving $(a^{\mu} b^{\nu} + b^{\mu} a^{\nu}) \eta _{\mu \nu}$ as is since eventually these will not be numbers and may not commute, thus will not be equal to $2 a^{\mu} b^{\nu} \eta _{\mu \nu}$)
\begin{equation*}
    \Big( a^{\mu} (p) a^{\nu} (- p) \e ^{2 i E (p) \tau} + a^{\mu} (p) b^{\nu} (p) + b^{\mu} (p) a^{\nu} (p) + b^{\mu} (p) b^{\nu} (- p) \e ^{- 2 i E (p) \tau} \Big) \eta _{\mu \nu} = 0 .
\end{equation*}

As for the other part of the constraint, let's start with the L.H.S
\begin{multline*}
    \partial _{\tau} X^{\mu} \partial _{\tau} X^{\nu} \eta _{\mu \nu} = - \frac{1}{(4 \pi)^2} \int \diff p \diff p' \Big( a^{\mu} (p) \e^{- i (- E (p) \tau + p \sigma)} - b^{\mu} (p) \e ^{i (- E (p) \tau + p \sigma)} \Big) \times \\
    \times \Big( a^{\nu} (p') \e^{- i (- E (p') \tau + p' \sigma)} - b^{\nu} (p') \e ^{i (- E (p') \tau + p' \sigma)} \Big) =
\end{multline*}
\begin{multline*}
    = - \frac{\eta _{\mu \nu}}{(4 \pi)^2} \int \diff p \diff p' \Big( a^{\mu} (p) a^{\nu} (p') \e ^{i (E (p) + E (p')) \tau} \e ^{- i (p + p') \sigma} - a^{\mu} (p) b^{\nu} (p') \e ^{i (E (p) - E (p')) \tau} \e ^{- i (p - p') \sigma} - \\
    - b^{\mu} (p) a^{\nu} (p') \e ^{- i (E (p) - E (p')) \tau} \e ^{i (p - p') \sigma} + b^{\mu} (p) b^{\nu} (p') \e ^{- i (E (p) + E (p')) \tau} \e ^{i (p + p') \sigma} \Big) ,
\end{multline*}
which we integrate over $\sigma$ to get
\begin{multline*}
    - \frac{\eta _{\mu \nu}}{8 \pi} \int \diff p \diff p' \Big( a^{\mu} (p) a^{\nu} (p') \e ^{i (E (p) + E (p')) \tau} \delta (p + p') - a^{\mu} (p) b^{\nu} (p') \e ^{i (E (p) - E (p')) \tau} \delta (p - p') - \\
    - b^{\mu} (p) a^{\nu} (p') \e ^{- i (E (p) - E (p')) \tau} \delta (p - p') + b^{\mu} (p) b^{\nu} (p') \e ^{- i (E (p) + E (p')) \tau} \delta (p + p') \Big) =
\end{multline*}
\begin{equation*}
    = - \frac{\eta _{\mu \nu}}{8 \pi} \int \diff p \Big( a^{\mu} (p) a^{\nu} (- p) \e ^{2 i E (p) \tau} - a^{\mu} (p) b^{\nu} (p) - b^{\mu} (p) a^{\nu} (p) + b^{\mu} (p) b^{\nu} (- p) \e ^{- 2 i E (p) \tau} \Big) ,
\end{equation*}
and now for the R.H.S
\begin{multline*}
    - \partial _{\sigma} X^{\mu} \partial _{\sigma} X^{\nu} \eta _{\mu \nu} = \frac{\eta _{\mu \nu}}{(4 \pi)^2} \int \diff p \diff p' \frac{p}{\sqrt{p^2 + m^2}} \frac{p'}{\sqrt{p^{\prime 2} + m^2}} \Big( - a^{\mu} (p) \e ^{- i (-E \tau + p \sigma)} + b^{\mu} (p) \e ^{i (- E \tau + p \sigma)} \Big) \times \\
    \times \Big( - a^{\nu} (p') \e ^{- i (- E (p') \tau + p' \sigma)} + b^{\nu} (p') \e ^{i (- E (p') \tau + p' \sigma)} \Big) =
\end{multline*}
\begin{multline*}
    = \frac{\eta _{\mu \nu}}{(4 \pi)^2} \int \diff p \diff p' \frac{p p'}{E (p) E (p')} \Big( a^{\mu} (p) a^{\nu} (p') \e ^{i (E (p) + E (p')) \tau} \e ^{- i (p + p') \sigma} - a^{\mu} (p) b^{\nu} (p') \e ^{i (E (p) - E (p')) \tau} \e ^{- i (p - p') \sigma} - \\
    - b^{\mu} (p) a^{\nu} (p') \e ^{- i (E (p) - E (p')) \tau} \e ^{i (p - p') \sigma} + b^{\mu} (p) b^{\nu} (p) \e ^{- i ( E (p) + E (p')) \tau} \e ^{i (p + p') \sigma} \Big) ,
\end{multline*}
which once again we integrate over $\sigma$,
\begin{multline*}
    \frac{\eta _{\mu \nu}}{8 \pi} \int \diff p \diff p' \frac{p p'}{E (p) E (p')} \Big( a^{\mu} (p) a^{\nu} (p') \e ^{i (E (p) + E (p')) \tau} \delta (p + p') - a^{\mu} (p) b^{\nu} (p') \e ^{i (E (p) - E (p')) \tau} \delta (p - p') - \\
    - b^{\mu} (p) a^{\nu} (p') \e ^{- i (E (p) - E (p')) \tau} \delta (p - p') + b^{\mu} (p) b^{\nu} (p) \e ^{- i ( E (p) + E (p')) \tau} \delta (p + p') \Big) =
\end{multline*}
\begin{equation*}
    = \frac{1}{8 \pi} \int \diff p \frac{p^2}{p^2 + m^2} \Big( a^{\mu} (p) a^{\nu} (- p) \e ^{2 i E (p) \tau} - a^{\mu} (p) b^{\nu} (p) - b^{\mu} (p) a^{\nu} (p) + b^{\mu} (p) b^{\nu} (- p) \e ^{- 2 i E (p) \tau} \Big) \eta _{\mu \nu} ,
\end{equation*}
which when equated to the L.H.S yields
\begin{equation*}
    \int \diff p \left( \frac{p^2}{p^2 + m^2} + 1 \right) \Big( a^{\mu} (p) a^{\nu} (- p) \e ^{2 i E (p) \tau} - a^{\mu} (p) b^{\nu} (p) - b^{\mu} (p) a^{\nu} (p) + b^{\mu} (p) b^{\nu} (- p) \e ^{- 2 i E (p) \tau} \Big) \eta _{\mu \nu} = 0 ,
\end{equation*}
which implies
\begin{equation*}
    \Big( a^{\mu} (p) a^{\nu} (- p) \e ^{2 i E (p) \tau} - a^{\mu} (p) b^{\nu} (p) - b^{\mu} (p) a^{\nu} (p) + b^{\mu} (p) b^{\nu} (- p) \e ^{- 2 i E (p) \tau} \Big) \eta _{\mu \nu} = 0 .
\end{equation*}
We can sum and subtract this with the previous condition to simplify and obtain
\begin{align*}
    \Big( a^{\mu} (p) a^{\nu} (- p) \e ^{2 i E (p) \tau} + b^{\mu} (p) b^{\nu} (- p) \e ^{- 2 i E (p) \tau} \Big) \eta _{\mu \nu} & = 0   \\
    \Big( a^{\mu} (p) b^{\nu} (p) + b^{\mu} (p) a^{\nu} (p) \Big) \eta _{\mu \nu}                                                 & = 0 ,
\end{align*}
and for while $a^{\mu}$ and $b^{\nu}$ are number-valued, the second condition can be further simplified to
\begin{equation*}
    a^{\mu} (p) b^{\nu} (p) \eta _{\mu \nu} = 0 ,
\end{equation*}
meaning $a^{\mu}$ and $b^{\nu}$ are orthogonal. In general, this condition states that $a^{\mu}$ and $b^{\nu}$ anti-commute w.r.t Lorentz inner product. The first condition must hold for all values of $\tau$, and since the exponentials are linearly independent, we thus have
\begin{align*}
    a^{\mu} (p) a^{\nu} (- p) \eta _{\mu \nu} & = 0   \\
    b^{\mu} (p) b^{\nu} (- p) \eta _{\mu \nu} & = 0 ,
\end{align*}
meaning that reflecting the argument of $a^{\mu}$ and $b^{\nu}$ creates orthogonal vectors. We can thus write
\begin{align*}
    a^{\mu} (p) & = \Lambda ^{\mu} _{\ \nu} \left( \frac{p}{4} \right) a^{\nu} (0)    \\
    b^{\mu} (p) & = \Upsilon ^{\mu} _{\ \nu} \left( \frac{p}{4} \right) b^{\nu} (0) ,
\end{align*}
where $\Lambda , \Upsilon \in \mathrm{SO}^{+} (1 , d)$. Also. since the orthogonality under reflection must hold for all values of $p$, in particular we have
\begin{align*}
    a^{\mu} (0) a^{\nu} (0) \eta _{\mu \nu} & = 0   \\
    b^{\mu} (0) b^{\nu} (0) \eta _{\mu \nu} & = 0 ,
\end{align*}
meaning that the initial $a^{\mu} (0)$ and $b^{\nu} (0)$ are null vectors. This means that $a^{\mu} (p)$ and $b^{\nu} (p)$ are also null vectors since they are related to the initial values by Lorentz transformation. With this, the solution becomes
\begin{equation*}
    X^{\mu} (\tau , \sigma) = \frac{1}{2 \pi} \int \diff p \frac{1}{2 E (p)} \left( \Lambda ^{\mu} _{\ \nu} \left( \frac{p}{4} \right) a^{\nu} (0) \e ^{- i (- E \tau + p \sigma)} + \Upsilon ^{\mu} _{\ \nu} \left( \frac{p}{4} \right) b^{\nu} (0) \e ^{i (- E \tau + p \sigma)} \right)
\end{equation*}
with $(a^{\mu} b^{\nu} + b^{\nu} a^{\mu}) \eta _{\mu \nu} = 0$. This condition implies
\begin{equation*}
    \left( \Lambda ^{\mu} _{\ \gamma} \left( \frac{p}{4} \right) a^{\gamma} (0) \Upsilon ^{\nu} _{\ \rho} \left( \frac{p}{4} \right) b^{\rho} (0) \right) \eta _{\mu \nu} = 0
\end{equation*}
\begin{equation*}
    \Upsilon ^{\mu} _{\ \nu} \left( \frac{p}{4} \right) = \Lambda ^{\ \mu} _{\nu} \left( \frac{p}{4} \right) = \Lambda ^{\mu} _{\ \nu} \left( - \frac{p}{4} \right) ,
\end{equation*}
thus turning the solution into
\begin{equation*}
    X^{\mu} (\tau , \sigma) = \frac{1}{2 \pi} \int \diff p \frac{1}{2 E (p)} \left( \Lambda ^{\mu} _{\ \nu} \left( \frac{p}{4} \right) a^{\nu} (0) \e ^{- i (- E \tau + p \sigma)} + \Lambda ^{\mu} _{\ \nu} \left( - \frac{p}{4} \right) b^{\nu} (0) \e ^{i (- E \tau + p \sigma)} \right) .
\end{equation*}

Next, we choose background coordinates s.t $a^{\mu} (0) = (a_{0} , a_{0}, 0 , ... , 0) = a_{0} x^{+}$ and $b^{\nu} (0) = (b_{0} , b_{0} , 0 , ... , 0) = b_{0} x^{+}$, simplifying the solution to
\begin{align*}
    X^{\mu} (\tau , \sigma) = \frac{1}{2 \pi} \int \diff p \frac{1}{2 E (p)} \left( a_{0} \Lambda ^{\mu} _{\ +} \left( \frac{p}{4} \right) \e ^{- i (- E \tau + p \sigma)} + b_{0} \Lambda ^{\mu} _{\ +} \left( - \frac{p}{4} \right) \e ^{i (- E \tau + p \sigma)} \right) x^{+}
\end{align*}

\newpage

\section{Solutions for the Polyakov KG eqn}

From the inverse-area corrected Polyakov action we have the equation
\begin{equation*}
    (\partial _{\sigma} ^2 - \partial _{\tau} ^2) X^{\mu} - m^2 X^{\mu} = 0
\end{equation*}
subjected to the constraint
\begin{equation*}
    \partial _{c} X^{\mu} \partial _{d} X^{\nu} \eta _{\mu \nu} - \frac{1}{2} g_{c d} g^{a b} \partial _{a} X^{\mu} \partial _{b} X^{\nu} \eta _{\mu \nu} = 0 ,
\end{equation*}
which with conformal gauge $g_{a b} = \phi \eta _{a b}$ becomes
\begin{equation*}
    \partial _{c} X^{\mu} \partial _{d} X^{\nu} \eta_{\mu \nu} = \frac{1}{2} \eta _{c d} \eta ^{a b} \partial _{a} X^{\mu} \partial _{b} X^{\nu} \eta _{\mu \nu}
\end{equation*}
or more explicitly,
\begin{align*}
    \partial _{\tau} X^{\mu} \partial _{\tau} X^{\nu} \eta _{\mu \nu}     & = - \frac{1}{2} \eta ^{a b} \partial _{a} X^{\mu} \partial _{b} X^{\nu} \eta _{\mu \nu} \\
    \partial _{\sigma} X^{\mu} \partial _{\sigma} X^{\nu} \eta _{\mu \nu} & = \frac{1}{2} \eta ^{a b} \partial _{a} X^{\mu} \partial _{b} X^{\nu} \eta _{\mu \nu}   \\
    \partial _{\tau} X^{\mu} \partial _{\sigma} X^{\nu} \eta_{\mu \nu}    & = \partial _{\sigma} X^{\mu} \partial _{\tau} X^{\nu} \eta _{\mu \nu} = 0 .
\end{align*}
These simplify further by expanding the summation on $a,b$:
\begin{equation*}
    \partial _{\tau} X^{\mu} \partial _{\tau} X^{\nu} \eta _{\mu \nu} = - \frac{1}{2} \left( - \partial _{\tau} X^{\mu} \partial _{\tau} X^{\nu} \eta _{\mu \nu} + \partial _{\sigma} X^{\mu} \partial _{\sigma} X^{\nu} \eta _{\mu \nu} \right)
\end{equation*}
\begin{equation*}
    \partial _{\tau} X^{\mu} \partial _{\tau} X^{\nu} \eta _{\mu \nu} + \partial _{\sigma} X^{\mu} \partial _{\sigma} X^{\nu} \eta _{\mu \nu} = 0 .
\end{equation*}
These can be combined by multiplying the previous one by 2 and summing, leaving us with
\begin{equation*}
    (\partial _{\tau} X^{\mu} + \partial _{\sigma} X^{\mu}) (\partial _{\tau} X^{\nu} + \partial _{\sigma} X^{\nu}) \eta _{\mu \nu} = 0 .
\end{equation*}

The EoM is just a one-dimensional finite-space Klein-Gordon equation, which has general solution given by Fourier transform
\begin{equation*}
    X^{\mu} (\tau , \sigma) = \frac{1}{2 \pi \alpha '} \sum _{n \in \mathbb{Z}} \frac{1}{2 E_{n}} \left( a^{\mu} _{n} \e ^{i (- E_{n} \tau + n \sigma)} + b^{\mu} _{n} \e ^{i (E_{n} \tau + n \sigma)} \right) ,
\end{equation*}
with $E_{n} = \sqrt{n^2 + m^2}$. The derivatives are
\begin{align*}
    \partial _{\tau} X^{\mu}   & = \frac{i}{4 \pi \alpha '} \sum _{n \in \mathbb{Z}} \left( - a^{\mu} _{n} \e ^{i (- E_{n} \tau + n \sigma)} + b^{\mu} _{n} \e ^{i (E_{n} \tau + n \sigma)} \right)                 \\
    \partial _{\sigma} X^{\mu} & = \frac{i}{4 \pi \alpha '} \sum _{n \in \mathbb{Z}} \frac{n}{E_{n}} \left( a^{\mu} _{n} \e ^{i (- E_{n} \tau + n \sigma)} + b^{\mu} _{n} \e ^{i (E_{n} \tau + n \sigma)} \right) .
\end{align*}

Reality of $X^{\mu}$ implies
\begin{align*}
    (X^{\mu})^{*} & = \left( \frac{1}{2 \pi \alpha '} \sum _{n \in \mathbb{Z}} \frac{1}{2 E_{n}} \left( a^{\mu} _{n} \e ^{i (- E_{n} \tau + n \sigma)} + b^{\mu} _{n} \e ^{i (E_{n} \tau + n \sigma)} \right) \right) ^{*} =            \\
                  & = \frac{1}{2 \pi \alpha '} \sum _{n \in \mathbb{Z}} \frac{1}{2 E_{n}} \left( (a^{\mu} _{n})^{*} \e ^{- i (- E_{n} \tau + n \sigma)} + (b^{\mu} _{n})^{*} \e ^{- i (E_{n} \tau + n \sigma)} \right) \equiv X^{\mu} ,
\end{align*}
giving us
\begin{align*}
    (a^{\mu} _{n})^{*} & = b^{\mu} _{- n}   \\
    (b^{\mu} _{n})^{*} & = a^{\mu} _{- n} ,
\end{align*}
so the solution becomes for closed strings
\begin{equation*}
    X^{\mu} (\tau , \sigma) = \frac{1}{2 \pi \alpha '} \sum _{n \in \mathbb{Z}} \frac{1}{2 E_{n}} \left( a^{\mu} _{n} \e ^{i (- E_{n} \tau + n \sigma)} + (a^{\mu} _{- n})^{*} \e ^{i (E_{n} \tau + n \sigma)} \right) .
\end{equation*}
For open strings with free endpoints B.C we have general solution
\begin{equation*}
    X^{\mu} (\tau , \sigma) = \frac{1}{2 \pi \alpha '} \sum _{n \in \mathbb{Z}} \frac{1}{2 E_{n}} \left( a^{\mu} _{n} \e ^{- i E_{n} \tau} + b^{\mu} _{n} \e ^{i E_{n} \tau} \right) \cos (n \sigma) ,
\end{equation*}
which reality condition implies $b^{\mu} _{n} = (a^{\mu} _{n})^{*}$, thus open string with free endpoint solution becomes
\begin{equation*}
    X^{\mu} (\tau , \sigma) = \frac{1}{2 \pi \alpha '} \sum _{n \in \mathbb{Z}} \frac{1}{2 E_{n}} \left( a^{\mu} _{n} \e ^{- i E_{n} \tau} + (a^{\mu} _{n})^{*} \e ^{i E_{n} \tau} \right) \cos (n \sigma) .
\end{equation*}

Analysing the constraints, for the closed string we have that
\begin{equation*}
    \partial _{\tau} X^{\mu} + \partial _{\sigma} X^{\mu} = \frac{i}{4 \pi \alpha '} \sum _{n \in \mathbb{Z}} \left( \left( \frac{n}{E_{n}} - 1 \right) a^{\mu} _{n} \e ^{i (- E_{n} \tau + n \sigma)} + \left( \frac{n}{E_{n}} + 1 \right) b^{\mu} _{n} \e ^{i (E_{n} \tau + n \sigma)} \right) ,
\end{equation*}
so the constraint reads
\begin{multline*}
    (\partial _{\tau} X + \partial _{\sigma} X)^2 = - \frac{\eta _{\mu \nu}}{(4 \pi \alpha ')^2} \sum _{n} \sum _{p} \bigg( \bigg( \frac{n}{E_{n}} - 1 \bigg) \bigg( \frac{p}{E_{p}} - 1 \bigg) a^{\mu} _{n} a^{\nu} _{p} \e ^{i ( - (E_{n} + E_{p}) \tau + (n + p) \sigma)} + \\
    + \bigg( \frac{n}{E_{n}} - 1 \bigg) \bigg( \frac{p}{E_{p}} + 1 \bigg) a^{\mu} _{n} (a^{\nu} _{- p})^{*} \e ^{i (- (E_{n} - E_{p}) \tau + (n + p) \sigma)} + \bigg( \frac{n}{E_{n}} + 1 \bigg) \bigg( \frac{p}{E_{p}} - 1 \bigg) (a^{\mu} _{- n})^{*} a^{\nu} _{p} \e ^{i ((E_{n} - E_{p}) \tau + (n + p) \sigma)} + \\
    + \bigg( \frac{n}{E_{n}} + 1 \bigg) \bigg( \frac{p}{E_{p}} + 1 \bigg) (a^{\mu} _{- n})^{*} (a^{\nu} _{- p})^{*} \e ^{i ((E_{n} + E_{p}) \tau + (n + p) \sigma)} \bigg) \equiv 0 .
\end{multline*}
Since the exponentials are all linearly independent and we can't decouple the $\tau$ exponential from any of the sums, we have 3 cases to investigate: $n = p$, $n = - p$ and $n \neq \pm p$. For $n \neq \pm p$, we get
\begin{align*}
    L^{1} _{n , p} & := \bigg( \frac{n}{E_{n}} - 1 \bigg) \bigg( \frac{p}{E_{p}} - 1 \bigg) a^{\mu} _{n} a^{\nu} _{p} \eta _{\mu \nu} = 0                   \\
    L^{2} _{n , p} & := \bigg( \frac{n}{E_{n}} - 1 \bigg) \bigg( \frac{p}{E_{p}} + 1 \bigg) a^{\mu} _{n} (a^{\nu} _{- p})^{*} \eta _{\mu \nu} = 0           \\
    L^{3} _{n , p} & := \bigg( \frac{n}{E_{n}} + 1 \bigg) \bigg( \frac{p}{E_{p}} - 1 \bigg) (a^{\mu} _{- n})^{*} a^{\nu} _{p} \eta _{\mu \nu} = 0           \\
    L^{4} _{n , p} & := \bigg( \frac{n}{E_{n}} + 1 \bigg) \bigg( \frac{p}{E_{p}} + 1 \bigg) (a^{\mu} _{- n})^{*} (a^{\nu} _{- p})^{*} \eta _{\mu \nu} = 0 ,
\end{align*}
where $L^{2} _{n , p}$ and $L^{3} _{n , p}$ are just complex conjugates of each other with $n \mapsto - n$ and $p \mapsto - p$, so they are redundant. The same is true for $L^{1} _{n , p}$ and $L^{4} _{n , p}$, so in the end we have
\begin{align*}
    L_{n , p}             & := \bigg( \frac{n}{E_{n}} - 1 \bigg) \bigg( \frac{p}{E_{p}} - 1 \bigg) a^{\mu} _{n} a^{\nu} _{p} \eta _{\mu \nu} = 0           \\
    \widetilde{L}_{n , p} & := \bigg( \frac{n}{E_{n}} - 1 \bigg) \bigg( \frac{p}{E_{p}} + 1 \bigg) a^{\mu} _{n} (a^{\nu} _{- p})^{*} \eta _{\mu \nu} = 0 .
\end{align*}
The coefficients of both expressions are actually not necessary, since for them to be $0$ we would need $n \rightarrow \pm \infty$ or $p \rightarrow \pm \infty$, so we can redefine
\begin{align*}
    L_{n , p}             & := a^{\mu} _{n} a^{\nu} _{p} \eta _{\mu \nu} = 0                          \\
    \widetilde{L}_{n , p} & := a^{\mu} _{n} (a^{\nu} _{- p})^{*} \eta _{\mu \nu} = 0 , n \neq \pm p .
\end{align*}
For $n = - p$, the first and last terms are complex conjugates of each other with the relabeling $n \mapsto - n$ and $p \mapsto - p$, while the exponentials in the 2 middle terms are reduced to $1$, so we get
\begin{align*}
    \left( \frac{n}{E_{n}} - 1 \right) ^2 a^{\mu} _{n} a^{\nu} _{- n} \eta _{\mu \nu}                                             & = 0   \\
    \sum _{n} \left( \left( \left( \frac{n}{E_{n}} \right) ^2 - 1 \right) a^{\mu} _{n} (a^{\nu} _{n})^{*} \eta _{\mu \nu} \right) & = 0 ,
\end{align*}
where in the first equation we can ignore the first factor once more. Finally for $n = p$, we get
\begin{align*}
    \left( \frac{n}{E_{n}} - 1 \right) ^2 a^{\mu} _{n} a^{\nu} _{n} \eta _{\mu \nu}                        & = 0   \\
    \left( \left( \frac{n}{E_{n}} \right) ^2 - 1 \right) a^{\mu} _{n} (a^{\nu} _{- n})^{*} \eta _{\mu \nu} & = 0 ,
\end{align*}
where we again ignore the first factor because it's only $0$ at infinity, thus collecting everything we have
\begin{align*}
    L_{n , p} := a^{\mu} _{n} a^{\nu} _{p} \eta _{\mu \nu}                                                & = 0 , \ \forall \ n , p \in \mathbb{Z} , \\
    \widetilde{L}_{n , p} := a^{\mu} _{n} (a^{\nu} _{- p})^{*} \eta _{\mu \nu}                            & = 0 , \ n \neq - p ,                     \\
    \sum _{n} \left( \left( \left( \frac{n}{E_{n}} \right) ^2 - 1 \right) \widetilde{L}_{n , - n} \right) & = 0 .
\end{align*}

One last thing before quantizing: the $n = 0$ term in the solution reads
\begin{equation*}
    \frac{1}{2 \pi \alpha '} \frac{1}{2 m} \left( a^{\mu} _{0} \e ^{- i m \tau} + (a^{\mu} _{0})^{*} \e ^{i m \tau} \right) ,
\end{equation*}
which is divergent in the $m \rightarrow 0$ limit unless $\mathfrak{Re}(a^{\mu} _{0}) \sim m$. Keeping the analogy to the $m \rightarrow 0$ regime, we conclude $a^{\mu} _{0} = 2 \pi \alpha ' m x^{\mu} + 4 \pi i (\alpha ')^2 p^{\mu}$. Since $a^{\mu} _{0}$ includes $p^{\mu}$, we have that
\begin{equation*}
    M^2 = - p^{\mu} p_{\mu} = \frac{1}{16 \pi ^2 (\alpha ')^4} \left( (2 \pi \alpha ')^{2} x^{\mu} x^{\nu} - a^{\mu} _{0} (a^{\nu} _{0})^{*} \right) \eta _{\mu \nu} ,
\end{equation*}
which after analysing closer the $n = - p$ condition
\begin{equation*}
    \sum _{n} \left( \left( \left( \frac{n}{E_{n}} \right) ^2 - 1 \right) \widetilde{L}_{n , - n} \right) = 0
\end{equation*}
\begin{equation*}
    \sum _{n \neq 0} \left( \left( \left( \frac{n}{E_{n}} \right) ^2 - 1 \right) \widetilde{L}_{n , - n} \right) - \widetilde{L}_{0 , 0} = 0
\end{equation*}
\begin{equation*}
    \sum _{n \neq 0} \left( \left( \left( \frac{n}{E_{n}} \right) ^2 - 1 \right) \widetilde{L}_{n , - n} \right) = a^{\mu} _{0} (a^{\nu} _{0})^{*} \eta _{\mu \nu} \equiv (4 \pi (\alpha ')^2)^2 p^{\mu} p_{\mu} + (2 \pi \alpha ' m)^{2} x^{\mu} x_{\mu} ,
\end{equation*}
we have the string mass in terms of vibrational modes... (and center of mass position?!?)
\begin{align*}
    M^2 & = \frac{1}{(4 \pi (\alpha ')^2)^2} \left( (2 \pi \alpha ' m)^{2} x^{\mu} x_{\mu} - \sum _{n \neq 0} \left( \left( \left( \frac{n}{E_{n}} \right) ^2 - 1 \right) \widetilde{L}_{n , - n} \right) \right) =                         \\
        & = \frac{1}{(4 \pi (\alpha ')^2)^2} \left( (2 \pi \alpha ' m)^{2} x^{\mu} x_{\mu} - \sum _{n \neq 0} \left( \left( \left( \frac{n}{E_{n}} \right) ^2 - 1 \right) a^{\mu} _{n} (a^{\nu} _{n})^{*} \eta _{\mu \nu} \right) \right) .
\end{align*}

As for the open string, its derivatives are
\begin{align*}
    \partial _{\tau} X^{\mu}   & = \frac{i}{4 \pi \alpha '} \sum _{n} \left( - a^{\mu} _{n} \e ^{- i E_{n} \tau} + (a^{\mu} _{n})^{*} \e ^{i E_{n} \tau} \right) \cos (n \sigma)                 \\
    \partial _{\sigma} X^{\mu} & = - \frac{1}{4 \pi \alpha '} \sum _{n} \frac{n}{E_{n}} \left( a^{\mu} _{n} \e ^{- i E_{n} \tau} + (a^{\mu} _{n})^{*} \e ^{i E_{n} \tau} \right) \sin (n \sigma)
\end{align*}
and the constraint reads
\begin{equation*}
    \partial _{\tau} X^{\mu} + \partial _{\sigma} X^{\mu} = \frac{1}{4 \pi \alpha '} \sum _{n} \left( - a^{\mu} _{n} \e ^{- i E_{n} \tau} \left( i \cos (n \sigma) + \frac{n}{E_{n}} \sin (n \sigma) \right) + (a^{\mu} _{n})^{*} \e ^{i E_{n} \tau} \left( i \cos (n \sigma) - \frac{n}{E_{n}} \sin (n \sigma) \right) \right)
\end{equation*}
\begin{multline*}
    (\partial _{\tau} X + \partial _{\sigma} X)^2 = \frac{\eta _{\mu \nu}}{(4 \pi \alpha ')^2} \sum _{n} \sum _{p} \bigg( a^{\mu} _{n} a^{\nu} _{p} \e ^{- i (E_{n} + E_{p}) \tau} \left( i \cos (n \sigma) + \frac{n}{E_{n}} \sin (n \sigma) \right) \left( i \cos (p \sigma) + \frac{p}{E_{p}} \sin (p \sigma) \right) - \\
    - a^{\mu} _{n} (a^{\nu} _{p})^{*} \e ^{- i (E_{n} - E_{p}) \tau} \left( i \cos (n \sigma) + \frac{n}{E_{n}} \sin (n \sigma) \right) \left( i \cos (p \sigma) - \frac{p}{E_{p}} \sin (p \sigma) \right) - \\
    - (a^{\mu} _{n})^{*} a^{\nu} _{p} \e ^{i (E_{n} - E_{p}) \tau} \left( i \cos (n \sigma) - \frac{n}{E_{n}} \sin (n \sigma) \right) \left( i \cos (p \sigma) + \frac{p}{E_{p}} \sin (p \sigma) \right) + \\
    + (a^{\mu} _{n})^{*} (a^{\nu} _{p})^{*} \e ^{i (E_{n} + E_{p}) \tau} \left( i \cos (n \sigma) - \frac{n}{E_{n}} \sin (n \sigma) \right) \left( i \cos (p \sigma) - \frac{p}{E_{p}} \sin (p \sigma) \right) \bigg) = 0 .
\end{multline*}
Once more, we see that the middle 2 terms and outer 2 terms are complex conjugate pairs. Unlike the closed string, here all cases lead to
\begin{align*}
    L_{n , p} : = a^{\mu} _{n} a^{\nu} _{p} \eta _{\mu \nu}                   & = 0   \\
    \widetilde{L}_{n , p} : = a^{\mu} _{n} (a^{\nu} _{p})^{*} \eta _{\mu \nu} & = 0 ,
\end{align*}
where again to be more in line with ST literature we remap $p \mapsto p - n$
\begin{align*}
    L_{n , p} = a^{\mu} _{n} a^{\nu} _{p - n} \eta _{\mu \nu}                   & = 0   \\
    \widetilde{L}_{n , p} = a^{\mu} _{n} (a^{\nu} _{p - n})^{*} \eta _{\mu \nu} & = 0 .
\end{align*}
Since we don't have a summed constraint, the mass of the open string can't be written in terms of vibrational modes, only in terms of the 0-th mode
\begin{align*}
    M^2 = - p^{\mu} p_{\mu} & = \frac{1}{(4 \pi (\alpha ')^2 )^2} a^{\mu} _{0} a^{\nu} _{0} \eta _{\mu \nu} = \\
                            & = \frac{1}{(4 \pi (\alpha ')^2 )^2} L_{0 , 0} \equiv 0
\end{align*}

\subsection{The Expansion Modes Algebra}

\subsubsection{Closed String}

Before starting this analysis, let's rename the vibrational modes to $\alpha ^{\mu} _{n}$ to be more in line with the literature

Start by finding the canonical momentum
\begin{equation*}
    \Pi _{\lambda} = \frac{\partial \Lagr}{\partial (\partial _{\tau} X^{\lambda})} = - \frac{T}{2} \left( \sqrt{- g} - \frac{k \Delta}{2 \sqrt{- g}} \right) g^{a b} \delta ^{\tau} _{a} \delta ^{\mu} _{\lambda} \partial _{b} X^{\nu} \eta _{\mu \nu} =
\end{equation*}
\begin{equation*}
    = - \frac{1}{4 \pi \alpha '} \left( 1 - \frac{k \Delta}{2 \phi ^2} \right) \eta ^{\tau b} \partial _{b} X^{\nu} \eta _{\lambda \nu} =
\end{equation*}
\begin{equation*}
    = \frac{1}{4 \pi \alpha '} F^{-} \partial _{\tau} X^{\nu} \eta_{\lambda \nu} = F^{-} \frac{i}{4 \pi \alpha '} \sum _{n \in \mathbb{Z}} \left( - \alpha ^{\nu} _{n} \e ^{i (- E_{n} \tau + n \sigma)} + (\alpha ^{\nu} _{- n})^{*} \e ^{i (E_{n} \tau + n \sigma)} \right) \eta _{\nu \lambda} .
\end{equation*}
We then proceed to the cannonical Poisson bracket relations
\begin{equation*}
    \left\{ X^{\mu} (\tau , \sigma) , \Pi ^{\nu} (\tau , \sigma ') \right\} = \delta (\sigma - \sigma ') \eta ^{\mu \nu}
\end{equation*}
\begin{equation*}
    \left\{ X^{\mu} (\tau , \sigma) , X^{\nu} (\tau , \sigma ') \right\} = \left\{ \Pi ^{\mu} (\tau , \sigma) , \Pi ^{\nu} (\tau , \sigma ') \right\} = 0 ,
\end{equation*}
where by expanding the first relation we get
\begin{multline*}
    \left\{ X^{\mu} (\tau , \sigma) , \Pi ^{\nu} (\tau , \sigma ') \right\} = \bigg\{ \frac{1}{2 \pi \alpha '} \sum _{n \in \mathbb{Z}} \frac{1}{2 E_{n}} \left( \alpha ^{\mu} _{n} \e ^{i (- E_{n} \tau + n \sigma)} + (\alpha ^{\mu} _{- n})^{*} \e ^{i (E_{n} \tau + n \sigma)} \right) , \\
    F^{-} \frac{i}{4 \pi \alpha '} \sum _{p \in \mathbb{Z}} \left( - \alpha ^{\nu} _{p} \e ^{i (- E_{p} \tau + p \sigma ')} + (\alpha ^{\nu} _{- p})^{*} \e ^{i (E_{p} \tau + p \sigma ')} \right) \bigg\} =
\end{multline*}
\begin{multline*}
    = \frac{i}{8 \pi ^2 (\alpha ')^2} \sum _{n} \sum _{p} \frac{1}{2 E_{n}} \bigg( - F^{-} \left\{ \alpha ^{\mu} _{n} , \alpha ^{\nu} _{p} \right\} \e ^{- i (E_{n} + E_{p}) \tau} \e ^{i (n \sigma + p \sigma ')} + F^{-} \left\{ \alpha ^{\mu} _{n} , (\alpha ^{\nu} _{- p})^{*} \right\} \e ^{- i (E_{n} - E_{p}) \tau} \e ^{i (n \sigma + p \sigma ')} - \\
    - F^{-} \left\{ (\alpha ^{\mu} _{- n})^{*} , \alpha ^{\nu} _{p} \right\} \e ^{i (E_{n} - E_{p}) \tau} \e ^{i (n \sigma + p \sigma ')} + F^{-} \left\{ (\alpha ^{\mu} _{- n})^{*} , (\alpha ^{\nu} _{- p})^{*} \right\} \e ^{i (E_{n} + E_{p}) \tau} \e ^{i (n \sigma + p \sigma ')} \bigg) = \delta (\sigma - \sigma ') \eta ^{\mu \nu}
\end{multline*}
\begin{equation*}
    \bigg\downarrow \frac{1}{2 \pi} \int _{0} ^{2 \pi} \diff \sigma \e ^{- i k \sigma}
\end{equation*}
\begin{multline*}
    \frac{i}{8 \pi ^2 (\alpha ')^2} \sum _{n} \sum _{p} \frac{1}{2 E_{n}} \bigg( - F^{-} \left\{ \alpha ^{\mu} _{n} , \alpha ^{\nu} _{p} \right\} \e ^{- i (E_{n} + E_{p}) \tau} \delta _{n , k} \e ^{i p \sigma '} + F^{-} \left\{ \alpha ^{\mu} _{n} , (\alpha ^{\nu} _{- p})^{*} \right\} \e ^{- i (E_{n} - E_{p}) \tau} \delta _{n , k} \e ^{i p \sigma '} - \\
    - F^{-} \left\{ (\alpha ^{\mu} _{- n})^{*} , \alpha ^{\nu} _{p} \right\} \e ^{i (E_{n} - E_{p}) \tau} \delta _{n , k} \e ^{i p \sigma '} + F^{-} \left\{ (\alpha ^{\mu} _{- n})^{*} , (\alpha ^{\nu} _{- p})^{*} \right\} \e ^{i (E_{n} + E_{p}) \tau} \delta _{n , k} \e ^{i p \sigma '} \bigg) = \frac{\eta ^{\mu \nu}}{2 \pi} \e ^{- i k \sigma '}
\end{multline*}
\begin{multline*}
    \frac{i}{8 \pi ^2 (\alpha ')^2} \sum _{p} \frac{1}{2 E_{k}} \bigg( - F^{-} \left\{ \alpha ^{\mu} _{k} , \alpha ^{\nu} _{p} \right\} \e ^{- i (E_{k} + E_{p}) \tau} \e ^{i p \sigma '} + F^{-} \left\{ \alpha ^{\mu} _{k} , (\alpha ^{\nu} _{- p})^{*} \right\} \e ^{- i (E_{k} - E_{p}) \tau} \e ^{i p \sigma '} - \\
    - F^{-} \left\{ (\alpha ^{\mu} _{- k})^{*} , \alpha ^{\nu} _{p} \right\} \e ^{i (E_{k} - E_{p}) \tau} \e ^{i p \sigma '} + F^{-} \left\{ (\alpha ^{\mu} _{- k})^{*} , (\alpha ^{\nu} _{- p})^{*} \right\} \e ^{i (E_{k} + E_{p}) \tau} \e ^{i p \sigma '} \bigg) = \frac{\eta ^{\mu \nu}}{2 \pi} \e ^{- i k \sigma '}
\end{multline*}
\begin{equation*}
    \bigg\downarrow \frac{1}{2 \pi} \int _{0} ^{2 \pi} \diff \sigma ' \e ^{i m \sigma '}
\end{equation*}
\begin{equation*}
    \frac{1}{2 \pi} \int _{0} ^{2 \pi} \diff \sigma ' F^{-} \e ^{i (p + m) \sigma '} , \ F^{-} = 1 - \frac{k \Delta}{2 \phi ^2} , \ \mathrm{assume} \ \phi = constant
\end{equation*}
\begin{multline*}
    \frac{i}{8 \pi ^2 (\alpha ')^2} \frac{F^{-}}{2 E_{k}} \bigg( - \left\{ \alpha ^{\mu} _{k} , \alpha ^{\nu} _{- m} \right\} \e ^{- i (E_{k} + E_{m}) \tau} + \left\{ \alpha ^{\mu} _{k} , (\alpha ^{\nu} _{m})^{*} \right\} \e ^{- i (E_{k} - E_{m}) \tau} - \\
    - \left\{ (\alpha ^{\mu} _{- k})^{*} , \alpha ^{\nu} _{- m} \right\} \e ^{i (E_{k} - E_{m}) \tau} + \left\{ (\alpha ^{\mu} _{- k})^{*} , (\alpha ^{\nu} _{m})^{*} \right\} \e ^{i (E_{k} + E_{m}) \tau} \bigg) = \frac{\eta ^{\mu \nu}}{2 \pi} \delta _{m , k} .
\end{multline*}
The only way for this to hold is
\begin{align*}
    \left\{ \alpha ^{\mu} _{k} , \alpha ^{\nu} _{- m} \right\}     & = \left\{ (\alpha ^{\mu} _{- k})^{*} , (\alpha ^{\nu} _{m})^{*} \right\} = 0                                                                            \\
    \left\{ \alpha ^{\mu} _{k} , (\alpha ^{\nu} _{m})^{*} \right\} & = - \left\{ (\alpha ^{\mu} _{- k})^{*} , \alpha ^{\nu} _{- m} \right\} = - i \frac{4 \pi (\alpha ')^{2}}{F^{-}} E_{k} \eta ^{\mu \nu} \delta _{k , m} ,
\end{align*}
thus rescaling we get harmonic oscillator bracket relations
\begin{align*}
    a^{\mu} _{k}       & : = \frac{1}{2 \alpha '} \sqrt{\frac{F^{-}}{\pi}} \frac{1}{\sqrt{E_{k}}} \alpha ^{\mu} _{k}        \\
    (a^{\mu} _{k})^{*} & : =  \frac{1}{2 \alpha '} \sqrt{\frac{F^{-}}{\pi}} \frac{1}{\sqrt{E_{k}}} (\alpha ^{\mu} _{k})^{*}
\end{align*}
\begin{equation*}
    \Downarrow
\end{equation*}
\begin{align*}
    \left\{ a^{\mu} _{k} , (a^{\nu} _{p})^{*} \right\} & = - i \eta ^{\mu \nu} \delta _{k , p}                            \\
    \left\{ a^{\mu} _{k} , a^{\nu} _{p} \right\}       & = \left\{ (a^{\mu} _{k})^{*} , (a^{\nu} _{p})^{*} \right\} = 0 .
\end{align*}
For the $k = p = 0$ case, we also get the expected relations for $x^{\mu}$ and $p^{\nu}$:
\begin{equation*}
    \left\{ \alpha ^{\mu} _{0} , (\alpha ^{\nu} _{0})^{*} \right\} = - i \frac{4 \pi (\alpha ')^{2}}{F^{-}} E_{0} \eta ^{\mu \nu}
\end{equation*}
\begin{equation*}
    \left\{ 2 \pi \alpha ' \mu x^{\mu} + 4 \pi i (\alpha ')^{2} p^{\mu} , 2 \pi \alpha ' \mu x^{\nu} - 4 \pi i (\alpha ')^{2} p^{\nu} \right\} = - i \frac{4 \pi (\alpha ')^{2}}{F^{-}} \mu \eta ^{\mu \nu}
\end{equation*}
\begin{equation*}
    (2 \pi \alpha ')^{2} \bigg(  \mu ^2 \left\{ x^{\mu} , x^{\nu} \right\} - 2 i \alpha ' \mu \left\{ x^{\mu} , p^{\nu} \right\} + 2 i \alpha ' \mu \left\{ p^{\mu} , x^{\nu} \right\} + (2 \alpha ') \left\{ p^{\mu} , p^{\nu} \right\} \bigg) = - i \frac{4 \pi (\alpha ')^2}{F^{-}} \mu \eta ^{\mu \nu}
\end{equation*}
\begin{equation*}
    \Downarrow
\end{equation*}
\begin{align*}
    \left\{ x^{\mu} , x^{\nu} \right\} & = \left\{ p^{\mu} , p^{\nu} \right\} = 0           \\
    \left\{ x^{\mu} , p^{\nu} \right\} & = \frac{1}{4 \pi \alpha ' F^{-}} \eta ^{\mu \nu} .
\end{align*}
Since $F^{-} = 1 - k \Delta / (2 \phi ^2)$, if we choose
\begin{equation*}
    \phi = \sqrt{\frac{k \Delta}{2 \left( 1 - \dfrac{1}{4 \pi \alpha '} \right)}}
\end{equation*}
then the bracket relation simplifies to the expected
\begin{equation*}
    \left\{ x^{\mu} , p^{\nu} \right\} = \eta ^{\mu \nu} .
\end{equation*}

\subsection{The Constraint Algebra}

With the bracket relations between the vibrational modes in hand, we can now calculate the algebra generated by the constraints. For simplicity, since $E_{n} = \sqrt{n^2 + \mu ^2} \neq 0$, we can switch the constraints to be between the normalized harmonic oscillator modes $a^{\mu} _{n}$ intead of the $\alpha ^{\mu} _{n}$.
\begin{equation*}
    \left\{ L_{n , m} , L_{k , p} \right\} = \left\{ a^{\mu} _{n} a^{\nu} _{m} \eta _{\mu \nu} , a^{\mu '} _{k} a^{\nu '} _{p} \eta _{\mu ' \nu '} \right\} = 0
\end{equation*}
\begin{align*}
    \left\{ L_{n , m} , \widetilde{L}_{k , p} \right\} & = \left\{ a^{\mu} _{n} a^{\nu} _{m} \eta _{\mu \nu} , a^{\mu '} _{k} (a^{\nu '} _{- p})^{*} \eta _{\mu ' \nu '} \right\} =                                                                                                     \\
                                                       & = \eta _{\mu \nu} \left( a^{\nu} _{m} \left\{ a^{\mu} _{n} , (a^{\nu '} _{- p})^{*} \right\} a^{\mu '} _{k} + a^{\mu} _{n} \left\{ a^{\nu} _{m} , (a^{\nu '} _{- p})^{*} \right\} a^{\mu '} _{k} \right) \eta _{\mu ' \nu '} = \\
                                                       & = \eta _{\mu \nu} \left( a^{\nu} _{m}(- i \eta ^{\mu \nu '} \delta _{n , - p}) a^{\mu '} _{k} + a^{\mu} _{n} (- i \eta ^{\nu \nu '} \delta _{m , - p}) a^{\mu '} _{k} \right) \eta _{\mu ' \nu '} =                            \\
                                                       & = - i \left( a^{\mu} _{m} a^{\nu} _{k} \eta _{\mu \nu} \delta _{n , - p} + a^{\mu} _{n} a^{\nu} _{k} \eta _{\mu \nu} \delta _{m , - p} \right) =                                                                               \\
                                                       & = - i \left( L_{m, k} \delta _{n , - p} + L_{n , k} \delta _{m , - p} \right)
\end{align*}
\begin{align*}
    \left\{ \widetilde{L}_{n , m} , \widetilde{L}_{k , p} \right\} & = \left\{ a^{\mu} _{n} (a^{\nu} _{- m})^{*} \eta _{\mu \nu} , a^{\mu '} _{k} (a^{\nu '} _{- p})^{*} \eta _{\mu ' \nu '} \right\} =                                                                                                             \\
                                                                   & = \eta _{\mu \nu} \left( (a^{\nu} _{- m})^{*} \left\{ a^{\mu} _{n} , (a^{\nu '} _{- p})^{*} \right\} a^{\mu '} _{k} + a^{\mu} _{n} \left\{ (a^{\nu} _{- m})^{*} , a^{\mu '} _{k} \right\} (a^{\nu '} _{- p})^{*} \right) \eta _{\mu ' \nu '} = \\
                                                                   & = \eta _{\mu \nu} \left( (a^{\nu} _{- m})^{*} (- i \eta ^{\mu \nu '} \delta _{n , - p}) a^{\mu '} _{k} - a^{\mu} _{n} (- i \eta ^{\nu \mu '} \delta _{- m , k}) (a^{\nu '} _{- p})^{*} \right) \eta _{\mu ' \nu '} =                           \\
                                                                   & = - i \left( a^{\mu} _{k} (a^{\nu} _{- m})^{*} \eta _{\mu \nu} \delta _{n , - p} - a^{\mu} _{n} (a^{\nu} _{- p})^{*} \eta _{\mu \nu} \delta _{- m , k} \right) =                                                                               \\
                                                                   & = - i \left( \widetilde{L}_{k , m} \delta _{n , - p} - \widetilde{L}_{n , p} \delta _{- m , k} \right) .
\end{align*}

\newpage

\section{Into the Quantum Realm}

We now proceed to quantize the KG string through canonical quantization. The string $X^{\mu}$ and it's momentum $\Pi ^{\nu}$ are promoted to operators $\Hat{X}^{\mu}$ and $\Hat{\Pi}^{\nu}$, and we turn the Poisson bracket $\{ . , . \}$ into commutator between operators $i \hbar [. , .]$. The equal time bracket relations turn into equal time commutation relations
\begin{align*}
    \left[ \Hat{X}^{\mu} (\tau , \sigma) , \Hat{\Pi}^{\nu} (\tau , \sigma ') \right] & = i \delta (\sigma - \sigma ') \eta ^{\mu \nu}                                       \\
    \left[ \Hat{X}^{\mu} , \Hat{X}^{\nu} \right]                                     & = \left[ \Hat{\Pi}^{\mu} , \Hat{\Pi}^{\nu} \right] = 0                               \\
    \left[ \Hat{a}^{\mu} _{n} , \Hat{a}^{\nu} _{m} \right]                           & = \left[ (\Hat{a}^{\mu} _{n})^{\dagger} , (\Hat{a}^{\nu} _{m})^{\dagger} \right] = 0 \\
    \left[ \Hat{a}^{\mu} _{n} , (\Hat{a}^{\nu} _{m})^{\dagger} \right]               & = \eta ^{\mu \nu} \delta _{n , m}                                                    \\
    \left[ \Hat{x}^{\mu} , \Hat{x}^{\nu} \right]                                     & = \left[ \Hat{p}^{\mu} , \Hat{p}^{\nu} \right] = 0                                   \\
    \left[ \Hat{x}^{\mu} , \Hat{p}^{\nu} \right]                                     & = i \eta ^{\mu \nu} .
\end{align*}
Thus, the rescaled vibrational mode operators $\Hat{a}^{\mu} _{n}$ and $(\Hat{a}^{\nu} _{m})^{\dagger}$ are annihilation and creations operators, respectively.

We define a vacuum state of the string to obey
\begin{equation*}
    \Hat{a}^{\mu} _{n} \ket{0} = 0 , \ \mathrm{for} \ n \neq 0 .
\end{equation*}
For $n = 0$, we have the center of mass position and momentum operators, so the vacuum also obeys
\begin{align*}
    \Hat{x}^{\mu} \ket{0 ; x} & = x^{\mu} \ket{0 ; x}                               \\
    \Hat{p}_{\mu} \ket{0 ; x} & = - i \frac{\partial}{\partial x^{\mu}} \ket{0 ; x}
\end{align*}
in position representation or alternatively
\begin{align*}
    \Hat{x}^{\mu} \ket{0 ; p} & = - i \frac{\partial}{\partial p_{\mu}} \ket{0 ; p} \\
    \Hat{p}_{\mu} \ket{0 ; p} & = p_{\mu} \ket{0 ; p}
\end{align*}
in momentum representation.

A generic state arises from a sequence of creation operators on the vacuum
\begin{equation*}
    ((a^{\mu _{1}} _{1})^{\dagger})^{n_{\mu _1}} ((a^{\mu _2} _{2})^{\dagger})^{n_{\mu _2}} ... ((a^{\nu _1} _{- 1})^{\dagger})^{n_{\nu _1}} ((a^{\mu _2} _{- 2})^{\dagger})^{n_{\nu _2}} ... \ket{0} .
\end{equation*}
This (should) give rise to particles.

As in regular ST, we have ghosts arising from the Minkowski metric
\begin{equation*}
    \Big[ \Hat{a}^{\mu} _{n} , (\Hat{a}^{\nu} _{m})^{\dagger} \Big] = \eta ^{\mu \nu} \delta _{n m} .
\end{equation*}

For constraints, classically we have
\begin{align*}
    L_{n , p}             & = a ^{\mu} _{n} a ^{\nu} _{p} \eta _{\mu \nu} = 0         \\
    \widetilde{L}_{n , p} & = a ^{\mu} _{n} (a ^{\nu} _{- p})^{*} \eta _{\mu \nu} = 0
\end{align*}
which because of the existence of ghosts we require to have vanishing matrix elements when sandwiched between physical states
\begin{equation*}
    \bra{\mathrm{phys} '} \Hat{L}_{n , p} \ket{\mathrm{phys}} = 0 = \bra{\mathrm{phys} '} \Hat{\widetilde{L}}_{n , p} \ket{\mathrm{phys}} .
\end{equation*}
When translating the constraints into quantum operators, in $L_{n , p}$ we have no ambiguity of ordering since $\Hat{a}$ commutes with itself. As for $\widetilde{L}_{n , p}$, there is an ambiguity for $p = - n \neq 0$, so we pick normal ordering with the annihilation operators moved to the right
\begin{equation*}
    \Hat{\widetilde{L}}_{n , - n} = (\Hat{\alpha} ^{\mu} _{n})^{\dagger} \Hat{\alpha} ^{\nu} _{n} \eta _{\mu \nu} .
\end{equation*}
The ambiguity manifests in the imposition of this constraint as
\begin{equation*}
    \bra{\mathrm{phys }'} \left( \sum _{n} \left( \left( \left( \frac{n}{E_{n}} \right) ^2 - 1 \right) \widetilde{L}_{n , - n} \right) - c \right) \ket{\mathrm{phys}} = 0 ,
\end{equation*}
for some costant $c$. Since classically
\begin{align*}
    M^2 & = \frac{1}{(4 \pi (\alpha ')^2)^2} \left( (2 \pi \alpha ' \mu)^{2} x^{\mu} x_{\mu} - \sum _{n \neq 0} \left( \left( \left( \frac{n}{E_{n}} \right) ^2 - 1 \right) \widetilde{L}_{n , - n} \right) \right) =                         \\
        & = \frac{1}{(4 \pi (\alpha ')^2)^2} \left( (2 \pi \alpha ' \mu)^{2} x^{\mu} x_{\mu} - \sum _{n \neq 0} \left( \left( \left( \frac{n}{E_{n}} \right) ^2 - 1 \right) a^{\mu} _{n} (a^{\nu} _{n})^{*} \eta _{\mu \nu} \right) \right) .
\end{align*}
we see that the string mass spectrum will be affected by this costant
\begin{equation*}
    \Hat{M}^2 = \frac{1}{(4 \pi (\alpha ')^2)^2} \left( (2 \pi \alpha ' \mu)^{2} \Hat{x}^{\mu} \Hat{x}_{\mu} - \left( \sum _{n \neq 0} \left( \left( \frac{n}{E_{n}} \right) ^2 - 1 \right) (\Hat{a}^{\mu} _{n})^{\dagger} \Hat{a}^{\nu} _{n} \eta _{\mu \nu} - c \right) \right)
\end{equation*}

The commutation relations between the $L$'s are inherited from the bracket relations
\begin{align*}
    \left[ \Hat{L}_{n , m} , \Hat{L}_{k , p} \right]                         & = 0                                                                                               \\
    \left[ \Hat{L}_{n , m} , \Hat{\widetilde{L}}_{k , p} \right]             & = \Hat{L}_{m , k} \delta _{n , - p} + \Hat{L}_{n , k} \delta _{m , - p}                           \\
    \left[ \Hat{\widetilde{L}}_{n , m} , \Hat{\widetilde{L}}_{k , p} \right] & = \Hat{\widetilde{L}}_{k , m} \delta _{n , - p} - \Hat{\widetilde{L}}_{n , p} \delta _{- m , k} .
\end{align*}
Since $[\Hat{L}_{n , m} , \Hat{\widetilde{L}}_{p , k}]$ only has annihilation operators, it has no ordering ambiguities. $[\Hat{\widetilde{L}}_{n , m} , \Hat{\widetilde{L}}_{p , k}]$ have ordering ambiguities for $m = - k$ and/or $p = - n$, so we add the anomalous terms
\begin{equation*}
    \left[ \Hat{\widetilde{L}}_{n , m} , \Hat{\widetilde{L}}_{k , p} \right] = \Hat{\widetilde{L}}_{k , m} \delta _{n , - p} - \Hat{\widetilde{L}}_{n , p} \delta _{- m , k} + C_{n} \delta _{n , -p} + D_{k} \delta _{- m , k}
\end{equation*}
Clearly, $C_{0} = 0 = D_{0}$, since in these cases the equivalent terms are composed by $\Hat{\widetilde{L}}_{0 , 0} \sim \Hat{x}^{\mu} \Hat{x}_{\mu} + \Hat{p}^{\mu} \Hat{p}_{\mu}$. Also, because of the $\delta$, we also have that $C_{- n} = - C_{n}$ and $D_{- k} = - D_{k}$. We get from the jacobi identity
\begin{equation*}
    \left[ \Hat{\widetilde{L}}_{n , m} , \left[ \Hat{\widetilde{L}}_{k , p} , \Hat{\widetilde{L}}_{r , s} \right] \right] + \left[ \Hat{\widetilde{L}}_{r , s} , \left[ \Hat{\widetilde{L}}_{n , m} , \Hat{\widetilde{L}}_{k , p} \right] \right] + \left[ \Hat{\widetilde{L}}_{k , p} , \left[ \Hat{\widetilde{L}}_{r , s} , \Hat{\widetilde{L}}_{n , m} \right] \right] = 0
\end{equation*}
\begin{multline*}
    \left[ \Hat{\widetilde{L}}_{n , m} , \left( \Hat{\widetilde{L}}_{r , p} \delta _{k , - s} - \Hat{\widetilde{L}}_{k , s} \delta _{- p , r} + C_{k} \delta _{k , - s} + D_{r} \delta _{- p , r} \right) \right] + \\
    + \left[ \Hat{\widetilde{L}}_{r , s} , \left( \Hat{\widetilde{L}}_{k , m} \delta _{n , - p} - \Hat{\widetilde{L}}_{n , p} \delta _{- m , k} + C_{n} \delta _{n , - p} + D_{k} \delta _{- m , k} \right) \right] + \\
    + \left[ \Hat{\widetilde{L}}_{k , p} , \left( \Hat{\widetilde{L}}_{n , s} \delta _{r , - m} - \Hat{\widetilde{L}}_{r , m} \delta _{- s , n} + C_{r} \delta _{r , - m} + D_{n} \delta _{- s , n} \right) \right] = 0
\end{multline*}
\begin{multline*}
    \left[ \Hat{\widetilde{L}}_{n , m} , \Hat{\widetilde{L}}_{r , p} \right] \delta _{k , - s} - \left[ \Hat{\widetilde{L}}_{n , m} , \Hat{\widetilde{L}}_{k , s} \right] \delta _{- p , r} + \left[ \Hat{\widetilde{L}}_{r , s} , \Hat{\widetilde{L}}_{k , m} \right] \delta _{n , - p} - \\
    - \left[ \Hat{\widetilde{L}}_{r , s} , \Hat{\widetilde{L}}_{n , p} \right] \delta _{- m , k} + \left[ \Hat{\widetilde{L}}_{k , p} , \Hat{\widetilde{L}}_{n , s} \right] \delta _{r , - m} - \left[ \Hat{\widetilde{L}}_{k , p} , \Hat{\widetilde{L}}_{r , m} \right] \delta _{- s , n} = 0
\end{multline*}
\begin{multline*}
    \left( \Hat{\widetilde{L}}_{r , m} \delta _{n , - p} - \Hat{\widetilde{L}}_{n , p} \delta _{- m , r} + C_{n} \delta _{n , - p} + D_{r} \delta _{- m , r} \right) \delta _{k , - s} - \left( \Hat{\widetilde{L}}_{k , m} \delta _{n , - s} - \Hat{\widetilde{L}}_{n , s} \delta _{- m , k} + C_{n} \delta _{n , - s} + D_{k} \delta _{- m , k} \right) \delta _{- p , r} + \\
    + \left( \Hat{\widetilde{L}}_{k , s} \delta _{r , - m} - \Hat{\widetilde{L}}_{r , m} \delta _{- s , k} + C_{r} \delta _{r , - m} + D_{k} \delta _{- s , k} \right) \delta _{n , - p} - \left( \Hat{\widetilde{L}}_{n , s} \delta _{r , - p} - \Hat{\widetilde{L}}_{r , p} \delta _{- s , n} + C_{r} \delta _{r , - p} + D_{n} \delta _{- s , n} \right) \delta _{- m , k} + \\
    + \left( \Hat{\widetilde{L}}_{n , p} \delta _{k , - s} - \Hat{\widetilde{L}}_{k , s} \delta _{- p , n} + C_{k} \delta _{k , - s} + D_{n} \delta _{- p , n} \right) \delta _{r , - m} - \left( \Hat{\widetilde{L}}_{r , p} \delta _{k , - m} - \Hat{\widetilde{L}}_{k , m} \delta _{- p , r} + C_{k} \delta _{k , - m} + D_{r} \delta _{- p , r} \right) \delta _{- s , n} = 0 .
\end{multline*}
For $k = - s$, $n = - p$ and $r = - m$
\begin{multline*}
    \left( \Hat{\widetilde{L}}_{r , - r} - \Hat{\widetilde{L}}_{n , - n} + C_{n} + D_{r} \right) - \left( \Hat{\widetilde{L}}_{k , - r} \delta _{p , s} - \Hat{\widetilde{L}}_{n , - k} \delta _{r , k} + C_{n} \delta _{n , k} + D_{k} \delta _{r , k} \right) \delta _{n , m} + \\
    + \left( \Hat{\widetilde{L}}_{k , - k} - \Hat{\widetilde{L}}_{r , - r} + C_{r} + D_{k} \right) - \left( \Hat{\widetilde{L}}_{n , - k} \delta _{r , n} - \Hat{\widetilde{L}}_{r , - n} \delta _{k , n} + C_{r} \delta _{r , n} + D_{n} \delta _{k , n} \right) \delta _{r , k} + \\
    + \left( \Hat{\widetilde{L}}_{n , - n} - \Hat{\widetilde{L}}_{k , - k} + C_{k} + D_{n} \right) - \left( \Hat{\widetilde{L}}_{r , - n} \delta _{k , r} - \Hat{\widetilde{L}}_{k , - r} \delta _{n , r} + C_{k} \delta _{k , r} + D_{r} \delta _{n , r} \right) \delta _{k , n} = 0 .
\end{multline*}
Assuming $n \neq m$, $r \neq k$ and $k \neq n$ leaves us with
\begin{equation*}
    C_{n} + C_{r} + C_{k} + D_{r} + D_{k} + D_{n} = 0 ,
\end{equation*}
which for $n + k + r = 0$ becomes
\begin{equation*}
    C_{n} + C_{k} + C_{- n - k} + D_{n} + D_{k} + D_{- n - k} = 0
\end{equation*}
\begin{equation*}
    C_{n} + C_{k} - C_{n + k} + D_{n} + D_{k} - D_{n + k} = 0 ,
\end{equation*}
and now setting $k = 1$ gives
\begin{equation*}
    C_{n} - C_{n + 1} + C_{1} + D_{n} - D_{n + 1} + D_{1} = 0
\end{equation*}
\begin{equation*}
    C_{n + 1} - C_{n} - C_{1} = - \left( D_{n + 1} - D_{n} - D_{1} \right) ,
\end{equation*}
meaning $C_{n}$ and $D_{n}$ are negatives of each other differing by some constant $\beta$, so it suffices to find $C_{n}$ and later determine the constant $\beta$. It is not difficult to see that the solution to this difference eqn is $C_{n} = A n$, so $D_{n} = - A n + \beta$. Since we know $D_{- n} = - D_{n}$, then $\beta = 0$, and $A = C_{1}$. We now need to determine the value of $C_{1}$. By now, the commutator of $\Hat{\widetilde{L}}$ is
\begin{equation*}
    \left[ \Hat{\widetilde{L}}_{n , m} , \Hat{\widetilde{L}}_{k , p} \right] = \Hat{\widetilde{L}}_{k , m} \delta _{n , - p} - \Hat{\widetilde{L}}_{n , p} \delta _{- m , k} + C_{1} n \delta _{n , - p} - C_{1} k \delta _{- m , k} .
\end{equation*}
Let's calculate the VEV of this commutator for $m = k \neq 0$ with $n = - p = 1$
\begin{equation*}
    \bra{0} \left[ \Hat{\widetilde{L}}_{1 , k} , \Hat{\widetilde{L}}_{k , - 1} \right] \ket{0} =\bra{0} \Hat{\widetilde{L}}_{1 , k} \Hat{\widetilde{L}}_{k , - 1} \ket{0}
\end{equation*}
\begin{equation*}
    \bra{0} \Hat{\widetilde{L}}_{k , k} + C_{1} \ket{0} = \bra{0} (a^{\mu} _{- k})^{\dagger} a^{\nu} _{1} \eta _{\mu \nu} (a^{\mu'} _{1})^{\dagger} a^{\nu '} _{k} \eta _{\mu ' \nu '} \ket{0}
\end{equation*}
\begin{equation*}
    \bra{0} C_{1} \ket{0} = 0 \iff C_{1} = 0 ,
\end{equation*}
so in actuallity the constraint algebra has no anomalies.

Let us now denote the ground state of momentum $p^{\mu}$ as $\ket{0 ; p}$. The mass-shell condition
\begin{equation*}
    \Hat{M}^2 = \frac{1}{(4 \pi (\alpha ')^2)^2} \left( (2 \pi \alpha ' \mu)^{2} \Hat{x}^{\mu} \Hat{x}_{\mu} - \left( \sum _{n \neq 0} \left( \left( \frac{n}{E_{n}} \right) ^2 - 1 \right) (\Hat{a}^{\mu} _{n})^{\dagger} \Hat{a}^{\nu} _{n} \eta _{\mu \nu} - c \right) \right)
\end{equation*}
implies that
\begin{equation*}
    \bra{0 ; p} \Hat{M}^2 \ket{0 ; p} = \frac{1}{(4 \pi (\alpha ')^2)^2} \left( (2 \pi \alpha ' \mu)^{2} \bra{0 ; p} \Hat{x}^{\mu} \Hat{x}_{\mu} \ket{0 ; p} - \left( \sum _{n \neq 0} \left( \left( \frac{n}{E_{n}} \right) ^2 - 1 \right) \bra{0 ; p} (\Hat{a}^{\mu} _{n})^{\dagger} \Hat{a}^{\nu} _{n} \eta _{\mu \nu} \ket{0 ; p} - \bra{0 ; p} c \ket{0 ; p} \right) \right)
\end{equation*}
\begin{equation*}
    M^2 = \frac{1}{(4 \pi (\alpha ')^2)^2} \left( (2 \pi \alpha ' \mu)^{2} \frac{\partial \psi ^{*} (p)}{\partial p_{\mu}} \frac{\partial \psi (p)}{\partial p^{\mu}} + c \right)
\end{equation*}
\begin{equation*}
    (4 \pi (\alpha ')^2)^2 p^2 = - \left( (2 \pi \alpha ' \mu)^{2} \left| \frac{\partial \psi (p)}{\partial p} \right| ^2 + c \right)
\end{equation*}
Now looking at the first excited state $\zeta _{\mu} (\Hat{a}^{\mu} _{1})^{\dagger} \ket{0 ; p}$ with $\zeta _{\mu} = \zeta _{\mu} (p)$ being the polarization covector, the mass-shell now reads
\begin{multline*}
    \bra{0 ; p} \Hat{a}^{\mu} _{1} \zeta _{\mu} \Hat{M}^2 \zeta _{\nu} (\Hat{a}^{\nu} _{1})^{\dagger} \ket{0 ; p} = \frac{1}{(4 \pi (\alpha ')^2)^2} \Bigg( (2 \pi \alpha ' \mu)^{2} \bra{0 ; p} \Hat{a}^{\mu} _{1} \zeta _{\mu} \Hat{x}^{\mu '} \Hat{x}_{\mu '} \zeta _{\nu} (\Hat{a}^{\nu} _{1})^{\dagger} \ket{0 ; p} - \\
    - \left( \sum _{n \neq 0} \left( \left( \frac{n}{E_{n}} \right) ^2 - 1 \right) \bra{0 ; p} \Hat{a}^{\mu} _{1} \zeta _{\mu} (\Hat{a}^{\mu '} _{n})^{\dagger} \Hat{a}^{\nu '} _{n} \eta _{\mu ' \nu '} \zeta _{\nu} (\Hat{a}^{\nu} _{1})^{\dagger} \ket{0 ; p} - \bra{0 ; p} \Hat{a}^{\mu} _{1} \zeta _{\mu} c \zeta _{\nu} (\Hat{a}^{\nu} _{1})^{\dagger} \ket{0 ; p} \right) \Bigg)
\end{multline*}
\begin{equation*}
    M^2 \zeta ^{\mu} \zeta _{\mu} = \frac{1}{(4 \pi (\alpha ')^2)^2} \left( (2 \pi \alpha ' \mu)^{2} \left| \frac{\partial \psi (p)}{\partial p} \right| ^2 \zeta ^{\mu} \zeta _{\mu} - \left( \frac{1}{1 + \mu ^2} - 1 \right) \zeta ^{\mu} \zeta _{\mu} + c \zeta ^{\mu} \zeta _{\mu} \right)
\end{equation*}
\begin{equation*}
    (4 \pi (\alpha ')^2)^2 p^2 = - \left( (2 \pi \alpha ' \mu)^{2} \left| \frac{\partial \psi (p)}{\partial p} \right| ^2 - \left( \frac{1}{1 + \mu ^2} - 1 \right) + c \right) .
\end{equation*}
The auxiliary $\Hat{\widetilde{L}}_{1 , 0} \zeta _{\mu '} (\Hat{a}^{\mu '} _{1})^{\dagger} \ket{0 ; p} = 0$ condition implies
\begin{equation*}
    \Hat{\widetilde{L}}_{1 , 0} \zeta _{\mu '} (\Hat{a}^{\mu '} _{1})^{\dagger} \ket{0 ; p} = 0
\end{equation*}
\begin{equation*}
    (\Hat{a}^{\mu} _{0})^{\dagger} \Hat{a}^{\nu} _{1} \eta _{\mu \nu} \zeta _{\mu '} (\Hat{a}^{\mu '} _{1})^{\dagger} \ket{0 ; p} = 0
\end{equation*}
\begin{equation*}
    \zeta _{\mu} (\Hat{a}^{\mu} _{0})^{\dagger} \ket{0 ; p} = 0
\end{equation*}
\begin{equation*}
    \zeta _{\mu} \left( 2 \pi \alpha ' \mu \Hat{x}^{\mu} - 4 \pi (\alpha ')^2 i \Hat{p}^{\mu} \right) \ket{0 ; p} = 0
\end{equation*}
\begin{equation*}
    - 2 \pi \alpha ' \mu i \zeta _{\mu} \frac{\partial \psi (p)}{\partial p_{\mu}} = 4 \pi (\alpha ')^2 i \zeta _{\mu} p^{\mu}
\end{equation*}
\begin{equation*}
    \zeta _{\mu} p^{\mu} = - \frac{\mu}{2 \alpha '} \zeta _{\mu} \frac{\partial \psi (p)}{\partial p_{\mu}} .
\end{equation*}
If the momentum wave function is chosen to be (need to justify this)
\begin{equation*}
    \psi (p) = \frac{\alpha '}{\mu} p^2 ,
\end{equation*}
we get that $\zeta _{\mu} p^{\mu} = 0$. Using this in the mass-shell condition for the first excited state yields
\begin{equation*}
    (4 \pi (\alpha ')^2)^2 p^2 = - \left( (2 \pi \alpha ' \mu)^{2} 4 \frac{(\alpha ')^2}{\mu ^2} p^2 - \left( \frac{1}{1 + \mu ^2} - 1 \right) + c \right)
\end{equation*}
\begin{equation*}
    2 (4 \pi (\alpha ')^2)^2 p^2 = \frac{1}{1 + \mu ^2} - 1 - c ,
\end{equation*}
and for the ground state
\begin{equation*}
    2 (4 \pi (\alpha ')^2)^2 p^2 = - c .
\end{equation*}
In general, the $n$-th excited state will have momentum given by
\begin{equation*}
    2 (4 \pi (\alpha ')^2)^2 p^2 = \frac{n^2}{n^2 + \mu ^2} - 1 - c .
\end{equation*}
The norm of the states is given by $\zeta _{\mu} \zeta ^{\mu}$, thus if we choose $p$ to lie on the $(0 , 1)$-plane, then the $D - 2$ spacelike polarizations normal to that plane clearly have positive norm. If we choose $c$ such that the $n = d$ , $d = (D - 1)$ (we choose this because the solution only has $n = 0 , 1 , 2 , ... , d$ terms by the $\widetilde{L}$ constraints) excited state is a tachyon with $p^2 > 0$, then $p$ can be taken to have no time component and thus $\zeta$ is time-like with negative norm. If $p^2 < 0$, $p$ can be chosen to ONLY have a time component, leaving $\zeta$ spacelike with positive norm again. At last, if $p^2 = 0$, then $\zeta$ is proportional to $p$ with zero norm. Thus, for the absence of ghosts we require
\begin{equation*}
    c \geq \frac{d^2}{d^2 + \mu ^2} - 1 .
\end{equation*}
In the boundary case, the 3rd excited vector state is massless and the rest (including the scalar ground state) are tachyons... This analysis, however, is to get spin 1 particles from the closed string. The actual massless particle we want from closed strings is a symmetric spin 2 particle, so let's consider now the state $\zeta _{\mu \nu} (\Hat{a}^{\mu} _{d})^{\dagger} (\Hat{a}^{\nu} _{d})^{\dagger} \ket{0 ; p}$ where $\zeta _{\mu \nu} = \zeta _{\nu \mu}$ is the polarization tensor. The mass-shell is
\begin{multline*}
    \bra{0 ; p} \Hat{a}^{\alpha} _{d} \Hat{a}^{\beta} _{d} \zeta _{\alpha \beta} \Hat{M}^2 \zeta _{\mu \nu} (\Hat{a}^{\mu} _{d})^{\dagger} (\Hat{a}^{\nu} _{d})^{\dagger} \ket{0 ; p} = \frac{1}{(4 \pi (\alpha ')^2)^2} \Bigg( (2 \pi \alpha ' \mu)^2 \bra{0 ; p} \Hat{a}^{\alpha} _{d} \Hat{a}^{\beta} _{d} \zeta _{\alpha \beta} \Hat{x}^{\mu '} \Hat{x}_{\mu '} \zeta _{\mu \nu} (\Hat{a}^{\mu} _{d})^{\dagger} (\Hat{a}^{\nu} _{d})^{\dagger} \ket{0 ; p} - \\
    - \Bigg( \sum _{n \neq 0} \left( \left( \frac{n}{E_{n}} \right) ^2 - 1 \right) \bra{0 ; p} \Hat{a}^{\alpha} _{d} \Hat{a}^{\beta} _{d} \zeta _{\alpha \beta} (\Hat{a}^{\mu '} _{n})^{\dagger} \Hat{a}^{\nu '} _{n} \eta _{\mu ' \nu '} \zeta _{\mu \nu} (\Hat{a}^{\mu} _{d})^{\dagger} (\Hat{a}^{\nu} _{d})^{\dagger} \ket{0 ; p} - \bra{0 ; p} \Hat{a}^{\alpha} _{d} \Hat{a}^{\beta} _{d} \zeta _{\alpha \beta} c \zeta _{\mu \nu} (\Hat{a}^{\mu} _{d})^{\dagger} (\Hat{a}^{\nu} _{d})^{\dagger} \ket{0 ; p} \Bigg) \Bigg)
\end{multline*}
\begin{equation*}
    M^2 \zeta _{\mu \nu} \zeta ^{\mu \nu} = \frac{1}{(4 \pi (\alpha ')^2)^2} \Bigg( (2 \pi \alpha ' \mu)^2 \left| \frac{\partial \psi (p)}{\partial p} \right| ^2 \zeta _{\mu \nu} \zeta ^{\mu \nu} - \left( 2 \left( \frac{d^2}{d^2 + \mu ^2} - 1 \right) \zeta _{\mu \nu} \zeta ^{\mu \nu} - c \zeta _{\mu \nu} \zeta ^{\mu \nu} \right) \Bigg)
\end{equation*}
\begin{equation*}
    (4 \pi (\alpha ')^2)^2 p^2 = - \Bigg( (2 \pi \alpha ' \mu)^2 \left| \frac{\partial \psi (p)}{\partial p} \right| ^2 - \left( 2 \left( \frac{d^2}{d^2 + \mu ^2} - 1 \right) - c \right) \Bigg)
\end{equation*}
\begin{equation*}
    2 (4 \pi (\alpha ')^2)^2 p^2 = 2 \left( \frac{d^2}{d^2 + \mu ^2} - 1 \right) - c .
\end{equation*}
Using the auxiliary $\Hat{\widetilde{L}}_{d , 0} \zeta _{\mu \nu} (\Hat{a}^{\mu} _{d})^{\dagger} (\Hat{a}^{\nu} _{d})^{\dagger} \ket{0 ; p} = 0$ condition implies
\begin{equation*}
    \Hat{\widetilde{L}}_{d , 0} \zeta _{\mu \nu} (\Hat{a}^{\mu} _{d})^{\dagger} (\Hat{a}^{\nu} _{d})^{\dagger} \ket{0 ; p} = 0
\end{equation*}
\begin{equation*}
    (\Hat{a}^{\mu '} _{0})^{\dagger} \Hat{a}^{\nu '} _{d} \eta _{\mu ' \nu '} \zeta _{\mu \nu} (\Hat{a}^{\mu} _{d})^{\dagger} (\Hat{a}^{\nu} _{d})^{\dagger} \ket{0 ; p} = 0
\end{equation*}
\begin{equation*}
    \left( (\Hat{a}^{\mu '} _{0})^{\dagger} \eta _{\mu ' \nu '} \zeta _{\mu \nu} \eta ^{\nu ' \mu} (a^{\nu} _{d})^{\dagger} + (\Hat{a}^{\mu '} _{0})^{\dagger} \eta _{\mu ' \nu '} \zeta _{\mu \nu} (\Hat{a}^{\mu} _{d})^{\dagger} \Hat{a}^{\nu '} _{d} (\Hat{a}^{\nu} _{d})^{\dagger} \right) \ket{0 ; p} = 0
\end{equation*}
\begin{equation*}
    \left( (\Hat{a}^{\mu '} _{0})^{\dagger} \eta _{\mu ' \nu '} \zeta _{\mu \nu} \eta ^{\nu ' \mu} (a^{\nu} _{d})^{\dagger} + (\Hat{a}^{\mu '} _{0})^{\dagger} \eta _{\mu ' \nu '} \zeta _{\mu \nu} (\Hat{a}^{\mu} _{d})^{\dagger} \eta ^{\nu ' \nu} \right) \ket{0 ; p} = 0
\end{equation*}
\begin{equation*}
    2 \zeta _{\mu \nu} (\Hat{a}^{\mu} _{0})^{\dagger} (\Hat{a}^{\nu} _{d})^{\dagger} \ket{0 ; p} = 0
\end{equation*}
\begin{equation*}
    \zeta _{\mu \nu} (\Hat{a}^{\nu} _{d})^{\dagger} \left( 2 \pi \alpha ' \mu \Hat{x}^{\mu} - 4 \pi (\alpha ')^2 i \Hat{p}^{\mu} \right) \ket{0 ; p} = 0
\end{equation*}
\begin{equation*}
    \zeta _{\mu \nu} \left( - 2 \pi \alpha ' \mu i \frac{\partial \psi (p)}{\partial p_{\mu}} - 4 \pi (\alpha ')^2 i p^{\mu} \right) (\Hat{a}^{\nu} _{d})^{\dagger} \ket{0} = 0
\end{equation*}
\begin{equation*}
    \zeta _{\mu \nu} p^{\mu} (\Hat{a}^{\nu} _{d})^{\dagger} \ket{0} = 0 ,
\end{equation*}
implying that
\begin{equation*}
    \zeta _{\mu \nu} p^{\mu} = \zeta _{\mu \nu} p^{\nu} = 0 .
\end{equation*}
Since the norm of the spin-2 states is given by $\zeta ^{\mu \nu} \zeta _{\mu \nu}$, we need those to not be negative. Since $\zeta$ is symmetric and we work in $(- , + , + , ...)$ signature, $\zeta ^{\mu \nu} \zeta _{\mu \nu} \geq 0$ always hold, so we just need to choose $c$ such that we get ate leats 1 massless spin-2 state. For that, we see that
\begin{equation*}
    c = 2 \left( \frac{n^2}{n^2 + \mu ^2} - 1 \right)
\end{equation*}
does the trick for the $n$-th state, and since for the non-existence of ghosts in the spin-1 states we required
\begin{equation*}
    c \geq \frac{d^2}{d^2 + \mu ^2} - 1 ,
\end{equation*}
which is negative, we can thus choose
\begin{equation*}
    c \equiv \frac{2 d^2}{d^2 + \mu ^2} - 2 ,
\end{equation*}
leaving the $d$-th spin-2 state massless, the rest being tachyons... This, however, is if we impose that one spin-2 state have to be massless. Since granular space-time affects wave propagation speed, in the effective limit gravitons might be effectively massive, thus we can get rid of the tachyons by letting $c > 0$ if need be.

Let's now analyse spurious states to see the conditions to maximize 0-norm states. A physical state $\ket{\phi}$ is a state satisfying the constraints
\begin{align*}
    \Hat{L}_{n , m} \ket{\phi}                                                                                                                & = 0 , \ \forall n , m \in \mathbb{Z} \\
    \Hat{\widetilde{L}}_{n , m} \ket{\phi}                                                                                                    & = 0 , \ n \neq - m                   \\
    \left( \sum _{n} \left( \left( \left( \frac{n}{E_{n}} \right) ^2 - 1 \right) \Hat{\widetilde{L}}_{n , - n} \right) - c \right) \ket{\phi} & = 0 .
\end{align*}
A spurious state $\ket{\psi}$ is a state satisfying the last constraint and orthogonal to all physical states, i.e,
\begin{equation*}
    \left( \sum _{n} \left( \left( \left( \frac{n}{E_{n}} \right) ^2 - 1 \right) \Hat{\widetilde{L}}_{n , - n} \right) - c \right) \ket{\psi} = 0 ,
\end{equation*}
\begin{equation*}
    \langle \phi | \psi \rangle = 0 .
\end{equation*}
Spurious states can be written in the from
\begin{equation*}
    \ket{\psi} = \sum _{n \neq - m} \Hat{\widetilde{L}}_{n , m} \ket{\chi _{n , m}}
\end{equation*}
with $\ket{\chi _{n , m}}$ satisfying
\begin{equation*}
    \left( \sum _{n} \left( \left( \left( \frac{n}{E_{n}} \right) ^2 - 1 \right) \Hat{\widetilde{L}}_{n , - n} \right) - c - k - p \right) \ket{\chi _{k , p}} = 0 .
\end{equation*}
To see that this state is indeed orthogonal to $\ket{\phi}$, we perform the inner product
\begin{equation*}
    \langle \phi | \psi \rangle = \sum _{n \neq - m} \bra{\phi} \Hat{\widetilde{L}}_{n , m} \ket{\chi _{n , m}} = \sum _{n \neq - m} \bra{\chi _{n , m}} \Hat{\widetilde{L}}_{- m , - n} \ket{\phi} ^{*} \equiv 0 .
\end{equation*}
The interesting case is when $\ket{\psi}$ is both physical AND spurious, since by construction those states have 0-norm, being orthogonal to every physical state, including themselves. Such states can be constructed by considering spurious states of the form
\begin{equation*}
    \ket{\psi _{- 2 , n}} = \sum _{m} \Hat{\widetilde{L}}_{n , m} \ket{\widetilde{\chi}_{- 2 , m}}
\end{equation*}
where $\ket{\widetilde{\chi}_{- 2 , m}}$ satisfies
\begin{align*}
    \Hat{\widetilde{L}}_{k , p} \ket{\widetilde{\chi}_{- 2 , m}}                                                                                                                 & = 0 , \ k \neq - p \\
    \left( \sum _{n'} \left( \left( \left( \frac{n'}{E_{n'}} \right) ^2 - 1 \right) \Hat{\widetilde{L}}_{n' , - n'} \right) - c + 2 - m \right) \ket{\widetilde{\chi}_{- 2 , m}} & = 0 .
\end{align*}
These states are annihilated by $\Hat{\widetilde{L}}_{k , p}$, for $p \neq - n$, and for this case we have that
\begin{equation*}
    \sum _{n} \left( \left( \left( \frac{n}{E_{n}} \right) ^2 - 1 \right) \Hat{\widetilde{L}}_{- p , - n} \ket{\psi _{- 2 , n}} \right) =
\end{equation*}
\begin{equation*}
    = \sum _{n} \left( \left( \left( \frac{n}{E_{n}} \right) ^2 - 1 \right) \Hat{\widetilde{L}}_{- 1 , - n} \sum _{m} \Hat{\widetilde{L}}_{n , m} \ket{\widetilde{\chi}_{- 2 , m}} \right)  =
\end{equation*}
\begin{equation*}
    = \sum _{n} \left( \left( \left( \frac{n}{E_{n}} \right) ^2 - 1 \right) \Hat{\widetilde{L}}_{n , - n} \right) \ket{\widetilde{\chi}_{- 2 , 1}} ,
\end{equation*}
which vanish if $c = 1$.

\newpage

\section{Polyakov Action in terms of LQG Variables}

Turn embbeding fields $X^{\mu}$ into vector in the spin-1 representation of $\mathrm{Spin}(d , 1)$, $X^{I}$, and promote partial derivative to covariant derivative $\partial _{a} \mapsto \mc{D}_{a}$ acting as
\begin{equation*}
    \mc{D}_{a} X^{I} = \partial _{a} X^{I} + k \mc{A}^{I} _{a J} X^{J} ,
\end{equation*}
where $\mc{A}^{I} _{a J} (x) = (A^{\alpha \beta} _{a} (x) T_{\alpha \beta})^{I} _{\ J}$ is the WS $\mathrm{Spin}(d , 1)$ connection, so the action becomes
\begin{equation*}
    S = - \frac{T}{2} \int \diff ^2 x \sqrt{- g} g^{a b} \mc{D}_{a} X^{I} \mc{D}_{b} X^{J} \eta _{I J} ,
\end{equation*}
where the WS metric can be recast in terms of auxiliary zweibein fields as
\begin{equation*}
    S = - \frac{T}{2} \int \diff ^2 x e e^{a} _{i} e^{b i} \mc{D}_{a} X^{I} \mc{D}_{b} X^{J} \eta _{I J} .
\end{equation*}

\subsection{EoMs}

\subsubsection{W.r.t $e$}

\begin{align*}
    \frac{\delta S}{\delta e^{c} _{k}} & = \frac{T}{2} \left( \frac{\partial}{\partial e^{c} _{k}} e e^{a} _{i} e^{b i} + e \frac{\partial}{\partial e^{c} _{k}} (e^{a} _{i} e^{b i}) \right) \mc{D}_{a} X^{I} \mc{D}_{b} X^{J} \eta _{I J} = \\
                                       & = T \left( - e e^{k} _{c} e^{a} _{i} e^{b i} + e \delta ^{a} _{c} \delta ^{k} _{i} e^{b i} \right) \mc{D}_{a} X^{I} \mc{D}_{b} X^{J} \eta _{I J} =                                                   \\
                                       & = T \left( - e e^{k} _{c} e^{a} _{i} e^{b i} \mc{D}_{a} X^{I} \mc{D}_{b} X^{J} \eta _{I J} + e e^{b k} \mc{D}_{c} X^{I} \mc{D}_{b} X^{J} \eta _{I J} \right) \overset{!}{=} 0
\end{align*}
\begin{equation*}
    T^{k} _{c} := e^{b k} \mc{D}_{c} X^{I} \mc{D}_{b} X^{J} \eta _{I J} - e^{k} _{c} e^{a} _{i} e^{b i} \mc{D}_{a} X^{I} \mc{D}_{b} X^{J} \eta _{I J} = 0
\end{equation*}
\begin{equation*}
    e^{k} _{c} = f e^{b k} \mc{D}_{c} X^{I} \mc{D}_{b} X^{J} \eta _{I J} ,
\end{equation*}
\begin{equation*}
    \frac{1}{f} = e^{a} _{i} e^{b i} \mc{D}_{a} X^{I} \mc{D}_{b} X^{J} \eta _{I J}
\end{equation*}

\subsubsection{W.r.t $A$}

\begin{align*}
    \frac{\delta S}{\delta \mc{A}^{K L} _{c}} & = \frac{T}{2} \left( e e^{a} _{i} e^{b i} \frac{\partial}{\partial \mc{A}^{K L} _{c}} \left( \mc{D}_{a} X^{I} \mc{D}_{b} X^{J} \right) \eta _{I J} \right) = \\
                                              & = T \left( e e^{a} _{i} e^{b i} \delta ^{c} _{a} \delta ^{I} _{[K} \eta _{L] I'} X^{I'} \mc{D}_{b} X^{J} \eta _{I J} \right) =                               \\
                                              & = T e e^{c} _{i} e^{b i} \mc{D}_{b} X_{[K} X_{L]} \overset{!}{=} 0
\end{align*}
\begin{equation*}
    \mc{T}^{I J} _{a} = \mc{D}_{a} X^{[I} X^{J]} = 0
\end{equation*}

\subsubsection{W.r.t $X$}

\begin{align*}
    \frac{\delta S}{\delta X^{K}} & = - \frac{T}{2} \left( \partial _{c} \left( e e^{a} _{i} e^{b i} \frac{\partial}{\partial (\partial _{c} X^{K})} \left( \mc{D}_{a} X^{I} \mc{D}_{b} X^{J} \right) \eta _{I J} \right) - e e^{a} _{i} e^{b i} \frac{\partial}{\partial X^{K}} \left( \mc{D}_{a} X^{I} \mc{D}_{b} X^{J} \right) \eta _{I J} \right) = \\
                                  & = - T \left( \partial _{c} \left( e e^{a} _{i} e^{b i} \delta ^{c} _{a} \delta ^{I} _{K} \mc{D}_{b} X^{J} \eta _{I J} \right) - e e^{a} _{i} e^{b i} k \mc{A}^{I} _{a I'} \delta ^{I'} _{K} \mc{D}_{b} X^{J} \eta _{I J} \right)=                                                                                   \\
                                  & = - T \left( \partial _{a} \left( e e^{a} _{i} e^{b i} \mc{D}_{b} X^{J} \eta _{K J} \right) - k \mc{A}^{I} _{a K} \left( e e^{a} _{i} e^{b i} \mc{D}_{b} X^{J} \eta _{I J} \right) \right) =                                                                                                                        \\
                                  & = - T \mc{D}_{a} \left( e e^{a} _{i} e^{b i} \mc{D}_{b} X_{K} \right) \overset{!}{=} 0
\end{align*}
\begin{equation*}
    \mc{D}_{a} \left( e e^{a} _{i} e^{b i} \mc{D}_{b} X^{I} \right) = 0
\end{equation*}
In conformal gauge $e e^{a} _{i} e^{b i} = \eta ^{a b}$, thus
\begin{equation*}
    \eta ^{a b} \mc{D}_{a} \mc{D}_{b} X^{I} = 0 .
\end{equation*}
More explicitly,
\begin{equation*}
    \eta ^{a b} \mc{D}_{a} \left( \partial _{b} X^{I} + k \mc{A}^{I} _{b J} X^{J} \right) = 0
\end{equation*}
\begin{equation*}
    \eta ^{a b} \left( \left( \partial _{a} \partial _{b} X^{I} + k \mc{A}^{I} _{a J} \partial _{b} X^{J} \right) + \left( k \partial _{a} \left( \mc{A}^{I} _{b J} X^{J} \right) + k^2 \mc{A}^{I} _{a I'} \mc{A}^{I'} _{b J} X^{J} \right) \right) = 0
\end{equation*}
\begin{equation*}
    \eta ^{a b} \left( \partial _{a} \partial _{b} X^{I} + 2 k \mc{A}^{I} _{a J} \partial _{b} X^{J} + k \partial _{a} \mc{A}^{I} _{b J} X^{J} + k^2 \mc{A}^{I} _{a I'} \mc{A}^{I'} _{b J} X^{J} \right) = 0
\end{equation*}
\begin{equation*}
    \eta ^{a b} \left( \delta ^{I} _{J} \partial _{a} \partial _{b} + 2 k \mc{A}^{I} _{a J} \partial _{b} \right) X^{J} = - \left( k \eta ^{a b} \partial _{a} \mc{A}^{I} _{b J} X^{J} + k^2 \eta ^{a b} \mc{A}^{I} _{a I'} \mc{A}^{I'} _{b J} X^{J} \right) ,
\end{equation*}
which has implicit solution given by
\begin{equation*}
    X^{K} (x) = - \int \diff ^2 x' G^{K} _{\ I} (x , x') \left( k \eta ^{a' b'} \partial _{a'} \mc{A}^{I} _{b' J} (x') X^{J} (x') + k^2 \eta ^{a' b'} \mc{A}^{I} _{a' I'} (x') \mc{A}^{I'} _{b' J} (x') X^{J} (x') \right) ,
\end{equation*}
where the Green's function matrix $G^{K} _{\ J} (x , x')$ satisfies
\begin{equation*}
    \eta ^{a b} \left( \delta ^{I} _{K} \partial _{a} \partial _{b} + 2 k \mc{A}^{I} _{a K} (x) \partial _{b} \right) G^{K} _{\ J} (x , x') = \delta ^{I} _{J} \delta (x , x') .
\end{equation*}

Boundary conditions come from
\begin{equation*}
    \delta S = \int \diff ^2 x \left( \frac{\partial \Lagr}{\partial X^{K}} \delta X^{K} + \frac{\partial \Lagr}{\partial (\partial _{c}X^{K})} \delta (\partial _{c} X^{K}) + ... \right) =
\end{equation*}
\begin{equation*}
    = \int \diff ^2 x \left( (\mathrm{EoM}) \delta X^{K} + \partial _{c} \left( \frac{\partial \Lagr}{\partial (\partial _{c} X^{K})} \delta X^{K} \right) + ... \right) =
\end{equation*}
\begin{equation*}
    = \int \diff ^2 x (\mathrm{EoM}) \delta X^{K} + \int \diff \tau \left( \frac{\partial \Lagr}{\partial (\partial _{\sigma} X^{K})} \delta X^{K} \right) \bigg| _{0} ^{\sigma _{1}} + ... = 0 ,
\end{equation*}
thus
\begin{equation*}
    \left( \frac{\partial \Lagr}{\partial (\partial _{\sigma} X^{K})} \delta X^{K} \right) \bigg| _{0} ^{\sigma _{1}} = 0
\end{equation*}
\begin{equation*}
    \left( \left( - T e e^{\sigma} _{i} e^{b i} \mc{D}_{b} X_{K} \right) \delta X^{K} \right) \big| _{0} ^{\sigma _{1}} = 0
\end{equation*}
\begin{equation*}
    \left( (\mc{D}_{\sigma} X_{K}) \delta X^{K} \right) \big| _{0} ^{\sigma _{1}} = 0 ,
\end{equation*}
so either $\delta X^{K} (\tau , \sigma _{*}) = 0 , \ \sigma _{*} = 0 , \sigma _{1}$ (Dirichlet B.C) or $\mc{D}_{\sigma} X^{K} (\tau , \sigma _{*}) = 0 , \ \sigma _{*} = 0 , \sigma _{1}$ (free end-point B.C). For closed strings, $X^{K} (\tau , \sigma) = X^{K} (\tau , \sigma + 2 \pi)$.

When connection is trivial EoM reduces to regular wave eqn, which have the regular string solution
\begin{equation*}
    X^{I} _{0} (\tau , \sigma) = X^{I} _{0 L} (\sigma ^{+}) + X^{I} _{0 R} (\sigma ^{-}) ,
\end{equation*}
with
\begin{equation*}
    X^{I} _{0 L} (\sigma ^{+}) = \frac{1}{2} x^{I} + \frac{1}{2} \alpha ' p^{I} \sigma ^{+} + i \sqrt{\frac{\alpha '}{2}} \sum _{n \neq 0} \frac{1}{n} \widetilde{\alpha}^{I} _{n} \e ^{- i n \sigma ^{+}} , \ \sigma ^{+} = \tau + \sigma
\end{equation*}
\begin{equation*}
    X^{I} _{0 R} (\sigma ^{-}) = \frac{1}{2} x^{I} + \frac{1}{2} \alpha ' p^{I} \sigma ^{-} + i \sqrt{\frac{\alpha '}{2}} \sum _{n \neq 0} \frac{1}{n} \alpha ^{I} _{n} \e ^{- i n \sigma ^{-}} , \ \sigma ^{-} = \tau - \sigma .
\end{equation*}
This, however, does not amount to any motion in actual space-time, since with trivial connection the vector does not change after parallel transport. For a more interesting case, let's consider $A_{\tau} ^{\alpha \beta} = A_{\sigma} ^{\alpha \beta} = a \varepsilon ^{\alpha \beta}$, which amounts to flat space-time. The EoM turn into
\begin{equation*}
    \left( \partial _{\sigma} ^2 - \partial _{\tau} ^2 \right) X^{I} + 2 k a (\varepsilon ^{\alpha \beta} T_{\alpha \beta})^{I} _{\ J} \left( \partial _{\sigma} - \partial _{\tau} \right) X^{J} = 0
\end{equation*}
\begin{equation*}
    \left( \partial _{\sigma} ^2 - \partial _{\tau} ^2 \right) \begin{bmatrix} X^{0} \\ X^{1} \\ X^{2} \\ \vdots \\ X^{d} \end{bmatrix} + 4 k a \left( \partial _{\sigma} - \partial _{\tau} \right) \begin{bmatrix} 0 & 1 & 1 & ... & 1 \\ 1 & 0 & 1 & ... & 1 \\ 1 & - 1 & 0 & ... & 1 \\ \vdots & \vdots & \vdots & \ddots & \vdots \\ 1 & -1 & -1 & ... & 0 \end{bmatrix} \begin{bmatrix} X^{0} \\ X^{1} \\ X^{2} \\ \vdots \\ X^{d} \end{bmatrix} = 0
\end{equation*}
\begin{equation*}
    \begin{bmatrix} \left( \partial _{\sigma} ^2 - \partial _{\tau} ^2 \right) X^{0} + 4 k a \left( \partial _{\sigma} - \partial _{\tau} \right) \sum _{i > 0} X^{i} \\ \left( \partial _{\sigma} ^2 - \partial _{\tau} ^2 \right) X^{1} + 4 k a \left( \partial _{\sigma} - \partial _{\tau} \right) \sum _{i \neq 1} X^{i} \\ \left( \partial _{\sigma} ^2 - \partial _{\tau} ^2 \right) X^{2} + 4 k a \left( \partial _{\sigma} - \partial _{\tau} \right) \left( X^{0} - X^{1} + \sum _{i > 2} X^{i} \right) \\ \vdots \\ \left( \partial _{\sigma} ^2 - \partial _{\tau} ^2 \right) X^{d} + 4 k a \left( \partial _{\sigma} - \partial _{\tau} \right) \left( X^{0} - \sum _{i = 1} ^{d - 1} X^{i} \right) \end{bmatrix} = 0
\end{equation*}
\begin{equation*}
    \left( \partial _{\sigma} ^2 - \partial _{\tau} ^2 \right) X^{I} + 4 k a \left( \partial _{\sigma} - \partial _{\tau} \right) \left( (\mathrm{sgn}(I))^2 X^{0} + \sum _{i > 0} \mathrm{sgn}(i - I) X^{i} \right) = 0
\end{equation*}
\begin{equation*}
    \left( \partial _{\sigma} - \partial _{\tau} \right) \left( \left( \partial _{\sigma} + \partial _{\tau} \right) X^{I} + 4 k a \left( (\mathrm{sgn}(I))^2 X^{0} + \sum _{i > 0} \mathrm{sgn}(i - I) X^{i} \right) \right) = 0
\end{equation*}
\begin{equation*}
    \left( \partial _{\sigma} + \partial _{\tau} \right) X^{I} + 4 k a \left( (\mathrm{sgn}(I))^2 X^{0} + \sum _{i > 0} \mathrm{sgn}(i - I) X^{i} \right) = C^{I} _{1} . (\tau + \sigma) , \ C^{I} \in \mathbb{C}
\end{equation*}
\begin{equation*}
    \left( \partial _{\sigma} + \partial _{\tau} \right) X^{I} = C^{I} _{1} . (\tau + \sigma) - 4 k a \left( (\mathrm{sgn}(I))^2 X^{0} + \sum _{i > 0} \mathrm{sgn}(i - I) X^{i} \right)
\end{equation*}
\begin{equation*}
    X^{I} = \frac{1}{2} C^{I} _{1} . \left( \tau ^2 + \sigma ^2 \right) + C^{I} _{2} . \left( \tau - \sigma \right) - 4 k a \int \diff ^2 x' \left( (\mathrm{sgn}(I))^2 X^{0} + \sum _{i > 0} \mathrm{sgn}(i - I) X^{i} \right) .
\end{equation*}
If $D = 2$, we get a symmetric system of eqns
\begin{align*}
    X^{0} & = \frac{1}{2} C^{0} _{1} . \left( \tau ^2 + \sigma ^2 \right) + C^{0} _{2} . \left( \tau - \sigma \right) - 4 k a \int \diff ^2 x' X^{1} \\
    X^{1} & = \frac{1}{2} C^{1} _{1} . \left( \tau ^2 + \sigma ^2 \right) + C^{1} _{2} . \left( \tau - \sigma \right) - 4 k a \int \diff ^2 x' X^{0}
\end{align*}
\begin{equation*}
    X^{0} & = \frac{1}{2} C^{0} _{1} . \left( \tau ^2 + \sigma ^2 \right) + C^{0} _{2} . \left( \tau - \sigma \right) - 4 k a \int \diff ^2 x' \left( \frac{1}{2} C^{1} _{1} . \left( \tau ^{\prime 2} + \sigma ^{\prime 2} \right) + C^{1} _{2} . \left( \tau ' - \sigma ' \right) - 4 k a \int \diff ^2 x'' X^{0} \right)
\end{equation*}
\begin{equation*}
    X^{0} - 16 k^2 a^2 \int \diff ^2 x' \int \diff ^2 x'' X^{0} = \frac{1}{2} C^{0} _{1} . \left( \tau ^2 + \sigma ^2 \right) + C^{0} _{2} . \left( \tau - \sigma \right) - \frac{1}{2} C^{1} _{1} \left( \frac{\tau ^3}{3} \sigma + \tau \frac{\sigma ^3}{3} \right) - C^{1} _{2} \left( \frac{\tau ^2}{2} \sigma - \tau \frac{\sigma ^2}{2} \right) .
\end{equation*}
This integral equation is tricky to deal with, but we can turn it into a differential equation by applying $\partial _{\sigma} \partial _{\tau}$ twice:
\begin{equation*}
    \left( \partial _{\sigma} \partial _{\tau} \right) ^2 X^{0} - 16 k^2 a^2 X^{0} = 0 .
\end{equation*}
This has general solution given by sum of plane waves and exponentials
\begin{equation*}
    X^{0} = \sum _{n \neq 0} \frac{1}{n^4} \left( a^{0} _{n} \e ^{i 2 \sqrt{k a} n (\tau + \sigma)} + b^{0} _{n} \e ^{i 2 \sqrt{k a} n (\tau - \sigma)} + c^{0} _{n} \e ^{2 \sqrt{k a} n (\tau + \sigma)} + d^{0} _{n} \e ^{2 \sqrt{k a} n (\tau - \sigma)} \right) .
\end{equation*}

\end{document}