\documentclass{article}
\usepackage[utf8]{inputenc}
\usepackage{indentfirst}
\usepackage{graphicx}
\usepackage{amssymb}
\usepackage{amsmath,amsthm,amsfonts}
\usepackage{esint}
\usepackage{upgreek}
\usepackage{tikz}
\usepackage{mathtools}
\usepackage{mathrsfs}
\usepackage{slashed}
\usepackage{hyperref}
\usepackage{bbold}
\usepackage[margin=2cm]{geometry}
\usepackage{titlesec}

\newcommand{\R}{\mathbb{R}}
\newcommand{\diff}{\mathrm{d}}
\newcommand{\inp}{\mathrm{i}}
\newcommand{\e}{\mathrm{e}}
\newcommand{\Lagr}{\mathscr{L}}
\newcommand{\PS}{\mathcal{P}}



\newcommand{\bite}{\begin{itemize}}
	\newcommand{\eat}{\end{itemize}}
\newcommand{\beq}{\begin{equation}}
	\newcommand{\eeq}{\end{equation}}
\newcommand{\rarrow}{\rightarrow}
\newcommand{\beqa}{\begin{align}}
	\newcommand{\eeqa}{\end{align}}
\newcommand{\barr}{\begin{array}}
	\newcommand{\earr}{\end{array}}
\newcommand{\del}{\partial}
\newcommand{\de}{\mathrm{d}}
%\newcommand{\mu\nu}{{\mu\nu}}
\renewcommand{\th}{\mathrm{th}}
\newcommand{\com}[1]{\begin{itemize}\color{RED}{{#1}}\end{itemize}}
\newcommand{\C}{\mathbb{C}}
\newcommand{\R}{\mathbb{R}}
\newcommand{\btw}[1]{\color{PURPLE}{{#1}}\color{BLACK}}
\newcommand{\cut}[1]{\color{RED}{{#1}}\color{BLACK}}

%text
\newcommand{\ie}{\textit{i.e.}~}
\newcommand{\eg}{\textit{e.g.}~}
\newcommand{\wrt}{\textit{w.r.t.}~}
\newcommand{\etc}{\textit{etc.}~}


\newcommand{\M}{\mathcal{M}}
\newcommand{\N}{\mathcal{N}}
% \newcommand{\H}{\mathcal{H}}

\newcommand{\bz}{\mathbf{z}}

\newcommand{\mb}[1]{\mathbf{#1}}
\newcommand{\mc}[1]{\mathcal{#1}}
\newcommand{\mbb}[1]{\mathbb{#1}}
\newcommand{\mf}[1]{\mathfrak{#1}}

\newcommand{\unit}[1]{\mathbf{\hat{#1}}}

%% Package bbold for bold identity symbol
\usepackage{bbold}

\newcommand{\id}{\mathbb{1}}

\newcommand{\utilde}[1]{\underaccent{\tilde}{#1}}

\newcommand{\vect}[1]{\boldsymbol{#1}}
\newcommand{\bvec}[1]{\boldsymbol{\vec #1}}
\newcommand{\expect}[1]{\langle #1\rangle}
\newcommand{\innerp}[2]{\langle #1 \vert #2 \rangle}
\newcommand{\expectop}[3]{\langle #1 \vert #2 \vert #3 \rangle}
\newcommand{\bra}[1]{\langle #1 \vert}
\newcommand{\ket}[1]{\vert #1 \rangle}
\newcommand{\supersc}[1]{$^{\textrm{#1}}$}
\newcommand{\subsc}[1]{$_{\textrm{#1}}$}
\newcommand{\sltwoc}{\mathfrak{sl}(2,\mathbb{C})}

\newcommand{\norm}[1]{\lVert #1 \rVert}

\newcommand{\rket}[1]{\vert #1 ]}
\newcommand{\rbra}[1]{[ #1 \vert}

\newcommand{\rinnerp}[2]{[ #1 \vert #2 ]}

\newcommand{\bket}[1]{\vert #1 )}
\newcommand{\bbra}[1]{( #1 \vert}

\newcommand{\binnerp}[2]{( #1 \vert #2 )}

\newcommand{\innerpA}[2]{\langle #1 \vert #2 ]}
\newcommand{\innerpB}[2]{[ #1 \vert #2 \rangle}

\newcommand{\onehalf}{\frac{1}{2}}

\newcommand{\Tr}{\mathrm{Tr}}

\newcommand{\Cyl}{\mathrm{Cyl}}

\newtheorem{theorem}{Theorem}[paragraph]
\newtheorem{proposition}{Proposition}[paragraph]
\newtheorem{corollary}{Corollary}[theorem]

\setcounter{secnumdepth}{4}

\titleformat{\paragraph}
{\normalfont\normalsize\bfseries}{\theparagraph}{1em}{}
\titlespacing*{\paragraph}
{0pt}{3.25ex plus 1ex minus .2ex}{1.5ex plus .2ex}

\title{Sketchboard}

\begin{document}

\maketitle

\newpage

\section{Basic concepts}

\subsection{With extra capital internal index on embbeding fields}

Promote embedding fields $X^{\mu}$ to have an internal group index with $D$ values
\begin{equation*}
    X^{\mu} \rightarrow X^{\mu I} \ , \ \ I = 0 , ... , d
\end{equation*}
\begin{equation*}
    g_{a b} = 2 f \partial _{a} X^{\mu} \partial _{b} X^{\nu} G_{\mu \nu} \rightarrow g_{a b} ^{I J} = 2 f D_{a} X^{\mu I} D_{b} X^{\nu J} G_{\mu \nu}
\end{equation*}
\begin{equation*}
    D_{a} X^{\mu I} = \partial _{a} X^{\mu I} + \omega ^{I} _{a J} X^{\mu J}
\end{equation*}
\begin{equation*}
    e^{i} _{a} e^{j} _{b} \eta _{i j} = g_{a b} \rightarrow e^{i} _{a} e^{j} _{b} \eta _{i j} = \Tr (g_{a b} ^{I J} T_{I} T_{J}) = g_{a b} ^{I J} \eta _{I J} \ , \ \ i , j = 0 , 1
\end{equation*}
\begin{equation*}
    g = \det (g_{a b}) \rightarrow g = \det (\Tr (g_{a b} ^{I J} T_{I} T_{J})) = \det (g_{a b} ^{I J} \eta _{I J}) = \det (e^{i} _{a} e^{j} _{b} \eta _{i j}) = - \det (e)^{2} \implies \sqrt{- g} = \det (e)
\end{equation*}

\subsection{With extra small internal index on embedding fields}

Promote embbeding fields to have a internal group index with 2 values
\begin{equation*}
    X^{\mu} \rightarrow X^{\mu i} \ , \ \ i = 0 , 1
\end{equation*}
\begin{equation*}
    g_{a b} = 2 f \partial _{a} X^{\mu} \partial _{b} X^{\nu} G_{\mu \nu} \rightarrow g_{a b} ^{i j} = 2 f D_{a} X^{\mu i} D_{b} X^{\nu j} G_{\mu \nu}
\end{equation*}
\begin{equation*}
    D_{a} X^{\mu i} = \partial _{a} X^{\mu i} + \omega ^{i} _{a j} X^{\mu j}
\end{equation*}
\begin{equation*}
    e^{i} _{a} e^{j} _{b} \eta _{i j} = g_{a b} \rightarrow e^{i} _{a} e^{j} _{b} \eta _{i j} = \Tr (g_{a b} ^{i j} T_{i} T_{j}) = g_{a b} ^{i j} \eta _{i j} \implies e^{i} _{a} e^{j} _{b} = g_{a b} ^{i j}
\end{equation*}
\begin{equation*}
    g = \det (g_{a b}) \rightarrow g = \det (\Tr (g_{a b} ^{i j} T_{i} T_{j})) = \det (g_{a b} ^{i j} \eta _{i j}) = \det (e^{i} _{a} e^{j} _{b} \eta _{i j}) = - \det (e)^{2} \implies \sqrt{- g} = \det (e)
\end{equation*}

\subsection{Without extra index on embbeding fields}

Change from WS metric to WS zweibein and connection (which vanishes since in 2d metric is conformally flat)
\begin{equation*}
    e_{a} ^{i} e_{b} ^{j} \eta _{i j} = g_{a b} = 2 f \partial _{a} X^{\mu} \partial _{b} X^{\nu} G_{\mu \nu}
\end{equation*}
\begin{equation*}
    g = \det (g_{a b}) = \det (e_{a} ^{i} e_{b} ^{j} \eta _{i j}) = - \det (e)^{2} \implies \sqrt{- g} = \det (e)
\end{equation*}

\section{Building an Action}

Start with Polyakov action in curved space-time
\begin{equation*}
    S_{P} = - \frac{T_{0}}{2} \int \diff \tau \wedge \diff \sigma \sqrt{- g} g^{a b} \partial _{a} X^{\mu} \partial _{b} X^{\nu} G_{\mu \nu} \ ,
\end{equation*}

\subsection{With capital internal index}

... and promote partial derivative $\partial _{a}$ to covariant derivative $D_{a}$, giving us our first attempt at modified Polyakov action
\begin{equation*}
    S_{M P 1} = - \frac{T_{0}}{2} \int \diff \tau \wedge \diff \sigma \det (e) \eta ^{i j} e^{a} _{i} e^{b} _{j} D_{a} X^{\mu I} D_{b} X^{\nu J} E_{\mu} ^{K} E_{\nu K} \eta _{I J}
\end{equation*}
\begin{equation*}
    \updownarrow
\end{equation*}
\begin{equation*}
    \Lagr _{M P 1} = - \frac{T_{0}}{2} \det (e) \eta ^{i j} e^{a} _{i} e^{b} _{j} D_{a} X^{\mu I} D_{b} X^{\nu J} E_{\mu} ^{K} E_{\nu K} \eta _{I J}
\end{equation*}

\subsection{With small internal index}

... and promote derivatives to covariant $D_{a}$, giving us another modified Polyakov action
\begin{equation*}
    S_{M P 2} = - \frac{T_{0}}{2} \int \diff \tau \wedge \diff \sigma \det (e) e^{a} _{i} e^{b} _{j} D_{a} X^{\mu i} D_{b} X^{\mu j} E_{\mu} ^{I} (X) E_{\nu I} (X)
\end{equation*}
\begin{equation*}
    \updownarrow
\end{equation*}
\begin{equation*}
    \Lagr _{M P 2} = - \frac{T_{0}}{2} \det (e) e^{a} _{i} e^{b} _{j} D_{a} X^{\mu i} D_{b} X^{\mu j} E_{\mu} ^{I} (X) E_{\nu I} (X)
\end{equation*}

\subsection{Without extra index}

... and swap to new set of variables giving us the dyad-Polyakov action
\begin{equation*}
    S_{D P} = - \frac{T_{0}}{2} \int \diff \tau \wedge \diff \sigma \det (e) e^{a} _{i} e^{b i} \partial _{a} X^{\mu} \partial _{b} X^{\nu} E_{\mu} ^{I} (X) E_{\nu I} (X) \ .
\end{equation*}
\begin{equation*}
    \updownarrow
\end{equation*}
\begin{equation*}
    \Lagr _{D P} = - \frac{T_{0}}{2} \det (e) e^{a} _{i} e^{b i} \partial _{a} X^{\mu} \partial _{b} X^{\nu} E_{\mu} ^{I} (X) E_{\nu I} (X) \ .
\end{equation*}

\section{EoMs}

\subsection{w.r.t $e$}

Start by writing $\det (e) = \frac{1}{2} \varepsilon ^{c d} \varepsilon _{m n} e_{c} ^{m} e_{d} ^{n}$
\begin{align*}
    \frac{\delta S_{D P}}{\delta e^{e} _{l}} & = \frac{D \Lagr _{D P}}{D e^{e} _{l}} = \left( \frac{\partial}{\partial e^{e} _{l}} - \partial _{f} \frac{\partial}{\partial (\partial _{f} e^{e} _{l})} \right) \left( - \frac{T_{0}}{2} \frac{1}{2} \varepsilon ^{c d} \varepsilon _{m n} e^{m} _{c} e^{n} _{d} \eta ^{i j} e^{a} _{i} e^{b} _{j} \partial _{a} X^{\mu} \partial _{b} X^{\nu} E^{I} _{\mu} E^{J} _{\nu} \eta _{I J} \right) = \\
                                       & = - \frac{T_{0}}{4}
\end{align*}

\end{document}