\documentclass{article}
\usepackage[utf8]{inputenc}
\usepackage{indentfirst}
\usepackage{graphicx}
\usepackage{amssymb}
\usepackage{amsmath,amsthm,amsfonts}
\usepackage{esint}
\usepackage{upgreek}
\usepackage{tikz}
\usepackage{mathtools}
\usepackage{mathrsfs}
\usepackage{slashed}
\usepackage{hyperref}
\usepackage{bbold}
\usepackage[margin=2cm]{geometry}
\usepackage{titlesec}

\newcommand{\diff}{\mathrm{d}}
\newcommand{\inp}{\mathrm{i}}
\newcommand{\e}{\mathrm{e}}
\newcommand{\Lagr}{\mathscr{L}}
\newcommand{\PS}{\mathcal{P}}



\newcommand{\bite}{\begin{itemize}}
	\newcommand{\eat}{\end{itemize}}
\newcommand{\beq}{\begin{equation}}
	\newcommand{\eeq}{\end{equation}}
\newcommand{\rarrow}{\rightarrow}
\newcommand{\beqa}{\begin{align}}
	\newcommand{\eeqa}{\end{align}}
\newcommand{\barr}{\begin{array}}
	\newcommand{\earr}{\end{array}}
\newcommand{\del}{\partial}
\newcommand{\de}{\mathrm{d}}
%\newcommand{\mu\nu}{{\mu\nu}}
\renewcommand{\th}{\mathrm{th}}
\newcommand{\com}[1]{\begin{itemize}\color{RED}{{#1}}\end{itemize}}
\newcommand{\C}{\mathbb{C}}
\newcommand{\R}{\mathbb{R}}
\newcommand{\btw}[1]{\color{PURPLE}{{#1}}\color{BLACK}}
\newcommand{\cut}[1]{\color{RED}{{#1}}\color{BLACK}}

%text
\newcommand{\ie}{\textit{i.e.}~}
\newcommand{\eg}{\textit{e.g.}~}
\newcommand{\wrt}{\textit{w.r.t.}~}
\newcommand{\etc}{\textit{etc.}~}


\newcommand{\M}{\mathcal{M}}
\newcommand{\N}{\mathcal{N}}
% \newcommand{\H}{\mathcal{H}}

\newcommand{\bz}{\mathbf{z}}

\newcommand{\mb}[1]{\mathbf{#1}}
\newcommand{\mc}[1]{\mathcal{#1}}
\newcommand{\mbb}[1]{\mathbb{#1}}
\newcommand{\mf}[1]{\mathfrak{#1}}

\newcommand{\unit}[1]{\mathbf{\hat{#1}}}

%% Package bbold for bold identity symbol
\usepackage{bbold}

\newcommand{\id}{\mathbb{1}}

\newcommand{\utilde}[1]{\underaccent{\tilde}{#1}}

\newcommand{\vect}[1]{\boldsymbol{#1}}
\newcommand{\bvec}[1]{\boldsymbol{\vec #1}}
\newcommand{\expect}[1]{\langle #1\rangle}
\newcommand{\innerp}[2]{\langle #1 \vert #2 \rangle}
\newcommand{\expectop}[3]{\langle #1 \vert #2 \vert #3 \rangle}
\newcommand{\bra}[1]{\langle #1 \vert}
\newcommand{\ket}[1]{\vert #1 \rangle}
\newcommand{\supersc}[1]{$^{\textrm{#1}}$}
\newcommand{\subsc}[1]{$_{\textrm{#1}}$}
\newcommand{\sltwoc}{\mathfrak{sl}(2,\mathbb{C})}

\newcommand{\norm}[1]{\lVert #1 \rVert}

\newcommand{\rket}[1]{\vert #1 ]}
\newcommand{\rbra}[1]{[ #1 \vert}

\newcommand{\rinnerp}[2]{[ #1 \vert #2 ]}

\newcommand{\bket}[1]{\vert #1 )}
\newcommand{\bbra}[1]{( #1 \vert}

\newcommand{\binnerp}[2]{( #1 \vert #2 )}

\newcommand{\innerpA}[2]{\langle #1 \vert #2 ]}
\newcommand{\innerpB}[2]{[ #1 \vert #2 \rangle}

\newcommand{\onehalf}{\frac{1}{2}}

\newcommand{\Tr}{\mathrm{Tr}}

\newcommand{\Cyl}{\mathrm{Cyl}}

\newtheorem{theorem}{Theorem}[paragraph]
\newtheorem{proposition}{Proposition}[paragraph]
\newtheorem{corollary}{Corollary}[theorem]

\setcounter{secnumdepth}{4}

\titleformat{\paragraph}
{\normalfont\normalsize\bfseries}{\theparagraph}{1em}{}
\titlespacing*{\paragraph}
{0pt}{3.25ex plus 1ex minus .2ex}{1.5ex plus .2ex}

\title{Sketchboard}

\begin{document}

\maketitle

\newpage

\section{Basic concepts}

\subsection{With extra capital internal index on embbeding fields}

Promote embedding fields $X^{\mu}$ to have an internal group index with $D$ values
\begin{equation*}
    X^{\mu} \rightarrow X^{\mu I} \ , \ \ I = 0 , ... , d
\end{equation*}
\begin{equation*}
    g_{a b} = 2 f \partial _{a} X^{\mu} \partial _{b} X^{\nu} G_{\mu \nu} \rightarrow g_{a b} ^{I J} \sim f D_{a} X^{\mu I} D_{b} X^{\nu J} G_{\mu \nu}
\end{equation*}
\begin{equation*}
    D_{a} X^{\mu I} = \partial _{a} X^{\mu I} + \omega ^{I} _{a J} X^{\mu J}
\end{equation*}
\begin{equation*}
    e^{i} _{a} e^{j} _{b} \eta _{i j} = g_{a b} \rightarrow e^{i} _{a} e^{j} _{b} \eta _{i j} = \Tr (g_{a b} ^{I J} T_{I} T_{J}) = g_{a b} ^{I J} \eta _{I J} \ , \ \ i , j = 0 , 1
\end{equation*}
\begin{equation*}
    g = \det (g_{a b}) \rightarrow g = \det (\Tr (g_{a b} ^{I J} T_{I} T_{J})) = \det (g_{a b} ^{I J} \eta _{I J}) = \det (e^{i} _{a} e^{j} _{b} \eta _{i j}) = - \det (e)^{2} \implies \sqrt{- g} = \det (e)
\end{equation*}

\subsection{With 1 extra small internal index on embedding fields}

Promote embbeding fields to have a internal group index with 2 values
\begin{equation*}
    X^{\mu} \rightarrow X^{\mu i} \ , \ \ i = 0 , 1
\end{equation*}
\begin{equation*}
    g_{a b} = 2 f \partial _{a} X^{\mu} \partial _{b} X^{\nu} G_{\mu \nu} \rightarrow g_{a b} ^{i j} \sim f D_{a} X^{\mu i} D_{b} X^{\nu j} G_{\mu \nu}
\end{equation*}
\begin{equation*}
    D_{a} X^{\mu i} = \partial _{a} X^{\mu i} + \omega ^{i} _{a j} X^{\mu j}
\end{equation*}
\begin{equation*}
    e^{i} _{a} e^{j} _{b} \eta _{i j} = g_{a b} \rightarrow e^{i} _{a} e^{j} _{b} \eta _{i j} = \Tr (g_{a b} ^{i j} T_{i} T_{j}) = g_{a b} ^{i j} \eta _{i j} \implies e^{i} _{a} e^{j} _{b} = g_{a b} ^{i j}
\end{equation*}
\begin{equation*}
    g = \det (g_{a b}) \rightarrow g = \det (\Tr (g_{a b} ^{i j} T_{i} T_{j})) = \det (g_{a b} ^{i j} \eta _{i j}) = \det (e^{i} _{a} e^{j} _{b} \eta _{i j}) = - \det (e)^{2} \implies \sqrt{- g} = \det (e)
\end{equation*}

\subsection{With 2 extra small indices}

Promote embbeding fields $X^{\mu}$ to have two internal indices with 2 values
\begin{equation*}
    X^{\mu} \rightarrow X^{\mu i j} \ , \ \ i , j = 0 , 1
\end{equation*}

\subsection{Without extra index on embbeding fields}

Change from WS metric to WS zweibein and connection (which vanishes since in 2d metric is conformally flat)
\begin{equation*}
    e_{a} ^{i} e_{b} ^{j} \eta _{i j} = g_{a b} = 2 f \partial _{a} X^{\mu} \partial _{b} X^{\nu} G_{\mu \nu}
\end{equation*}
\begin{equation*}
    g = \det (g_{a b}) = \det (e_{a} ^{i} e_{b} ^{j} \eta _{i j}) = - \det (e)^{2} \implies \sqrt{- g} = \det (e)
\end{equation*}

\section{Building an Action}

Start with Polyakov action in curved space-time
\begin{equation*}
    S_{P} = - \frac{T_{0}}{2} \int \diff \tau \wedge \diff \sigma \sqrt{- g} g^{a b} \partial _{a} X^{\mu} \partial _{b} X^{\nu} G_{\mu \nu} \ ,
\end{equation*}

\subsection{With capital internal index}

... and promote partial derivative $\partial _{a}$ to covariant derivative $D_{a}$, giving us our first attempt at modified Polyakov action
\begin{equation*}
    S_{M P 1} = - \frac{T_{0}}{2} \int \diff \tau \wedge \diff \sigma \det (e) \eta ^{i j} e^{a} _{i} e^{b} _{j} D_{a} X^{\mu I} D_{b} X^{\nu J} E_{\mu} ^{K} E_{\nu K} \eta _{I J}
\end{equation*}
\begin{equation*}
    \updownarrow
\end{equation*}
\begin{equation*}
    \Lagr _{M P 1} = - \frac{T_{0}}{2} \det (e) \eta ^{i j} e^{a} _{i} e^{b} _{j} D_{a} X^{\mu I} D_{b} X^{\nu J} E_{\mu} ^{K} E_{\nu K} \eta _{I J}
\end{equation*}

\subsection{With small internal index}

\subsubsection{Without extra field, 1 internal index}

... and promote derivatives to covariant $D_{a}$, giving us another modified Polyakov action
\begin{equation*}
    S_{M P 2} = - \frac{T_{0}}{2} \int \diff \tau \wedge \diff \sigma \det (e) e^{a} _{i} e^{b} _{j} D_{a} X^{\mu i} D_{b} X^{\mu j} E_{\mu} ^{I} (X) E_{\nu I} (X)
\end{equation*}
\begin{equation*}
    \updownarrow
\end{equation*}
\begin{equation*}
    \Lagr _{M P 2} = - \frac{T_{0}}{2} \det (e) e^{a} _{i} e^{b} _{j} D_{a} X^{\mu i} D_{b} X^{\mu j} E_{\mu} ^{I} (X) E_{\nu I} (X)
\end{equation*}

\subsubsection{With extra field}

..., promote partial derivative to covariant and add extra internal WS field $v^{i}$ leading us to
\begin{equation*}
    S_{M P 3} = - \frac{T}{2} \int \diff \tau \wedge \diff \sigma \det (e) \eta ^{i j} e^{a} _{i} e^{b} _{j} D_{a} X^{\mu} _{k} v^{k} D_{b} X^{\nu} _{l} v^{l} E_{\mu} ^{I} E_{\nu} ^{J} \eta _{I J}
\end{equation*}
\begin{equation*}
    \updownarrow
\end{equation*}
\begin{equation*}
    \Lagr _{M P 3} = - \frac{T}{2} \det (e) \eta ^{i j} e^{a} _{i} e^{b} _{j} D_{a} (X^{\mu} _{k} v^{k}) D_{b} (X^{\nu} _{l} v^{l}) E_{\mu} ^{I} E_{\nu} ^{J} \eta _{I J}
\end{equation*}

\subsubsection{Without extra field, 2 internal indices}

... and promote derivatives to covariant $D_{a}$,
\begin{equation*}
    S_{M P 4} = - \frac{T}{2} \int \diff \tau \wedge \diff \sigma e e^{a} _{i} e^{b i} D_{a} X^{\mu j j'} \eta _{j j'} \eta _{k k'} D_{b} X^{\nu k k'} E^{I} _{\mu} E_{\nu I}
\end{equation*}
\begin{equation*}
    \Lagr _{M P 4} = - \frac{T}{2} e e^{a} _{i} e^{b} _{j} D_{a} X^{\mu i k} \eta _{k l} D_{b} X^{\nu l j} E^{I} _{\mu} E_{\nu I}
\end{equation*}

\subsection{Without extra index}

... and swap to new set of variables giving us the dyad-Polyakov action
\begin{equation*}
    S_{D P} = - \frac{T_{0}}{2} \int \diff \tau \wedge \diff \sigma \det (e) e^{a} _{i} e^{b i} \partial _{a} X^{\mu} \partial _{b} X^{\nu} E_{\mu} ^{I} (X) E_{\nu I} (X) \ .
\end{equation*}
\begin{equation*}
    \updownarrow
\end{equation*}
\begin{equation*}
    \Lagr _{D P} = - \frac{T_{0}}{2} \det (e) e^{a} _{i} e^{b i} \partial _{a} X^{\mu} \partial _{b} X^{\nu} E_{\mu} ^{I} (X) E_{\nu I} (X) \ .
\end{equation*}

\subsection{Linear Polyakov action}

..., promote partial derivative to covariant derivative and build a linear action with inclusion of $D$-dimensional gamma matrices and bulk spinors
\begin{equation*}
    S_{L P} = - \frac{T}{2} \int \diff \tau \wedge \diff \sigma \overline{\psi} e e^{a} _{i} D_{a} X^{\mu i} E^{I} _{\mu} \gamma _{I} \psi
\end{equation*}
\begin{equation*}
    \updownarrow
\end{equation*}
\begin{equation*}
    \Lagr _{L P} = - \frac{T}{2} \overline{\psi} e e^{a} _{i} D_{a} X^{\mu i} E^{I} _{\mu} \gamma _{I} \psi
\end{equation*}

\section{EoMs}

Start by writing $\det (e) = \frac{1}{2} \varepsilon ^{c d} \varepsilon _{m n} e_{c} ^{m} e_{d} ^{n}$ and $g_{a b} = g_{a b} ^{i j} \eta _{i j}$

\subsection{With small internal index}

\subsubsection{Without extra field, w.r.t $e$}

\begin{align*}
    \frac{\delta S_{M P 2}}{\delta e^{e} _{l}} & = \frac{D \Lagr _{M P 2}}{D e^{e} _{l}} = \left( \frac{\partial}{\partial e^{e} _{l}} - \partial _{f} \frac{\partial}{\partial (\partial _{f} e^{e} _{l})} \right) \left( - \frac{T}{2} \frac{1}{2} \varepsilon ^{c d} \varepsilon _{m n} e^{m} _{c} e^{n} _{d} e^{a} _{i} e^{b} _{j} D_{a} X^{\mu i} D_{b} X^{\nu j} E_{\mu} ^{I} E_{\nu I} \right) = \\
                                               & = - \frac{T}{4} \varepsilon ^{c d} \varepsilon _{m n} ( \eta ^{m l} g_{c e} e^{n} _{d} e^{a} _{i} e^{b} _{j} + e^{m} _{c} \eta ^{n l} g_{d e} e^{a} _{i} e^{b} _{j} +                                                                                                                                                                                  \\
                                               & + e^{m} _{c} e^{n} _{d} \delta ^{a} _{e} \delta ^{l} _{i} e^{b} _{j} + e^{m} _{c} e^{n} _{d} e^{a} _{i} \delta ^{b} _{e} \delta ^{l} _{j} ) D_{a} X^{\mu i} D_{b} X^{\nu j} E_{\mu} ^{I} E_{\nu I} =                                                                                                                                                   \\
                                               & = - \frac{T}{2} \left( \varepsilon ^{c d} \varepsilon _{m n} \eta ^{m l} g_{c e} e^{n} _{d} e^{a} _{i} e^{b} _{j} + 2 \det (e) e^{a} _{i} \delta ^{b} _{e} \delta ^{l} _{j} \right) D_{a} X^{\mu i} D_{b} X^{\nu j} E_{\mu} ^{I} E_{\nu I} =                                                                                                           \\
                                               & = - \frac{T}{2} (\varepsilon ^{c d} \varepsilon _{m n} \eta ^{m l} e^{k} _{c} e_{e k} e^{n} _{d} e^{a} _{i} e^{b} _{j} + 2 \det (e) e^{a} _{i} \delta ^{b} _{e} \delta ^{l} _{j}) D_{a} X^{\mu i} D_{b} X^{\nu j} E_{\mu} ^{I} E_{\nu I} =                                                                                                             \\
                                               & = - \frac{T}{2} (\varepsilon ^{c d} \varepsilon _{m n} e^{m} _{c} e^{n} _{d} e^{l} _{e} e^{a} _{i} e^{b} _{j} + 2 \det (e) e^{a} _{i} \delta ^{b} _{e} \delta ^{l} _{j}) D_{a} X^{\mu i} D_{b} X^{\nu j} E_{\mu} ^{I} E_{\nu I} =                                                                                                                      \\
                                               & = -T (- \det (e) e^{l} _{e} e^{a} _{i} e^{b} _{j} + \det (e) e^{a} _{i} \delta ^{b} _{e} \delta ^{l} _{j}) D_{a} X^{\mu i} D_{b} X^{\nu j} E_{\mu} ^{I} E_{\nu I} \overset{!}{=} 0
\end{align*}
\begin{equation*}
    \downarrow
\end{equation*}
\begin{equation*}
    T^{l} _{e} := E_{\mu} ^{I} E_{\nu I} (e^{a} _{i} D_{a} X^{\mu i} D_{e} X^{\nu l} - e^{l} _{e} e^{a} _{i} e^{b} _{j} D_{a} X^{\mu i} D_{b} X^{\nu j}) = 0
\end{equation*}
\begin{equation*}
    \Downarrow
\end{equation*}
\begin{equation*}
    e^{l} _{e} = f e^{a} _{i} D_{a} X^{\mu i} D_{e} X^{\nu l} E_{\mu} ^{I} E_{\nu I} \ ,
\end{equation*}
\begin{equation*}
    \frac{1}{f} = e^{a} _{i} e^{b} _{j} D_{a} X^{\mu i} D_{b} X^{\nu j} E_{\mu} ^{I} E_{\nu I}
\end{equation*}
\begin{equation*}
    \downarrow
\end{equation*}
\begin{align*}
    e^{l} _{e} e_{f l} & = (f e^{a} _{i} D_{a} X^{\mu i} D_{e} X^{\nu l} E^{I} _{\mu} E_{\nu I}) (f e^{a'} _{i'} D_{a'} X^{\mu ' i'} D_{f} X^{\nu '} _{l} E^{I'} _{\mu '} E_{\nu ' I'}) = \\
                       & = f^{2} D_{e} X^{\nu l} D_{f} X^{\nu '} _{l} E^{I'} _{\nu} E_{\nu ' I'} (e^{a} _{i} e^{a'} _{i'} D_{a} X^{\mu i} D_{a'} X^{\mu ' i'} E^{I} _{\mu} E_{\mu ' I}) = \\
                       & = f D_{e} X^{\mu i} D_{f} X^{\nu} _{i} E^{I} _{\mu} E_{\nu I}
\end{align*}
\begin{equation*}
    \Downarrow
\end{equation*}
\begin{equation*}
    D_{e} X^{\mu i} D_{f} X^{\nu} _{i} E^{I} _{\mu} E_{\nu I} = 2 \partial _{e} X^{\mu} \partial _{f} X^{\nu} G_{\mu \nu}
\end{equation*}
\begin{equation*}
    e^{a} _{i} e^{b} _{j} D_{a} X^{\mu i} D_{b} X^{\nu j} E^{I} _{\mu} E_{\nu I} = g^{a b} \partial _{a} X^{\mu} \partial _{b} X^{\nu} G_{\mu \nu}
\end{equation*}
\begin{equation*}
    \Downarrow
\end{equation*}
\begin{equation*}
    D_{a} X^{\mu 0} = - i \partial _{a} X^{\mu}
\end{equation*}
\begin{equation*}
    D_{a} X^{\mu 1} = \partial _{a} X^{\mu}
\end{equation*}
\begin{equation*}
    \Downarrow
\end{equation*}
\begin{equation*}
    X^{\mu 0} = - i X^{\mu 1}
\end{equation*}

\subsubsection{With extra field, w.r.t $e$}

\begin{align*}
    \frac{\delta S_{M P 3}}{\delta e^{e} _{o}} & = \frac{D \Lagr _{M P 3}}{D e^{e} _{o}} = \left( \frac{\partial}{\partial e^{e} _{o}} - \partial _{f} \frac{\partial}{\partial (\partial _{f} e^{e} _{o})} \right) \left( - \frac{T}{2} \frac{1}{2} \varepsilon ^{c d} \varepsilon _{m n} e^{m} _{c} e^{n} _{d} \eta ^{i j} e^{a} _{i} e^{b} _{j} D_{a} (X^{\mu} _{k} v^{k}) D_{b} (X^{\nu} _{l} v^{l}) E_{\mu} ^{I} E_{\nu} ^{J} \eta _{I J} \right) = \\
                                               & = - \frac{T}{4} \varepsilon ^{c d} \varepsilon _{m n} (g_{c e} \eta ^{m o} e^{n} _{d} \eta ^{i j} e^{a} _{i} e^{b} _{j} + e^{m} _{c} g_{d e} \eta ^{n o} \eta ^{i j} e^{a} _{i} e^{b} _{j} +                                                                                                                                                                                                            \\
                                               & + e^{m} _{c} e^{n} _{d} \eta ^{i j} \delta ^{a} _{e} \delta ^{o} _{i} e^{b} _{j} + e^{m} _{c} e^{n} _{d} \eta ^{i j} e^{a} _{i} \delta ^{b} _{e} \delta ^{o} _{j}) D_{a} (X^{\mu} _{k} v^{k}) D_{b} (X^{\nu} _{l} v^{l}) E^{I} _{\mu} E^{J} _{\nu} \eta _{I J} =                                                                                                                                        \\
                                               & = - \frac{T}{2} \varepsilon ^{c d} \varepsilon _{m n} (e_{c} ^{p} e_{e p} e^{n} _{d} \eta ^{m o} \eta ^{i j} e^{a} _{i} e^{b} _{j} + 2 \det (e) e^{a o} \delta ^{b} _{e})  D_{a} (X^{\mu} _{k} v^{k}) D_{b} (X^{\nu} _{l} v^{l}) E^{I} _{\mu} E^{J} _{\nu} \eta _{I J} =                                                                                                                                \\
                                               & = - T (- \det (e) e^{o} _{e} \eta ^{i j} e^{a} _{i} e^{b} _{j} + \det (e) e^{a o} \delta ^{b} _{e}) D_{a} (X^{\mu} _{k} v^{k}) D_{b} (X^{\nu} _{l} v^{l}) E^{I} _{\mu} E^{J} _{\nu} \eta _{I J} \overset{!}{=} 0
\end{align*}
\begin{equation*}
    \downarrow
\end{equation*}
\begin{equation*}
    \boxed{T^{o} _{e} := (e^{a o} D_{a} (X^{\mu} _{k} v^{k}) D_{e} (X^{\nu} _{l} v^{l}) - e^{o} _{e} \eta ^{i j} e^{a} _{i} e^{b} _{j} D_{a} (X^{\mu} _{k} v^{k}) D_{b} (X^{\nu} _{l} v^{l})) E^{I} _{\mu} E^{J} _{\nu} \eta _{I J} = 0}
\end{equation*}
\begin{equation*}
    \Downarrow
\end{equation*}
\begin{equation*}
    e^{o} _{e} = f \, e^{a o} D_{a} (X^{\mu} _{k} v^{k}) D_{e} (X^{\nu} _{l} v^{l}) E^{I} _{\mu} E^{J} _{\nu} \eta _{I J} \ ,
\end{equation*}
\begin{equation*}
    \frac{1}{f} = \eta ^{i j} e^{a} _{i} e^{b} _{j} D_{a} (X^{\mu} _{k} v^{k}) D_{b} (X^{\nu} _{l} v^{l}) E^{I} _{\mu} E^{J} _{\nu} \eta _{I J}
\end{equation*}

\subsubsection{With extra field, w.r.t $\omega$}

\begin{align*}
    \frac{\delta S_{M P 3}}{\delta \omega ^{m n} _{c}} & = \frac{D \Lagr _{M P 3}}{D \omega ^{m n} _{c}} = \left( \frac{\partial}{\partial \omega ^{m n} _{c}} - \partial _{d} \frac{\partial}{\partial (\partial _{d} \omega ^{m n} _{c})} \right) \left( - \frac{T}{2} \det (e) \eta ^{i j} e^{a} _{i} e^{b} _{j} D_{a} X^{\mu} _{k} v^{k} D_{b} X^{\nu} _{l} v^{l} E_{\mu} ^{I} E_{\nu} ^{J} \eta _{I J} \right) = \\
                                                       & = - \frac{T}{4} e g^{a b} ((- \delta ^{c} _{a} \delta ^{k'} _{[m} \eta _{n] k} X^{\mu} _{k'} v^{k}) D_{b} X^{\nu} _{l} v^{l} + D_{a} X^{\mu} _{k} v^{k} (- \delta ^{c} _{b} \delta ^{l'} _{[m} \eta _{n] l} X^{\nu} _{l'} v^{l})) G_{\mu \nu} =                                                                                                              \\
                                                       & = \frac{T}{4} e (g^{c a} X^{\mu} _{[m} v_{n]} D_{a} X^{\nu} _{k} v^{k} + g^{a c} D_{a} X^{\mu} _{k} v^{k} X^{\nu} _{[m} v_{n]}) G_{\mu \nu} =                                                                                                                                                                                                                \\
                                                       & = \frac{T}{2} e g^{a c} D_{a} X^{\mu} _{k} v^{k} X^{\nu} _{[m} v_{n]} G_{\mu \nu} \overset{!}{=} 0
\end{align*}
\begin{equation*}
    \downarrow
\end{equation*}
\begin{equation*}
    \boxed{\mathcal{T}^{i} _{a b} := e^{c i} D_{c} X^{\mu} _{k} v^{k} X^{\nu} _{[m} v_{n]} e^{m} _{a} e^{n} _{b} G_{\mu \nu} = 0}
\end{equation*}

\subsubsection{With extra field, w.r.t $v$}

\begin{align*}
    \frac{\delta S_{M P 3}}{\delta v^{l}} & = \frac{D \Lagr _{M P 3}}{D v^{l}} = \left( \frac{\partial}{\partial v^{l}} - \partial _{c} \frac{\partial}{\partial (\partial _{c} v^{l})} \right) \left( - \frac{T}{2} e e^{a} _{i} e^{b i} D_{a} X^{\mu} _{j} v^{j} D_{b} X^{\nu} _{k} v^{k} E^{I} _{\mu} E_{\nu I} \right) = \\
                                          & = - \frac{T}{2} g^{a b} (D_{a} X^{\mu} _{j} \delta ^{j} _{l} D_{b} X^{\nu} _{k} v^{k} + D_{a} X^{\mu} _{j} v^{j} D_{b} X^{\nu} _{k} \delta ^{k} _{l}) G_{\mu \nu} =                                                                                                              \\
                                          & = - T g^{a b} D_{a} X^{\mu} _{l} D_{b} X^{\nu} _{j} v^{j} G_{\mu \nu} \overset{!}{=} 0
\end{align*}
\begin{equation*}
    \downarrow
\end{equation*}
\begin{equation*}
    \boxed{g^{a b} D_{a} X^{\mu} _{l} D_{b} X^{\nu} _{j} v^{j} G_{\mu \nu} = 0}
\end{equation*}

\subsubsection{With extra field, w.r.t $X$}

\begin{align*}
    \frac{\delta S_{M P 3}}{\delta X^{\lambda} _{l}} & = \frac{D \Lagr _{M P 3}}{D X^{\lambda} _{l}} = \left( \frac{\partial}{\partial X^{\lambda} _{l}} - \partial _{c} \frac{\partial}{\partial (\partial _{c} X^{\lambda} _{l})} \right) \left( - \frac{T}{2} e e^{a} _{i} e^{b i} D_{a} X^{\mu} _{j} v^{j} D_{b} X^{\nu} _{k} v^{k} E^{I} _{\mu} E_{\nu I}  \right) = \\
                                                     & = - \frac{T}{2} \Bigg( e g^{a b} \Bigg( (- \omega ^{j'} _{a j} \delta ^{\mu} _{\lambda} \delta ^{l} _{j'} v^{j} D_{b} X^{\nu} _{k} v^{k} - D_{a} X^{\mu} _{j} v^{j} \omega ^{k'} _{b k} \delta ^{\nu} _{\lambda} \delta ^{l} _{k'} v^{k}) E^{I} _{\mu} E_{\nu I} +                                                 \\
                                                     & + e g^{a b} D_{a} X^{\mu} _{j} v^{j} D_{b} X^{\nu} _{k} v^{k} \left( \frac{\partial E^{I} _{\mu}}{\partial X^{\lambda} _{l}} E_{\nu I} + E^{I} _{\mu} \frac{\partial E_{\nu I}}{\partial X^{\lambda} _{l}} \right) \Bigg) -                                                                                        \\
                                                     & - \partial _{c} (e g^{a b} (\delta ^{c} _{a} \delta ^{\mu} _{\lambda} \delta ^{l} _{j} v^{j} D_{b} X^{\nu} _{k} v^{k} + D_{a} X^{\mu} _{j} v^{j} \delta ^{c} _{b} \delta ^{\nu} _{\lambda} \delta ^{l} _{k} v^{k}) G_{\mu \nu}) \Bigg) =                                                                           \\
                                                     & = - T \Bigg( - e g^{a b} D_{a} X^{\mu} _{j} v^{j} \omega ^{l} _{b k} v^{k} G_{\mu \lambda} + e g^{a b} D_{a} X^{\mu} _{j} v^{j} D_{b} X^{\nu} _{k} v^{k} \frac{\partial E^{I} _{\mu}}{\partial X^{\lambda} _{l}} E_{\nu I} - \partial _{c} (e g^{a c} v^{l} D_{a} X^{\mu} _{j} v^{j} G_{\mu \lambda}) \Bigg)       \\
                                                     & = - T \Bigg( e g^{a b} D_{a} X^{\mu} _{j} v^{j} D_{b} X^{\nu} _{k} v^{k} \frac{\partial E^{I} _{\mu}}{\partial X^{\lambda} _{l}} E_{\nu I} - D_{b} (e g^{a b} v^{l} D_{a} X^{\mu} _{j} v^{j} E^{I} _{\mu} E_{\lambda I}) \Bigg) \overset{!}{=} 0
\end{align*}
\begin{equation*}
    \downarrow
\end{equation*}
\begin{equation*}
    \boxed{D_{b} (e e^{a} _{i} e^{b i} v^{l} D_{a} X^{\mu} _{j} v^{j} E^{I} _{\mu} E_{\lambda I}) = e e^{a} _{i} e^{b i} D_{a} X^{\mu} _{j} v^{j} D_{b} X^{\nu} _{k} v^{k} \frac{\partial E^{I} _{\mu}}{\partial X^{\lambda} _{l}} E_{\nu I}}
\end{equation*}

\subsection{Without extra index}

\subsubsection{w.r.t $e$}

\begin{align*}
    \frac{\delta S_{D P}}{\delta e^{e} _{l}} & = \frac{D \Lagr _{D P}}{D e^{e} _{l}} = \left( \frac{\partial}{\partial e^{e} _{l}} - \partial _{f} \frac{\partial}{\partial (\partial _{f} e^{e} _{l})} \right) \left( - \frac{T_{0}}{2} \frac{1}{2} \varepsilon ^{c d} \varepsilon _{m n} e^{m} _{c} e^{n} _{d} \eta ^{i j} e^{a} _{i} e^{b} _{j} \partial _{a} X^{\mu} \partial _{b} X^{\nu} E^{I} _{\mu} E^{J} _{\nu} \eta _{I J} \right) = \\
                                             & = - \frac{T_{0}}{4}
\end{align*}

\subsection{Linear action}

\subsubsection{w.r.t $e$}

\begin{align*}
    \frac{\delta S_{L P}}{\delta \delta e^{b} _{j}} & = \frac{D \Lagr _{L P}}{D e^{b} _{j}} = \left( \frac{\partial}{\partial e^{b} _{j}} - \partial _{e} \frac{\partial}{\partial (\partial _{e} e^{b} _{j})} \right) \left( - \frac{T}{2} \overline{\psi} e e^{a} _{i} D_{a} X^{\mu i} E^{I} _{\mu} \gamma _{I} \psi \right) = \\
                                                    & = - \frac{T}{4} \overline{\psi} \varepsilon ^{c d} \varepsilon _{m n} ((g_{c b} \eta ^{m j} e^{n} _{d} + e^{m} _{c} g_{d b} \eta ^{n j}) e^{a} _{i} + e^{m} _{c} e^{n} _{d} \delta ^{a} _{b} \delta ^{j} _{i}) D_{a} X^{\mu i} E^{I} _{\mu} \gamma _{I} \psi =             \\
                                                    & = - \frac{T}{2} \overline{\psi} (- e e^{j} _{b} e^{a} _{i} D_{a} X^{\mu i} E^{I} _{\mu} \gamma _{I} + e D_{b} X^{\mu j} E^{I} _{\mu} \gamma _{I}) \psi \overset{!}{=} 0
\end{align*}
\begin{equation*}
    \downarrow
\end{equation*}
\begin{equation*}
    T^{j} _{b} := \overline{\psi} (D_{b} X^{\mu j} E^{I} _{\mu} \gamma _{I} - e^{j} _{b} e^{a} _{i} D_{a} X^{\mu i} E^{I} _{\mu} \gamma _{I}) \psi = 0
\end{equation*}
\begin{equation*}
    \downarrow
\end{equation*}
\begin{equation*}
    e^{j} _{b} = F \overline{\psi} D_{b} X^{\mu j} E^{I} _{\mu} \gamma _{I} \psi ,
\end{equation*}
\begin{equation*}
    \frac{1}{F} = \overline{\psi} e^{a} _{i} D_{a} X^{\mu i} E^{I} _{\mu} \gamma _{I} \psi
\end{equation*}
\begin{equation*}
    \Downarrow
\end{equation*}
\begin{equation*}
    \overline{\psi} \psi = 1 , \ \psi \overline{\psi} = \mathbb{1}
\end{equation*}
\begin{equation*}
    e^{a} _{i} e^{b} _{j} D_{a} X^{\mu i} D_{b} X^{\nu j} E^{I} _{\mu} E^{J} _{\nu} \eta _{I J} = g^{a b} \partial _{a} X^{\mu} \partial _{b} X^{\nu} G_{\mu \nu}
\end{equation*}

\section{Inverse Area Action}

\subsection{Nambu-Goto}

Start from Nambu-Goto action
\begin{equation*}
    S_{N G} = - T \int \diff ^2 x \sqrt{- h} ,
\end{equation*}
and make quantum geometry correction
\begin{equation*}
    \sqrt{- h} \rightarrow \sqrt{- (h + g \Delta (x))} \approx \sqrt{- h} \left( 1 - \frac{g \Delta (x)}{2 (- h)} + \mathcal{O} \left( \frac{g^2}{h^2} \right) \right) ,
\end{equation*}
\begin{equation*}
    \Delta (x) \rightarrow \Delta (x ') = J^2 \Delta (x) , \ \Delta = \det (\Delta _{a b})
\end{equation*}
leading to modified NG action
\begin{equation*}
    S_{M N G} = - T \int \diff ^2 x \left( \sqrt{- h} - \frac{g \Delta (x)}{2 \sqrt{- h}} \right) = S_{N G} + S_{I A}
\end{equation*}
\begin{equation*}
    \downarrow
\end{equation*}
\begin{equation*}
    \Lagr _{M N G} = - T \left( \sqrt{- h} - \frac{g \Delta}{2 \sqrt{- h}} \right)
\end{equation*}

\subsubsection{EoMs}

\begin{align*}
    \frac{\delta S_{M N G}}{\delta \Delta ^{a b}} & = - \frac{T g}{2 \sqrt{- h}} \frac{\del \Delta}{\del \Delta ^{a b}} = \\
                                                  & = - \frac{T g}{2 \sqrt{- h}} \Delta \Delta _{a b} = 0
\end{align*}
\begin{equation*}
    \frac{g \Delta}{\sqrt{- h}} \Delta _{a b} = 0
\end{equation*}

Let $\PS ^{\tau} _{\mu} \equiv \partial \Lagr _{M N G} / \partial \Dot{X}^{\mu}$ and $\PS ^{\sigma} _{\mu} \equiv \partial \Lagr _{M N G} / \partial X^{\prime \mu}$

\begin{align*}
    \delta S_{M N G} & = \int \diff ^2 x \left( \frac{\partial \Lagr _{M N G}}{\partial \Dot{X}^{\mu}} \delta \Dot{X}^{\mu} + \frac{\partial \Lagr _{M N G}}{\partial X^{\prime \mu}} \delta X^{\prime \mu} \right) =                                                   \\
                     & = - \int \diff ^2 x \left( \partial _{\tau} \PS ^{\tau} _{\mu} + \partial _{\sigma} \PS ^{\sigma} _{\mu} \right) \delta X^{\mu} + \int \diff \tau \PS ^{\sigma} _{\mu} \delta X^{\mu} \big| _{\sigma = 0} ^{\sigma = \sigma _1} \overset{!}{=} 0
\end{align*}
\begin{equation*}
    \downarrow
\end{equation*}
\begin{equation*}
    \mathrm{EoM:} \partial _{\tau} \PS ^{\tau} _{\mu} + \partial _{\sigma} \PS ^{\sigma} _{\mu} = 0
\end{equation*}
\begin{equation*}
    \mathrm{B.C.:} \PS ^{\sigma} _{\mu} \delta X^{\mu} \big| _{\sigma = 0} ^{\sigma = \sigma _1} = 0 ,
\end{equation*}
where
\begin{equation*}
    \PS ^{\tau} _{\mu} = \frac{\partial \Lagr _{M N G}}{\partial \Dot{X}^{\mu}} = - T \left( \frac{(\Dot{X} \cdot X') X' _{\mu} - (X')^2 \Dot{X}_{\mu}}{\sqrt{- h}} + \frac{g \Delta}{2} \frac{(\Dot{X} \cdot X') X' _{\mu} - (X')^2 \Dot{X}_{\mu}}{(- h)^{3/2}} \right) =
\end{equation*}
\begin{equation*}
    = - T \frac{(\Dot{X} \cdot X') X' _{\mu} - (X')^2 \Dot{X}_{\mu}}{\sqrt{- h}} \left( 1 + \frac{g \Delta}{2 (- h)} \right) = \PS ^{\tau} _{\mu (N G)} \left( 1 + \frac{g \Delta}{2 (- h)} \right)
\end{equation*}
\begin{equation*}
    \PS ^{\sigma} _{\mu} = \frac{\partial \Lagr _{M N G}}{\partial X^{\prime \mu}} = - T \left( \frac{(\Dot{X} \cdot X') \Dot{X}_{\mu} - (\Dot{X})^2 X' _{\mu}}{\sqrt{- h}} + \frac{g \Delta}{2} \frac{(\Dot{X} \cdot X') \Dot{X}_{\mu} - (\Dot{X})^2 X' _{\mu}}{(- h)^{3/2}} \right) =
\end{equation*}
\begin{equation*}
    = - T \frac{(\Dot{X} \cdot X') \Dot{X}_{\mu} - (\Dot{X})^2 X' _{\mu}}{\sqrt{- h}} \left( 1 + \frac{g \Delta}{2 (- h)} \right) = \PS ^{\sigma} _{\mu (N G)} \left( 1 + \frac{g \Delta}{2 (- h)} \right)
\end{equation*}
gauge fixing static gauge $\tau = t$ and transverse gauge $\frac{\partial X}{\partial \tau} \cdot \frac{\partial X}{\partial s} \frac{\diff s}{\diff \sigma} = 0$ ($s = \mathrm{length \ along \ string}$)
\begin{equation*}
    \downarrow
\end{equation*}
\begin{equation*}
    \PS ^{\tau \mu} _{(N G)} = T \frac{\diff s}{\diff \sigma} \gamma _{v_{\perp}} \frac{\partial X^{\mu}}{\partial t}
\end{equation*}
\begin{equation*}
    \PS ^{\sigma \mu} _{(N G)} = \frac{T}{\gamma _{v_{\perp}}} \frac{\partial X^{\mu}}{\partial s}
\end{equation*}
\begin{equation*}
    - h = \frac{1}{\gamma _{v_{\perp}} ^2} \left( \frac{\diff s}{\diff \sigma} \right) ^2
\end{equation*}
string energy get's redefined to
\begin{equation*}
    \frac{\partial}{\partial t} \left( T \frac{\diff s}{\diff \sigma} \gamma _{v_{\perp}} \left( 1 + \frac{g \Delta}{2 (- h)} \right) \right) = 0,
\end{equation*}
and for the spatial part we have
\begin{equation*}
    \partial _{\tau} \Vec{\PS}^{\tau} + \partial _{\sigma} \Vec{\PS}^{\sigma} = 0
\end{equation*}
\begin{equation*}
    \downarrow
\end{equation*}
\begin{equation*}
    \frac{\partial}{\partial t} \left[ T \frac{\diff s}{\diff \sigma} \gamma _{v_{\perp}} \frac{\partial \Vec{X}}{\partial t} \left( 1 + \frac{g \Delta}{2 (- h)} \right) \right] + \frac{\diff s}{\diff \sigma} \frac{\partial}{\partial s} \left[ - \frac{T}{\gamma _{v_{\perp}}} \frac{\partial \Vec{X}}{\partial s} \left( 1 + \frac{g \Delta}{2 (- h)} \right) \right] = 0
\end{equation*}
\begin{equation*}
    \downarrow
\end{equation*}
\begin{equation*}
    \mu \gamma _{v_{\perp}} \left( 1 + \frac{g \Delta}{2 (- h)} \right) \frac{\partial ^2 \Vec{X}}{\partial t^2} - \frac{\partial}{\partial s} \left[ \frac{T}{\gamma _{v_{\perp}}} \left( 1 + \frac{g \Delta}{2 (- h)} \right) \frac{\partial \Vec{X}}{\partial s} \right] = 0
\end{equation*}
$\implies$ effective mass density becomes $\mu _{eff} = \mu \gamma _{v_{\perp}} \left( 1 - \frac{g \Delta}{2 (- h)} \right)$ and effective tension becomes $T_{eff} = \frac{T}{\gamma _{v_{\perp}}} \left( 1 - \frac{g \Delta}{2 (- h)} \right)$
\begin{equation*}
    \downarrow
\end{equation*}
\begin{equation*}
    \mu _{eff} \frac{\partial ^2 \Vec{X}}{\partial t^2} - \frac{\partial}{\partial s} \left[ T_{eff} \frac{\partial \Vec{X}}{\partial s} \right] = 0
\end{equation*}
\begin{align*}
    \mathscr{H} & = \Dot{\Vec{X}} \cdot \Vec{\pi} - \Lagr =                                                                                                                                                                                                                                           \\
                & = \Vec{v}_{\perp} \cdot \left( T \frac{\diff s}{\diff \sigma} \gamma _{v_{\perp}} \left( 1 + \frac{g \Delta}{2 (- h)} \right) \Vec{v}_{\perp} \right) - \left( - T \frac{\diff s}{\diff \sigma} \frac{1}{\gamma _{v_{\perp}}} \left( 1 + \frac{g \Delta}{2 (- h)} \right) \right) = \\
                & = T \frac{\diff s}{\diff \sigma} \left( 1 + \frac{g \Delta}{2 (- h)} \right) \left( \gamma _{v_{\perp}} v_{\perp} ^2 + \frac{1}{\gamma _{v_{\perp}}} \right) =                                                                                                                      \\
                & = T \frac{\diff s}{\diff \sigma} \gamma _{v_{\perp}} \left( 1 + \frac{g \Delta}{2 (- h)} \right)
\end{align*}
let $\left( 1 + \frac{g \Delta}{2 (- h)} \right) = F$
\begin{equation*}
    \frac{\partial ^2 \Vec{X}}{\partial t^2} - \frac{1}{F \gamma _{v_{\perp}}} \frac{\diff \sigma}{\diff s} \frac{\partial}{\partial \sigma} \left[ \frac{1}{\gamma _{v_{\perp}}} F \frac{\diff \sigma}{\diff s} \frac{\partial \Vec{X}}{\partial \sigma} \right] = 0
\end{equation*}
\begin{equation*}
    \downarrow
\end{equation*}
\begin{equation*}
    A (\sigma) = \frac{\gamma _{v_{\perp}}}{F} \frac{\diff s}{\diff \sigma} \overset{!}{=} 1
\end{equation*}
\begin{equation*}
    \downarrow
\end{equation*}
\begin{equation*}
    \diff \sigma = \frac{\gamma _{v_{\perp}}}{F} \diff s = \frac{1}{T F^2} \diff E \implies \sigma (q) = \frac{1}{T} \int _{0} ^{q} \frac{1}{F^2} \diff E
\end{equation*}
\begin{equation*}
    \downarrow
\end{equation*}
\begin{equation*}
    F^2 \frac{\partial ^2 \Vec{X}}{\partial t^2} - \frac{\partial ^2 \Vec{X}}{\partial \sigma ^2} = 0
\end{equation*}
$\implies$ speed of wave on the string gets modified by correction factor $v = c / F$.

\subsection{Translation and Lorentz symmetry}

Starting with global translations, define the transformation by
\begin{equation*}
    X^{\mu} \mapsto X^{\mu} + \epsilon ^{\mu} .
\end{equation*}
This does not change the action since it only depends on derivatives of $X^{\mu}$ and so $\delta (\partial _{a} X^{\mu}) = \partial _{a} (\delta X^{\mu}) = \partial _{a} \epsilon ^{\mu} = 0$.

As for Lorentz transformations, the infinitesimal for is
\begin{equation*}
    X^{\mu} \mapsto X^{\mu} + \epsilon ^{\mu \nu} X_{\nu} ,
\end{equation*}
with $\epsilon ^{\mu \nu}$ anti-symmetric.

\subsection{Verifying if there is longitudinal velocity}



\begin{equation*}
    - h = \frac{1}{\gamma _{v_{\perp}} ^2} \left( \frac{\diff s}{\diff \sigma} \right) ^2 = \frac{F^2}{\gamma _{v_{\perp}} ^4} = F^2 \left( 1 - v_{\perp} ^2 \right) ^2
\end{equation*}
\begin{equation*}
    \downarrow
\end{equation*}
\begin{equation*}
    - h = \left( 1 + \frac{g \Delta}{2 (- h)} \right) ^2 \left( 1 - v_{\perp} ^2 \right) ^2
\end{equation*}
\begin{equation*}
    \downarrow
\end{equation*}
\begin{equation*}
    \frac{- h}{\left( 1 + \frac{g \Delta}{2 (- h)} \right) ^2} = \left( 1 - v_{\perp} ^2 \right) ^2
\end{equation*}
\begin{equation*}
    \ \ \ \ \ \ \ \ \ \ \ \ \ \ \ \ \ \ \ \ \ \ \ \downarrow (\mathrm{WolframAlpha})
\end{equation*}
\begin{equation*}
    - h = \frac{(1 - v_{\perp}^2)^2}{3} - f_1 - f_2 ,
\end{equation*}
where (let $a = \frac{g \Delta}{2}$ and $b = (1 - v_{\perp}^2)^2$)
\begin{equation*}
    f_1 = \frac{\sqrt[3]{- 27 a^2 b + 3 \sqrt{3} \sqrt{27 a^4 b^2 + 4 a^3 b^3} - 18 a b^2 - 2 b^3}}{3 \sqrt[3]{2}}
\end{equation*}
\begin{equation*}
    f_2 = \frac{(6 a b + b^2)}{6 f_1}
\end{equation*}
\begin{equation*}
    \downarrow
\end{equation*}
\begin{equation*}
    F = 1 + \frac{g \Delta}{2 (- h)} = 1 + \frac{g \Delta}{2 \left( \frac{b}{3} - f_1 - f_2 \right)}
\end{equation*}
\begin{equation*}
    \downarrow
\end{equation*}
\begin{equation*}
    v = \frac{1}{F} = \frac{1}{1 + \dfrac{g \Delta}{2 \left( \dfrac{b}{3} - f_1 - f_2 \right)}}
\end{equation*}

\newpage

\section{Bimetric Polyakov}

Start with Polyakov action
\begin{equation*}
    S_{P} = - \frac{T}{2} \int \diff ^2 x \sqrt{- g} g^{a b} \partial _{a} X^{\mu} \partial _{b} X^{\nu} G_{\mu \nu},
\end{equation*}
and promote to bimetric action
\begin{equation*}
    \downarrow
\end{equation*}
\begin{equation*}
    S_{B P} = - \frac{T}{2} \int \diff ^2 x \sqrt{- g} g^{a b} \partial _{a} X^{\mu} \partial _{b} X^{\nu} G_{\mu \nu} - \frac{T'}{2} \int \diff ^2 x \sqrt{- h} h^{a b} \partial _{a} X^{\mu} \partial _{b} X^{\nu} H_{\mu \nu}
\end{equation*}

\subsection{EoMs}

\begin{align*}
    \frac{\delta S_{B P}}{\delta g^{c d}} & = - \frac{T}{2} \left( \frac{\partial \sqrt{- g}}{\partial g^{c d}} g^{a b} + \sqrt{- g} \frac{\partial g^{a b}}{\partial g^{c d}} \right) \partial _{a} X^{\mu} \partial _{b} X^{\nu} G_{\mu \nu} + 0 = \\
                                          & = - \frac{T}{2} \left( - \frac{1}{2} \sqrt{- g} g_{c d} g^{a b} + \sqrt{- g} \delta ^{a} _{(c} \delta ^{b} _{d)} \right) \partial _{a} X^{\mu} \partial _{b} X^{\nu} G_{\mu \nu} \overset{!}{=} 0
\end{align*}
\begin{equation*}
    T^{(G)} _{c d} := \left( \partial _{c} X^{\mu} \partial _{d} X^{\nu} - \frac{1}{2} g_{c d} g^{a b} \partial _{a} X^{\mu} \partial _{b} X^{\nu} \right) G_{\mu \nu} = 0
\end{equation*}
\begin{equation*}
    \Downarrow
\end{equation*}
\begin{equation*}
    g_{c d} = 2 f^{(G)} \partial _{c} X^{\mu} \partial _{d} X^{\nu} G_{\mu \nu} ,
\end{equation*}
\begin{equation*}
    \frac{1}{f^{(G)}} = g^{a b} \partial _{a} X^{\mu} \partial _{b} X^{\nu} G_{\mu \nu}
\end{equation*}

\begin{align*}
    \frac{\delta S_{B P}}{\delta h^{c d}} & = ... =                                                                                                                                                                                            \\
                                          & = - \frac{T'}{2} \left( - \frac{1}{2} \sqrt{- h} h_{c d} h^{a b} + \sqrt{- h} \delta ^{a} _{(c} \delta ^{b} _{d)} \right) \partial _{a} X^{\mu} \partial _{b} X^{\nu} H_{\mu \nu} \overset{!}{=} 0
\end{align*}
\begin{equation*}
    \downarrow
\end{equation*}
\begin{equation*}
    T^{(H)} _{c d} := \left( \partial _{c} X^{\mu} \partial _{d} X^{\mu} - \frac{1}{2} h_{c d} h^{a b} \partial _{a} X^{\mu} \partial _{b} X^{\nu} \right) H_{\mu \nu} = 0
\end{equation*}
\begin{equation*}
    \Downarrow
\end{equation*}
\begin{equation*}
    h_{c d} = 2 f^{(H)} \partial _{c} X^{\mu} \partial _{d} X^{\nu} H_{\mu \nu} ,
\end{equation*}
\begin{equation*}
    \frac{1}{f^{(H)}} = h^{a b} \partial _{a} X^{\mu} \partial _{b} X^{\nu} H_{\mu \nu}
\end{equation*}
If the auxiliary metrics $g$ and $h$ concide with the induced metrics from $G$ and $H$, then $f^{(G)} = f^{(H)} = 1/2$ and the action reduces to
\begin{equation*}
    S = - T \int \diff ^2 x \sqrt{- g} - T' \int \diff ^2 x \sqrt{- h} ,
\end{equation*}
where if $T' = - T k / 2$ and $h = \Delta ^{2} / g$ we recover the NG area corrected action
\begin{equation*}
    S = - T \int \diff ^2 x \left( \sqrt{- g} - \frac{k \Delta}{2 \sqrt{- g}} \right) .
\end{equation*}
The condition $h = \Delta ^{2} / g$ can be achieved by
\begin{equation*}
    h_{a b} = \Delta _{a c} \Delta _{b d} g^{c d} .
\end{equation*}
This also implies
\begin{align*}
    \partial _{a} X^{\mu} \partial _{b} X^{\nu} H_{\mu \nu}                                       & = \Delta _{a c} \Delta _{b d} g^{c d} =                                                                                                                      \\
                                                                                                  & = \Delta _{a c} \Delta _{b d} \left( \frac{1}{2 g} \varepsilon ^{c c'} \varepsilon ^{d d'} g_{c' d'} \right) =                                               \\
    \delta _{a} ^{c'} \delta _{b} ^{d'} \partial _{c'} X^{\mu} \partial _{d'} X^{\nu} H_{\mu \nu} & = \Delta _{a c} \Delta _{b d} \left( \frac{1}{2 g} \varepsilon ^{c c'} \varepsilon ^{d d'} \partial _{c'} X^{\mu} \partial _{d'} X^{\nu} G_{\mu \nu} \right)
\end{align*}
\begin{equation*}
    \left( \delta _{a} ^{c'} \delta _{b} ^{d'} H_{\mu \nu} - \frac{1}{2 g} \Delta _{a c} \varepsilon ^{c c'} \Delta _{b d} \varepsilon ^{d d'} G_{\mu \nu} \right) \del _{c'} X^{\mu} \del _{d'} X^{\nu} = 0 .
\end{equation*}
One way this can hold is if the expression in parenthesis is equal to 0, which can be achieved if $\Delta _{a c} = \sqrt{\Delta} \varepsilon _{a c}$ and also
\begin{equation*}
    \delta _{a} ^{c'} \delta _{b} ^{d'} H_{\mu \nu} - \frac{\Delta}{2 g} \delta _{a} ^{c'} \delta _{b} ^{d'} G_{\mu \nu} = 0
\end{equation*}
\begin{equation*}
    H_{\mu \nu} = \frac{\Delta}{2 g} G_{\mu \nu} .
\end{equation*}

\begin{align*}
    \frac{\delta S_{B P}}{\delta X^{\lambda}} & = \left( - \frac{T}{2} \sqrt{- g} g^{a b} \partial _{\lambda} G_{\mu \nu} - \frac{T'}{2} \sqrt{- h} h^{a b} \partial _{\lambda} H_{\mu \nu} \right) \partial _{a} X^{\mu} \partial _{b} X^{\nu} -                                                               \\
                                              & - \partial _{c} \left( - \frac{T}{2} \sqrt{- g} g^{a b} G_{\mu \nu} - \frac{T'}{2} \sqrt{- h} h^{a b} H_{\mu \nu} \right) (\delta ^{c} _{a} \delta ^{\mu} _{\lambda} \partial _{b} X^{\nu} + \partial _{a} X^{\mu} \delta ^{c} _{b} \delta ^{\nu} _{\lambda}) = \\
                                              & = \partial _{a} \left( T \sqrt{- g} g^{a b} G_{\mu \lambda} + T' \sqrt{- h} h^{a b} H_{\mu \lambda} \right) \partial _{b} X^{\mu} -                                                                                                                             \\
                                              & - \left( \frac{T}{2} \sqrt{- g} g^{a b} \partial _{\lambda} G_{\mu \nu} + \frac{T'}{2} \sqrt{- h} h^{a b} \partial _{\lambda} H_{\mu \nu} \right) \partial _{a} X^{\mu} \partial _{b} X^{\nu} \overset{!}{=} 0
\end{align*}
\begin{equation*}
    \downarrow
\end{equation*}
\begin{equation*}
    \partial _{a} \left( F^{a b} _{\mu \lambda} \partial _{b} X^{\mu} \right) = \frac{1}{2} \partial _{\lambda} F^{a b} _{\mu \nu} \partial _{a} X^{\mu} \partial _{b} X^{\nu} ,
\end{equation*}
\begin{equation*}
    F^{a b} _{\mu \nu} = T \sqrt{- g} g^{a b} G_{\mu \nu} + T' \sqrt{- h} h^{a b} H_{\mu \nu} .
\end{equation*}
Assuming the condition for $h = \Delta ^2 / g$, $T' = - T k / 2$ and flat background $G_{\mu \nu} = \eta _{\mu \nu}$, the EoM reduces to
\begin{equation*}
    \partial _{a} \left( \left( \sqrt{- g} g^{a b} - \frac{k \Delta ^2}{4 g} \sqrt{- g}^{-1} (\sqrt{\Delta}^{-1} \varepsilon ^{a c}) (\sqrt{\Delta}^{-1} \varepsilon ^{b d}) g_{c d} \right) \partial _{b} X^{\mu} \right) = 0
\end{equation*}
\begin{equation*}
    \del _{a} \left( \left( \sqrt{- g} g^{a b} - \frac{k \Delta}{2 \sqrt{- g}} g^{a b} \right) \del _{b} X^{\mu} \right) = 0
\end{equation*}
\begin{equation*}
    \del _{a} \left( \left( 1 - \frac{k \Delta}{2 (- g)} \right) \sqrt{- g} g^{a b} \del _{b} X^{\mu} \right) = 0 ,
\end{equation*}
which in conformal gauge $g_{a b} = \e ^{\phi} \eta _{a b}$ reduces further
\begin{equation*}
    \partial _{a} \left( \left( 1 - \frac{k \Delta}{2 \e ^{2 \phi}} \right) \eta ^{a b} \partial _{b} X^{\mu} \right) = 0
\end{equation*}
\begin{equation*}
    \eta ^{a b} \partial _{a} \left( F^{-} \partial _{b} X^{\mu} \right) = 0
\end{equation*}
\begin{equation*}
    F^{-} \eta ^{a b} \partial _{a} \partial _{b} X^{\mu} + \eta ^{a b} \partial _{a} F^{-} \partial _{b} X^{\mu} = 0
\end{equation*}
\begin{equation*}
    \eta ^{a b} \partial _{a} \partial _{b} X^{\mu} + \eta ^{a b} \partial _{a} \ln (F^{-}) \partial _{b} X^{\mu} = 0 .
\end{equation*}
This is the exact same equation derived in the next section which leads to the Klein-Gordon eqn. In fact, plugging the condition for $h = \Delta ^2 / g$ and $T' = - T k / 2$ in the action reproduces the action in next section:
\begin{equation*}
    S = - \frac{T}{2} \int \diff ^2 x \sqrt{- g} g^{a b} \del _{a} X^{\mu} \del _{b} X^{\nu} G_{\mu \nu} + \frac{T k}{4} \int \diff ^2 x \Delta \sqrt{- g}^{-1} (\sqrt{\Delta}^{-1} \varepsilon ^{a c}) (\sqrt{\Delta}^{-1} \varepsilon ^{b d}) g_{c d} \del _{a} X^{\mu} \del _{b} X^{\nu} \frac{\Delta}{2 g} G_{\mu \nu} =
\end{equation*}
\begin{equation*}
    = - \frac{T}{2} \int \diff ^2 x \left( \sqrt{- g} g^{a b} \del _{a} X^{\mu} \del _{b} X^{\nu} G_{\mu \nu} - \frac{k \Delta}{2 \sqrt{- g}} g^{a b} \del _{a} X^{\mu} \del _{b} X^{\nu} G_{\mu \nu} \right) =
\end{equation*}
\begin{equation*}
    = - \frac{T}{2} \int \diff ^2 x \left( \sqrt{- g} - \frac{k \Delta}{2 \sqrt{- g}} \right) g^{a b} \del _{a} X^{\mu} \del _{b} X^{\nu} G_{\mu \nu} ,
\end{equation*}
so the area corrected Polyakov action of next section is just a special case of the bimetric Polyakov action.

\newpage

\section{Inverse Area Polyakov}

Start with Polyakov action
\begin{equation*}
    S_{P} = - \frac{T}{2} \int \diff ^2 x \sqrt{- g} g^{a b} \partial _{a} X^{\mu} \partial _{b} X^{\nu} G_{\mu \nu}
\end{equation*}
and make quantum geometry correction
\begin{equation*}
    \sqrt{- g} \rightarrow \sqrt{- (g + k \Delta (x))} \approx \sqrt{- g} \left( 1 + \frac{k \Delta (x)}{2 g} + \mathcal{O} \left( \frac{k^2}{g^2} \right) \right) ,
\end{equation*}
\begin{equation*}
    \Delta (x) \rightarrow \Delta (x') = J^2 \Delta (x) , \ \Delta = \det (\Delta _{a b})
\end{equation*}
leading to
\begin{equation*}
    S_{I A P} = - \frac{T}{2} \int \diff ^{2} x \left( \sqrt{- g} - \frac{k \Delta}{2 \sqrt{- g}} \right) g^{a b} \partial _{a} X^{\mu} \partial _{b} X^{\nu} G_{\mu \nu}
\end{equation*}

\subsection{EoMs}

\begin{align*}
    \frac{\delta S_{I A P}}{\delta \Delta ^{c d}} & = - \frac{T k}{4 \sqrt{- g}} \frac{\del \Delta}{\del \Delta ^{c d}} g^{a b} \del _{a} X^{\mu} \del _{b} X^{\nu} G_{\mu \nu} = \\
                                                  & = - \frac{T k}{4 \sqrt{- g}} \Delta \Delta _{c d} g^{a b} \del _{a} X^{\mu} \del _{b} X^{\nu} G_{\mu \nu} = 0
\end{align*}
\begin{equation*}
    \frac{k \Delta}{\sqrt{- g}} \Delta _{c d} = 0
\end{equation*}

\begin{align*}
    \frac{\delta S_{I A P}}{\delta g^{c d}} & = \frac{\partial}{\partial g^{c d}} \left( \sqrt{- g} - \frac{k \Delta}{2 \sqrt{- g}} \right) g^{a b} \partial _{a} X^{\mu} \partial _{b} X^{\nu} G_{\mu \nu} + \left( \sqrt{- g} - \frac{k \Delta}{2 \sqrt{- g}} \right) \delta ^{a} _{c} \delta ^{b} _{d} \partial _{a} X^{\mu} \partial _{b} X^{\nu} G_{\mu \nu} = \\
                                            & = \left( - \frac{1}{2} \sqrt{- g} g_{c d} + \frac{k \Delta}{4 \sqrt{- g}} g_{c d} \right) g^{a b} \partial _{a} X^{\mu} \partial _{b} X^{\nu} G_{\mu \nu} + \left( \sqrt{- g} - \frac{k \Delta}{2 \sqrt{- g}} \right) \partial _{c} X^{\mu} \partial _{d} X^{\nu} G_{\mu \nu} =                                       \\
                                            & = - \frac{1}{2} \left( \sqrt{- g} - \frac{k \Delta}{2 \sqrt{- g}} \right) g_{c d} g^{a b} \partial _{a} X^{\mu} \partial _{b} X^{\nu} G_{\mu \nu} + \left( \sqrt{- g} - \frac{k \Delta}{2 \sqrt{- g}} \right) \partial _{c} X^{\mu} \partial _{d} X^{\nu} G_{\mu \nu} \overset{!}{=} 0
\end{align*}
\begin{equation*}
    \downarrow
\end{equation*}
\begin{equation*}
    T_{c d} := \partial _{c} X^{\mu} \partial _{d} X^{\nu} G_{\mu \nu} - \frac{1}{2} g_{c d} g^{a b} \partial _{a} X^{\mu} \partial _{b} X^{\nu} G_{\mu \nu} = 0
\end{equation*}
\begin{equation*}
    \downarrow
\end{equation*}
\begin{equation*}
    g_{c d} = 2 f \partial _{c} X^{\mu} \partial _{d} X^{\nu} G_{\mu \nu} ,
\end{equation*}
\begin{equation*}
    \frac{1}{f} = g^{a b} \partial _{a} X^{\mu} \partial _{b} X^{\nu} G_{\mu \nu}
\end{equation*}

\begin{align*}
    \frac{\delta S_{I A P}}{\delta X^{\lambda}} & = \left( \sqrt{- g} - \frac{k \Delta}{2 \sqrt{- g}} \right) g^{a b} \partial _{a} X^{\mu} \partial _{b} X^{\nu} \partial _{\lambda} G_{\mu \nu} - \partial _{c} \left( \left( \sqrt{- g} - \frac{k \Delta}{2 \sqrt{- g}} \right) 2 g^{a b} \delta ^{c} _{a} \delta ^{\mu} _{\lambda} \partial _{b} X^{\nu} G_{\mu \nu} \right) = \\
                                                & = \left( \sqrt{- g} - \frac{k \Delta}{2 \sqrt{- g}} \right) g^{a b} \partial _{a} X^{\mu} \partial _{b} X^{\nu} \partial _{\lambda} G_{\mu \nu} - 2 \left( \sqrt{- g} - \frac{k \Delta}{2 \sqrt{- g}} \right) g^{a b} \partial _{a} \partial _{b} X^{\nu} G_{\lambda \nu} -                                                      \\
                                                & - 2 \partial _{a} \left( \left( \sqrt{- g} - \frac{k \Delta}{2 \sqrt{- g}} \right) g^{a b} \right) \partial _{b} X^{\nu} G_{\lambda \nu} =                                                                                                                                                                                       \\
                                                & = \left( 1 - \frac{k \Delta}{2 (- g)} \right) \sqrt{- g} g^{a b} \partial _{a} X^{\mu} \partial _{b} X^{\nu} \partial _{\lambda} G_{\mu \nu} - 2 \left( 1 - \frac{k \Delta}{2 (- g)} \right) \sqrt{- g} g^{a b} \partial _{a} \partial _{b} X^{\nu} G_{\lambda \nu} -                                                            \\
                                                & - 2 \partial _{a} \left( \left( 1 - \frac{k \Delta}{2 (- g)} \right) \sqrt{- g} g^{a b} \right) \partial _{b} X^{\nu} G_{\lambda \nu} \overset{!}{=} 0
\end{align*}
\begin{equation*}
    \downarrow
\end{equation*}
\begin{multline*}
    \left( 1 - \frac{k \Delta}{2 (- g)} \right) \sqrt{- g} g^{a b} \partial _{a} \partial _{b} X^{\nu} G_{\lambda \nu} + \partial _{a} \left( \left( 1 - \frac{k \Delta}{2 (- g)} \right) \sqrt{- g} g^{a b} \right) \partial _{b} X^{\nu} G_{\lambda \nu} = \\
    \frac{1}{2} \left( 1 - \frac{k \Delta}{2 (- g)} \right) \sqrt{- g} g^{a b} \partial _{a} X^{\mu} \partial _{b} X^{\nu} \partial _{\lambda} G_{\mu \nu} ,
\end{multline*}
imposing conformal symmetry $g^{a b} = \e ^{- \phi (x)} \eta ^{a b}$ and let $F' = \left( 1 - \frac{k \Delta}{2 \e ^{2 \phi}} \right)$ leads to
\begin{equation*}
    \eta ^{a b} \partial _{a} \partial _{b} X^{\nu} G_{\lambda \nu} + \frac{1}{F'} \eta ^{a b} \partial _{a} F' \partial _{b} X^{\nu} G_{\lambda \nu} = \eta ^{a b} \partial _{a} X^{\mu} \partial _{b} X^{\nu} \partial _{\lambda} G_{\mu \nu} ,
\end{equation*}
assuming flat background $G_{\mu \nu} = \eta _{\mu \nu}$
\begin{equation*}
    \eta ^{a b} \partial _{a} \partial _{b} X^{\nu} \eta _{\lambda \nu} + \frac{1}{F'} \eta ^{a b} \partial _{a} F' \partial _{b} X^{\nu} \eta _{\lambda \nu} = 0
\end{equation*}
\begin{equation*}
    \eta ^{a b} \partial _{a} \partial _{b} X^{\mu} + \eta ^{a b} \frac{k \Delta \e ^{- 2 \phi} \partial _{a} \phi}{1 - \dfrac{k \Delta \e ^{- 2 \phi}}{2}} \partial _{b} X^{\mu} - \eta ^{a b} \frac{1}{\left( 1 - \dfrac{k \Delta}{2 \e ^{2 \phi}} \right)} \frac{k \partial _{a} \Delta}{2 \e ^{2 \phi}} \del _{b} X^{\mu} = 0
\end{equation*}
\begin{equation*}
    \eta ^{a b} \partial _{a} \partial _{b} X^{\mu} + \eta ^{a b} \frac{k \Delta \partial _{a} \phi}{\e ^{2 \phi} - \dfrac{k \Delta}{2}} \partial _{b} X^{\mu} - \eta ^{a b} \frac{\partial _{a} \Delta}{\dfrac{2 \e ^{2 \phi}}{k} - \Delta} \del _{b} X^{\mu} = 0
\end{equation*}
\begin{equation*}
    \eta ^{a b} \partial _{a} \partial _{b} X^{\mu} + \eta ^{a b} \frac{\left( 2 \del _{a} \phi - \del _{a} \ln (\Delta) \right)}{\dfrac{2 \e ^{2 \phi}}{k \Delta} - 1} \del _{b} X^{\mu} = 0
\end{equation*}
\begin{equation*}
    \eta ^{a b} \partial _{a} \partial _{b} X^{\mu} + \eta ^{a b} \frac{\del _{a} \left( 2 \phi - \ln (\Delta) \right)}{\dfrac{2 \e ^{2 \phi}}{k \Delta} - 1} \del _{b} X^{\mu} = 0
\end{equation*}
\begin{equation*}
    \eta ^{a b} \partial _{a} \partial _{b} X^{\mu} + \eta ^{a b} \frac{\del _{a} \left( \ln (\e ^{2 \phi}) - \ln (\Delta) \right)}{\dfrac{2 \e ^{2 \phi}}{k \Delta} - 1} \del _{b} X^{\mu} = 0
\end{equation*}
\begin{equation*}
    \eta ^{a b} \partial _{a} \partial _{b} X^{\mu} + \eta ^{a b} \frac{\del _{a} \ln \left( \e ^{2 \phi} / \Delta \right)}{\dfrac{2 \e ^{2 \phi}}{k \Delta} - 1} \del _{b} X^{\mu} = 0
\end{equation*}
This reveals a wave eqn sourced by a coupling between the conformal factor $\phi$, quantum of area field $\Delta$ and the tangent vectors to the string, which could hint towards a connection between the conformal factor and the dilaton field... perhaps $\phi (x)$ can be viewed as a \textquotedblleft WS dilaton"? Perhaps, relating to the analysis of emergent dimensions in next section, the conformal factor is a WS dilaton which sets the string scale, which in turn dictates how the extra massive dimensions are hidden since they are inaccessible until they become effectively massless, but the WS still sees them all.

Going back to the equation before plugging in the definition of $F'$, we can simplify it as
\begin{equation*}
    \eta ^{a b} \partial _{a} \partial _{b} X^{\mu} + \eta ^{a b} \partial _{a} \ln (F') \partial _{b} X^{\mu} = 0 ,
\end{equation*}
then use plane-wave ansatz $X^{\mu} = X^{\mu} _{0} \e ^{- i (E \tau - p \sigma)}$
\begin{equation*}
    (E^{2} - p^{2}) X^{\mu} + \partial _{\tau} \ln (F') i E X^{\mu} - \partial _{\sigma} \ln (F') i p X^{\mu} = 0
\end{equation*}
\begin{equation*}
    \downarrow
\end{equation*}
\begin{equation*}
    (E^{2} - p^{2}) X^{\mu} + i (E \partial _{\tau} \ln (F') - p \partial _{\sigma} \ln (F')) X^{\mu} = 0
\end{equation*}
\begin{equation*}
    \downarrow
\end{equation*}
\begin{equation*}
    i (E \partial _{\tau} \ln (F') - p \partial _{\sigma} \ln (F')) = - m^2 ,
\end{equation*}
\begin{equation*}
    - m^{2} = - E^{2} + p^{2}
\end{equation*}
\begin{equation*}
    \Downarrow
\end{equation*}
\begin{equation*}
    \partial _{\tau} \ln (F') = i E , \ \partial _{\sigma} \ln (F') = i p
\end{equation*}
\begin{equation*}
    \Downarrow
\end{equation*}
\begin{equation*}
    \ln (F') = i (E \tau + p \sigma) + a , \ a \in \mathbb{C}
\end{equation*}
\begin{equation*}
    \Downarrow
\end{equation*}
\begin{equation*}
    F' = A \e ^{i (E \tau + p \sigma)} , \ A \in \C
\end{equation*}
\begin{equation*}
    \Downarrow
\end{equation*}
\begin{equation*}
    \e ^{\phi (x)} = \sqrt{\frac{k \Delta}{2 \left( 1 - A \e ^{i (E \tau + p \sigma)} \right)}}
\end{equation*}
\begin{equation*}
    \downarrow
\end{equation*}
\begin{equation*}
    (\partial _{\sigma} ^{2} - \partial _{\tau} ^{2}) X^{\mu} - m^{2} X^{\mu} = 0 .
\end{equation*}
The imposition
\begin{equation*}
    i (E \partial _{\tau} \ln (F') - p \partial _{\sigma} \ln (F')) = - m^2
\end{equation*}
gives us a way to relate the mass $m$ directly to the discrete geometry parameter $k \Delta$:
\begin{equation*}
    \frac{i}{F'} (E \partial _{\tau} F' - p \partial _{\sigma} F') = - m^2
\end{equation*}
\begin{equation*}
    \frac{i}{\dfrac{2 \e ^{2 \phi}}{k \Delta} - 1} \left( E \del _{\tau} (2 \phi - \ln (\Delta)) - p \del _{\sigma} (2 \phi - \ln (\Delta)) \right) = - m^2
\end{equation*}
\begin{equation*}
    \frac{i}{\dfrac{2 \e ^{2 \phi}}{k \Delta} - 1} \left( p \del _{\sigma} \ln \left( \e^{2 \phi} / \Delta \right) - E \del _{\tau} \ln \left( \e^{2 \phi} / \Delta \right) \right) = m^2 .
\end{equation*}

\newpage

\section{Emergent dimensions through LQG-strings}

Since quantum geometry is expected to violate Lorentz invariance at the Planck scale by modifying the dispersion relation as in
\begin{equation*}
    E^{2} = m^{2} + p^{2} + \sum _{n \geq 3} c_{n} \frac{p^{n}}{E_{pl} ^{n - 2}} ,
\end{equation*}
where $E_{pl}$ is the Planck energy, we make a correction to Polyakov action in the form
\begin{equation*}
    S = - \frac{1}{4 \pi \alpha '} \int \diff ^2 x \sqrt{- g} \left( g^{a b} \del _{a} X^{\mu} \del _{b} X^{\nu} \eta _{\mu \nu} + \mu _{0} ^{2} \e ^{- m_{\mu} / T} X^{\mu} X^{\nu} \eta _{\mu \nu} \right)
\end{equation*}
where $T$ is the temperature of target-space, (in SI units) $\mu _{0} ^2$ has units of inverse area (perhaps $\mu _{0} ^{2} \sim (\mathrm{quantum \ of}$ $\mathrm{area})^{-1}$? Maybe $\mu _{0} ^{2} = \mu ^{2} / (M_{pl} ^{2} A_{j})$, where $A_{j}$ is the quantum of area with spin $j$) and $m_{0} >> m_{1} >> ... >> m_{d}$. This action has manifest WS reparameterisation invariance. As for background Poincaré symmetry and WS Weyl invariance, those hold approximately for the fields with $m_{\mu} >> T$. The masses $m_{\mu}$ are constrained by the dispersion relations
\begin{equation*}
    \widetilde{E}_{\alpha} ^{2} = \widetilde{p}_{\alpha} ^2 + \sum _{n = 3} ^{d} \frac{m_{n}}{T} \frac{\widetilde{p}_{\alpha} ^{n}}{E_{pl} ^{n - 2}}
\end{equation*}
where $\alpha = 3 , 4 , ... , d$ and $\widetilde{E}_{\alpha} ^2$ ($\widetilde{p}_{\alpha} ^2$) does not mean $\widetilde{E}^{\alpha} \widetilde{E}_{\alpha}$ ($\widetilde{p}^{\alpha} \widetilde{p}_{\alpha}$), rather it is $\widetilde{E}_{3} ^{2} (\widetilde{p}_{3} ^2) , \widetilde{E}_{4} ^2 (\widetilde{p}_{4} ^2) , ... , \widetilde{E}_{d} ^2 (\widetilde{p}_{d} ^2)$, that is, the squared WS energy (momentum) of each field $X^{\mu}$. Since $m_0 , m_1$ and $m_2$ remain unconstrained after quantum geometry is considered, this might mean that the true quantum gravitational field is actually $(2 + 1)$-dimensional, since precisely the masses of $X^{0} , X^{1}$ and $X^{2}$ remain unconstrained,  such that those may be regarded as \textquotedblleft fundamental" dimensions, while the other dimensions have their masses constrained by the modified dispersion relations.

\subsection{EoMs}

\begin{align*}
    \frac{\delta S}{\delta g^{c d}} & \propto \frac{\del \sqrt{- g}}{\del g^{c d}} \left( g^{a b} \del _{a} X^{\mu} \del _{b} X^{\nu} \eta _{\mu \nu} + \mu _{0} ^{2} \e ^{- m_{\mu} / T} X^{\mu} X^{\nu} \eta _{\mu \nu} \right) + \sqrt{- g} \delta ^{a} _{c} \delta ^{b} _{d} \del _{a} X^{\mu} \del _{b} X^{\nu} \eta _{\mu \nu} = \\
                                    & = - \frac{1}{2} \sqrt{- g} g_{c d} \left( g^{a b} \del _{a} X^{\mu} \del _{b} X^{\nu} \eta _{\mu \nu} + \mu _{0} ^{2} \e ^{- m_{\mu} / T} X^{\mu} X^{\nu} \eta _{\mu \nu} \right) + \sqrt{- g} \del _{c} X^{\mu} \del _{d} X^{\nu} \eta _{\mu \nu} = 0
\end{align*}
\begin{equation*}
    T_{c d} := \del _{c} X^{\mu} \del _{d} X^{\nu} \eta _{\mu \nu} - \frac{1}{2} g_{c d} \left( g^{a b} \del _{a} X^{\mu} \del _{b} X^{\nu} \eta _{\mu \nu} + \mu _{0} ^{2} \e ^{- m_{\mu} / T} X^{\mu} X^{\nu} \eta _{\mu \nu} \right) = 0
\end{equation*}
\begin{equation*}
    g_{c d} = 2 f \del _{c} X^{\mu} \del _{d} X^{\nu} \eta _{\mu \nu} ,
\end{equation*}
\begin{equation*}
    \frac{1}{f} = \left( g^{a b} \del _{a} X^{\mu} \del _{b} X^{\nu} \eta _{\mu \nu} + \mu _{0} ^{2} \e ^{- m_{\mu} / T} X^{\mu} X^{\nu} \eta _{\mu \nu} \right) .
\end{equation*}
From this, it is clear that $\e ^{- m_{\mu} / T} X^{\mu} X^{\nu} \eta _{\mu \nu} = 0$, thus the base mass $\mu _{0} ^2$ is a Lagrange multiplier, the auxiliary metric reduces to the induced metric and the action simplify back to regular NG action.

\begin{align*}
    \frac{\delta S}{\delta X^{\lambda}} & \propto + 2 \sqrt{- g} \mu _{0} ^2 \e ^{- m_{\mu} / T} \delta ^{\mu} _{\lambda} X^{\nu} \eta _{\mu \nu} - 2 \del _{c} \left( \sqrt{- g} g^{a b} \delta ^{c} _{a} \delta ^{\mu} _{\lambda} \del _{b} X^{\nu} \eta _{\mu \nu} \right) = \\
                                        & = - 2 \left( \del _{a} \left( \sqrt{- g} g^{a b} \del _{b} X^{\nu} \eta _{\lambda \nu} \right) - \sqrt{- g} \mu _{0} ^2 \e ^{- m_{\lambda} / T} X^{\nu} \eta _{\lambda \nu} \right) = 0
\end{align*}
\begin{equation*}
    \del _{a} \left( \sqrt{- g} g^{a b} \del _{b} X^{\mu} \right) - \sqrt{- g} \mu _{0} ^2 \e ^{- m_{\mu} / T} X^{\mu} = 0
\end{equation*}
\begin{equation*}
    \frac{1}{\sqrt{- g}} \del _{a} \left( \sqrt{- g} g^{a b} \del _{b} X^{\mu} \right) - \mu _{0} ^2 \e ^{- m_{\mu} / T} X^{\mu} = 0
\end{equation*}
\begin{equation*}
    g^{a b} \nabla _{a} \nabla _{b} X^{\mu} - \mu _{0} ^2 \e ^{- m_{\mu} / T} X^{\mu} = 0 .
\end{equation*}

From the $X$ EoM before simplifying, we can use conformal gauge $g_{a b} = \e ^{\phi (x)} \eta _{a b}$ to get a variable-coefficient KG eqn
\begin{equation*}
    \eta ^{a b} \del _{a} \del _{b} X^{\mu} - \e ^{\phi (x)} \mu _{0} ^{2} \e ^{- m_{\mu} / T} X^{\mu} = 0
\end{equation*}
\begin{equation*}
    ((\del _{\sigma})^2 - (\del _{\tau})^2) X^{\mu} - \mu _{0} ^{2} \e ^{\phi _{\mu} (x)} X^{\mu} = 0 ,
\end{equation*}
where $\phi _{\mu} (x) \equiv \phi (x) - m_{\mu} / T$.

\newpage

\section{Relating NG and Polyakov analysis}

From NG analysis
\begin{equation*}
    (F^{+})^2 \frac{\partial ^2 X^{\mu}}{\partial t^2} - \frac{\partial ^2 X^{\mu}}{\partial x^2} = 0 ,
\end{equation*}
make change of variables $\tau = \tau (t , x)$ and $\sigma = \sigma (t , x)$
\begin{equation*}
    (F^{+})^2 \left( \frac{\partial \tau}{\partial t} \frac{\partial}{\partial \tau} + \frac{\partial \sigma}{\partial t} \frac{\partial}{\partial \sigma} \right) \left( \frac{\partial \tau}{\partial t} \frac{\partial X^{\mu}}{\partial \tau} + \frac{\partial \sigma}{\partial t} \frac{\partial X^{\mu}}{\partial \sigma} \right) - \left( \frac{\partial \tau}{\partial x} \frac{\partial}{\partial \tau} + \frac{\partial \sigma}{\partial x} \frac{\partial}{\partial \sigma} \right) \left( \frac{\partial \tau}{\partial x} \frac{\partial X^{\mu}}{\partial \tau} + \frac{\partial \sigma}{\partial x} \frac{\partial X^{\mu}}{\partial \sigma} \right) = 0
\end{equation*}
\begin{multline*}
    (F^{+})^2 \left( \left( \frac{\partial \tau}{\partial t} \right) ^2 \frac{\partial ^2 X^{\mu}}{\partial \tau ^2} + 2 \frac{\partial \tau}{\partial t} \frac{\partial \sigma}{\partial t} \frac{\partial ^2 X^{\mu}}{\partial \tau \partial \sigma} + \left( \frac{\partial \sigma}{\partial t} \right) ^2 \frac{\partial ^2 X^{\mu}}{\partial \sigma ^2} \right) - \\
    - \left( \left( \frac{\partial \tau}{\partial x} \right) ^2 \frac{\partial ^2 X^{\mu}}{\partial \tau ^2} + 2 \frac{\partial \tau}{\partial x} \frac{\partial \sigma}{\partial x} \frac{\partial ^2 X^{\mu}}{\partial \tau \partial \sigma} + \left( \frac{\partial \sigma}{\partial x} \right) ^2 \frac{\partial ^2 X^{\mu}}{\partial \sigma ^2} \right) + \\
    + \left( (F^{+})^2 \frac{\partial ^2 \tau}{\partial t^2} - \frac{\partial ^2 \tau}{\partial x^2} \right) \frac{\partial X^{\mu}}{\partial \tau} + \left( (F^{+})^2 \frac{\partial ^2 \sigma}{\partial t^2} - \frac{\partial ^2 \sigma}{\partial x^2} \right) \frac{\partial X^{\mu}}{\partial \sigma} = 0
\end{multline*}
\begin{multline*}
    \left( (F^{+})^2 (\partial _{t} \tau)^2 - (\partial _{x} \tau)^2 \right) \partial _{\tau} ^2 X^{\mu} + \left( (F^{+})^2 (\partial _{t} \sigma)^2 - (\partial _{x} \sigma)^2 \right) \partial _{\sigma} ^2 X^{\mu} + \\
    + 2 \left( (F^{+})^2 \partial _{t} \tau \partial _{t} \sigma - \partial _{x} \tau \partial _{x} \sigma \right) \partial _{\tau} \partial _{\sigma} X^{\mu} + \left( (F^{+})^2 \partial _{t} ^2 \tau - \partial _{x} ^2 \tau \right) \partial _{\tau} X^{\mu} + \left( (F^{+})^2 \partial _{t} ^2 \sigma - \partial _{x} ^2 \sigma \right) \partial _{\sigma} X^{\mu} = 0
\end{multline*}
Without loss of generality, we can set (just rescale/rotate the coordinates)
\begin{align*}
    (F^{+})^2 (\partial _t \sigma)^2 - (\partial _x \sigma)^2 & = (\partial _x \tau)^2 - (F^{+})^2 (\partial _t \tau)^2 \\
    (F^{+})^2 \partial _t \tau \partial _t \sigma             & = \partial _x \tau \partial _x \sigma ,
\end{align*}
or more compactly
\begin{align*}
    \partial _{t} \tau   & = v \partial _{x} \sigma \\
    \partial _{t} \sigma & = v \partial _{x} \tau ,
\end{align*}
where $v = 1 / F^{+}$, which is equivalent to
\begin{align*}
    \partial _{t} ^2 \tau   & = \partial _{t} v \partial _{x} \sigma + v \partial _{x} v \partial _{x} \tau + v^2 \partial _{x} ^2 \tau      \\
    \partial _{t} ^2 \sigma & = \partial _{t} v \partial _{x} \tau + v \partial _{x} v \partial _{x} \sigma + v ^2 \partial _{x} ^2 \sigma ,
\end{align*}
reducing the big equation to
\begin{equation*}
    \partial _{\tau} ^2 X^{\mu} - \partial _{\sigma} ^2 X^{\mu} + \frac{\partial _{t} v \partial _{x} \sigma + v \partial _{x} v \partial _{x} \tau}{\left( (\partial _{t} \tau)^2 - v^2 (\partial _{x} \tau)^2 \right)} \partial _{\tau} X^{\mu} + \frac{\partial _{t} v \partial _{x} \tau + v \partial _{x} v \partial _{x} \sigma}{\left( (\partial _{t} \tau)^2 - v^2 (\partial _{x} \tau)^2 \right)} \partial _{\sigma} X^{\mu} = 0 .
\end{equation*}
Next introduce $X^{\mu} = \kappa (\tau , \sigma) Y^{\mu} (\tau , \sigma)$ s.t terms proportional to $\partial _{\tau} Y^{\mu}$ and $\partial _{\sigma} Y^{\mu}$ vanish:
\begin{align*}
    \partial _{\tau} X^{\mu}      & = \partial _{\tau} \kappa Y^{\mu} + \kappa \partial _{\tau} Y^{\mu}                                                                    \\
    \partial _{\sigma} X^{\mu}    & = \partial _{\sigma} \kappa Y^{\mu} + \kappa \partial _{\sigma} Y^{\mu}                                                                \\
    \partial _{\tau} ^2 X^{\mu}   & = \partial _{\tau} ^2 \kappa Y^{\mu} + 2 \partial _{\tau} \kappa \partial _{\tau} Y^{\mu} + \kappa \partial _{\tau} ^2 Y^{\mu}         \\
    \partial _{\sigma} ^2 X^{\mu} & = \partial _{\sigma} ^2 \kappa Y^{\mu} + 2 \partial _{\sigma} \kappa \partial _{\sigma} Y^{\mu} + \kappa \partial _{\sigma} ^2 Y^{\mu}
\end{align*}
\begin{align*}
    \kappa \frac{\partial _{t} v \partial _{x} \sigma + v \partial _{x} v \partial _{x} \tau}{\left( (\partial _{t} \tau)^2 - v^2 (\partial _{x} \tau)^2 \right)} + 2 \partial _{\tau} \kappa   & \overset{!}{=} 0 \\
    \kappa \frac{\partial _{t} v \partial _{x} \tau + v \partial _{x} v \partial _{x} \sigma}{\left( (\partial _{t} \tau)^2 - v^2 (\partial _{x} \tau)^2 \right)} - 2 \partial _{\sigma} \kappa & \overset{!}{=} 0
\end{align*}
\begin{align*}
    \partial _{\tau} \kappa   & = - \frac{\kappa (\partial _{t} v \partial _{x} \sigma + v \partial _{x} v \partial _{x} \tau)}{2 \left( (\partial _{t} \tau)^2 - v^2 (\partial _{x} \tau)^2 \right)} \\
    \partial _{\sigma} \kappa & = \frac{\kappa (\partial _{t} v \partial _{x} \tau + v \partial _{x} v \partial _{x} \sigma)}{2 \left( (\partial _{t} \tau)^2 - v^2 (\partial _{x} \tau)^2 \right)} ,
\end{align*}
undoing chain rule for $t$ by multiplying first eq by $\partial _{t} \tau$ and second by $\partial _{t} \sigma$ and summing gives
\begin{align*}
    \partial _{t} \kappa & = \partial _{t} \tau \partial _{\tau} \kappa + \partial _{t} \sigma \partial _{\sigma} \kappa =                                                                                                                                                                                                                                                                                                                        \\
                         & = \partial _{t} \tau \left( - \frac{\kappa (\partial _{t} v \partial _{x} \sigma + v \partial _{x} v \partial _{x} \tau)}{2 \left( (\partial _{t} \tau)^2 - v^2 (\partial _{x} \tau)^2 \right)} \right) + \partial _{t} \sigma \left( \frac{\kappa (\partial _{t} v \partial _{x} \tau + v \partial _{x} v \partial _{x} \sigma)}{2 \left( (\partial _{t} \tau)^2 - v^2 (\partial _{x} \tau)^2 \right)}  \right) =     \\
                         & = v \partial _{x} \sigma \left( - \frac{\kappa (\partial _{t} v \partial _{x} \sigma + v \partial _{x} v \partial _{x} \tau)}{2 \left( (\partial _{t} \tau)^2 - v^2 (\partial _{x} \tau)^2 \right)} \right) + v \partial _{x} \tau \left( \frac{\kappa (\partial _{t} v \partial _{x} \tau + v \partial _{x} v \partial _{x} \sigma)}{2 \left( (\partial _{t} \tau)^2 - v^2 (\partial _{x} \tau)^2 \right)}  \right) = \\
                         & = \frac{v \kappa \partial _{t} v \left( (\partial _{x} \tau)^2 - (\partial _{x} \sigma)^2 \right)}{2 \left( (\partial _{t} \tau)^2 - v^2 (\partial _{x} \tau)^2 \right)} =                                                                                                                                                                                                                                             \\
                         & = \frac{v \kappa \partial _{t} v \left( (\partial _{x} \tau)^2 - (\partial _{t} \tau / v)^2 \right)}{2 \left( (\partial _{t} \tau)^2 - v^2 (\partial _{x} \tau)^2 \right)} =                                                                                                                                                                                                                                           \\
                         & = - \frac{\kappa \partial _{t} v}{2 v} ,
\end{align*}
or simply
\begin{equation*}
    \frac{\partial _{t} \kappa}{\kappa} = - \frac{1}{2} \frac{\partial _{t} v}{v}
\end{equation*}
\begin{equation*}
    \partial _{t} \ln (\kappa) = - \frac{1}{2} \partial _{t} \ln (v)
\end{equation*}
\begin{equation*}
    \ln (\kappa) = - \frac{1}{2} \ln (v) + f(x) .
\end{equation*}
To determine $f(x)$, we undo the chain rule for $x$ now:
\begin{align*}
    \partial _{x} \kappa & = \partial _{x} \tau \partial _{\tau} \kappa + \partial _{x} \sigma \partial _{\sigma} \kappa =                                                                                                                                                                                                                                                                                                                    \\
                         & = \partial _{x} \tau \left( - \frac{\kappa (\partial _{t} v \partial _{x} \sigma + v \partial _{x} v \partial _{x} \tau)}{2 \left( (\partial _{t} \tau)^2 - v^2 (\partial _{x} \tau)^2 \right)} \right) + \partial _{x} \sigma \left( \frac{\kappa (\partial _{t} v \partial _{x} \tau + v \partial _{x} v \partial _{x} \sigma)}{2 \left( (\partial _{t} \tau)^2 - v^2 (\partial _{x} \tau)^2 \right)}  \right) = \\
                         & = \frac{\kappa v \partial _{x} v \left( (\partial _{x} \sigma)^2 - (\partial _{x} \tau)^2 \right)}{2 \left( (\partial _{t} \tau)^2 - v^2 (\partial _{x} \tau)^2 \right)} =                                                                                                                                                                                                                                         \\
                         & = \frac{\kappa v \partial _{x} v \left( (\partial _{t} \tau / v)^2 - (\partial _{x} \tau)^2 \right)}{2 \left( (\partial _{t} \tau)^2 - v^2 (\partial _{x} \tau)^2 \right)} =                                                                                                                                                                                                                                       \\
                         & = \frac{\kappa \partial _{x} v}{2 v}
\end{align*}
\begin{equation*}
    \frac{\partial _{x} \kappa}{ \kappa} = \frac{1}{2} \frac{\partial _{x} v}{v}
\end{equation*}
\begin{equation*}
    \partial _{x} \ln (\kappa) = \frac{1}{2} \partial _{x} \ln (v)
\end{equation*}
\begin{equation*}
    \ln (\kappa) = \frac{1}{2} \ln (v) ,
\end{equation*}
\begin{equation*}
    \partial _{x} \left( - \frac{1}{2} \ln (v) + f(x) \right) = \frac{1}{2} \partial _{x} \ln (v)
\end{equation*}
\begin{equation*}
    f' (x) = \partial _{x} \ln (v) \ \mathrm{(perhaps \ this \ means \ somewhere \ in \ the \ composition \ of \ functions \ time \ dependence \ is \ lost?)}
\end{equation*}
\begin{equation*}
    f (x) = \ln (v)
\end{equation*}
giving us
\begin{equation*}
    \kappa = v^{\frac{1}{2}} , \ v = v (x) .
\end{equation*}
Substituting this into the earlier equation yields
\begin{equation*}
    \partial _{\tau} ^2 Y^{\mu} - \partial _{\sigma} ^2 Y^{\mu} + m^2 Y^{\mu} = 0 ,
\end{equation*}
with
\begin{align*}
    m^2 & = \frac{\partial _{\tau} ^2 \kappa - \partial _{\sigma} ^2 \kappa}{\kappa} + \frac{(\partial _{t} v \partial _{x} \sigma + v \partial _{x} v \partial _{x} \tau) \partial _{\tau} \kappa + (\partial _{t} v \partial _{x} \tau + v \partial _{x} v \partial _{x} \sigma) \partial _{\sigma} \kappa}{\kappa \left( (\partial _{t} \tau)^2 - v^2 (\partial _{x} \tau)^2 \right)} = \\
        & = \frac{\partial _{\tau} ^2 \kappa - \partial _{\sigma} ^2 \kappa}{\kappa} + \frac{\partial _{t} v (\partial _{x} \sigma \partial _{\tau} \kappa + \partial _{x} \tau \partial _{\sigma} \kappa) + v \partial _{x} v \partial _{x} \kappa}{\kappa \left( (\partial _{t} \tau)^2 - v^2 (\partial _{x} \tau)^2 \right)} =                                                          \\
        & = \frac{\partial _{\tau} ^2 \kappa - \partial _{\sigma} ^2 \kappa}{\kappa} + \frac{(\partial _{x} v)^2}{\left( (\partial _{t} \tau)^2 - v^2 (\partial _{x} \tau)^2 \right)} ,
\end{align*}
where by use of chain rule we have that
\begin{equation*}
    \partial _{t} ^2 \kappa - v^2 \partial _{x} ^2 \kappa - v \partial _{x} v \partial _{x} \kappa = ((\partial _{t} \tau)^2 - v^2 (\partial _{x} \tau)^2) (\partial _{\tau} ^2 \kappa - \partial _{\sigma} ^2 \kappa) ,
\end{equation*}
where since $\kappa = v^{\frac{1}{2}}$, $m^2$ reduces to
\begin{equation*}
    m^2 = \frac{(\partial _{x} v)^2 - 2 v \partial _{x} ^2 v}{4 \left( (\partial _{t} \tau)^2 - v^2 (\partial _{x} \tau)^2 \right)} ,
\end{equation*}
with $\tau$ satisfying
\begin{align*}
    \partial _{t} \tau   & = v \partial _{x} \sigma \\
    v \partial _{x} \tau & = \partial _{t} \sigma
\end{align*}
such that $- m^2$ is indeed constant.

\newpage

\section{Solutions for the Polyakov KG eqn (WRONG ONE)}

From the inverse-area corrected Polyakov action we have the equation
\begin{equation*}
    (\partial _{\sigma} ^2 - \partial _{\tau} ^2) X^{\mu} - m^2 X^{\mu} = 0
\end{equation*}
subjected to the constraint
\begin{equation*}
    \partial _{c} X^{\mu} \partial _{d} X^{\nu} \eta _{\mu \nu} - \frac{1}{2} g_{c d} g^{a b} \partial _{a} X^{\mu} \partial _{b} X^{\nu} \eta _{\mu \nu} = 0 ,
\end{equation*}
which with conformal gauge $g_{a b} = \phi \eta _{a b}$ becomes
\begin{equation*}
    \partial _{c} X^{\mu} \partial _{d} X^{\nu} \eta_{\mu \nu} = \frac{1}{2} \eta _{c d} \eta ^{a b} \partial _{a} X^{\mu} \partial _{b} X^{\nu} \eta _{\mu \nu}
\end{equation*}
or more explicitly,
\begin{align*}
    \partial _{\tau} X^{\mu} \partial _{\tau} X^{\nu} \eta _{\mu \nu}     & = - \frac{1}{2} \eta ^{a b} \partial _{a} X^{\mu} \partial _{b} X^{\nu} \eta _{\mu \nu} \\
    \partial _{\sigma} X^{\mu} \partial _{\sigma} X^{\nu} \eta _{\mu \nu} & = \frac{1}{2} \eta ^{a b} \partial _{a} X^{\mu} \partial _{b} X^{\nu} \eta _{\mu \nu}   \\
    \partial _{\tau} X^{\mu} \partial _{\sigma} X^{\nu} \eta_{\mu \nu}    & = \partial _{\sigma} X^{\mu} \partial _{\tau} X^{\nu} \eta _{\mu \nu} = 0 .
\end{align*}
These simplify further by expanding the summation on $a,b$:
\begin{equation*}
    \partial _{\tau} X^{\mu} \partial _{\tau} X^{\nu} \eta _{\mu \nu} = - \frac{1}{2} \left( - \partial _{\tau} X^{\mu} \partial _{\tau} X^{\nu} \eta _{\mu \nu} + \partial _{\sigma} X^{\mu} \partial _{\sigma} X^{\nu} \eta _{\mu \nu} \right)
\end{equation*}
\begin{equation*}
    \partial _{\tau} X^{\mu} \partial _{\tau} X^{\nu} \eta _{\mu \nu} = - \partial _{\sigma} X^{\mu} \partial _{\sigma} X^{\nu} \eta _{\mu \nu} .
\end{equation*}
The EoM is just a one-dimensional Klein-Gordon equation, which has general solution given by Fourier transform
\begin{equation*}
    X^{\mu} (\tau , \sigma) = \frac{1}{2 \pi} \int \diff p \frac{1}{2 E (p)} \left( a^{\mu} (p) \e ^{- i (- E \tau + p \sigma)} + b^{\mu} (p) \e ^{i (- E \tau + p \sigma)} \right) ,
\end{equation*}
where $E (p) = \sqrt{p^2 + m^2}$. The derivatives are
\begin{align*}
    \partial _{\tau} X^{\mu}   & = \frac{i}{4 \pi} \int \diff p \left( a^{\mu} (p) \e^{- i (- E \tau + p \sigma)} - b^{\mu} (p) \e ^{i (- E \tau + p \sigma)} \right)                                \\
    \partial _{\sigma} X^{\mu} & = \frac{i}{4 \pi} \int \diff p \frac{p}{\sqrt{p^2 + m^2}} \left( - a^{\mu} (p) \e ^{-i (- E \tau + p \sigma)} + b^{\mu} (p) \e ^{i (- E \tau + p \sigma)} \right) .
\end{align*}
Starting with the mixed constraint,
\begin{multline*}
    \partial _{\tau} X^{\mu} \partial _{\sigma} X^{\nu} \eta _{\mu \nu} = - \frac{\eta _{\mu \nu}}{(4 \pi)^2} \int \diff p \left( a^{\mu} (p) \e^{- i (- E (p) \tau + p \sigma)} - b^{\mu} (p) \e ^{i (- E (p) \tau + p \sigma)} \right) \times \\
    \times \int \diff p' \frac{p'}{\sqrt{p^{\prime 2} + m^2}} \left( - a^{\nu} (p') \e ^{-i (- E (p') \tau + p' \sigma)} + b^{\nu} (p') \e ^{i (- E (p') \tau + p' \sigma)} \right) =
\end{multline*}
\begin{multline*}
    = - \frac{\eta _{\mu \nu}}{(4 \pi)^2} \int \diff p \diff p' \frac{p'}{\sqrt{p^{\prime 2} + m^2}} \left( a^{\mu} (p) \e^{- i (- E (p) \tau + p \sigma)} - b^{\mu} (p) \e ^{i (- E (p) \tau + p \sigma)} \right) \times \\
    \times \left( - a^{\nu} (p') \e ^{-i (- E (p') \tau + p' \sigma)} + b^{\nu} (p') \e ^{i (- E (p') \tau + p' \sigma)} \right) =
\end{multline*}
\begin{multline*}
    = - \frac{\eta _{\mu \nu}}{(4 \pi)^2} \int \diff p \diff p' \frac{p'}{\sqrt{p^{\prime 2} + m^2}} \Big( - a^{\mu} (p) a^{\nu} (p') \e ^{i (E (p) + E (p')) \tau} \e ^{- i (p + p') \sigma} + a^{\mu} (p) b^{\nu} (p') \e ^{i (E (p) - E (p')) \tau} \e ^{- i (p - p') \sigma} + \\
    + b^{\mu} (p) a^{\nu} (p') \e ^{- i (E (p) - E (p')) \tau} \e ^{i (p - p') \sigma} - b^{\mu} (p) b^{\nu} (p') \e ^{- i (E (p) + E (p')) \tau} \e ^{i (p + p') \sigma} \Big) = 0 .
\end{multline*}
If we integrate this expression over $\sigma$ we obtain
\begin{multline*}
    - \frac{\eta _{\mu \nu}}{(4 \pi)^2} \int \diff p \diff p' \frac{p'}{\sqrt{p^{\prime 2} + m^2}} \Big( - a^{\mu} (p) a^{\nu} (p') \e ^{i (E (p) + E (p')) \tau} \int \diff \sigma \e ^{-i (p + p') \sigma} + \\
    + a^{\mu} (p) b^{\nu} (p') \e ^{i (E (p) - E (p')) \tau} \int \diff \sigma \e ^{-i (p - p') \sigma} + b^{\mu} (p) a^{\nu} (p') \e ^{- i (E (p) - E (p')) \tau} \int \diff \sigma \e ^{i (p - p') \sigma} - \\
    - b^{\mu} (p) b^{\nu} (p') \e ^{- i (E (p) + E (p')) \tau} \int \diff \sigma \e ^{i (p + p') \sigma} \Big) =
\end{multline*}
\begin{multline*}
    = - \frac{\eta _{\mu \nu}}{8 \pi} \int \diff p \diff p' \frac{p'}{\sqrt{p^{\prime 2} + m^2}} \Big( - a^{\mu} (p) a^{\nu} (p') \e ^{i (E (p) + E (p')) \tau} \delta (p + p') + a^{\mu} (p) b^{\nu} (p') \e ^{i (E (p) - E (p')) \tau} \delta (p - p') + \\
    + b^{\mu} (p) a^{\nu} (p') \e ^{- i (E (p) - E (p')) \tau} \delta (p - p') - b^{\mu} (p) b^{\nu} (p') \e ^{- i (E (p) + E (p')) \tau} \delta (p + p') \Big) =
\end{multline*}
\begin{equation*}
    = - \frac{\eta _{\mu \nu}}{8 \pi} \int \diff p \frac{p}{\sqrt{p^2 + m^2}} \Big( a^{\mu} (p) a^{\nu} (- p) \e ^{2 i E (p) \tau} + a^{\mu} (p) b^{\nu} (p) + b^{\mu} (p) a^{\nu} (p) + b^{\mu} (p) b^{\nu} (- p) \e ^{- 2 i E (p) \tau} \Big) = 0
\end{equation*}
which implies the whole expression inside the integral must vanish, thus we have (leaving $(a^{\mu} b^{\nu} + b^{\mu} a^{\nu}) \eta _{\mu \nu}$ as is since eventually these will not be numbers and may not commute, thus will not be equal to $2 a^{\mu} b^{\nu} \eta _{\mu \nu}$)
\begin{equation*}
    \Big( a^{\mu} (p) a^{\nu} (- p) \e ^{2 i E (p) \tau} + a^{\mu} (p) b^{\nu} (p) + b^{\mu} (p) a^{\nu} (p) + b^{\mu} (p) b^{\nu} (- p) \e ^{- 2 i E (p) \tau} \Big) \eta _{\mu \nu} = 0 .
\end{equation*}

As for the other part of the constraint, let's start with the L.H.S
\begin{multline*}
    \partial _{\tau} X^{\mu} \partial _{\tau} X^{\nu} \eta _{\mu \nu} = - \frac{1}{(4 \pi)^2} \int \diff p \diff p' \Big( a^{\mu} (p) \e^{- i (- E (p) \tau + p \sigma)} - b^{\mu} (p) \e ^{i (- E (p) \tau + p \sigma)} \Big) \times \\
    \times \Big( a^{\nu} (p') \e^{- i (- E (p') \tau + p' \sigma)} - b^{\nu} (p') \e ^{i (- E (p') \tau + p' \sigma)} \Big) =
\end{multline*}
\begin{multline*}
    = - \frac{\eta _{\mu \nu}}{(4 \pi)^2} \int \diff p \diff p' \Big( a^{\mu} (p) a^{\nu} (p') \e ^{i (E (p) + E (p')) \tau} \e ^{- i (p + p') \sigma} - a^{\mu} (p) b^{\nu} (p') \e ^{i (E (p) - E (p')) \tau} \e ^{- i (p - p') \sigma} - \\
    - b^{\mu} (p) a^{\nu} (p') \e ^{- i (E (p) - E (p')) \tau} \e ^{i (p - p') \sigma} + b^{\mu} (p) b^{\nu} (p') \e ^{- i (E (p) + E (p')) \tau} \e ^{i (p + p') \sigma} \Big) ,
\end{multline*}
which we integrate over $\sigma$ to get
\begin{multline*}
    - \frac{\eta _{\mu \nu}}{8 \pi} \int \diff p \diff p' \Big( a^{\mu} (p) a^{\nu} (p') \e ^{i (E (p) + E (p')) \tau} \delta (p + p') - a^{\mu} (p) b^{\nu} (p') \e ^{i (E (p) - E (p')) \tau} \delta (p - p') - \\
    - b^{\mu} (p) a^{\nu} (p') \e ^{- i (E (p) - E (p')) \tau} \delta (p - p') + b^{\mu} (p) b^{\nu} (p') \e ^{- i (E (p) + E (p')) \tau} \delta (p + p') \Big) =
\end{multline*}
\begin{equation*}
    = - \frac{\eta _{\mu \nu}}{8 \pi} \int \diff p \Big( a^{\mu} (p) a^{\nu} (- p) \e ^{2 i E (p) \tau} - a^{\mu} (p) b^{\nu} (p) - b^{\mu} (p) a^{\nu} (p) + b^{\mu} (p) b^{\nu} (- p) \e ^{- 2 i E (p) \tau} \Big) ,
\end{equation*}
and now for the R.H.S
\begin{multline*}
    - \partial _{\sigma} X^{\mu} \partial _{\sigma} X^{\nu} \eta _{\mu \nu} = \frac{\eta _{\mu \nu}}{(4 \pi)^2} \int \diff p \diff p' \frac{p}{\sqrt{p^2 + m^2}} \frac{p'}{\sqrt{p^{\prime 2} + m^2}} \Big( - a^{\mu} (p) \e ^{- i (-E \tau + p \sigma)} + b^{\mu} (p) \e ^{i (- E \tau + p \sigma)} \Big) \times \\
    \times \Big( - a^{\nu} (p') \e ^{- i (- E (p') \tau + p' \sigma)} + b^{\nu} (p') \e ^{i (- E (p') \tau + p' \sigma)} \Big) =
\end{multline*}
\begin{multline*}
    = \frac{\eta _{\mu \nu}}{(4 \pi)^2} \int \diff p \diff p' \frac{p p'}{E (p) E (p')} \Big( a^{\mu} (p) a^{\nu} (p') \e ^{i (E (p) + E (p')) \tau} \e ^{- i (p + p') \sigma} - a^{\mu} (p) b^{\nu} (p') \e ^{i (E (p) - E (p')) \tau} \e ^{- i (p - p') \sigma} - \\
    - b^{\mu} (p) a^{\nu} (p') \e ^{- i (E (p) - E (p')) \tau} \e ^{i (p - p') \sigma} + b^{\mu} (p) b^{\nu} (p) \e ^{- i ( E (p) + E (p')) \tau} \e ^{i (p + p') \sigma} \Big) ,
\end{multline*}
which once again we integrate over $\sigma$,
\begin{multline*}
    \frac{\eta _{\mu \nu}}{8 \pi} \int \diff p \diff p' \frac{p p'}{E (p) E (p')} \Big( a^{\mu} (p) a^{\nu} (p') \e ^{i (E (p) + E (p')) \tau} \delta (p + p') - a^{\mu} (p) b^{\nu} (p') \e ^{i (E (p) - E (p')) \tau} \delta (p - p') - \\
    - b^{\mu} (p) a^{\nu} (p') \e ^{- i (E (p) - E (p')) \tau} \delta (p - p') + b^{\mu} (p) b^{\nu} (p) \e ^{- i ( E (p) + E (p')) \tau} \delta (p + p') \Big) =
\end{multline*}
\begin{equation*}
    = \frac{1}{8 \pi} \int \diff p \frac{p^2}{p^2 + m^2} \Big( a^{\mu} (p) a^{\nu} (- p) \e ^{2 i E (p) \tau} - a^{\mu} (p) b^{\nu} (p) - b^{\mu} (p) a^{\nu} (p) + b^{\mu} (p) b^{\nu} (- p) \e ^{- 2 i E (p) \tau} \Big) \eta _{\mu \nu} ,
\end{equation*}
which when equated to the L.H.S yields
\begin{equation*}
    \int \diff p \left( \frac{p^2}{p^2 + m^2} + 1 \right) \Big( a^{\mu} (p) a^{\nu} (- p) \e ^{2 i E (p) \tau} - a^{\mu} (p) b^{\nu} (p) - b^{\mu} (p) a^{\nu} (p) + b^{\mu} (p) b^{\nu} (- p) \e ^{- 2 i E (p) \tau} \Big) \eta _{\mu \nu} = 0 ,
\end{equation*}
which implies
\begin{equation*}
    \Big( a^{\mu} (p) a^{\nu} (- p) \e ^{2 i E (p) \tau} - a^{\mu} (p) b^{\nu} (p) - b^{\mu} (p) a^{\nu} (p) + b^{\mu} (p) b^{\nu} (- p) \e ^{- 2 i E (p) \tau} \Big) \eta _{\mu \nu} = 0 .
\end{equation*}
We can sum and subtract this with the previous condition to simplify and obtain
\begin{align*}
    \Big( a^{\mu} (p) a^{\nu} (- p) \e ^{2 i E (p) \tau} + b^{\mu} (p) b^{\nu} (- p) \e ^{- 2 i E (p) \tau} \Big) \eta _{\mu \nu} & = 0   \\
    \Big( a^{\mu} (p) b^{\nu} (p) + b^{\mu} (p) a^{\nu} (p) \Big) \eta _{\mu \nu}                                                 & = 0 ,
\end{align*}
and for while $a^{\mu}$ and $b^{\nu}$ are number-valued, the second condition can be further simplified to
\begin{equation*}
    a^{\mu} (p) b^{\nu} (p) \eta _{\mu \nu} = 0 ,
\end{equation*}
meaning $a^{\mu}$ and $b^{\nu}$ are orthogonal. In general, this condition states that $a^{\mu}$ and $b^{\nu}$ anti-commute w.r.t Lorentz inner product. The first condition must hold for all values of $\tau$, and since the exponentials are linearly independent, we thus have
\begin{align*}
    a^{\mu} (p) a^{\nu} (- p) \eta _{\mu \nu} & = 0   \\
    b^{\mu} (p) b^{\nu} (- p) \eta _{\mu \nu} & = 0 ,
\end{align*}
meaning that reflecting the argument of $a^{\mu}$ and $b^{\nu}$ creates orthogonal vectors. We can thus write
\begin{align*}
    a^{\mu} (p) & = \Lambda ^{\mu} _{\ \nu} \left( \frac{p}{4} \right) a^{\nu} (0)    \\
    b^{\mu} (p) & = \Upsilon ^{\mu} _{\ \nu} \left( \frac{p}{4} \right) b^{\nu} (0) ,
\end{align*}
where $\Lambda , \Upsilon \in \mathrm{SO}^{+} (1 , d)$. Also. since the orthogonality under reflection must hold for all values of $p$, in particular we have
\begin{align*}
    a^{\mu} (0) a^{\nu} (0) \eta _{\mu \nu} & = 0   \\
    b^{\mu} (0) b^{\nu} (0) \eta _{\mu \nu} & = 0 ,
\end{align*}
meaning that the initial $a^{\mu} (0)$ and $b^{\nu} (0)$ are null vectors. This means that $a^{\mu} (p)$ and $b^{\nu} (p)$ are also null vectors since they are related to the initial values by Lorentz transformation. With this, the solution becomes
\begin{equation*}
    X^{\mu} (\tau , \sigma) = \frac{1}{2 \pi} \int \diff p \frac{1}{2 E (p)} \left( \Lambda ^{\mu} _{\ \nu} \left( \frac{p}{4} \right) a^{\nu} (0) \e ^{- i (- E \tau + p \sigma)} + \Upsilon ^{\mu} _{\ \nu} \left( \frac{p}{4} \right) b^{\nu} (0) \e ^{i (- E \tau + p \sigma)} \right)
\end{equation*}
with $(a^{\mu} b^{\nu} + b^{\nu} a^{\mu}) \eta _{\mu \nu} = 0$. This condition implies
\begin{equation*}
    \left( \Lambda ^{\mu} _{\ \gamma} \left( \frac{p}{4} \right) a^{\gamma} (0) \Upsilon ^{\nu} _{\ \rho} \left( \frac{p}{4} \right) b^{\rho} (0) \right) \eta _{\mu \nu} = 0
\end{equation*}
\begin{equation*}
    \Upsilon ^{\mu} _{\ \nu} \left( \frac{p}{4} \right) = \Lambda ^{\ \mu} _{\nu} \left( \frac{p}{4} \right) = \Lambda ^{\mu} _{\ \nu} \left( - \frac{p}{4} \right) ,
\end{equation*}
thus turning the solution into
\begin{equation*}
    X^{\mu} (\tau , \sigma) = \frac{1}{2 \pi} \int \diff p \frac{1}{2 E (p)} \left( \Lambda ^{\mu} _{\ \nu} \left( \frac{p}{4} \right) a^{\nu} (0) \e ^{- i (- E \tau + p \sigma)} + \Lambda ^{\mu} _{\ \nu} \left( - \frac{p}{4} \right) b^{\nu} (0) \e ^{i (- E \tau + p \sigma)} \right) .
\end{equation*}

Next, we choose background coordinates s.t $a^{\mu} (0) = (a_{0} , a_{0}, 0 , ... , 0) = a_{0} x^{+}$ and $b^{\nu} (0) = (b_{0} , b_{0} , 0 , ... , 0) = b_{0} x^{+}$, simplifying the solution to
\begin{align*}
    X^{\mu} (\tau , \sigma) = \frac{1}{2 \pi} \int \diff p \frac{1}{2 E (p)} \left( a_{0} \Lambda ^{\mu} _{\ +} \left( \frac{p}{4} \right) \e ^{- i (- E \tau + p \sigma)} + b_{0} \Lambda ^{\mu} _{\ +} \left( - \frac{p}{4} \right) \e ^{i (- E \tau + p \sigma)} \right) x^{+}
\end{align*}

\newpage

\section{Solutions for the Polyakov KG eqn}

From the inverse-area corrected Polyakov action we have the equation
\begin{equation*}
    (\partial _{\sigma} ^2 - \partial _{\tau} ^2) X^{\mu} - m^2 X^{\mu} = 0
\end{equation*}
subjected to the constraint
\begin{equation*}
    \partial _{c} X^{\mu} \partial _{d} X^{\nu} \eta _{\mu \nu} - \frac{1}{2} g_{c d} g^{a b} \partial _{a} X^{\mu} \partial _{b} X^{\nu} \eta _{\mu \nu} = 0 ,
\end{equation*}
which with conformal gauge $g_{a b} = \phi \eta _{a b}$ becomes
\begin{equation*}
    \partial _{c} X^{\mu} \partial _{d} X^{\nu} \eta_{\mu \nu} = \frac{1}{2} \eta _{c d} \eta ^{a b} \partial _{a} X^{\mu} \partial _{b} X^{\nu} \eta _{\mu \nu}
\end{equation*}
or more explicitly,
\begin{align*}
    \partial _{\tau} X^{\mu} \partial _{\tau} X^{\nu} \eta _{\mu \nu}     & = - \frac{1}{2} \eta ^{a b} \partial _{a} X^{\mu} \partial _{b} X^{\nu} \eta _{\mu \nu} \\
    \partial _{\sigma} X^{\mu} \partial _{\sigma} X^{\nu} \eta _{\mu \nu} & = \frac{1}{2} \eta ^{a b} \partial _{a} X^{\mu} \partial _{b} X^{\nu} \eta _{\mu \nu}   \\
    \partial _{\tau} X^{\mu} \partial _{\sigma} X^{\nu} \eta_{\mu \nu}    & = \partial _{\sigma} X^{\mu} \partial _{\tau} X^{\nu} \eta _{\mu \nu} = 0 .
\end{align*}
These simplify further by expanding the summation on $a,b$:
\begin{equation*}
    \partial _{\tau} X^{\mu} \partial _{\tau} X^{\nu} \eta _{\mu \nu} = - \frac{1}{2} \left( - \partial _{\tau} X^{\mu} \partial _{\tau} X^{\nu} \eta _{\mu \nu} + \partial _{\sigma} X^{\mu} \partial _{\sigma} X^{\nu} \eta _{\mu \nu} \right)
\end{equation*}
\begin{equation*}
    \partial _{\tau} X^{\mu} \partial _{\tau} X^{\nu} \eta _{\mu \nu} + \partial _{\sigma} X^{\mu} \partial _{\sigma} X^{\nu} \eta _{\mu \nu} = 0 .
\end{equation*}
These can be combined by multiplying the previous one by 2 and summing/subtracting, leaving us with
\begin{equation*}
    (\partial _{\tau} X \pm \partial _{\sigma} X)^2 = 0 .
\end{equation*}

The EoM is just a one-dimensional finite-space Klein-Gordon equation, which has general solution given by Fourier transform
\begin{equation*}
    X^{\mu} (\tau , \sigma) = \sqrt{\frac{\alpha '}{2}} \sum _{n \in \mathbb{Z}} \frac{1}{2 E_{n}} \left( a^{\mu} _{n} \e ^{i (E_{n} \tau - n \sigma)} + b^{\mu} _{n} \e ^{- i (E_{n} \tau - n \sigma)} \right) ,
\end{equation*}
with $E_{n} = \sqrt{n^2 + m^2}$.

Reality of $X^{\mu}$ implies
\begin{align*}
    (X^{\mu})^{*} & = \left( \sqrt{\frac{\alpha '}{2}} \sum _{n \in \mathbb{Z}} \frac{1}{2 E_{n}} \left( a^{\mu} _{n} \e ^{i (E_{n} \tau - n \sigma)} + b^{\mu} _{n} \e ^{- i (E_{n} \tau - n \sigma)} \right) \right) ^{*} =        \\
                  & = \sqrt{\frac{\alpha '}{2}} \sum _{n \in \mathbb{Z}} \frac{1}{2 E_{n}} \left( (a^{\mu} _{n})^{*} \e ^{- i (E_{n} \tau - n \sigma)} + (b^{\mu} _{n})^{*} \e ^{i (E_{n} \tau - n \sigma)} \right) \equiv X^{\mu} ,
\end{align*}
giving us
\begin{align*}
    (a^{\mu} _{n})^{*} & = b^{\mu} _{n}   \\
    (b^{\mu} _{n})^{*} & = a^{\mu} _{n} ,
\end{align*}
thus the solution for closed string becomes
\begin{equation*}
    X^{\mu} = \sqrt{\frac{\alpha '}{2}} \sum _{n} \frac{1}{2 E_{n}} \left( a^{\mu} _{n} \e ^{i (E_{n} \tau - n \sigma)} + (a^{\mu} _{n})^{*} \e ^{- i (E_{n} \tau - n \sigma)} \right) .
\end{equation*}

Analysing the constraints, for the closed string we have the derivatives
\begin{align*}
    \partial _{\tau} X^{\mu}   & = \frac{i}{2} \sqrt{\frac{\alpha '}{2}} \sum _{n} \left( a^{\mu} _{n} \e ^{i (E_{n} \tau - n \sigma)} - (a^{\mu} _{n})^{*} \e ^{- i (E_{n} \tau - n \sigma)} \right)                     \\
    \partial _{\sigma} X^{\mu} & = - \frac{i}{2} \sqrt{\frac{\alpha '}{2}} \sum _{n} \frac{n}{E_{n}} \left( a^{\mu} _{n} \e ^{i (E_{n} \tau - n \sigma)} - (a^{\mu} _{n})^{*} \e ^{- i (E_{n} \tau - n \sigma)} \right) ,
\end{align*}
\begin{equation*}
    \partial _{\tau} X^{\mu} \pm \partial _{\sigma} X^{\mu} = \frac{i}{2} \sqrt{\frac{\alpha '}{2}} \sum _{n} \left( 1 \mp \frac{n}{E_{n}} \right) \left( a^{\mu} _{n} \e ^{i (E_{n} \tau - n \sigma)} - (a^{\mu} _{n})^{*} \e ^{- i (E_{n} \tau - n \sigma)} \right)
\end{equation*}
so the constraint reads
\begin{multline*}
    (\partial _{\tau} X \pm \partial _{\sigma} X)^2 = - \frac{\eta _{\mu \nu} \alpha '}{8} \sum _{n} \sum _{p} \left( 1 \mp \frac{n}{E_{n}} \right) \left( 1 \mp \frac{p}{E_{p}} \right) \bigg( a^{\mu} _{n} a^{\nu} _{p} \e ^{i (E_{n} + E_{p}) \tau} \e ^{- i (n + p) \sigma} - \\
    - a^{\mu} _{n} (a^{\nu} _{p})^{*} \e ^{i (E_{n} - E_{p}) \tau} \e ^{- i (n - p) \sigma} - (a^{\mu} _{n})^{*} a^{\nu} _{p} \e ^{- i (E_{n} - E_{p}) \tau} \e ^{i (n - p) \sigma} + (a^{\mu} _{n})^{*} (a^{\nu} _{p})^{*} \e ^{- i (E_{n} + E_{p}) \tau} \e ^{i (n + p) \sigma} \bigg) \equiv 0
\end{multline*}
Since the exponentials are all linearly independent and we can't decouple the $\tau$ exponential from any of the sums, we have 2 cases to investigate: $n = p$ and $n \neq p$. For $n \neq p$, we get
\begin{align*}
    L^{1} _{n , p} & := a^{\mu} _{n} a^{\nu} _{p} \eta _{\mu \nu} = 0               \\
    L^{2} _{n , p} & := a^{\mu} _{n} (a^{\nu} _{p})^{*} \eta _{\mu \nu} = 0         \\
    L^{3} _{n , p} & := (a^{\mu} _{n})^{*} a^{\nu} _{p} \eta _{\mu \nu} = 0         \\
    L^{4} _{n , p} & := (a^{\mu} _{n})^{*} (a^{\nu} _{p})^{*} \eta _{\mu \nu} = 0 ,
\end{align*}
where $L^{2} _{n , p}$ and $L^{3} _{n , p}$ are just complex conjugates of each other, so they are redundant. The same is true for $L^{1} _{n , p}$ and $L^{4} _{n , p}$, so in the end we have
\begin{align*}
    L_{n , p}             & := a^{\mu} _{n} a^{\nu} _{p} \eta _{\mu \nu} = 0         \\
    \widetilde{L}_{n , p} & := a^{\mu} _{n} (a^{\nu} _{p})^{*} \eta _{\mu \nu} = 0 .
\end{align*}
For $n = p$, the first and last terms are complex conjugates of each other, while the exponentials in the 2 middle terms are reduced to $1$, so we get
\begin{align*}
    a^{\mu} _{n} a^{\nu} _{n} \eta _{\mu \nu}                                                                        & = 0   \\
    \sum _{n} \left( \left( 1 \mp \frac{n}{E_{n}} \right) ^2 a^{\mu} _{n} (a^{\nu} _{n})^{*} \eta _{\mu \nu} \right) & = 0 .
\end{align*}
Collecting everything we have
\begin{align*}
    L_{n , p} := a^{\mu} _{n} a^{\nu} _{p} \eta _{\mu \nu}                                                           & = 0 , \ \forall \ n , p \in \mathbb{Z} , \\
    \widetilde{L}_{n , p} := a^{\mu} _{n} (a^{\nu} _{p})^{*} \eta _{\mu \nu}                                         & = 0 , \ n \neq p ,                       \\
    \sum _{n} \left( \left( 1 \mp \frac{n}{E_{n}} \right) ^2 a^{\mu} _{n} (a^{\nu} _{n})^{*} \eta _{\mu \nu} \right) & = 0 .
\end{align*}

One last thing before quantizing: the $n = 0$ term in the solution reads
\begin{equation*}
    \frac{1}{2 \pi \alpha '} \frac{1}{2 m} \left( a^{\mu} _{0} \e ^{- i m \tau} + (a^{\mu} _{0})^{*} \e ^{i m \tau} \right) ,
\end{equation*}
which is divergent in the $m \rightarrow 0$ limit unless $\mathfrak{Re}(a^{\mu} _{0}) \sim m$. Keeping the analogy to the $m \rightarrow 0$ regime, we conclude $a^{\mu} _{0} = c_{1} m x^{\mu} + i c_{2} p^{\mu}$, where $c_{1}$ and $c_{2}$ are to be determined. We also have that
\begin{equation*}
    M^2 = - p^{\mu} p_{\mu} = - \frac{1}{(c_{2})^2} (a^{\mu} _{0} (a^{\nu} _{0})^{*} - (c_{1} m)^2 x^{\mu} x^{\nu}) \eta _{\mu \nu} ,
\end{equation*}
which after analysing closer the $n = p$ condition
\begin{equation*}
    \sum _{n} \left( \left( 1 \mp \frac{n}{E_{n}} \right) ^2 \widetilde{L}_{n , n} \right) = 0
\end{equation*}
\begin{equation*}
    \sum _{n \neq 0} \left( \left( 1 \mp \frac{n}{E_{n}} \right) ^2 \widetilde{L}_{n , n} \right) + \widetilde{L}_{0 , 0} = 0
\end{equation*}
\begin{equation*}
    \sum _{n \neq 0} \left( \left( 1 \mp \frac{n}{E_{n}} \right) ^2 \widetilde{L}_{n , n} \right) = - a^{\mu} _{0} (a^{\nu} _{0})^{*} \eta _{\mu \nu} \equiv - \left( (c_{1} m)^2 x^{\mu} x_{\mu} + (c_{2})^2 p^{\mu} p_{\mu} \right) ,
\end{equation*}
we have the string mass in terms of vibrational modes and center of mass position
\begin{align*}
    M^2 & = \frac{1}{(c_{2})^2} \left( \sum _{n \neq 0} \left( \left( 1 \mp \frac{n}{E_{n}} \right) ^2 \widetilde{L}_{n , n} \right) - (c_{1} m)^2 x^{\mu} x_{\mu} \right) =                           \\
        & = \frac{1}{(c_{2})^2} \left( \sum _{n \neq 0} \left( \left( 1 \mp \frac{n}{E_{n}} \right) ^2 a^{\mu} _{n} (a^{\nu} _{n})^{*} \eta _{\mu \nu} \right) - (c_{1} m)^2 x^{\mu} x_{\mu} \right) .
\end{align*}

\subsection{The Expansion Modes Algebra}

\subsubsection{Closed String}

Before starting this analysis, let's rename the vibrational modes to $\alpha ^{\mu} _{n}$ to be more in line with the literature

Start by finding the canonical momentum
\begin{equation*}
    \Pi _{\lambda} = \frac{\partial \Lagr}{\partial (\partial _{\tau} X^{\lambda})} = - \frac{T}{2} \left( \sqrt{- g} - \frac{k \Delta}{2 \sqrt{- g}} \right) g^{a b} \delta ^{\tau} _{a} \delta ^{\mu} _{\lambda} \partial _{b} X^{\nu} \eta _{\mu \nu} =
\end{equation*}
\begin{equation*}
    = - \frac{1}{4 \pi \alpha '} \left( 1 - \frac{k \Delta}{2 \phi ^2} \right) \eta ^{\tau b} \partial _{b} X^{\nu} \eta _{\lambda \nu} =
\end{equation*}
\begin{equation*}
    = \frac{1}{4 \pi \alpha '} F^{-} \partial _{\tau} X^{\nu} \eta_{\lambda \nu} = F^{-} \frac{i}{8 \pi \alpha '} \sum _{n \in \mathbb{Z}} \left( \alpha ^{\nu} _{n} \e ^{i (E_{n} \tau - n \sigma)} - (\alpha ^{\nu} _{n})^{*} \e ^{- i (E_{n} \tau - n \sigma)} \right) \eta _{\nu \lambda} ,
\end{equation*}
where $F^{-} = \sum _{l} (A_{l} \e ^{i (E_{l} \tau + l \sigma)} + (A_{l})^{*} \e ^{- i (E_{l} \tau + l \sigma)})$. We then proceed to the equal time canonical Poisson bracket relations
\begin{equation*}
    \left\{ X^{\mu} (\tau , \sigma) , \Pi ^{\nu} (\tau , \sigma ') \right\} = \delta (\sigma - \sigma ') \eta ^{\mu \nu}
\end{equation*}
\begin{equation*}
    \left\{ X^{\mu} (\tau , \sigma) , X^{\nu} (\tau , \sigma ') \right\} = \left\{ \Pi ^{\mu} (\tau , \sigma) , \Pi ^{\nu} (\tau , \sigma ') \right\} = 0 ,
\end{equation*}
where by expanding the first relation at $\tau = 0$ we get
\begin{multline*}
    \left\{ X^{\mu} (0 , \sigma) , \Pi ^{\nu} (0 , \sigma ') \right\} = \bigg\{ \frac{1}{2} \sum _{n} \frac{1}{E_{n}} \left( \alpha ^{\mu} _{n} \e ^{- i n \sigma} + (\alpha ^{\mu} _{n})^{*} \e ^{i n \sigma} \right) , \\
    \frac{i}{8 \pi \alpha '} F^{-} (0 , \sigma ') \sum _{p} \left( \alpha ^{\nu} _{p} \e ^{- i p \sigma '} - (\alpha ^{\nu} _{p})^{*} \e ^{i p \sigma '} \right) \bigg\} = \delta (\sigma - \sigma ') \eta ^{\mu \nu}
\end{multline*}
\begin{multline*}
    \frac{i}{16 \pi \alpha '} F^{-} (0 , \sigma ') \sum _{n} \frac{1}{E_{n}} \sum _{p} \bigg( \left\{ \alpha ^{\mu} _{n} , \alpha ^{\nu} _{p} \right\} \e ^{- i n \sigma} \e ^{- i p \sigma '} - \left\{ \alpha ^{\mu} _{n} , (\alpha ^{\nu} _{p})^{*} \right\} \e ^{- i n \sigma} \e ^{i p \sigma '} + \\
    + \left\{ (\alpha ^{\mu} _{n})^{*} , \alpha ^{\nu} _{p} \right\} \e ^{i n \sigma} \e ^{- i p \sigma '} - \left\{ (\alpha ^{\mu} _{n})^{*} , (\alpha ^{\nu} _{p})^{*} \right\} \e ^{i n \sigma} \e ^{i p \sigma '} \bigg) = \delta (\sigma - \sigma ') \eta ^{\mu \nu}
\end{multline*}
\begin{equation*}
    \Bigg\downarrow \frac{1}{2 \pi} \int _{0} ^{2 \pi} \diff \sigma \e ^{i m \sigma}
\end{equation*}
\begin{multline*}
    \frac{i}{16 \pi \alpha '} F^{-} (0 , \sigma ') \frac{1}{E_{m}} \sum _{p} \bigg( \left\{ \alpha ^{\mu} _{m} , \alpha ^{\nu} _{p} \right\} \e ^{- i p \sigma '} - \left\{ \alpha ^{\mu} _{m} , (\alpha ^{\nu} _{p})^{*} \right\} \e ^{i p \sigma '} + \\
    + \left\{ (\alpha ^{\mu} _{- m})^{*} , \alpha ^{\nu} _{p} \right\} \e ^{- i p \sigma '} - \left\{ (\alpha ^{\mu} _{- m})^{*} , (\alpha ^{\nu} _{p})^{*} \right\} \e ^{i p \sigma '} \bigg) = \frac{1}{2 \pi} \e ^{i m \sigma '} \eta ^{\mu \nu}
\end{multline*}
\begin{multline*}
    \frac{i}{8 \alpha ' E_{m}} \sum _{l} \sum _{p} \bigg( \left( A_{l} \e ^{i l \sigma '} + (A_{l})^{*} \e ^{- i l \sigma '} \right) \left\{ \alpha ^{\mu} _{m} , \alpha ^{\nu} _{p} \right\} \e ^{- i p \sigma '} - \\
    - \left( A_{l} \e ^{i l \sigma '} + (A_{l})^{*} \e ^{- i l \sigma '} \right) \left\{ \alpha ^{\mu} _{m} , (\alpha ^{\nu} _{p})^{*} \right\} \e ^{i p \sigma '} + \\
    + \left( A_{l} \e ^{i l \sigma '} + (A_{l})^{*} \e ^{- i l \sigma '} \right) \left\{ (\alpha ^{\mu} _{- m})^{*} , \alpha ^{\nu} _{p} \right\} \e ^{- i p \sigma '} - \\
    - \left( A_{l} \e ^{i l \sigma '} + (A_{l})^{*} \e ^{- i l \sigma '} \right) \left\{ (\alpha ^{\mu} _{- m})^{*} , (\alpha ^{\nu} _{p})^{*} \right\} \e ^{i p \sigma '} \bigg) = \e ^{i m \sigma '} \eta ^{\mu \nu}
\end{multline*}
\begin{multline*}
    \frac{i}{8 \alpha ' E_{m}} \sum _{l} \sum _{p} \bigg( A_{l} \left\{ \alpha ^{\mu} _{m} , \alpha ^{\nu} _{p} \right\} \e ^{i (l - p) \sigma '} + (A_{l})^{*} \left\{ \alpha ^{\mu} _{m} , \alpha ^{\nu} _{p} \right\} \e ^{- i (l + p) \sigma '} - \\
    - A_{l} \left\{ \alpha ^{\mu} _{m} , (\alpha ^{\nu} _{p})^{*} \right\} \e ^{i (l + p) \sigma '} - (A_{l})^{*} \left\{ \alpha ^{\mu} _{m} , (\alpha ^{\nu} _{p})^{*} \right\} \e ^{i (p - l) \sigma '} + \\
    + A_{l} \left\{ (\alpha ^{\mu} _{- m})^{*} , \alpha ^{\nu} _{p} \right\} \e ^{i (l - p) \sigma '} + (A_{l})^{*} \left\{ (\alpha ^{\mu} _{- m})^{*} , \alpha ^{\nu} _{p} \right\} \e ^{- i (l + p) \sigma '} - \\
    - A_{l} \left\{ (\alpha ^{\mu} _{- m})^{*} , (\alpha ^{\nu} _{p})^{*} \right\} \e ^{i (l + p) \sigma '} - (A_{l})^{*} \left\{ (\alpha ^{\mu} _{- m})^{*} , (\alpha ^{\nu} _{p})^{*} \right\} \e ^{i (p - l) \sigma '} \bigg) = \e ^{i m \sigma '} \eta ^{\mu \nu}
\end{multline*}
\begin{equation*}
    \Bigg\downarrow \frac{1}{2 \pi} \int _{0} ^{2 \pi} \diff \sigma '
\end{equation*}
\begin{multline*}
    \frac{i}{8 \alpha ' E_{m}} \sum _{l} \bigg( A_{l} \left\{ \alpha ^{\mu} _{m} , \alpha ^{\nu} _{l} \right\} + (A_{l})^{*} \left\{ \alpha ^{\mu} _{m} , \alpha ^{\nu} _{- l} \right\} - A_{l} \left\{ \alpha ^{\mu} _{m} , (\alpha ^{\nu} _{- l})^{*} \right\} - (A_{l})^{*} \left\{ \alpha ^{\mu} _{m} , (\alpha ^{\nu} _{l})^{*} \right\} + \\
    + A_{l} \left\{ (\alpha ^{\mu} _{- m})^{*} , \alpha ^{\nu} _{l} \right\} + (A_{l})^{*} \left\{ (\alpha ^{\mu} _{- m})^{*} , \alpha ^{\nu} _{- l} \right\} - A_{l} \left\{ (\alpha ^{\mu} _{- m})^{*} , (\alpha ^{\nu} _{- l})^{*} \right\} - (A_{l})^{*} \left\{ (\alpha ^{\mu} _{- m})^{*} , (\alpha ^{\nu} _{l})^{*} \right\} \bigg) = \delta _{m , 0} \eta ^{\mu \nu} .
\end{multline*}
If we fix the expansion modes of $F^{-}$ to satisfy
\begin{align*}
    \mathfrak{Re}(A_{0}) = - 2 \alpha ' ,
\end{align*}
we get the desired
\begin{align*}
    \left\{ \alpha ^{\mu} _{m} , \alpha ^{\nu} _{l} \right\}       & = 0 = \left\{ (\alpha ^{\mu} _{m})^{*} , (\alpha ^{\nu} _{l})^{*} \right\}                                                            \\
    \left\{ \alpha ^{\mu} _{m} , (\alpha ^{\nu} _{l})^{*} \right\} & = - i E_{m} \delta _{|m| , |l|} \eta ^{\mu \nu} = - \left\{ (\alpha ^{\mu} _{m})^{*} , \alpha ^{\nu} _{l} \right\} , \ m , l \geq 0 ,
\end{align*}
where the absolute value comes from the fact that $\alpha ^{\mu} _{- n} = \alpha ^{\mu} _{n}$. We can rescale the vibrational modes to get harmonic oscillator bracket relations
\begin{align*}
    a^{\mu} _{n}       & := \frac{1}{\sqrt{E_{n}}} \alpha ^{\mu} _{n}       \\
    (a^{\mu} _{n})^{*} & := \frac{1}{\sqrt{E_{n}}} (\alpha ^{\mu} _{n})^{*}
\end{align*}
\begin{align*}
    \left\{ a^{\mu} _{n} , a^{\nu} _{p} \right\}       & = 0 = \left\{ (a^{\mu} _{n})^{*} , (a^{\nu} _{p})^{*} \right\} \\
    \left\{ a^{\mu} _{n} , (a^{\nu} _{p})^{*} \right\} & = - i \delta _{|n| , |p|} \eta ^{\mu \nu} .
\end{align*}
This can be used to calculate the bracket relation between $x^{\mu}$ and $p^{\nu}$:
\begin{equation*}
    \left\{ \alpha ^{\mu} _{0} , (\alpha ^{\nu} _{0})^{*} \right\} = \left\{ (c_{1} \mu x^{\mu} + i c_{2} p^{\mu}) , (c_{1} \mu x^{\nu} - i c_{2} p^{\nu}) \right\} = - i \mu \eta ^{\mu \nu}
\end{equation*}
\begin{equation*}
    (c_{1} \mu)^2 \left\{ x^{\mu} , x^{\nu} \right\} + (c_{2})^{2} \left\{ p^{\mu} , p^{\nu} \right\} - i c_{1} c_{2} \mu \left( \left\{ x^{\mu} , p^{\nu} \right\} - \left\{ p^{\mu} , x^{\nu} \right\} \right) = - i \mu \eta ^{\mu \nu}
\end{equation*}
\begin{equation*}
    \left\{ \alpha ^{\mu} _{0} , \alpha ^{\nu} _{0} \right\} = \left\{ (c_{1} \mu x^{\mu} + i c_{2} p^{\mu}) , (c_{1} \mu x^{\nu} + i c_{2} p^{\nu}) \right\} = 0
\end{equation*}
\begin{equation*}
    (c_{1} \mu)^2 \left\{ x^{\mu} , x^{\nu} \right\} - (c_{2})^2 \left\{ p^{\mu} , p^{\nu} \right\} + i c_{1} c_{2} \mu \left( \left\{ x^{\mu} , p^{\nu} \right\} + \left\{ p^{\mu} , x^{\nu} \right\} \right) = 0 .
\end{equation*}
By summing and subtracting these we get that
\begin{align*}
    \left\{ x^{\mu} , x^{\nu} \right\} & = 0 = \left\{ p^{\mu} , p^{\nu} \right\}                                         \\
    \left\{ x^{\mu} , p^{\nu} \right\} & = \frac{1}{2 c_{1} c_{2}} \eta ^{\mu \nu} = - \left\{ p^{\mu} , x^{\nu} \right\}
\end{align*}
so we get that $c_{2} = 1 / (2 c_{1})$, and the 0-th mode becomes $\alpha ^{\mu} _{0} = c_{1} \mu x^{\mu} + i p^{\mu} / (2 c_{1})$. Keeping the analogy to the $\mu \rightarrow 0$ limit, we see that $c_{1} = \sqrt{2} / \sqrt{\alpha '}$.

\subsection{The Constraint Algebra}

With the bracket relations between the vibrational modes in hand, we can now calculate the algebra generated by the constraints. For simplicity, since $E_{n} = \sqrt{n^2 + \mu ^2} > 0$, we can focus instead on the algebra of the normalized harmonic oscillator modes $a^{\mu} _{n}$ instead of the expansion modes $\alpha ^{\mu} _{n}$.
\begin{equation*}
    \left\{ L_{n , m} , L_{k , p} \right\} = \left\{ a^{\mu} _{n} a^{\nu} _{m} \eta _{\mu \nu} , a^{\mu '} _{k} a^{\nu '} _{p} \eta _{\mu ' \nu '} \right\} = 0
\end{equation*}
\begin{align*}
    \left\{ L_{n , m} , \widetilde{L}_{k , p} \right\} & = \left\{ a^{\mu} _{n} a^{\nu} _{m} \eta _{\mu \nu} , a^{\mu '} _{k} (a^{\nu '} _{p})^{*} \eta _{\mu ' \nu '} \right\} =                                                                                                   \\
                                                       & = \eta _{\mu \nu} \left( a^{\nu} _{m} \left\{ a^{\mu} _{n} , (a^{\nu '} _{p})^{*} \right\} a^{\mu '} _{k} + a^{\mu} _{n} \left\{ a^{\nu} _{m} , (a^{\nu '} _{p})^{*} \right\} a^{\mu '} _{k} \right) \eta _{\mu ' \nu '} = \\
                                                       & = \eta _{\mu \nu} \left( a^{\nu} _{m}(- i \eta ^{\mu \nu '} \delta _{n , p}) a^{\mu '} _{k} + a^{\mu} _{n} (- i \eta ^{\nu \nu '} \delta _{m , p}) a^{\mu '} _{k} \right) \eta _{\mu ' \nu '} =                            \\
                                                       & = - i \left( a^{\mu} _{m} a^{\nu} _{k} \eta _{\mu \nu} \delta _{n , p} + a^{\mu} _{n} a^{\nu} _{k} \eta _{\mu \nu} \delta _{m , p} \right) =                                                                               \\
                                                       & = - i \left( L_{m, k} \delta _{n , p} + L_{n , k} \delta _{m , p} \right)
\end{align*}
\begin{align*}
    \left\{ \widetilde{L}_{n , m} , \widetilde{L}_{k , p} \right\} & = \left\{ a^{\mu} _{n} (a^{\nu} _{m})^{*} \eta _{\mu \nu} , a^{\mu '} _{k} (a^{\nu '} _{p})^{*} \eta _{\mu ' \nu '} \right\} =                                                                                                         \\
                                                                   & = \eta _{\mu \nu} \left( (a^{\nu} _{m})^{*} \left\{ a^{\mu} _{n} , (a^{\nu '} _{p})^{*} \right\} a^{\mu '} _{k} + a^{\mu} _{n} \left\{ (a^{\nu} _{m})^{*} , a^{\mu '} _{k} \right\} (a^{\nu '} _{p})^{*} \right) \eta _{\mu ' \nu '} = \\
                                                                   & = \eta _{\mu \nu} \left( (a^{\nu} _{m})^{*} (- i \eta ^{\mu \nu '} \delta _{n , p}) a^{\mu '} _{k} - a^{\mu} _{n} (- i \eta ^{\nu \mu '} \delta _{m , k}) (a^{\nu '} _{p})^{*} \right) \eta _{\mu ' \nu '} =                           \\
                                                                   & = - i \left( a^{\mu} _{k} (a^{\nu} _{m})^{*} \eta _{\mu \nu} \delta _{n , p} - a^{\mu} _{n} (a^{\nu} _{p})^{*} \eta _{\mu \nu} \delta _{m , k} \right) =                                                                               \\
                                                                   & = - i \left( \widetilde{L}_{k , m} \delta _{n , p} - \widetilde{L}_{n , p} \delta _{m , k} \right) .
\end{align*}
These relations reveal that $\alpha ^{\mu} _{- n}$ = $\alpha ^{\mu} _{n}$, since the middle relation is only invariant to $p \mapsto - p$ if the expansion modes are symmetric.

\newpage

\section{Into the Quantum Realm}

We now proceed to quantize the KG string through canonical quantization. The string $X^{\mu}$ and it's momentum $\Pi ^{\nu}$ are promoted to operators $\Hat{X}^{\mu}$ and $\Hat{\Pi}^{\nu}$, and we turn the Poisson bracket $\{ . , . \}$ into commutator between operators $[. , .] = i \hbar \{ . , . \}$. The equal time bracket relations turn into equal time commutation relations
\begin{align*}
    \left[ \Hat{X}^{\mu} (\tau , \sigma) , \Hat{\Pi}^{\nu} (\tau , \sigma ') \right] & = i \delta (\sigma - \sigma ') \eta ^{\mu \nu}                                       \\
    \left[ \Hat{X}^{\mu} , \Hat{X}^{\nu} \right]                                     & = \left[ \Hat{\Pi}^{\mu} , \Hat{\Pi}^{\nu} \right] = 0                               \\
    \left[ \Hat{a}^{\mu} _{n} , \Hat{a}^{\nu} _{m} \right]                           & = \left[ (\Hat{a}^{\mu} _{n})^{\dagger} , (\Hat{a}^{\nu} _{m})^{\dagger} \right] = 0 \\
    \left[ \Hat{a}^{\mu} _{n} , (\Hat{a}^{\nu} _{m})^{\dagger} \right]               & = \eta ^{\mu \nu} \delta _{n , m}                                                    \\
    \left[ \Hat{x}^{\mu} , \Hat{x}^{\nu} \right]                                     & = \left[ \Hat{p}^{\mu} , \Hat{p}^{\nu} \right] = 0                                   \\
    \left[ \Hat{x}^{\mu} , \Hat{p}^{\nu} \right]                                     & = i \eta ^{\mu \nu} .
\end{align*}
Thus, the rescaled vibrational mode operators $\Hat{a}^{\mu} _{n}$ and $(\Hat{a}^{\nu} _{m})^{\dagger}$ are annihilation and creations operators, respectively.

We define a vacuum state of the string to obey
\begin{equation*}
    \Hat{a}^{\mu} _{n} \ket{0} = 0 , \ \mathrm{for} \ n \neq 0 .
\end{equation*}
For $n = 0$, we have the center of mass position and momentum operators, so the vacuum also obeys
\begin{align*}
    \Hat{x}^{\mu} \ket{0 ; x} & = x^{\mu} \ket{0 ; x}                               \\
    \Hat{p}_{\mu} \ket{0 ; x} & = - i \frac{\partial}{\partial x^{\mu}} \ket{0 ; x}
\end{align*}
in position representation or alternatively
\begin{align*}
    \Hat{x}^{\mu} \ket{0 ; p} & = - i \frac{\partial}{\partial p_{\mu}} \ket{0 ; p} \\
    \Hat{p}_{\mu} \ket{0 ; p} & = p_{\mu} \ket{0 ; p}
\end{align*}
in momentum representation.

A generic state arises from a sequence of creation operators on the vacuum
\begin{equation*}
    ((\Hat{a}^{\mu _{1}} _{1})^{\dagger})^{n_{\mu _1}} ((\Hat{a}^{\mu _2} _{2})^{\dagger})^{n_{\mu _2}} ... ((\Hat{a}^{\nu _1} _{- 1})^{\dagger})^{n_{\nu _1}} ((\Hat{a}^{\nu _2} _{- 2})^{\dagger})^{n_{\nu _2}} ... \ket{0} .
\end{equation*}
This (should) give rise to particles.

As in regular ST, we have ghosts arising from the Minkowski metric
\begin{equation*}
    \Big[ \Hat{a}^{\mu} _{n} , (\Hat{a}^{\nu} _{m})^{\dagger} \Big] = \eta ^{\mu \nu} \delta _{n m} .
\end{equation*}

For constraints, classically we have
\begin{align*}
    L_{n , p}             & = \alpha ^{\mu} _{n} \alpha ^{\nu} _{p} \eta _{\mu \nu} = 0                    \\
    \widetilde{L}_{n , p} & = \alpha ^{\mu} _{n} (\alpha ^{\nu} _{p})^{*} \eta _{\mu \nu} = 0 , \ n \neq p
\end{align*}
which because of the existence of ghosts we require to have vanishing matrix elements when sandwiched between physical states
\begin{equation*}
    \bra{\mathrm{phys} '} \Hat{L}_{n , p} \ket{\mathrm{phys}} = 0 = \bra{\mathrm{phys} '} \Hat{\widetilde{L}}_{n , p} \ket{\mathrm{phys}} .
\end{equation*}
When translating the constraints into quantum operators, in $L_{n , p}$ we have no ambiguity of ordering since $\Hat{a}$ commutes with itself. As for $\widetilde{L}_{n , p}$, there is an ambiguity for $p = n \neq 0$, so we pick normal ordering with the annihilation operators moved to the right
\begin{equation*}
    \Hat{\widetilde{L}}_{n , n} = (\Hat{\alpha} ^{\mu} _{n})^{\dagger} \Hat{\alpha} ^{\nu} _{n} \eta _{\mu \nu} .
\end{equation*}
The ambiguity manifests in the imposition of this constraint as
\begin{equation*}
    \bra{\mathrm{phys }'} \left( \sum _{n \neq 0} \left( \left( 1 \mp \frac{n}{E_{n}} \right) ^2 \Hat{\widetilde{L}}_{n , n} \right) - c \right) \ket{\mathrm{phys}} = 0 ,
\end{equation*}
for some costant $c$. Since classically
\begin{align*}
    M^2 & = \frac{8}{\alpha '} \left( \sum _{n \neq 0} \left( \left( 1 \mp \frac{n}{E_{n}} \right) ^2 \widetilde{L}_{n , n} \right) - \frac{2 \mu ^2}{\alpha '} x^{\mu} x_{\mu} \right) =                                       \\
        & = \frac{8}{\alpha '} \left( \sum _{n \neq 0} \left( \left( 1 \mp \frac{n}{E_{n}} \right) ^2 \alpha ^{\mu} _{n} (\alpha ^{\nu} _{n})^{*} \eta _{\mu \nu} \right) - \frac{2 \mu ^2}{\alpha '} x^{\mu} x_{\mu} \right) ,
\end{align*}
we see that the string mass spectrum will be affected by this constant
\begin{equation*}
    \Hat{M}^2 = \frac{8}{\alpha '} \left( \sum _{n \neq 0} \left( \left( 1 \mp \frac{n}{E_{n}} \right) ^2 E_{n} (\Hat{a}^{\mu} _{n})^{\dagger} \Hat{a}^{\nu} _{n} \eta _{\mu \nu} \right) - c - \frac{2 \mu ^2}{\alpha '} \Hat{x}^{\mu} \Hat{x}_{\mu} \right) .
\end{equation*}
By enforcing normal-ordering, we can get the value of this constant:
\begin{equation*}
    \sum _{n \neq 0} \left( \left( 1 \mp \frac{n}{E_{n}} \right) ^2 E_{n} \Hat{a}^{\mu} _{n} (\Hat{a}^{\nu} _{n})^{\dagger} \eta _{\mu \nu} \right) =
\end{equation*}
\begin{equation*}
    = \sum _{n \neq 0} \left( \left( 1 + \frac{n}{E_{n}} \right) ^2 E_{n} \left( \eta ^{\mu \nu} \delta _{|n| , |n|} + (\Hat{a}^{\mu} _{n})^{\dagger} \Hat{a}^{\nu} _{n} \right) \eta _{\mu \nu} \right)
\end{equation*}
\begin{equation*}
    = \sum _{n \neq 0} \left( \left( 1 + \frac{n}{E_{n}} \right) ^2 E_{n} (\Hat{a}^{\mu} _{n})^{\dagger} \Hat{a}^{\nu} _{n} \eta _{\mu \nu} \right) + D \sum _{n \neq 0} \left( 1 + \frac{n}{E_{n}} \right) ^2 E_{n}
\end{equation*}
giving us
\begin{equation*}
    c = - D \sum _{n \neq 0} \left( 1 + \frac{n}{E_{n}} \right) ^2 E_{n} = - D \sum _{n > 0} \left( 1 + \frac{n^2}{E_{n} ^2} \right) E_{n} .
\end{equation*}

The commutation relations between the $L$'s are inherited from the bracket relations
\begin{align*}
    \left[ \Hat{L}_{n , m} , \Hat{L}_{k , p} \right]                         & = 0                                                                                           \\
    \left[ \Hat{L}_{n , m} , \Hat{\widetilde{L}}_{k , p} \right]             & = \Hat{L}_{m , k} \delta _{n , p} + \Hat{L}_{n , k} \delta _{m , p}                           \\
    \left[ \Hat{\widetilde{L}}_{n , m} , \Hat{\widetilde{L}}_{k , p} \right] & = \Hat{\widetilde{L}}_{k , m} \delta _{n , p} - \Hat{\widetilde{L}}_{n , p} \delta _{m , k} .
\end{align*}
Since $[\Hat{L}_{n , m} , \Hat{\widetilde{L}}_{p , k}]$ only has annihilation operators, it has no ordering ambiguities. $[\Hat{\widetilde{L}}_{n , m} , \Hat{\widetilde{L}}_{k , p}]$ have ordering ambiguities for $m = k$ and/or $p = n$, so we add the anomalous terms
\begin{equation*}
    \left[ \Hat{\widetilde{L}}_{n , m} , \Hat{\widetilde{L}}_{k , p} \right] = \Hat{\widetilde{L}}_{k , m} \delta _{n , p} - \Hat{\widetilde{L}}_{n , p} \delta _{m , k} + C_{n} \delta _{n , p} + D_{k} \delta _{m , k}
\end{equation*}
Clearly $C_{0} = 0 = D_{0}$, since if all indices are equal the commutator should vanish. Also, by the symmetry of the $L$'s and because of the $\delta$, we also have that $C_{- n} = C_{n}$ and $D_{- k} = D_{k}$. We get from the Jacobi identity
\begin{equation*}
    \left[ \Hat{\widetilde{L}}_{n , m} , \left[ \Hat{\widetilde{L}}_{k , p} , \Hat{\widetilde{L}}_{r , s} \right] \right] + \left[ \Hat{\widetilde{L}}_{r , s} , \left[ \Hat{\widetilde{L}}_{n , m} , \Hat{\widetilde{L}}_{k , p} \right] \right] + \left[ \Hat{\widetilde{L}}_{k , p} , \left[ \Hat{\widetilde{L}}_{r , s} , \Hat{\widetilde{L}}_{n , m} \right] \right] = 0
\end{equation*}
\begin{multline*}
    \left[ \Hat{\widetilde{L}}_{n , m} , \left( \Hat{\widetilde{L}}_{r , p} \delta _{k , s} - \Hat{\widetilde{L}}_{k , s} \delta _{p , r} + C_{k} \delta _{k , s} + D_{r} \delta _{p , r} \right) \right] + \\
    + \left[ \Hat{\widetilde{L}}_{r , s} , \left( \Hat{\widetilde{L}}_{k , m} \delta _{n , p} - \Hat{\widetilde{L}}_{n , p} \delta _{m , k} + C_{n} \delta _{n , p} + D_{k} \delta _{m , k} \right) \right] + \\
    + \left[ \Hat{\widetilde{L}}_{k , p} , \left( \Hat{\widetilde{L}}_{n , s} \delta _{r , m} - \Hat{\widetilde{L}}_{r , m} \delta _{s , n} + C_{r} \delta _{r , m} + D_{n} \delta _{s , n} \right) \right] = 0
\end{multline*}
\begin{multline*}
    \left[ \Hat{\widetilde{L}}_{n , m} , \Hat{\widetilde{L}}_{r , p} \right] \delta _{k , s} - \left[ \Hat{\widetilde{L}}_{n , m} , \Hat{\widetilde{L}}_{k , s} \right] \delta _{p , r} + \left[ \Hat{\widetilde{L}}_{r , s} , \Hat{\widetilde{L}}_{k , m} \right] \delta _{n , p} - \\
    - \left[ \Hat{\widetilde{L}}_{r , s} , \Hat{\widetilde{L}}_{n , p} \right] \delta _{m , k} + \left[ \Hat{\widetilde{L}}_{k , p} , \Hat{\widetilde{L}}_{n , s} \right] \delta _{r , m} - \left[ \Hat{\widetilde{L}}_{k , p} , \Hat{\widetilde{L}}_{r , m} \right] \delta _{s , n} = 0
\end{multline*}
\begin{multline*}
    \left( \Hat{\widetilde{L}}_{r , m} \delta _{n , p} - \Hat{\widetilde{L}}_{n , p} \delta _{m , r} + C_{n} \delta _{n , p} + D_{r} \delta _{m , r} \right) \delta _{k , s} - \left( \Hat{\widetilde{L}}_{k , m} \delta _{n , s} - \Hat{\widetilde{L}}_{n , s} \delta _{m , k} + C_{n} \delta _{n , s} + D_{k} \delta _{m , k} \right) \delta _{p , r} + \\
    + \left( \Hat{\widetilde{L}}_{k , s} \delta _{r , m} - \Hat{\widetilde{L}}_{r , m} \delta _{s , k} + C_{r} \delta _{r , m} + D_{k} \delta _{s , k} \right) \delta _{n , p} - \left( \Hat{\widetilde{L}}_{n , s} \delta _{r , p} - \Hat{\widetilde{L}}_{r , p} \delta _{s , n} + C_{r} \delta _{r , p} + D_{n} \delta _{s , n} \right) \delta _{m , k} + \\
    + \left( \Hat{\widetilde{L}}_{n , p} \delta _{k , s} - \Hat{\widetilde{L}}_{k , s} \delta _{p , n} + C_{k} \delta _{k , s} + D_{n} \delta _{p , n} \right) \delta _{r , m} - \left( \Hat{\widetilde{L}}_{r , p} \delta _{k , m} - \Hat{\widetilde{L}}_{k , m} \delta _{p , r} + C_{k} \delta _{k , m} + D_{r} \delta _{p , r} \right) \delta _{s , n} = 0 .
\end{multline*}
For $k = s$, $n = p$ and $r = m$,
\begin{multline*}
    \left( \Hat{\widetilde{L}}_{r , r} - \Hat{\widetilde{L}}_{n , n} + C_{n} + D_{r} \right) - \left( \Hat{\widetilde{L}}_{k , r} \delta _{p , s} - \Hat{\widetilde{L}}_{n , k} \delta _{r , k} + C_{n} \delta _{n , k} + D_{k} \delta _{r , k} \right) \delta _{n , m} + \\
    + \left( \Hat{\widetilde{L}}_{k , k} - \Hat{\widetilde{L}}_{r , r} + C_{r} + D_{k} \right) - \left( \Hat{\widetilde{L}}_{n , k} \delta _{r , n} - \Hat{\widetilde{L}}_{r , n} \delta _{k , n} + C_{r} \delta _{r , n} + D_{n} \delta _{k , n} \right) \delta _{r , k} + \\
    + \left( \Hat{\widetilde{L}}_{n , n} - \Hat{\widetilde{L}}_{k , k} + C_{k} + D_{n} \right) - \left( \Hat{\widetilde{L}}_{r , n} \delta _{k , r} - \Hat{\widetilde{L}}_{k , r} \delta _{n , r} + C_{k} \delta _{k , r} + D_{r} \delta _{n , r} \right) \delta _{k , n} = 0 .
\end{multline*}
Assuming $n \neq m$, $r \neq k$ and $k \neq n$ leaves us with
\begin{equation*}
    C_{n} + C_{r} + C_{k} + D_{r} + D_{k} + D_{n} = 0 ,
\end{equation*}
which for $n + k + r = 0$ becomes
\begin{equation*}
    C_{n} + C_{k} + C_{- n - k} + D_{n} + D_{k} + D_{- n - k} = 0
\end{equation*}
\begin{equation*}
    C_{n} + C_{k} + C_{n + k} + D_{n} + D_{k} + D_{n + k} = 0 ,
\end{equation*}
and now setting $k = 1$ gives
\begin{equation*}
    C_{n} + C_{n + 1} + C_{1} + D_{n} + D_{n + 1} + D_{1} = 0 .
\end{equation*}
Since this is a linear difference eqn, we can define $F_{n} = C_{n} + D_{n}$ and so we have the eqn
\begin{equation*}
    F_{n + 1} + F_{n} + F_{1} = 0
\end{equation*}
The solution to this difference eqn with $F_{0} = 0$ is
\begin{equation*}
    F_{n} = \frac{F_{1}}{2} \cos \left( \frac{n \pi}{2} \right) - \frac{F_{1}}{2} = \frac{F_{1}}{2} \left( \cos \left( \frac{n \pi}{2} \right) - 1 \right) .
\end{equation*}
Since $C_{0} = D_{0} = 0$, we conclude that $C_{n}$ and $D_{n}$ are both equal to half $F_{n}$
\begin{align*}
    C_{n} & = \frac{F_{1}}{4} \left( \cos \left( \frac{n \pi}{2} \right) - 1 \right) \\
    D_{n} & = \frac{F_{1}}{4} \left( \cos \left( \frac{n \pi}{2} \right) - 1 \right)
\end{align*}
By now, the commutator of $\Hat{\widetilde{L}}$ is
\begin{equation*}
    \left[ \Hat{\widetilde{L}}_{n , m} , \Hat{\widetilde{L}}_{k , p} \right] = \Hat{\widetilde{L}}_{k , m} \delta _{n , p} - \Hat{\widetilde{L}}_{n , p} \delta _{m , k} + \frac{F_{1}}{4} \left( \cos \left( \frac{n \pi}{2} \right) - 1 \right) \delta _{n , p} + \frac{F_{1}}{4} \left( \cos \left( \frac{k \pi}{2} \right) - 1 \right) \delta _{m , k} .
\end{equation*}
Let's calculate the VEV of this commutator for $m = - k \neq 0$ with $n = p = 1$
\begin{equation*}
    \bra{0} \left[ \Hat{\widetilde{L}}_{1 , - k} , \Hat{\widetilde{L}}_{k , 1} \right] \ket{0} =\bra{0} \Hat{\widetilde{L}}_{1 , - k} \Hat{\widetilde{L}}_{k , 1} \ket{0}
\end{equation*}
\begin{equation*}
    \bra{0} \Hat{\widetilde{L}}_{k , - k} - \frac{F_{1}}{4} \ket{0} = \bra{0} (a^{\mu} _{- k})^{\dagger} a^{\nu} _{1} \eta _{\mu \nu} (a^{\mu'} _{1})^{\dagger} a^{\nu '} _{k} \eta _{\mu ' \nu '} \ket{0}
\end{equation*}
\begin{equation*}
    \bra{0} - \frac{F_{1}}{4} \ket{0} = 0 \iff F_{1} = 0 ,
\end{equation*}
so in actuality, the constraint algebra has no anomalies.

Let us now denote the ground state of momentum $p^{\mu}$ as $\ket{0 ; p}$. The mass-shell condition
\begin{equation*}
    \Hat{M}^2 = - \frac{8}{\alpha '} \left( \frac{2 \mu ^2}{\alpha '} \Hat{x}^{\mu} \Hat{x}_{\mu} - \left( \sum _{n \neq 0} \left( \left( 1 \mp \frac{n}{E_{n}} \right) ^2 E_{n} (\Hat{a}^{\mu} _{n})^{\dagger} \Hat{a}^{\nu} _{n} \eta _{\mu \nu} \right) - c \right) \right)
\end{equation*}
implies that
\begin{equation*}
    \bra{0 ; p} \Hat{M}^2 \ket{0 ; p} = - \frac{8}{\alpha '} \left( \frac{2 \mu ^2}{\alpha '} \bra{0 ; p} \Hat{x}^{\mu} \Hat{x}_{\mu} \ket{0 ; p} - \left( \sum _{n \neq 0} \left( \left( 1 \mp \frac{n}{E_{n}} \right) ^2 E_{n} \right) \bra{0 ; p} (\Hat{a}^{\mu} _{n})^{\dagger} \Hat{a}^{\nu} _{n} \eta _{\mu \nu} \ket{0 ; p} - \bra{0 ; p} c \ket{0 ; p} \right) \right)
\end{equation*}
\begin{equation*}
    M^2 = - \frac{8}{\alpha '} \left( \frac{2 \mu ^2}{\alpha '} \frac{\partial \psi ^{*} (p)}{\partial p_{\mu}} \frac{\partial \psi (p)}{\partial p^{\mu}} + c \right)
\end{equation*}
\begin{equation*}
    \frac{\alpha '}{8} p^2 = \left( \frac{2 \mu ^2}{\alpha '} \left| \frac{\partial \psi (p)}{\partial p} \right| ^2 + c \right)
\end{equation*}
Now looking at the first excited state $\zeta _{\mu} (\Hat{a}^{\mu} _{1})^{\dagger} \ket{0 ; p}$ with $\zeta _{\mu} = \zeta _{\mu} (p)$ being the polarization covector, the mass-shell now reads
\begin{multline*}
    \bra{0 ; p} \Hat{a}^{\mu} _{1} \zeta _{\mu} \Hat{M}^2 \zeta _{\nu} (\Hat{a}^{\nu} _{1})^{\dagger} \ket{0 ; p} = - \frac{8}{\alpha '} \Bigg( \frac{2 \mu ^2}{\alpha '} \bra{0 ; p} \Hat{a}^{\mu} _{1} \zeta _{\mu} \Hat{x}^{\mu '} \Hat{x}_{\mu '} \zeta _{\nu} (\Hat{a}^{\nu} _{1})^{\dagger} \ket{0 ; p} - \\
    - \left( \sum _{n \neq 0} \left( \left( 1 \mp \frac{n}{E_{n}} \right) ^2 E_{n} \right) \bra{0 ; p} \Hat{a}^{\mu} _{1} \zeta _{\mu} (\Hat{a}^{\mu '} _{n})^{\dagger} \Hat{a}^{\nu '} _{n} \eta _{\mu ' \nu '} \zeta _{\nu} (\Hat{a}^{\nu} _{1})^{\dagger} \ket{0 ; p} - \bra{0 ; p} \Hat{a}^{\mu} _{1} \zeta _{\mu} c \zeta _{\nu} (\Hat{a}^{\nu} _{1})^{\dagger} \ket{0 ; p} \right) \Bigg)
\end{multline*}
\begin{multline*}
    M^2 \zeta ^{\mu} \zeta _{\mu} = - \frac{8}{\alpha '} \Bigg( \frac{2 \mu ^2}{\alpha '} \left| \frac{\partial \psi (p)}{\partial p} \right| ^2 \zeta ^{\mu} \zeta _{\mu} - \Bigg( \sum _{n \neq 0} \left( \left( 1 + \frac{n}{E_{n}} \right) ^2 E_{n} \right) \bra{0 ; p} \Hat{a}^{\mu} _{1} \zeta _{\mu} (\Hat{a}^{\mu '} _{n})^{\dagger} \eta ^{\nu ' \nu} \delta _{|n| , 1} \eta _{\mu ' \nu '} \zeta _{\nu} \ket{0 ; p} - c \zeta ^{\mu} \zeta _{\mu} \Bigg) \Bigg)
\end{multline*}
\begin{equation*}
    M^2 \zeta ^{\mu} \zeta _{\mu} = - \frac{8}{\alpha '} \left( \frac{2 \mu ^2}{\alpha '} \left| \frac{\partial \psi (p)}{\partial p} \right| ^2 \zeta ^{\mu} \zeta _{\mu} - \left( \left( 1 - \frac{1}{\sqrt{1 + \mu ^2}} \right) ^2 + \left( 1 + \frac{1}{\sqrt{1 + \mu ^2}} \right) ^2 \right) \sqrt{1 + \mu ^2} \zeta ^{\mu} \zeta _{\mu} + c \zeta ^{\mu} \zeta _{\mu} \right)
\end{equation*}
\begin{equation*}
    M^2 = - \frac{8}{\alpha '} \left( \frac{2 \mu ^2}{\alpha '} \left| \frac{\partial \psi (p)}{\partial p} \right| ^2 - \left( 1 + \frac{1}{1 + \mu ^2} \right) \sqrt{1 + \mu ^2} + c \right)
\end{equation*}
The auxiliary $\bra{0 ; p} \Hat{\widetilde{L}}_{1 , 0} \zeta _{\mu '} (\Hat{a}^{\mu '} _{1})^{\dagger} \ket{0 ; p} = 0$ condition implies
\begin{equation*}
    \bra{0 ; p} \Hat{\widetilde{L}}_{1 , 0} \zeta _{\mu '} (\Hat{a}^{\mu '} _{1})^{\dagger} \ket{0 ; p} = 0
\end{equation*}
\begin{equation*}
    \bra{0 ; p} (\Hat{a}^{\mu} _{0})^{\dagger} \Hat{a}^{\nu} _{1} \eta _{\mu \nu} \zeta _{\mu '} (\Hat{a}^{\mu '} _{1})^{\dagger} \ket{0 ; p} = 0
\end{equation*}
\begin{equation*}
    \bra{0 ; p} \zeta _{\mu} (\Hat{a}^{\mu} _{0})^{\dagger} \ket{0 ; p} = 0
\end{equation*}
\begin{equation*}
    \bra{0 ; p} \zeta _{\mu} \left( \sqrt{\frac{2}{\alpha '}} \mu \Hat{x}^{\mu} - \frac{1}{2} \sqrt{\frac{\alpha '}{2}} i \Hat{p}^{\mu} \right) \ket{0 ; p} = 0
\end{equation*}
\begin{equation*}
    - \sqrt{\frac{2}{\alpha '}} \mu i \zeta _{\mu} \frac{\partial \psi (p)}{\partial p_{\mu}} = \frac{1}{2} \sqrt{\frac{\alpha '}{2}} i \zeta _{\mu} p^{\mu}
\end{equation*}
\begin{equation*}
    \zeta _{\mu} p^{\mu} = - \frac{4 \mu}{\alpha '} \zeta _{\mu} \frac{\partial \psi (p)}{\partial p_{\mu}} .
\end{equation*}
If the momentum wave function is chosen to be (need to justify this)
\begin{equation*}
    \psi (p) = \frac{\alpha '}{8 \sqrt{2} \mu} p^{\mu} p_{\mu} ,
\end{equation*}
we get that $\zeta _{\mu} p^{\mu} = 0$. Using this in the mass-shell condition for the first excited state yields
\begin{equation*}
    \frac{\alpha '}{8} p^2 = \frac{2 \mu ^2}{\alpha '} \frac{(\alpha ')^2}{32 \mu ^2} p^2 - \left( 1 + \frac{1}{1 + \mu ^2} \right) \sqrt{1 + \mu ^2} + c
\end{equation*}
\begin{equation*}
    \frac{\alpha '}{16} p^2 = c - \left( 1 + \frac{1}{1 + \mu ^2} \right) \sqrt{1 + \mu ^2} ,
\end{equation*}
and for the ground state
\begin{equation*}
    \frac{\alpha '}{16} p^2 = c
\end{equation*}
\begin{equation*}
    \frac{\alpha '}{16} p^2 = - D \sum _{k > 0} \left( 1 + \frac{k^2}{E_{k} ^2} \right) E_{k} .
\end{equation*}
In general, the $n$-th excited state will have momentum given by
\begin{equation*}
    \frac{\alpha '}{16} p^2 = c - \left( 1 + \frac{n^2}{n^2 + \mu ^2} \right) \sqrt{n^2 + \mu ^2}
\end{equation*}
\begin{equation*}
    \frac{\alpha '}{16} p^2 = - D \sum _{k > 0} \left( 1 + \frac{k^2}{E_{k} ^2} \right) E_{k} - \left( 1 + \frac{n^2}{n^2 + \mu ^2} \right) \sqrt{n^2 + \mu ^2} .
\end{equation*}
Since the norm of these states is given by $\zeta ^{\mu} \zeta _{\mu}$ and $\zeta _{\mu} p^{\mu} = 0$, we see that, in fact, we have no negative norm states since the R.H.S of the above equation is always negative, and the ground state scalar is a non-tachyonic massive particle. This analysis, however, is to get spin 1 particles from the closed string. The actual particle we want from closed strings is a symmetric spin 2 particle, so let's consider now the state $\zeta _{\mu \nu} (\Hat{a}^{\mu} _{d})^{\dagger} (\Hat{a}^{\nu} _{d})^{\dagger} \ket{0 ; p}$ where $\zeta _{\mu \nu} = \zeta _{\nu \mu}$ with $\eta ^{\mu \nu} \zeta _{\mu \nu}$ is the polarization tensor. The mass-shell is
\begin{multline*}
    \bra{0 ; p} \Hat{a}^{\alpha} _{d} \Hat{a}^{\beta} _{d} \zeta _{\alpha \beta} \Hat{M}^2 \zeta _{\mu \nu} (\Hat{a}^{\mu} _{d})^{\dagger} (\Hat{a}^{\nu} _{d})^{\dagger} \ket{0 ; p} = - \frac{8}{\alpha '} \Bigg( \frac{2 \mu ^2}{\alpha '} \bra{0 ; p} \Hat{a}^{\alpha} _{d} \Hat{a}^{\beta} _{d} \zeta _{\alpha \beta} \Hat{x}^{\mu '} \Hat{x}_{\mu '} \zeta _{\mu \nu} (\Hat{a}^{\mu} _{d})^{\dagger} (\Hat{a}^{\nu} _{d})^{\dagger} \ket{0 ; p} - \\
    - \Bigg( \sum _{n \neq 0} \left( \left( 1 + \frac{n}{E_{n}} \right) ^2 E_{n} \right) \bra{0 ; p} \Hat{a}^{\alpha} _{d} \Hat{a}^{\beta} _{d} \zeta _{\alpha \beta} (\Hat{a}^{\mu '} _{n})^{\dagger} \Hat{a}^{\nu '} _{n} \eta _{\mu ' \nu '} \zeta _{\mu \nu} (\Hat{a}^{\mu} _{d})^{\dagger} (\Hat{a}^{\nu} _{d})^{\dagger} \ket{0 ; p} - \bra{0 ; p} \Hat{a}^{\alpha} _{d} \Hat{a}^{\beta} _{d} \zeta _{\alpha \beta} c \zeta _{\mu \nu} (\Hat{a}^{\mu} _{d})^{\dagger} (\Hat{a}^{\nu} _{d})^{\dagger} \ket{0 ; p} \Bigg) \Bigg)
\end{multline*}
\begin{equation*}
    \sum _{n \neq 0} \left( \left( 1 + \frac{n}{E_{n}} \right) ^2 E_{n} \right) \bra{0 ; p} \Hat{a}^{\alpha} _{d} \Hat{a}^{\beta} _{d} \zeta _{\alpha \beta} (\Hat{a}^{\mu '} _{n})^{\dagger} \Hat{a}^{\nu '} _{n} \eta _{\mu ' \nu '} \zeta _{\mu \nu} (\Hat{a}^{\mu} _{d})^{\dagger} (\Hat{a}^{\nu} _{d})^{\dagger} \ket{0 ; p} =
\end{equation*}
\begin{equation*}
    = \sum _{n \neq 0} \left( \left( 1 + \frac{n}{E_{n}} \right) ^2 E_{n} \right) \bra{0 ; p} \Hat{a}^{\alpha} _{d} \Hat{a}^{\beta} _{d} \zeta _{\alpha \beta} (\Hat{a}^{\mu '} _{n})^{\dagger} \eta _{\mu ' \nu '} \zeta _{\mu \nu} (\eta ^{\nu ' \mu} \delta _{|n| , |d|} + (\Hat{a}^{\mu} _{d})^{\dagger} \Hat{a}^{\nu '} _{n}) (\Hat{a}^{\nu} _{d})^{\dagger} \ket{0 ; p} =
\end{equation*}
\begin{equation*}
    = 2 \left( \left( 1 + \frac{d^2}{E_{d} ^2} \right) E_{d} \right) \bra{0 ; p} \Hat{a}^{\alpha} _{d} \Hat{a}^{\beta}_{d} \zeta _{\alpha \beta} (\Hat{a}^{\mu} _{d})^{\dagger} (\Hat{a}^{\nu} _{d})^{\dagger} \zeta _{\mu \nu} \ket{0 ; p} =
\end{equation*}
\begin{equation*}
    = 2 \left( \left( 1 + \frac{d^2}{E_{d} ^2} \right) E_{d} \right) \bra{0 ; p} \Hat{a}^{\alpha} _{d} \zeta _{\alpha \beta} (\eta ^{\beta \mu} + (\Hat{a}^{\mu} _{d})^{\dagger} \Hat{a}^{\beta} _{d}) (\Hat{a}^{\nu} _{d})^{\dagger} \zeta _{\mu \nu} \ket{0 ; p} =
\end{equation*}
\begin{equation*}
    = 4 \left( \left( 1 + \frac{d^2}{E_{d} ^2} \right) E_{d} \right) \bra{0 ; p} \Hat{a}^{\alpha} _{d} (\Hat{a}^{\mu})^{\dagger} \zeta _{\alpha \beta} \zeta _{\mu \nu} \eta ^{\beta \nu} \ket{0 ; p} =
\end{equation*}
\begin{equation*}
    = 4 \left( \left( 1 + \frac{d^2}{E_{d} ^2} \right) E_{d} \right) \zeta _{\mu \nu} \zeta ^{\mu \nu}
\end{equation*}
\begin{equation*}
    M^2 \zeta _{\mu \nu} \zeta ^{\mu \nu} = - \frac{8}{\alpha '} \Bigg( \frac{2 \mu ^2}{\alpha '} \left| \frac{\partial \psi (p)}{\partial p} \right| ^2 \zeta _{\mu \nu} \zeta ^{\mu \nu} - \Bigg( 4 \left( \left( 1 + \frac{d^2}{d^2 + \mu ^2} \right) \sqrt{d^2 + \mu ^2} \right) \zeta _{\mu \nu} \zeta ^{\mu \nu} - c \zeta _{\mu \nu} \zeta ^{\mu \nu} \Bigg) \Bigg)
\end{equation*}
\begin{equation*}
    \frac{\alpha '}{8} p^2 = \Bigg( \frac{2 \mu ^2}{\alpha '} \left| \frac{\partial \psi (p)}{\partial p} \right| ^2 - \left( 4 \left( \left( 1 + \frac{d^2}{d^2 + \mu ^2} \right) \sqrt{d^2 + \mu ^2} \right) - c \right) \Bigg)
\end{equation*}
\begin{equation*}
    \frac{\alpha '}{16} p^2 = c - 4 \left( \left( 1 + \frac{d^2}{d^2 + \mu ^2} \right) \sqrt{d^2 + \mu ^2} \right)
\end{equation*}
\begin{equation*}
    \frac{\alpha '}{16} p^2 = - D \sum _{k > 0} \left( 1 + \frac{k^2}{E_{k} ^2} \right) E_{k} - 4 \left( \left( 1 + \frac{d^2}{d^2 + \mu ^2} \right) \sqrt{d^2 + \mu ^2} \right) .
\end{equation*}
Using the auxiliary $\bra{0 ; p} \Hat{\widetilde{L}}_{d , 0} \zeta _{\mu \nu} (\Hat{a}^{\mu} _{d})^{\dagger} (\Hat{a}^{\nu} _{d})^{\dagger} \ket{0 ; p} = 0$ condition implies
\begin{equation*}
    \bra{0 ; p} \Hat{\widetilde{L}}_{d , 0} \zeta _{\mu \nu} (\Hat{a}^{\mu} _{d})^{\dagger} (\Hat{a}^{\nu} _{d})^{\dagger} \ket{0 ; p} = 0
\end{equation*}
\begin{equation*}
    \bra{0 ; p} (\Hat{a}^{\mu '} _{0})^{\dagger} \Hat{a}^{\nu '} _{d} \eta _{\mu ' \nu '} \zeta _{\mu \nu} (\Hat{a}^{\mu} _{d})^{\dagger} (\Hat{a}^{\nu} _{d})^{\dagger} \ket{0 ; p} = 0
\end{equation*}
\begin{equation*}
    \bra{0 ; p} \left( (\Hat{a}^{\mu '} _{0})^{\dagger} \eta _{\mu ' \nu '} \zeta _{\mu \nu} \eta ^{\nu ' \mu} (a^{\nu} _{d})^{\dagger} + (\Hat{a}^{\mu '} _{0})^{\dagger} \eta _{\mu ' \nu '} \zeta _{\mu \nu} (\Hat{a}^{\mu} _{d})^{\dagger} \Hat{a}^{\nu '} _{d} (\Hat{a}^{\nu} _{d})^{\dagger} \right) \ket{0 ; p} = 0
\end{equation*}
\begin{equation*}
    \bra{0 ; p} \left( (\Hat{a}^{\mu '} _{0})^{\dagger} \eta _{\mu ' \nu '} \zeta _{\mu \nu} \eta ^{\nu ' \mu} (a^{\nu} _{d})^{\dagger} + (\Hat{a}^{\mu '} _{0})^{\dagger} \eta _{\mu ' \nu '} \zeta _{\mu \nu} (\Hat{a}^{\mu} _{d})^{\dagger} \eta ^{\nu ' \nu} \right) \ket{0 ; p} = 0
\end{equation*}
\begin{equation*}
    \bra{0 ; p} 2 \zeta _{\mu \nu} (\Hat{a}^{\mu} _{0})^{\dagger} (\Hat{a}^{\nu} _{d})^{\dagger} \ket{0 ; p} = 0
\end{equation*}
\begin{equation*}
    \bra{0 ; p} \zeta _{\mu \nu} (\Hat{a}^{\nu} _{d})^{\dagger} \left( \sqrt{\frac{2}{\alpha '}} \mu \Hat{x}^{\mu} - \frac{1}{2} \sqrt{\frac{\alpha '}{2}} i \Hat{p}^{\mu} \right) \ket{0 ; p} = 0
\end{equation*}
\begin{equation*}
    \bra{0} \zeta _{\mu \nu} \left( - \sqrt{\frac{2}{\alpha '}} \mu i \frac{\partial \psi (p)}{\partial p_{\mu}} - \frac{1}{2} \sqrt{\frac{\alpha '}{2}} i p^{\mu} \right) (\Hat{a}^{\nu} _{d})^{\dagger} \ket{0} = 0
\end{equation*}
\begin{equation*}
    \bra{0} \zeta _{\mu \nu} p^{\mu} (\Hat{a}^{\nu} _{d})^{\dagger} \ket{0} = 0 ,
\end{equation*}
implying that
\begin{equation*}
    \zeta _{\mu \nu} p^{\mu} = \zeta _{\mu \nu} p^{\nu} = 0 .
\end{equation*}
Since the norm of the spin-2 states is given by $\zeta ^{\mu \nu} \zeta _{\mu \nu}$, we need those to not be negative. Since $p^2 < 0$ and $\zeta _{\mu \nu} p^{\mu} = 0$,

The fact that $\zeta _{\mu \nu} = \zeta _{\nu \mu}$ leaves $\zeta _{\mu \nu}$ with $D (D + 1) / 2$ independent components. The traceless condition $\eta ^{\mu \nu} \zeta _{\mu \nu} = 0$ takes out 1 from that, leaving $(D (D + 1) / 2) - 1$ allowed polarizations. Finally, the condition $\zeta _{\mu \nu} p ^{\mu} = 0$ takes $D$ out of those, leaving $(D (D - 1) / 2) - 1$ allowed polarizations.
\begin{equation*}
    \frac{D (D - 1)}{2} - 1 = \frac{D^2 - D - 2}{2} = \frac{(D - 2) (D + 1)}{2} .
\end{equation*}
This leaves 5 polarizations in $D = 4$.

We now consider the commutators
\begin{equation*}
    \left[ \sum _{n} \Hat{\widetilde{L}}_{n , n} , \Hat{X}^{\mu} \right] = \left[ \sum _{n} (\Hat{\alpha}^{\mu '} _{n})^{\dagger} \Hat{\alpha}^{\nu '} _{n} \eta _{\mu ' \nu '} , \sqrt{\frac{\alpha '}{2}} \sum _{p} \frac{1}{2 E_{p}} \left( \Hat{\alpha}^{\mu} _{p} \e ^{i (E_{p} \tau - p \sigma)} + (\Hat{\alpha}^{\mu} _{p})^{\dagger} \e ^{- i (E_{p} \tau - p \sigma)} \right) \right] =
\end{equation*}
\begin{equation*}
    = \sqrt{\frac{\alpha '}{2}} \sum _{n} \sum _{p} \frac{1}{2 E_{p}} \Bigg( \left[ (\Hat{\alpha}^{\mu '} _{n})^{\dagger} , \Hat{\alpha}^{\mu} _{p} \right] \Hat{\alpha}^{\nu '} _{n} \eta _{\mu ' \nu '} \e ^{i (E_{p} \tau - p \sigma)} + (\Hat{a}^{\mu '} _{n})^{\dagger} \left[ \Hat{\alpha}^{\nu '} _{n} , (\Hat{\alpha}^{\mu} _{p})^{\dagger} \right] \eta _{\mu ' \nu '} \e ^{- i (E_{p} \tau - p \sigma)} \Bigg) =
\end{equation*}
\begin{equation*}
    = \sqrt{\frac{\alpha '}{2}} \sum _{n} \sum _{p} \frac{1}{2 E_{p}} \Bigg( (- E_{p} \eta ^{\mu ' \mu} \delta _{|n| , |p|}) \Hat{\alpha}^{\nu '} _{n} \eta _{\mu ' \nu '} \e ^{i (E_{p} \tau - p \sigma)} + (\Hat{\alpha}^{\mu '} _{n})^{\dagger} (E_{n} \eta ^{\nu ' \mu} \delta _{|n| , |p|}) \eta _{\mu ' \nu '} \e ^{- i (E_{p} \tau - p \sigma)} \Bigg) =
\end{equation*}
\begin{equation*}
    = \sqrt{\frac{\alpha '}{2}} \sum _{p} \Bigg( - \Hat{\alpha}^{\mu} _{p} \e ^{i (E_{p} \tau - p \sigma)} + (\Hat{\alpha}^{\mu} _{p})^{\dagger} \e ^{- i (E_{p} \tau - p \sigma)} \Bigg) = 2 i \del _{\tau} \Hat{X}^{\mu}
\end{equation*}
\begin{equation*}
    \Downarrow
\end{equation*}
\begin{equation*}
    \left[ - \frac{1}{2} \sum _{n} \Hat{\widetilde{L}}_{n , n} , \Hat{X}^{\mu} \right] = - i \del _{\tau} \Hat{X}^{\mu}
\end{equation*}

\begin{equation*}
    \left[ - \frac{1}{2} \sum _{n} \frac{n}{E_{n}} \Hat{\widetilde{L}}_{n , n} , \Hat{X}^{\mu} \right] = i \del _{\sigma} \Hat{X}^{\mu} ,
\end{equation*}
thus we see that
\begin{equation*}
    \Hat{H} = - \frac{1}{2} \sum _{n} \Hat{\widetilde{L}}_{n , n}
\end{equation*}
is the WS Hamiltonian generating $\tau$ translations by $\Hat{X}^{\mu} (\tau + \tau _{0} , \sigma) = \e ^{- i \Hat{H} \tau _{0}} \Hat{X}^{\mu} (\tau , \sigma) \e ^{i \Hat{H} \tau _{0}}$ and
\begin{equation*}
    \Hat{P} = - \frac{1}{2} \sum _{n} \frac{n}{E_{n}} \Hat{\widetilde{L}}_{n , n}
\end{equation*}
is the generator of $\sigma$ translations by $\Hat{X}^{\mu} (\tau , \sigma + \sigma _{0}) = \e ^{- i \Hat{P} \sigma _{0}} \Hat{X}^{\mu} (\tau , \sigma) \e ^{i \Hat{P} \sigma _{0}}$.

\newpage

\section{WS Kalb-Ramond}

Considering the more complete action
\begin{equation*}
    S = - \frac{1}{4 \pi \alpha '} \int \diff ^2 x \left( \sqrt{- g} - \frac{1}{2} \frac{k \Delta}{\sqrt{- g}} \right) \left( g^{a b} \del _{a} X^{\mu} \del _{b} X^{\nu} G_{\mu \nu} - \widetilde{\varepsilon}^{a b} \del _{a} X^{\mu} \del _{b} X^{\nu} B_{\mu \nu} \right) =
\end{equation*}
\begin{equation*}
    = - \frac{1}{4 \pi \alpha '} \int \diff ^2 x \left( \sqrt{- g} - \frac{1}{2} \frac{k \Delta}{\sqrt{- g}} \right) \left( g^{a b} \del _{a} X^{\mu} \del _{b} X^{\nu} G_{\mu \nu} - \sqrt{\Delta} \Delta ^{a b} \del _{a} X^{\mu} \del _{b} X^{\nu} B_{\mu \nu} \right) .
\end{equation*}

\begin{align*}
    \frac{\delta S}{\delta \Delta ^{c d}} & \propto - \frac{1}{2} \frac{k}{\sqrt{- g}} \Delta \Delta _{c d} \left( g^{a b} \del _{a} X^{\mu} \del _{b} X^{\nu} G_{\mu \nu} - \sqrt{\Delta} \Delta ^{a b} \del _{a} X^{\mu} \del _{b} X^{\nu} B_{\mu \nu} \right) -                                                                           \\
                                          & - \left( \sqrt{- g} - \frac{1}{2} \frac{k \Delta}{\sqrt{- g}} \right) \left( \frac{1}{2} \sqrt{\Delta} \Delta _{c d} \Delta ^{a b} \del _{a} X^{\mu} \del _{b} X^{\nu} B_{\mu \nu} + \sqrt{\Delta} \delta ^{a} _{c} \delta ^{b} _{d} \del _{a} X^{\mu} \del _{b} X^{\nu} B_{\mu \nu} \right) = 0
\end{align*}
\begin{multline*}
    \frac{k}{2} \sqrt{\Delta} \Delta _{c d} \left( g^{a b} \del _{a} X^{\mu} \del _{b} X^{\nu} G_{\mu \nu} - \sqrt{\Delta} \Delta ^{a b} \del _{a} X^{\mu} \del _{b} X^{\nu} B_{\mu \nu} \right) - \\
    - \left( g + \frac{k \Delta}{2} \right) \left( \frac{1}{2} \Delta _{c d} \Delta ^{a b} \del _{a} X^{\mu} \del _{b} X^{\nu} B_{\mu \nu} + \del _{c} X^{\mu} \del _{d} X^{\nu} B_{\mu \nu} \right) = 0
\end{multline*}
\begin{equation*}
    \Delta _{c d} \left( \frac{k \sqrt{\Delta}}{2} g^{a b} \del _{a} X^{\mu} \del _{b} X^{\nu} G_{\mu \nu} - \left( g + \frac{3 k \Delta}{4} \right) \Delta ^{a b} \del _{a} X^{\mu} \del _{b} X^{\nu} B_{\mu \nu} \right) - \left( g + \frac{k \Delta}{2} \right) \del _{c} X^{\mu} \del _{d} X^{\nu} B_{\mu \nu} = 0
\end{equation*}
\begin{equation*}
    \Delta _{c d} = \widetilde{f} \del _{c} X^{\mu} \del _{d} X^{\nu} B_{\mu \nu} ,
\end{equation*}
\begin{equation*}
    \frac{1}{\widetilde{f}} = \frac{1}{g + \dfrac{k \Delta}{2}} \left( \frac{k \sqrt{\Delta}}{2} g^{a b} \del _{a} X^{\mu} \del _{b} X^{\nu} G_{\mu \nu} - \left( g + \frac{3 k \Delta}{4} \right) \Delta ^{a b} \del _{a} X^{\mu} \del _{b} X^{\nu} B_{\mu \nu} \right) ,
\end{equation*}
if $g_{a b} = \del _{a} X^{\mu} \del _{b} X^{\nu} G_{\mu \nu}$ then
\begin{equation*}
    \frac{1}{\widetilde{f}} = \frac{1}{g + \dfrac{k \Delta}{2}} \left( {k \sqrt{\Delta}} - \left( g + \frac{3 k \Delta}{4} \right) \Delta ^{a b} \del _{a} X^{\mu} \del _{b} X^{\nu} B_{\mu \nu} \right)
\end{equation*}
\begin{equation*}
    \frac{1}{\widetilde{f}} = \frac{1}{g + \dfrac{k \Delta}{2}} \left( {k \sqrt{\Delta}} - \left( g + \frac{3 k \Delta}{4} \right) \Delta ^{a b} \frac{1}{\widetilde{f}} \Delta _{a b} \right)
\end{equation*}
\begin{equation*}
    \frac{1}{\widetilde{f}} \left( g + \frac{k \Delta}{2} \right) - \frac{1}{\widetilde{f}} \left( 2 g + \frac{3 k \Delta}{2} \right) = k \sqrt{\Delta}
\end{equation*}
\begin{equation*}
    - \frac{1}{\widetilde{f}} \left( g + k \Delta \right) = k \sqrt{\Delta}
\end{equation*}
\begin{equation*}
    \frac{1}{\widetilde{f}} = - \frac{k \sqrt{\Delta}}{g + k \Delta}
\end{equation*}
\begin{equation*}
    \widetilde{f} = - \frac{g + k \Delta}{k \sqrt{\Delta}} .
\end{equation*}
From this, one can see that the pullback of the Kalb-Ramond field is
\begin{equation*}
    b_{a b} = \del _{a} X^{\mu} \del _{b} X^{\nu} B_{\mu \nu} = - \frac{k \Delta}{g + k \Delta} \widetilde{\varepsilon}_{a b} = - \frac{1}{\dfrac{g}{k \Delta} + 1} \widetilde{\varepsilon}_{a b} ,
\end{equation*}
and also that
\begin{equation*}
    \Delta _{a b} = \sqrt{\Delta} \widetilde{\varepsilon}_{a b} = - \sqrt{\Delta} \left( \frac{g}{k \Delta} + 1 \right) b_{a b}
\end{equation*}

\newpage

\section{Polyakov Action in terms of LQG Variables}

Turn embbeding fields $X^{\mu}$ into vector in the spin-1 representation of $\R ^{d + 1} \rtimes \mathrm{Spin}(d , 1)$, $X^{I}$, and promote partial derivative to covariant derivative $\partial _{a} \mapsto \mc{D}_{a}$ acting as
\begin{equation*}
    \mc{D}_{a} X^{I} = \partial _{a} X^{I} + k \mc{A}^{I} _{a J} X^{J} ,
\end{equation*}
where the WS $\R ^{d + 1} \rtimes \mathrm{Spin}(d , 1)$ connection $\mc{A}_{a}$ with $I,J = 0 , ... , d , d + 1$ is given by
\begin{equation*}
    \mc{A}_{a} X = \begin{bmatrix} 0                         & \beta ^{0 1} _{a}       & \beta ^{0 2} _{a}       & ... & \beta ^{0 d} _{a}       & p^{0} _{a} / l \\ \beta ^{0 1} _{a} & 0 & - \theta ^{1 2} _{a} & ... & - \theta ^{1 d} _{a} & p^{1} _{a} / l \\ \beta ^{0 2} _{a} & \theta ^{1 2} _{a} & 0 & ... & - \theta ^{2 d} _{a} & p^{2} _{a} / l \\ \vdots & \vdots & \vdots & \ddots & \vdots & \vdots \\ \beta ^{0 d} _{a} & \theta ^{1 d} _{a} & \theta ^{2 d} _{a} & ... & 0 & p^{d} _{a} / l \\
                - \epsilon p^{0} _{a} / l & \epsilon p^{1} _{a} / l & \epsilon p^{2} _{a} / l & ... & \epsilon p^{d} _{a} / l & 0\end{bmatrix} \begin{bmatrix} X^{0} \\ X^{1} \\ X^{2} \\ \vdots \\ X^{d} \\ 1 \end{bmatrix} = \left( \beta ^{0 \alpha} _{a} B_{0 \alpha} + \frac{1}{2} \theta ^{\alpha \beta} _{a} J_{\alpha \beta} + \frac{p^{\mu} _{a}}{l} P_{\mu} \right) X ,
\end{equation*}
where $\epsilon = \mathrm{sgn} (\Lambda)$, i.e the sign of the background cosmological constant with $\alpha , \beta = 1 , ... , d$ and $\mu = 0,...,d$, so the action becomes
\begin{equation*}
    S = - \frac{T}{2} \int \diff ^2 x \sqrt{- g} g^{a b} \mc{D}_{a} X^{I} \mc{D}_{b} X^{J} \eta _{I J} ,
\end{equation*}
where the metrics can be recast in terms of auxiliary and bulk vielbein fields as in
\begin{equation*}
    S = - \frac{T}{2} \int \diff ^2 x e e^{a} _{I'} e^{b} _{J'} \eta ^{I' J'} \mc{D}_{a} X^{I} \mc{D}_{b} X^{J} \eta _{I J} ,
\end{equation*}
where
\begin{equation*}
    e := \sqrt{- \det (e_{a} ^{I} e_{b} ^{J} \eta _{I J})} = \sqrt{- \frac{1}{2} \varepsilon ^{f f'} \varepsilon ^{g g'} (e^{M} _{f} e^{N} _{g} \eta _{M N}) (e^{M'} _{f'} e^{N'} _{g'} \eta _{M' N'})} .
\end{equation*}

\subsection{Checking the derivative}

Since the connection is non-abelian, it should transform as
\begin{equation*}
    \mc{A}' _{a} = \mc{P} \mc{A}_{a} \mc{P}^{-1} - \frac{1}{k} (\partial _{a} \mc{P}) \mc{P}^{-1} ,
\end{equation*}
where $\mc{P} = \e ^{\beta ^{0 \alpha} B_{0 \alpha} + \theta ^{\alpha \beta} J_{\alpha \beta} / 2 + p^{\mu} P_{\mu} / l}$. Putting this into a gauge transformed derivative we have
\begin{equation*}
    \mc{D}' _{a} X' = \partial _{a} X' + k \mc{A}' _{a} X' =
\end{equation*}
\begin{equation*}
    = \partial _{a} (\mc{P} X) + k (\mc{P} \mc{A}_{a} \mc{P}^{-1} - \frac{1}{k} (\partial _{a} \mc{P}) \mc{P}^{-1}) (\mc{P} X) =
\end{equation*}
\begin{equation*}
    = \partial _{a} \mc{P} X + \mc{P} \partial _{a} X + k \mc{P} \mc{A}_{a} X - \partial _{a} \mc{P} X \equiv \mc{P} \mc{D}_{a} X ,
\end{equation*}
so this is a proper covariant derivative for the Poincaré group.

Since under gauge transformations $\mc{D}_{a} X$ transforms covariantly, the action is not invariant under diffeomorphisms. In fact, it transforms as
\begin{equation*}
    S' = - \frac{T}{2} \int \diff ^2 x e' e^{\prime a} _{I'} e^{\prime b} _{J'} \eta ^{I' J'} \mc{D}' _{a} X^{\prime I} \mc{D}' _{b} X^{\prime J} \eta _{I J} =
\end{equation*}
\begin{equation*}
    = - \frac{T}{2} \int \diff ^2 x e e^{a} _{K'} (\PS ^{-1})^{K'} _{\ \ I'} e^{b} _{L'} (\PS ^{-1})^{L'} _{\ \ J'} \eta ^{I' J'} \PS ^{I} _{\ K} \mc{D}_{a} X^{K} \PS ^{J} _{\ L} \mc{D}_{b} X^{L} \eta _{I J} ,
\end{equation*}
so we promote the killing forms $\eta _{I J}$ and $\eta ^{I' J'}$ to doubly co(ntra)-variant forms $K_{I J}$ and $K^{I' J'}$ such that $K' _{I J} = (\PS ^{-1})^{M} _{\ \ I} (\PS ^{-1})^{N} _{\ \ J} K_{M N}$ and $K^{\prime I' J'} = \PS ^{I'} _{\ M'} \PS ^{J'} _{\ N'} K^{M' N'}$, so the new action is
\begin{equation*}
    S = - \frac{T}{2} \int \diff ^2 x e e^{a} _{I'} e^{b} _{J'} K^{I' J'} \mc{D}_{a} X^{I} \mc{D}_{b} X^{J} K_{I J} .
\end{equation*}
For consistency, let's assume $K$ is a background field $K = K (X)$ like the metric $G$.

\subsection{EoMs}

\subsubsection{W.r.t $e$}

\begin{align*}
    \frac{\delta S}{\delta e^{c} _{K}} & \propto \left( \frac{\partial e}{\partial e^{c} _{K}} e^{a} _{I'} e^{b} _{J'} K^{I' J'} + e \frac{\partial}{\partial e^{c} _{K}} (e^{a} _{I'} e^{b} _{J'} K^{I' J'}) \right) \mc{D}_{a} X^{I} \mc{D}_{b} X^{J} K_{I J} = \\
                                       & = \left( - e e^{K} _{c} e^{a} _{I'} e^{b} _{J'} K^{I' J'} + 2 e \delta ^{a} _{c} \delta ^{K} _{I'} e^{b} _{J'} K^{I' J'} \right) \mc{D}_{a} X^{I} \mc{D}_{b} X^{J} K_{I J} =                                             \\
                                       & = - e e^{K} _{c} e^{a} _{I'} e^{b} _{J'} K^{I' J'} \mc{D}_{a} X^{I} \mc{D}_{b} X^{J} K_{I J} + 2 e e^{b} _{J'} K^{K J'} \mc{D}_{c} X^{I} \mc{D}_{b} X^{J} K_{I J} = 0
\end{align*}
\begin{equation*}
    T^{K} _{c} := 2 e^{b} _{J'} K^{K J'} \mc{D}_{c} X^{I} \mc{D}_{b} X^{J} K_{I J} - e^{K} _{c} e^{a} _{I'} e^{b} _{J'} K^{I' J'} \mc{D}_{a} X^{I} \mc{D}_{b} X^{J} K_{I J} = 0
\end{equation*}
\begin{equation*}
    e^{K} _{c} = 2 f e^{b K} \mc{D}_{c} X^{I} \mc{D}_{b} X^{J} K_{I J} ,
\end{equation*}
\begin{equation*}
    \frac{1}{f} = e^{a} _{I'} e^{b} _{J'} K^{I' J'} \mc{D}_{a} X^{I} \mc{D}_{b} X^{J} K_{I J}
\end{equation*}

\subsubsection{W.r.t $A$}

\begin{align*}
    \frac{\delta S}{\delta \mc{A}^{K L} _{c}} & = \frac{T}{2} \left( e e^{a} _{I'} e^{b} _{J'} K^{I' J'} \frac{\partial}{\partial \mc{A}^{K L} _{c}} \left( \mc{D}_{a} X^{I} \mc{D}_{b} X^{J} \right) K_{I J} \right) = \\
                                              & = T \left( e e^{a} _{I'} e^{b} _{J'} K^{I' J'} \delta ^{c} _{a} \delta ^{I} _{[K} \eta _{L] I'} X^{I'} \mc{D}_{b} X^{J} K_{I J} \right) =                               \\
                                              & = T e e^{c} _{I'} e^{b} _{J'} K^{I' J'} \mc{D}_{b} X_{[K} X_{L]} \overset{!}{=} 0
\end{align*}
\begin{equation*}
    \mc{T}^{I J} _{a} = \mc{D}_{a} X^{[I} X^{J]} = 0
\end{equation*}

\subsubsection{W.r.t $X$}

\begin{align*}
    \frac{\delta S}{\delta X^{K}} & = - \frac{T}{2} \Bigg( \partial _{c} \left( e e^{a} _{I'} e^{b} _{J'} K^{I' J'} \frac{\partial}{\partial (\partial _{c} X^{K})} \left( \mc{D}_{a} X^{I} \mc{D}_{b} X^{J} \right) K_{I J} \right) - e e^{a} _{I'} e^{b} _{J'} K^{I' J'} \frac{\partial}{\partial X^{K}} \left( \mc{D}_{a} X^{I} \mc{D}_{b} X^{J} \right) K_{I J} - \\
                                  & - e e^{a} _{I'} e^{b} _{J'} \frac{\del K^{I' J'}}{\del X^{K}} \mc{D}_{a} X^{I} \mc{D}_{b} X^{J} K_{I J} - e e^{a} _{I'} e^{b} _{J'} K^{I' J'} \mc{D}_{a} X^{I} \mc{D}_{b} X^{J} \frac{\del K_{I J}}{\del X^{K}} \Bigg) =                                                                                                          \\
                                  & = - T \bigg( \left( \partial _{c} \left( e e^{a} _{I'} e^{b} _{J'} K^{I' J'} \delta ^{c} _{a} \delta ^{I} _{K} \mc{D}_{b} X^{J} K_{I J} \right) - e e^{a} _{I'} e^{b} _{J'} K^{I' J'} k \mc{A}^{I} _{a L} \delta ^{I'} _{L} \mc{D}_{b} X^{J} K_{I J} \right) -                                                                    \\
                                  & - e e^{a} _{I'} e^{b} _{J'} \del _{K} K^{I' J'} \mc{D}_{a} X^{I} \mc{D}_{b} X^{J} K_{I J} - e e^{a} _{I'} e^{b} _{J'} K^{I' J'} \mc{D}_{a} X^{I} \mc{D}_{b} X^{J} \del _{K} K_{I J} \bigg) =                                                                                                                                      \\
                                  & = - T \left( \mc{D}_{a} \left( e e^{a} _{I'} e^{b} _{J'} K^{I' J'} \mc{D}_{b} X^{I} K_{I K} \right) - e e^{a} _{I'} e^{b} _{J'} \left( \del _{K} K^{I' J'} K_{I J} + K^{I' J'} \del _{K} K_{I J} \right) \mc{D}_{a} X^{I} \mc{D}_{b} X^{J} \right) \overset{!}{=} 0
\end{align*}
\begin{equation*}
    \mc{D}_{a} \left( e e^{a} _{I'} e^{b} _{J'} K^{I' J'} \mc{D}_{b} X^{I} K_{I K} \right) = e e^{a} _{I'} e^{b} _{J'} \left( \del _{K} K^{I' J'} K_{I J} + K^{I' J'} \del _{K} K_{I J} \right) \mc{D}_{a} X^{I} \mc{D}_{b} X^{J}
\end{equation*}
In \textquotedblleft constant gauge" $K_{I J} (X) = \eta _{I J}$ and in conformal gauge $e e^{a} _{I'} e^{b} _{J'} \eta ^{I' J'} = \eta ^{a b}$, this reduces to
\begin{equation*}
    \eta ^{a b} \mc{D}_{a} \mc{D}_{b} X^{I} = 0 .
\end{equation*}
More explicitly,
\begin{equation*}
    \eta ^{a b} \mc{D}_{a} \left( \partial _{b} X^{I} + k \mc{A}^{I} _{b J} X^{J} \right) = 0
\end{equation*}
\begin{equation*}
    \eta ^{a b} \left( \left( \partial _{a} \partial _{b} X^{I} + k \mc{A}^{I} _{a J} \partial _{b} X^{J} \right) + \left( k \partial _{a} \left( \mc{A}^{I} _{b J} X^{J} \right) + k^2 \mc{A}^{I} _{a I'} \mc{A}^{I'} _{b J} X^{J} \right) \right) = 0
\end{equation*}
\begin{equation*}
    \eta ^{a b} \left( \partial _{a} \partial _{b} X^{I} + 2 k \mc{A}^{I} _{a J} \partial _{b} X^{J} + k \partial _{a} \mc{A}^{I} _{b J} X^{J} + k^2 \mc{A}^{I} _{a I'} \mc{A}^{I'} _{b J} X^{J} \right) = 0
\end{equation*}
\begin{equation*}
    \eta ^{a b} \left( \delta ^{I} _{J} \partial _{a} \partial _{b} + 2 k \mc{A}^{I} _{a J} \partial _{b} \right) X^{J} = - \left( k \eta ^{a b} \partial _{a} \mc{A}^{I} _{b J} X^{J} + k^2 \eta ^{a b} \mc{A}^{I} _{a I'} \mc{A}^{I'} _{b J} X^{J} \right) ,
\end{equation*}
which has implicit solution given by
\begin{equation*}
    X^{K} (x) = - \int \diff ^2 x' G^{K} _{\ I} (x , x') \left( k \eta ^{a' b'} \partial _{a'} \mc{A}^{I} _{b' J} (x') X^{J} (x') + k^2 \eta ^{a' b'} \mc{A}^{I} _{a' I'} (x') \mc{A}^{I'} _{b' J} (x') X^{J} (x') \right) ,
\end{equation*}
where the Green's function matrix $G^{K} _{\ J} (x , x')$ satisfies
\begin{equation*}
    \eta ^{a b} \left( \delta ^{I} _{K} \partial _{a} \partial _{b} + 2 k \mc{A}^{I} _{a K} (x) \partial _{b} \right) G^{K} _{\ J} (x , x') = \delta ^{I} _{J} \delta (x , x') .
\end{equation*}

Boundary conditions come from
\begin{equation*}
    \delta S = \int \diff ^2 x \left( \frac{\partial \Lagr}{\partial X^{K}} \delta X^{K} + \frac{\partial \Lagr}{\partial (\partial _{c}X^{K})} \delta (\partial _{c} X^{K}) + ... \right) =
\end{equation*}
\begin{equation*}
    = \int \diff ^2 x \left( (\mathrm{EoM}) \delta X^{K} + \partial _{c} \left( \frac{\partial \Lagr}{\partial (\partial _{c} X^{K})} \delta X^{K} \right) + ... \right) =
\end{equation*}
\begin{equation*}
    = \int \diff ^2 x (\mathrm{EoM}) \delta X^{K} + \int \diff \tau \left( \frac{\partial \Lagr}{\partial (\partial _{\sigma} X^{K})} \delta X^{K} \right) \bigg| _{0} ^{\sigma _{1}} + ... = 0 ,
\end{equation*}
thus
\begin{equation*}
    \left( \frac{\partial \Lagr}{\partial (\partial _{\sigma} X^{K})} \delta X^{K} \right) \bigg| _{0} ^{\sigma _{1}} = 0
\end{equation*}
\begin{equation*}
    \left( \left( - T e e^{\sigma} _{i} e^{b i} \mc{D}_{b} X_{K} \right) \delta X^{K} \right) \big| _{0} ^{\sigma _{1}} = 0
\end{equation*}
\begin{equation*}
    \left( (\mc{D}_{\sigma} X_{K}) \delta X^{K} \right) \big| _{0} ^{\sigma _{1}} = 0 ,
\end{equation*}
so either $\delta X^{K} (\tau , \sigma _{*}) = 0 , \ \sigma _{*} = 0 , \sigma _{1}$ (Dirichlet B.C) or $\mc{D}_{\sigma} X^{K} (\tau , \sigma _{*}) = 0 , \ \sigma _{*} = 0 , \sigma _{1}$ (free end-point B.C). For closed strings, $X^{K} (\tau , \sigma) = X^{K} (\tau , \sigma + 2 \pi)$.

When connection is trivial EoM reduces to regular wave eqn, which have the regular string solution
\begin{equation*}
    X^{I} _{0} (\tau , \sigma) = X^{I} _{0 L} (\sigma ^{+}) + X^{I} _{0 R} (\sigma ^{-}) ,
\end{equation*}
with
\begin{equation*}
    X^{I} _{0 L} (\sigma ^{+}) = \frac{1}{2} x^{I} + \frac{1}{2} \alpha ' p^{I} \sigma ^{+} + i \sqrt{\frac{\alpha '}{2}} \sum _{n \neq 0} \frac{1}{n} \widetilde{\alpha}^{I} _{n} \e ^{- i n \sigma ^{+}} , \ \sigma ^{+} = \tau + \sigma
\end{equation*}
\begin{equation*}
    X^{I} _{0 R} (\sigma ^{-}) = \frac{1}{2} x^{I} + \frac{1}{2} \alpha ' p^{I} \sigma ^{-} + i \sqrt{\frac{\alpha '}{2}} \sum _{n \neq 0} \frac{1}{n} \alpha ^{I} _{n} \e ^{- i n \sigma ^{-}} , \ \sigma ^{-} = \tau - \sigma .
\end{equation*}
This, however, does not amount to any motion in actual space-time, since with trivial connection the vector does not change after parallel transport. For a more interesting case, let's consider $A_{\tau} ^{\alpha \beta} = A_{\sigma} ^{\alpha \beta} = a \varepsilon ^{\alpha \beta}$, which amounts to flat space-time. The EoM turn into
\begin{equation*}
    \left( \partial _{\sigma} ^2 - \partial _{\tau} ^2 \right) X^{I} + 2 k a (\varepsilon ^{\alpha \beta} T_{\alpha \beta})^{I} _{\ J} \left( \partial _{\sigma} - \partial _{\tau} \right) X^{J} = 0
\end{equation*}
\begin{equation*}
    \left( \partial _{\sigma} ^2 - \partial _{\tau} ^2 \right) \begin{bmatrix} X^{0} \\ X^{1} \\ X^{2} \\ \vdots \\ X^{d} \end{bmatrix} + 4 k a \left( \partial _{\sigma} - \partial _{\tau} \right) \begin{bmatrix} 0 & 1 & 1 & ... & 1 \\ 1 & 0 & 1 & ... & 1 \\ 1 & - 1 & 0 & ... & 1 \\ \vdots & \vdots & \vdots & \ddots & \vdots \\ 1 & -1 & -1 & ... & 0 \end{bmatrix} \begin{bmatrix} X^{0} \\ X^{1} \\ X^{2} \\ \vdots \\ X^{d} \end{bmatrix} = 0
\end{equation*}
\begin{equation*}
    \begin{bmatrix} \left( \partial _{\sigma} ^2 - \partial _{\tau} ^2 \right) X^{0} + 4 k a \left( \partial _{\sigma} - \partial _{\tau} \right) \sum _{i > 0} X^{i} \\ \left( \partial _{\sigma} ^2 - \partial _{\tau} ^2 \right) X^{1} + 4 k a \left( \partial _{\sigma} - \partial _{\tau} \right) \sum _{i \neq 1} X^{i} \\ \left( \partial _{\sigma} ^2 - \partial _{\tau} ^2 \right) X^{2} + 4 k a \left( \partial _{\sigma} - \partial _{\tau} \right) \left( X^{0} - X^{1} + \sum _{i > 2} X^{i} \right) \\ \vdots \\ \left( \partial _{\sigma} ^2 - \partial _{\tau} ^2 \right) X^{d} + 4 k a \left( \partial _{\sigma} - \partial _{\tau} \right) \left( X^{0} - \sum _{i = 1} ^{d - 1} X^{i} \right) \end{bmatrix} = 0
\end{equation*}
\begin{equation*}
    \left( \partial _{\sigma} ^2 - \partial _{\tau} ^2 \right) X^{I} + 4 k a \left( \partial _{\sigma} - \partial _{\tau} \right) \left( (\mathrm{sgn}(I))^2 X^{0} + \sum _{i > 0} \mathrm{sgn}(i - I) X^{i} \right) = 0
\end{equation*}
\begin{equation*}
    \left( \partial _{\sigma} - \partial _{\tau} \right) \left( \left( \partial _{\sigma} + \partial _{\tau} \right) X^{I} + 4 k a \left( (\mathrm{sgn}(I))^2 X^{0} + \sum _{i > 0} \mathrm{sgn}(i - I) X^{i} \right) \right) = 0
\end{equation*}
\begin{equation*}
    \left( \partial _{\sigma} + \partial _{\tau} \right) X^{I} + 4 k a \left( (\mathrm{sgn}(I))^2 X^{0} + \sum _{i > 0} \mathrm{sgn}(i - I) X^{i} \right) = C^{I} _{1} . (\tau + \sigma) , \ C^{I} \in \mathbb{C}
\end{equation*}
\begin{equation*}
    \left( \partial _{\sigma} + \partial _{\tau} \right) X^{I} = C^{I} _{1} . (\tau + \sigma) - 4 k a \left( (\mathrm{sgn}(I))^2 X^{0} + \sum _{i > 0} \mathrm{sgn}(i - I) X^{i} \right)
\end{equation*}
\begin{equation*}
    X^{I} = \frac{1}{2} C^{I} _{1} . \left( \tau ^2 + \sigma ^2 \right) + C^{I} _{2} . \left( \tau - \sigma \right) - 4 k a \int \diff ^2 x' \left( (\mathrm{sgn}(I))^2 X^{0} + \sum _{i > 0} \mathrm{sgn}(i - I) X^{i} \right) .
\end{equation*}
If $D = 2$, we get a symmetric system of eqns
\begin{align*}
    X^{0} & = \frac{1}{2} C^{0} _{1} . \left( \tau ^2 + \sigma ^2 \right) + C^{0} _{2} . \left( \tau - \sigma \right) - 4 k a \int \diff ^2 x' X^{1} \\
    X^{1} & = \frac{1}{2} C^{1} _{1} . \left( \tau ^2 + \sigma ^2 \right) + C^{1} _{2} . \left( \tau - \sigma \right) - 4 k a \int \diff ^2 x' X^{0}
\end{align*}
\begin{equation*}
    X^{0} & = \frac{1}{2} C^{0} _{1} . \left( \tau ^2 + \sigma ^2 \right) + C^{0} _{2} . \left( \tau - \sigma \right) - 4 k a \int \diff ^2 x' \left( \frac{1}{2} C^{1} _{1} . \left( \tau ^{\prime 2} + \sigma ^{\prime 2} \right) + C^{1} _{2} . \left( \tau ' - \sigma ' \right) - 4 k a \int \diff ^2 x'' X^{0} \right)
\end{equation*}
\begin{equation*}
    X^{0} - 16 k^2 a^2 \int \diff ^2 x' \int \diff ^2 x'' X^{0} = \frac{1}{2} C^{0} _{1} . \left( \tau ^2 + \sigma ^2 \right) + C^{0} _{2} . \left( \tau - \sigma \right) - \frac{1}{2} C^{1} _{1} \left( \frac{\tau ^3}{3} \sigma + \tau \frac{\sigma ^3}{3} \right) - C^{1} _{2} \left( \frac{\tau ^2}{2} \sigma - \tau \frac{\sigma ^2}{2} \right) .
\end{equation*}
This integral equation is tricky to deal with, but we can turn it into a differential equation by applying $\partial _{\sigma} \partial _{\tau}$ twice:
\begin{equation*}
    \left( \partial _{\sigma} \partial _{\tau} \right) ^2 X^{0} - 16 k^2 a^2 X^{0} = 0 .
\end{equation*}
This has general solution given by sum of plane waves and exponentials
\begin{equation*}
    X^{0} = \sum _{n \neq 0} \frac{1}{n^4} \left( a^{0} _{n} \e ^{i 2 \sqrt{k a} n (\tau + \sigma)} + b^{0} _{n} \e ^{i 2 \sqrt{k a} n (\tau - \sigma)} + c^{0} _{n} \e ^{2 \sqrt{k a} n (\tau + \sigma)} + d^{0} _{n} \e ^{2 \sqrt{k a} n (\tau - \sigma)} \right) .
\end{equation*}
In actuallity, only the $n = 1$ and $n = - 1$ terms satisfy the differential equation, so we have just
\begin{multline*}
    X^{0} = a^{0} _{1} \e ^{i 2 \sqrt{k a} (\tau + \sigma)} + b^{0} _{1} \e ^{i 2 \sqrt{k a} (\tau - \sigma)} + c^{0} _{1} \e ^{2 \sqrt{k a} (\tau + \sigma)} + d^{0} _{1} \e ^{2 \sqrt{k a} (\tau - \sigma)} + \\
    + a^{0} _{-1} \e ^{- i 2 \sqrt{k a} (\tau + \sigma)} + b^{0} _{-1} \e ^{- i 2 \sqrt{k a} (\tau - \sigma)} + c^{0} _{-1} \e ^{- 2 \sqrt{k a} (\tau + \sigma)} + d^{0} _{-1} \e ^{- 2 \sqrt{k a} (\tau - \sigma)}
\end{multline*}

\newpage

\section{Polyakov-LQG V2}

Same idea as before, but now contract the dyads directly with the $\mc{D}_{a} X^{I}$ factors
\begin{equation*}
    S = - \frac{T}{4} \int \diff ^2 x e e^{a} _{I} e^{b} _{J} \mc{D}_{a} X^{I} \mc{D}_{b} X^{J} .
\end{equation*}

\subsection{EoMs}

\subsubsection{W.r.t $e$}

\begin{align*}
    \frac{\delta S}{\delta e^{c} _{K}} & \propto \left( \frac{\partial e}{\partial e^{c} _{K}} e^{a} _{I} e^{b} _{J} + e \frac{\partial}{\partial e^{c} _{K}} (e^{a} _{I} e^{b} _{J}) \right) \mc{D}_{a} X^{I} \mc{D}_{b} X^{J} = \\
                                       & = \left( - e e^{K} _{c} e^{a} _{I} e^{b} _{J} + 2 e \delta ^{K} _{I} \delta ^{a} _{c} e^{b} _{J} \right) \mc{D}_{a} X^{I} \mc{D}_{b} X^{J} =                                             \\
                                       & = - e e^{K} _{c} e^{a} _{I} e^{b} _{J} \mc{D}_{a} X^{I} \mc{D}_{b} X^{J} + 2 e e^{b} _{J} \mc{D}_{c} X^{K} \mc{D}_{b} X^{J} = 0
\end{align*}
\begin{equation*}
    T_{c} ^{K} := 2 e^{b} _{J} \mc{D}_{c} X^{K} \mc{D}_{b} X^{J} - e^{K} _{c} e^{a} _{I} e^{b} _{J} \mc{D}_{a} X^{I}\mc{D}_{b} X^{J} = 0 ,
\end{equation*}
\begin{equation*}
    e^{K} _{c} = 2 f \mc{D}_{c} X^{K} ,
\end{equation*}
\begin{equation*}
    \frac{1}{f} = e^{b} _{J} \mc{D}_{b} X^{J} .
\end{equation*}

\subsubsection{W.r.t $A$}

\begin{align*}
    \frac{\delta S}{\delta \mc{A}^{K L} _{c}} & \propto e e^{a} _{I} e^{b} _{J} \frac{\partial}{\partial \mc{A}^{K L} _{c}} (\mc{D}_{a} X^{I} \mc{D}_{b}X^{J}) = \\
                                              & = 2 e e^{a} _{I} e^{b} _{J} \delta _{a} ^{c} \delta ^{I} _{[K} \eta _{L] I'} X^{I'} \mc{D}_{b} X^{J} =           \\
                                              & = 2 e e^{c} _{[K} X_{L]} e^{b} _{J} \mc{D}_{b} X^{J} = 0
\end{align*}
\begin{equation*}
    \mc{T}^{c} _{K L} := e^{c} _{[K} X_{L]} = 0 .
\end{equation*}

\subsubsection{W.r.t $X$}

\begin{align*}
    \frac{\delta S}{\delta X^{K}} & \propto 2 e e^{a} _{I} e^{b} _{J} k \mc{A}^{I} _{a I'} \delta ^{I'} _{K} \mc{D}_{b} X^{J} - 2 \partial _{c} \left( e e^{a} _{I} e^{b} _{J} \delta ^{c} _{a} \delta ^{I} _{K} \mc{D}_{b} X^{J} \right) = \\
                                  & = 2 e e^{a} _{I} e^{b} _{J} k \mc{A}^{I} _{a K} \mc{D}_{b} X^{J} - 2 \partial _{a} \left( e e^{a} _{K} e^{b} _{J} \mc{D}_{b} X^{J} \right) = 0
\end{align*}
\begin{equation*}
    \mc{D}_{a} \left( e e^{a} _{K} e^{b} _{J} \mc{D}_{b} X^{J} \right) = 0 .
\end{equation*}
If conformal gauge makes it so that $e e^{a} _{K} e^{b} _{J} = \eta ^{a b} \eta _{K J}$, this reduces to
\begin{equation*}
    \eta ^{a b} \mc{D}_{a} \mc{D}_{b} X^{I} = 0
\end{equation*}

\newpage

\section{Gauged Polyakov}

Turn embbeding fields $X^{\mu}$ into vectors in fundamental representation of gauge group $\mc{G}$, whose generators are
\begin{equation*}
    T = \begin{bmatrix} 0 & \beta ^{1} & \beta ^{2} & \beta ^{3} & p^{0} / l \\ \beta ^{1} & 0 & \theta ^{3} & - \theta ^{2} & p^{1} / l \\ \beta ^{2} & - \theta ^{3} & 0 & \theta ^{1} & p^{2} / l \\ \beta ^{3} & \theta ^{2} & - \theta ^{1} & 0 & p^{3} / l \\ \epsilon p^{0} / l & - \epsilon p^{1} / l & - \epsilon p^{2} / l & - \epsilon p^{3} / l & 0 \end{bmatrix} ,
\end{equation*}
where $\epsilon$ is the sign of the cosmological constant and if $\epsilon = - 1 \implies \mc{G} = \mathrm{SO} (3 , 2)$, $\epsilon = 0 \implies \mc{G} = \mathrm{ISO} (3 , 1)$ and $\epsilon = + 1 \implies \mc{G} = \mathrm{SO} (4 , 1)$. Thus, partial derivatives becomes covariant derivatives $\partial _{a} \rightarrow \mc{D}_{a}$,
\begin{equation*}
    \mc{D}_{a} X^{A} = \partial _{a} X^{A} + k \mc{A}^{A} _{a B} X^{B} ,
\end{equation*}
with $X^{A} = (X^{\mu} , X^{r})$. Connection is thus
\begin{equation*}
    \mc{A}_{a} = \partial _{a} X^{\mu} \mc{A}_{\mu} = \partial _{a} X^{\mu} \begin{bmatrix} 0 & \omega ^{0 1} _{\mu} & \omega ^{0 2} _{\mu} & \omega ^{0 3} _{\mu} & E^{0} _{\mu} \\
        \omega ^{0 1} _{\mu} & 0 & \omega ^{1 2} _{\mu} & - \omega ^{1 3} _{\mu} & E^{1} _{\mu} \\
        \omega ^{0 2} _{\mu} & - \omega ^{1 2} _{\mu} & 0 & \omega ^{2 3} _{\mu} & E^{2} _{\mu} \\
        \omega ^{0 3} _{\mu} & \omega ^{1 3} _{\mu} & - \omega ^{2 3} _{\mu} & 0 & E^{3} _{\mu} \\
                          \epsilon E^{0} _{\mu} & - \epsilon E^{1} _{\mu} & - \epsilon E^{2} _{\mu} & - \epsilon E^{3} _{\mu} & 0 \end{bmatrix} ,
\end{equation*}
where $\omega ^{I J} _{\mu}$ is the bulk spin connection and $E^{I} _{\mu}$ is the bulk vierbein. Action thus becomes
\begin{equation*}
    S = - \frac{T}{2} \int \diff ^2 x \sqrt{- g} g^{a b} \mc{D}_{a} X^{A} \mc{D}_{b} X^{B} G_{A B} ,
\end{equation*}
where $G_{A B}$ is the expanded metric, which we'll separate as
\begin{equation*}
    [G_{A B}] = \begin{bmatrix} [G_{\mu \nu}] & [G_{\mu r}] \\ [G_{\mu r}]^{T} & G_{r r} \end{bmatrix} ,
\end{equation*}
with $G_{\mu r} = \epsilon V_{\mu}$ and $G_{r r} = \epsilon V$, where $V_{\mu}$ is a vector analogous to the shift vector and $V$ is a function analogous to the lapse function. This separates the action into $S = S [X^{\mu}] + S_{int} [X^{\mu} , X^{r}] + S [X^{r}]$
\begin{align*}
    S [X^{\mu}] & = - \frac{T}{2} \int \diff ^2 x \sqrt{- g} g^{a b} \mc{D}_{a} X^{\mu} \mc{D}_{b} X^{\nu} G_{\mu \nu} \\
    S_{int} [X^{\mu} , X^{r}] & = - \epsilon T \int \diff ^2 x \sqrt{- g} g^{a b} \mc{D}_{a} X^{\mu} \mc{D}_{b} X^{r} V_{\mu} \\
    S [X^{r}] & = - \epsilon \frac{T}{2} \int \diff ^2 x \sqrt{- g} g^{a b} \mc{D}_{a} X^{r} \mc{D}_{b} X^{r} V .
\end{align*}

\subsection{EoMs}

\subsubsection{W.r.t $g$}

\begin{align*}
    \frac{\delta S}{\delta g^{c d}} & = - \frac{T}{2} \frac{\delta}{\delta g^{c d}} (\sqrt{- g} g^{a b} \mc{D}_{a} X^{A} \mc{D}_{b} X^{B} G_{A B}) = \\
    & = - \frac{T}{2} \left( \frac{\del \sqrt{- g}}{\del g^{c d}} g^{a b} + \sqrt{- g} \frac{\del g^{a b}}{\del g^{c d}} \right) \mc{D}_{a} X^{A} \mc{D}_{b} X^{B} G_{A B} = \\
    & = - \frac{T}{2} \left( - \frac{1}{2} \sqrt{- g} g_{c d} g^{a b} + \sqrt{- g} \delta ^{a} _{(c} \delta ^{b} _{d)} \right) \mc{D}_{a} X^{A} \mc{D}_{b} X^{B} G_{A B} = 0
\end{align*}
\begin{equation*}
    g_{c d} = 2 f \mc{D}_{c} X^{A} \mc{D}_{d} X^{B} G_{A B} ,
\end{equation*}
\begin{equation*}
    \frac{1}{f} = g^{a b} \mc{D}_{a} X^{A} \mc{D}_{b} X^{B} G_{A B}
\end{equation*}

\subsubsection{W.r.t $\mc{A}$}

\begin{align*}
    \frac{\delta S}{\delta \mc{A}^{C D} _{c}} & = - \frac{T}{2} \frac{\delta}{\delta \mc{A}^{C D} _{c}} (\sqrt{- g} g^{a b} \mc{D}_{a} X^{A} \mc{D}_{b} X^{B} G_{A B}) = \\
    & = - T \sqrt{- g} g^{a b} \frac{\del}{\del \mc{A}^{C D} _{c}} \mc{D}_{a} X^{A} \mc{D}_{b} X^{B} G_{A B} = \\
    & = - T \sqrt{- g} g^{a b} (k \delta ^{c} _{a} \delta ^{A} _{[C} \delta ^{A'} _{D]} X_{A'}) \mc{D}_{b} X^{B} G_{A B} = \\
    & = - T k \sqrt{- g} g^{c b} \mc{D}_{b} X^{B} G_{B [C} X_{D]} = 0
\end{align*}
\begin{equation*}
    \sigma _{c} ^{C D} = \mc{D}_{c} X^{[C} X^{D]} = 0
\end{equation*}

\subsubsection{W.r.t $X$}

\begin{align*}
    \frac{\delta S}{\delta X^{C}} & = - \frac{T}{2} \frac{\delta}{\delta X^{C}} (\sqrt{- g} g^{a b} \mc{D}_{a} X^{A} \mc{D}_{b} X^{B} G_{A B}) = \\
    & = T (\mc{D}_{c} (\sqrt{- g} g^{a b} \delta _{a} ^{c} \delta ^{A} _{C} \mc{D}_{b} X^{B} G_{A B}) - \sqrt{- g} g^{a b} \mc{D}_{a} X^{A} \mc{D}_{b} X^{B} \del _{C} G_{A B}) = \\
    & = T (\mc{D}_{a} (\sqrt{- g} g^{a b} \mc{D}_{b} X^{B} G_{B C}) - \sqrt{- g} g^{a b} \mc{D}_{a} X^{A} \mc{D}_{b} X^{B} \del _{C} G_{A B}) = 0
\end{align*}
\begin{equation*}
    \mc{D}_{a} (\sqrt{- g} g^{a b} \mc{D}_{b} X^{B} G_{B C}) = \sqrt{- g} g^{a b} \mc{D}_{a} X^{A} \mc{D}_{b} X^{B} \del _{C} G_{A B} .
\end{equation*}
In conformal gauge $g^{a b} = \e ^{- \phi} \eta ^{a b}$ and assuming $G_{A B} = \eta ' _{A B}$, where (with $\epsilon \neq 0$. For $\epsilon = 0$, this collapses the EoM back into normal wave eqns for $X^{\mu}$, thus restoring regular ST)
\begin{equation*}
    [\eta ' _{A B}] = \begin{bmatrix} [\eta _{\mu \nu}] & 0 \\ 0 & \epsilon \end{bmatrix}
\end{equation*}
\begin{equation*}
    \eta ^{a b} \mc{D}_{a} (\mc{D}_{b} X^{B} G_{B C}) = 0
\end{equation*}
\begin{equation*}
    \eta ^{a b} \mc{D}_{a} \mc{D}_{b} X^{A} = 0 .
\end{equation*}
\begin{equation*}
    \eta ^{a b} \mc{D}_{a} (\del _{b} X^{A} + k \mc{A}^{A} _{b C} X^{C}) = 0
\end{equation*}
\begin{equation*}
    \eta ^{a b} ((\del _{a} \del _{b} X^{A} + k \mc{A}^{A} _{a C} \del _{b} X^{C}) + k \mc{D}_{a} [\mc{A}^{A} _{b C} X^{C}]) = 0
\end{equation*}
\begin{equation*}
    \eta ^{a b} ((\del _{a} \del _{b} X^{A} + k \mc{A}^{A} _{a C} \del _{b} X^{C}) + k (\mc{D}_{a} \mc{A}^{A} _{b C} X^{C} + \mc{A}^{A} _{b C} \mc{D}_{a} X^{C})) = 0
\end{equation*}
\begin{equation*}
    \eta ^{a b} ((\del _{a} \del _{b} X^{A} + k \mc{A}^{A} _{a C} \del _{b} X^{C}) + k ((\del _{a} \mc{A}^{A} _{b C} + k \mc{A}^{A} _{a D} \mc{A}^{D} _{b C} - k \mc{A}^{D} _{a C} \mc{A}^{A} _{b D}) X^{C} + \mc{A}^{A} _{b C} (\del _{a} X^{C} + k \mc{A}^{C} _{a D} X^{D}))) = 0
\end{equation*}
\begin{equation*}
    \eta ^{a b} (\del _{a} \del _{b} X^{A} + 2 k \mc{A}^{A} _{a C} \del _{b} X^{C} + k \del _{a} \mc{A}^{A} _{b C} X^{C} + k^2 \mc{A}^{A} _{a D} \mc{A}^{D} _{b C} X^{C}) = 0
\end{equation*}
To produce this geometry, the bulk connection can be taken as
\begin{equation*}
    \mc{A}_{\mu} = 
    \begin{bmatrix} 0 & 0 & 0 & 0 & \delta ^{0} _{\mu} \\ 
                    0 & 0 & 0 & 0 & \delta ^{1} _{\mu} \\
                    0 & 0 & 0 & 0 & \delta ^{2} _{\mu} \\
                    0 & 0 & 0 & 0 & \delta ^{3} _{\mu} \\
                    \epsilon \delta ^{0} _{\mu} & - \epsilon \delta ^{1} _{\mu} & - \epsilon \delta ^{2} _{\mu} & - \epsilon \delta ^{3} _{\mu} & 0 \end{bmatrix}
\end{equation*}
\begin{equation*}
    \downarrow
\end{equation*}
\begin{equation*}
    \mc{A}_{a} = \partial _{a} X^{\mu} \mc{A}_{\mu} = 
    \begin{bmatrix} 0 & 0 & 0 & 0 & \del _{a} X^{0} \\ 
                    0 & 0 & 0 & 0 & \del _{a} X^{1} \\
                    0 & 0 & 0 & 0 & \del _{a} X^{2} \\
                    0 & 0 & 0 & 0 & \del _{a} X^{3} \\
                    \epsilon \del _{a} X^{0} & - \epsilon \del _{a} X^{1} & - \epsilon \del _{a} X^{2} & - \epsilon \del _{a} X^{3} & 0 \end{bmatrix}
\end{equation*}
\begin{equation*}
    \downarrow
\end{equation*}
\begin{equation*}
    \mc{A}_{a} = 
    \begin{bmatrix} 0 & 0 & 0 & 0 & n_{a} ^{0} \\ 
                    0 & 0 & 0 & 0 & n_{a} ^{1} \\
                    0 & 0 & 0 & 0 & n_{a} ^{2} \\
                    0 & 0 & 0 & 0 & n_{a} ^{3} \\
                    \epsilon n_{a} ^{0} & - \epsilon n_{a} ^{1} & - \epsilon n_{a} ^{2} & - \epsilon n_{a} ^{3} & 0 \end{bmatrix} ,
\end{equation*}
where $n_{a} ^{I}$ is some family of diff functions on the WS (dyad, but not the covariant dyad $e^{A} _{a} = \mc{D}_{a} X^{A}$. Covariant dyad $e^{A} _{a}$ generates covariant WS metric $e^{A} _{a} e^{B} _{b} G_{A B} = g_{a b} = \mc{D}_{a} X^{A} \mc{D}_{b} X^{B} G_{A B}$ whilst this family of functions generates induced metric $n^{I} _{a} n^{J} _{b} \eta _{I J} = \gamma _{a b} = \del _{a} X^{\mu} \del _{b} X^{\nu} G_{\mu \nu}$). For $\epsilon = - 1$ (flat adS background) we have thus
\begin{equation*}
    \mc{A}_{a} X = 
    \begin{bmatrix} n^{0} _{a} X^{r} \\
                    n^{1} _{a} X^{r} \\
                    n^{2} _{a} X^{r} \\
                    n^{3} _{a} X^{r} \\
                    n^{\mu} _{a} X_{\mu} \end{bmatrix}
\end{equation*}
\begin{equation*}
    \mc{A}_{a} \del _{b} X = \begin{bmatrix} n^{0} _{a} \del _{b} X^{r} \\
        n^{1} _{a} \del _{b} X^{r} \\
        n^{2} _{a} \del _{b} X^{r} \\
        n^{3} _{a} \del _{b} X^{r} \\
        n^{\mu} _{a} \del _{b} X_{\mu} \end{bmatrix} = 
        \begin{bmatrix} n^{0} _{a} \del _{b} X^{r} \\
            n^{1} _{a} \del _{b} X^{r} \\
            n^{2} _{a} \del _{b} X^{r} \\
            n^{3} _{a} \del _{b} X^{r} \\
            n^{\mu} _{a} n^{\nu} _{b} \eta _{\mu \nu} \end{bmatrix}
\end{equation*}
\begin{equation*}
    \del _{a} \mc{A}_{b} X = \begin{bmatrix} \del _{a} n^{0} _{b} X^{r} \\
        \del _{a} n^{1} _{b} X^{r} \\
        \del _{a} n^{2} _{b} X^{r} \\
        \del _{a} n^{3} _{b} X^{r} \\
        \del _{a} n^{\mu} _{b} X_{\mu} \end{bmatrix}
\end{equation*}
\begin{equation*}
    \mc{A}_{a} \mc{A}_{b} X = \mc{A}_{a} \begin{bmatrix} n^{0} _{b} X^{r} \\
        n^{1} _{b} X^{r} \\
        n^{2} _{b} X^{r} \\
        n^{3} _{b} X^{r} \\
        n^{\mu} _{b} X_{\mu} \end{bmatrix} = \begin{bmatrix} n^{0} _{a} n^{\mu} _{b} X_{\mu} \\
            n^{1} _{a} n^{\mu} _{b} X_{\mu} \\
            n^{2} _{a} n^{\mu} _{b} X_{\mu} \\
            n^{3} _{a} n^{\mu} _{b} X_{\mu} \\
            n^{\mu} _{a} n^{\nu} _{b} \eta _{\mu \nu} X^{r} \end{bmatrix}
\end{equation*}
\begin{equation*}
    \downarrow
\end{equation*}
\begin{equation*}
    \eta ^{a b} (\del _{a} \del _{b} X^{\mu} + 2 k \del _{a} X^{\mu} \del _{b} X^{r} + k \del_{a} \del_{b} X^{\mu} X^{r} + k^2 \del _{a} X^{\mu} \del _{b} X^{\nu} X_{\nu}) = 0
\end{equation*}
\begin{equation*}
    \eta ^{a b} (\del _{a} \del _{b} X^{r} + 2 k \del _{a} X^{\mu} \del _{b} X_{\mu} + k \del _{a} \del _{b} X^{\mu} X_{\mu} + k^2 \del _{a} X^{\mu} \del _{b} X_{\mu} X^{r}) = 0
\end{equation*}
\begin{equation*}
    \downarrow \del _{a} X^{\mu} \del _{b} X_{\mu} = \gamma _{a b} = \e ^{\varphi (x)} \eta _{a b}
\end{equation*}
\begin{equation*}
    \eta ^{a b} ((1 + k X^{r}) \del _{a} \del _{b} X^{\mu} + k (2 \del _{b} X^{r} + k \del _{b} X^{\nu} X_{\nu}) \del _{a} X^{\mu}) = 0
\end{equation*}
\begin{equation*}
    \eta ^{a b} \del _{a} \del _{b} X^{r} + 4 k \e ^{\varphi} + k \eta ^{a b} \del _{a} \del _{b} X^{\mu} X_{\mu} + 2 k^2 \e ^{\varphi} X^{r} = 0
\end{equation*}
\begin{equation*}
    \downarrow
\end{equation*}
\begin{equation*}
    \eta ^{a b} \left( \del _{a} \del _{b} X^{\mu} + \frac{k}{1 + k X^{r}} (2 \del _{a} X^{r} + k \del _{a} X^{\nu} X_{\nu}) \del _{b} X^{\mu} \right) = 0
\end{equation*}
\begin{equation*}
    \eta ^{a b} \del _{a} \del _{b} X^{r} + 2 k^2 \e ^{\varphi} X^{r} + 4 k \e ^{\varphi} - \frac{k^2 \eta ^{a b}}{1 + k X^{r}} (2 \del _{a} X^{r} + k \del _{a} X^{\nu} X_{\nu}) \del _{b} X^{\mu} X_{\mu} = 0 .
\end{equation*}
Assuming $|X^{I}| \ll 1$ leads to
\begin{equation*}
    \eta ^{a b} \del _{a} \del _{b} X^{\mu} = 0
\end{equation*}
\begin{equation*}
    \eta ^{a b} \del _{a} \del _{b} X^{r} + 2 k^{2} e^{\varphi} X^{r} = - 4 k e^{\varphi} .
\end{equation*}
Renaming the coupling constant $k$ to $\mu$ since $2 \mu ^2 \e ^{\varphi}$ is sort of like a mass term, we write the solution for $X^{r}$ as a perturbation around the homogenous wave eqn solution using the 1d wave eqn Green's function
\begin{equation*}
    X^{r} (x) = X^{r} _{0} (x) - 2 \mu \int \diff ^{2} x' G (x ; x') \e ^{\varphi (x')} (2 + \mu X^{r} (x')) = X^{r} _{0} + \delta X^{r} ,
\end{equation*}
\begin{equation*}
    (\partial _{\sigma} ^{2} - \partial _{\tau} ^{2}) X^{r} _{0} = 0
\end{equation*}
\begin{equation*}
    (\partial _{\sigma} ^{2} - \partial _{\tau} ^{2}) G (x ; x') = \delta ^{2} (x , x') .
\end{equation*}
This can be used to build a iterative recursion for $X^{r}$
\begin{equation*}
    X^{r} _{n + 1} (x) = X^{r} _{0} (x) - 2 \mu \int \diff ^2 x' G (x ; x') \e ^{\varphi (x')} (2 + \mu X^{r} _{n} (x'))
\end{equation*}
\begin{equation*}
    \downarrow
\end{equation*}
\begin{equation*}
    X^{r} (x) = X^{r} _{0} (x) + \sum _{k \geq 1} (- 2)^{k} (\mu)^{2k - 1} \int \prod _{n = 1} ^{k} \diff ^2 x^{(n)} G (x^{(n - 1)} ; x^{(n)}) \e ^{\varphi (x^{(n)})} (2 + \mu X^{r} _{0} (x^{(k)})) ,
\end{equation*}
with $x^{(0)} \equiv x$. The wave eqn Green's function $G (x ; x') = \frac{1}{2} \Theta (\tau - \tau ' - |\sigma - \sigma '|)$ just enforces causality on the integrals, thus the perturbation term $\delta X^{r}$ is a series of repeated integrals of a solution to the homogenous wave eqn times the conformal factor $e^{\varphi}$ induced from the 4d sector, with causality enforced.

\end{document}