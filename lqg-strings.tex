\RequirePackage{snapshot}
\RequirePackage[final]{graphicx}
\documentclass[twocolumn,draft,aps,prd,10pt,nofootinbib,nobibnotes]{revtex4-2}

% revtex4 imports the natbib package which conflicts with biblatex. To remedy the resulting name conflicts the following recipe is taken from https://tex.stackexchange.com/a/37077
% Start of 'ignore natbib' hack
%\let\bibhang\relax
%\let\citename\relax
%\let\bibfont\relax
%\let\Citeauthor\relax
%\let\textcite\relax
%\makeatletter
%\DeclareRobustCommand{\MakeUppercase}[1]{{%
%      \def\i{I}\def\j{J}%
%      \def\reserved@a##1##2{\let##1##2\reserved@a}%
%      \expandafter\reserved@a\@uclclist\reserved@b{\reserved@b\@gobble}%
%      \protected@edef\reserved@a{\uppercase{#1}}%
%      \reserved@a
%   }}
%\DeclareRobustCommand{\MakeLowercase}[1]{{%
%      \def\reserved@a##1##2{\let##2##1\reserved@a}%
%      \expandafter\reserved@a\@uclclist\reserved@b{\reserved@b\@gobble}%
%      \protected@edef\reserved@a{\lowercase{#1}}%
%      \reserved@a
%   }}
%\makeatother
%\expandafter\let\csname ver@natbib.sty\endcsname\relax
% End of 'ignore natbib' hack

%\usepackage{graphicx}

\usepackage{amsmath,amssymb,amsfonts,amsthm}
%\usepackage[sort&compress,numbers]{natbib}
%\usepackage[english]{babel}

\usepackage{enumerate}
\usepackage{enumitem}
\usepackage{epsfig}
\usepackage{latexsym}
\usepackage{xcolor}
\usepackage{mathrsfs}
\usepackage[colorinlistoftodos, shadow, textsize=small, obeyDraft,loadshadowlibrary]{todonotes}

\setlength{\emergencystretch}{2em}

\usepackage[colorlinks=true,citecolor=blue,linkcolor=blue,final]{hyperref}
%\usepackage{hyperref}

% Mar 25, 2023
% Added parskip package for spacing between paragraphs.
\usepackage{parskip}
\setlength{\parindent}{15pt}

% Nov 20, 2023
% Added package mciteplus to collapse multiple papers into one item in bibliography.
\usepackage{mciteplus}

%\usepackage{hypernat}

\usepackage{comment}

% Biblatex setup
% The default is the 'numerical' style.
%\usepackage[backend=biber,url=true,eprint=true,doi=true,sorting=none,backref=true,style=phys]{biblatex}
% We use the following bibliography databases:
%\usepackage{bibtex}
%\addbibresource{../bib_library.bib}
% End biblatex


%%% BEGIN Custom commands %%%
\newcommand{\bite}{\begin{itemize}}
	\newcommand{\eat}{\end{itemize}}
\newcommand{\beq}{\begin{equation}}
	\newcommand{\eeq}{\end{equation}}
\newcommand{\rarrow}{\rightarrow}
\newcommand{\beqa}{\begin{align}}
	\newcommand{\eeqa}{\end{align}}
\newcommand{\barr}{\begin{array}}
	\newcommand{\earr}{\end{array}}
\newcommand{\del}{\partial}
\newcommand{\de}{\mathrm{d}}
%\newcommand{\mu\nu}{{\mu\nu}}
\renewcommand{\th}{\mathrm{th}}
\newcommand{\com}[1]{\begin{itemize}\color{RED}{{#1}}\end{itemize}}
\newcommand{\C}{\mathbb{C}}
\newcommand{\R}{\mathbb{R}}
\newcommand{\btw}[1]{\color{PURPLE}{{#1}}\color{BLACK}}
\newcommand{\cut}[1]{\color{RED}{{#1}}\color{BLACK}}

%text
\newcommand{\ie}{\textit{i.e.}~}
\newcommand{\eg}{\textit{e.g.}~}
\newcommand{\wrt}{\textit{w.r.t.}~}
\newcommand{\etc}{\textit{etc.}~}


\newcommand{\M}{\mathcal{M}}
\newcommand{\N}{\mathcal{N}}
% \newcommand{\H}{\mathcal{H}}

\newcommand{\bz}{\mathbf{z}}

\newcommand{\mb}[1]{\mathbf{#1}}
\newcommand{\mc}[1]{\mathcal{#1}}
\newcommand{\mf}[1]{\mathfrak{#1}}
\newcommand{\mbb}[1]{\mathbb{#1}}

\newcommand{\unit}[1]{\mathbf{\hat{#1}}}

%% Package bbold for bold identity symbol
\usepackage{bbold}

\newcommand{\id}{\mathbb{1}}

\newcommand{\utilde}[1]{\underaccent{\tilde}{#1}}

\newcommand{\vect}[1]{\boldsymbol{#1}}
\newcommand{\bvec}[1]{\boldsymbol{\vec #1}}
\newcommand{\expect}[1]{\langle #1\rangle}
\newcommand{\innerp}[2]{\langle #1 \vert #2 \rangle}
\newcommand{\expectop}[3]{\langle #1 \vert #2 \vert #3 \rangle}
\newcommand{\bra}[1]{\langle #1 \vert}
\newcommand{\ket}[1]{\vert #1 \rangle}
\newcommand{\supersc}[1]{$^{\textrm{#1}}$}
\newcommand{\subsc}[1]{$_{\textrm{#1}}$}
\newcommand{\sltwoc}{\mathfrak{sl}(2,\mathbb{C})}

\newcommand{\norm}[1]{\lVert #1 \rVert}

\newcommand{\rket}[1]{\vert #1 ]}
\newcommand{\rbra}[1]{[ #1 \vert}

\newcommand{\rinnerp}[2]{[ #1 \vert #2 ]}

\newcommand{\bket}[1]{\vert #1 )}
\newcommand{\bbra}[1]{( #1 \vert}

\newcommand{\binnerp}[2]{( #1 \vert #2 )}

\newcommand{\innerpA}[2]{\langle #1 \vert #2 ]}
\newcommand{\innerpB}[2]{[ #1 \vert #2 \rangle}

\newcommand{\onehalf}{\frac{1}{2}}

\newcommand{\Tr}{\mathrm{Tr}}

\newcommand{\Cyl}{\mathrm{Cyl}}

%Luigi's custom commands
% \R is already defined as the same above.
% \newcommand{\R}{\mathbb{R}}
\newcommand{\diff}{\mathrm{d}}
\newcommand{\inp}{\mathrm{i}}
\newcommand{\e}{\mathrm{e}}
\newcommand{\Lagr}{\mathscr{L}}
\newcommand{\PS}{\mathcal{P}}
%%% END Custom commands %%%

\graphicspath{{figures/}}

%opening
\begin{document}

\preprint{This line only printed with preprint option}

\title{Quantizing the Bosonic String on a Loop Quantum Gravity Background}

\author{Deepak Vaid}
\email{dvaid79@gmail.com}
\affiliation{National Institute of Technology, Karnataka, India}
\author{Luigi Teixeira de Sousa}
\email{luigi.tiraque@gmail.com}
\affiliation{Universidade Federal de São Carlos, Brasil}

\date{\today}

\begin{abstract}

	With the goal of understanding whether or not it is possible to construct a string theory which is consistent with loop quantum gravity (LQG), we study alternate versions of the Nambu-Goto action for a bosonic string. We consider two types of modifications. The first is a phenomenological action based on the observation that LQG tells us that areas of two-surfaces are operators in quantum geometry and are bounded from below. This leads us to a string action which is similar to that of bimetric gravity. We provide formulations of the bimetric string for both the Nambu-Goto (second order) and Polyakov (first order) formulations. We explore the classical solutions of this action and its quantization and relate it to the conventional string solutions.
	
	The second is an action in which the background geometry is described in terms of the pullback of the connection which describes the bulk geometry to the worldsheet. The resulting action is in the form of a gauged sigma model, where the spacetime co-ordinates are now SO(D,1) vectors. We find that for the particular case of a constant background connection the action reduces to the bimetric action discussed above. We discuss classical solutions and quantization strategies for this action and its implications for the broader program of unifying string theory and loop quantum gravity.
\end{abstract}

\maketitle

\tableofcontents

\listoftodos

\section{Introduction}\label{sec:intro}

``Quantum gravity'' refers to the broad enterprise dedicated to finding a complete, consistent theory in which quantum mechanics coexists peacefully with general relativity. There are two major approaches to quantum gravity which are widely recognized for their success in describing various aspects of physics at the Planck scale. The first of these is string theory, the quantum theory of a one dimensional object which is said to contain within it a complete description of all particles, forces and their interactions, including gravity. However, where precisely in the vast space of effective field theories which can possibly arise as in the low energy limit of a conjectured ``M-Theory'', our Universe with its collection of particles, forces and coupling constants, exists, is still a matter of vigorous debate.

One of the primary criticisms of string theory is its apparent lack of ``background independence''. This is more general than the principle of \emph{general covariance} which states that physics should be independent of the choice of co-ordinates. Background independence is the statement that any \emph{quantum} theory of gravity - which, presumably, should provide a description not only of classical spacetimes, but also of fluctuating, semiclassical geometries and those in the deep quantum regime which do not have any sensible description in terms of any Riemannian geometry - should provide a description of physics which is independent, not only of the choice of co-ordinates, but also of the choice of background manifold on which those co-ordinates live. On the face of it string theory, at least in its conventional form, does not satisfy this principle.

It is sometimes argued that string theory is, in fact, background independent since the equations of motion which arise when the (Bosnic) string is coupled to the metric of the ``worldvolume'' (the background manifold in which the string is propagating) turn out to include Einstein's field equations for gravity. In  other words, Weyl invariance - one of the fundamental symmetries of the string worldsheet, \emph{requires} \todo{insert refs} that the background geometry must satisfy Einstein's equations. While technically correct, it begs the question, if general relativity ``emerges'' from string theory then why is it not possible to quantize the string on an arbitrary background to begin with? As anybody who has studied the basics of string theory knows, the standard quantization procedure \emph{assumes} that the worldvolume has a flat metric. It is, thus far, not technically feasible to do away with this assumption.

There is another sense in which string theory falls short of being a theory of quantum gravity. This, however, rests upon a philosophical perspective, about the nature of quantum gravity, which one may or may not choose to subscribe to. This is the viewpoint that in any quantum theory of gravity, geometry itself must be quantized. Again, one might say that gravitons - the quantum perturbations of the metric - are ``quanta of geometry''. However, gravitons \emph{live} upon some background manifold. Moreover, as mentioned, they are \emph{perturbative} by construction. Gravitons cannot provide a quantum description of geometry, anymore than quantizing water waves can provide a picture of the molecular structure of water\footnote{This viewpoint has been most clearly articulated by Jacobson \cite{Jacobson1995Thermodynamics}}. These considerations bring us to the second major\footnote{the characterization of LQG as a ``major'' approach is the our choice and not necessarily reflective of the consensus in the broader quantum gravity community.} approach towards a quantum theory of gravity, known as Loop Quantum Gravity or LQG for short.

LQG arose almost by accident, something it has in common with string theory. Traditionally, there are two ways in which a classical theory can be quantized. These are the Hamiltonian and Lagrangian approaches. The Lagrangian or path integral approach follows the prescription first suggested by Dirac and then made concrete by Feynman. There one views the classical action associated to a given evolution as corresponding to a \emph{phase angle} which determines the complex weight of the associated evolution. This approach respects spacetime covariance since here the central object is the action which is invariant under spacetime transformations by construction. The Hamiltonian approach, on the other hand, involves making a choice of a spacelike surface $\Sigma_t$ and a corresponding timelike vector $t^\mu$, normal to $\Sigma_t$ (here $t$ is a continuous parameter which labels the family of surfaces). This approach, therefore, necessarily does not respect spacetime covariance since it involves making a specific choice of the foliation of the background into spacelike surfaces.

In the decades prior to the advent of string theory a great deal of effort was put into attempting to quantize gravity via both the Lagrangian and Hamiltonian approaches. The Hamiltonian approach involves, as stated earlier, a choice of the foliation of the background geometry into a collection of spacelike surfaces $\Sigma_t$. Central to this approach was the ADM (Arnowitt-Deser-Misner) formalism which allows one to construct a Hamiltonian $H_{GR}$ for general relativity starting from the Einstein-Hilbert action $S_{EH}$. The resulting phase is co-ordinatized by a configuration variable which is the 3-metric $h_{ab}$ of the ``leaves'' $\Sigma_t$ of the foliation and a momentum variable which is a function of a the extrinsic curvature $k_{ab}$ of the leaves\footnote{Our notational convention is the following. Lowercase Greek letters from the middle of the alphabet $\mu, \nu, \ldots$ are spacetime indices which run from $(0 \ldots 3)$, whereas lowercase Roman letters $a,b,\ldots$ are spatial indices for quantities which live solely on the surfaces $\Sigma_t$ and take values in $(1,2,3)$}. The gravity Hamiltonian turns out to be a sum of two constraints known as the diffeomorphism (or ``momentum'') constraint $H_{diff}$ and the Hamiltonian constraint $H_{ham}$:

\begin{equation}
	H_{GR} = H_{diff} + H_{ham},
\end{equation}

and these are, in turn, functionals on the phase space of general relativity written in terms of $h_{ab}$ and $k_{ab}$. The idea then is that physical states of the theory $\ket{\Psi_{phys}} $, which can be written as functionals $\Psi[h_{ab}]$ of the 3-metric $h_{ab}$ , must be annihilated by these constraints:

\begin{equation}
	H_{GR} \ket{\Psi_{phys}} \equiv 0
\end{equation}

While this procedure is straightforward in principle, in practice is was impossible to implement in the quantum theory due to highly complicated non-polynomial dependence of the diffeomorphism and Hamiltonian constraints on the configuration and momentum variables. This unfortunate state of affairs persisted until the 1980s when Abhay Ashtekar recast general relativity as a theory of a Minkowski tetrad $E^I_\mu$ and a self-dual $\sltwoc$ connection $A_\mu^{IJ}$. Ashtekar realised that the constraints of general relativity, when expressed in terms of these ``new variables'', simplified drastically and became polynomial functions of the tetrad and connection variables. It was quickly realised by various researchers that the resulting theory could be quantized using the same methods used to quantize Yang-Mills gauge theory and that the diffeomorphism constraint could be solved exactly in terms of so-called \emph{spin-network} states.

Following this, work by Rovelli and Smolin \todo{insert ref} and by Ashtekar, Rovelli and Smolin demonstrated the most remarkable feature of this theory, which came to be known as ``Loop Quantum Gravity'', was that one could construct quantum operators for geometric quantities such as areas of two dimensional surfaces and volumes of three dimensional regions. Moreover, these operators could be diagonlized exactly in the spin-network basis and a lower bound on the smallest possible quantum of area and quantum of volume could be derived. This was the first time that physicists had discovered the ``atoms of space'' or the ``quanta of geometry'' in the true sense of the expression.

\todo[inline]{Insert brief intro to spin nets}

The following is the outline of the paper. In \autoref{sec:intro} we recall the Nambu-Goto action and its symmetries. We explain how the bulk (embedding manifold) geometry is coupled to the worldsheet geometry and the conventional argument for the statement that string theory describes not only the object but also its background (the geometry it is propagating) in and is therefore a complete theory of quantum gravity. We critique this claim and explain in what way LQG might provide a resolution of some of the puzzles of string theory. In \autoref{sec:modified-ng} we present a phenomenological modification of the Nambu-Goto action based on the observation from LQG that there exists a non-zero, positive, eigenvalue of the area operator. We show that this action can be expanded in terms of a small coupling constant $g = l_{pl}/l_s$ where the Planck length - the length at which quantum geometric effects become important - $l_{pl}$ is much smaller than the string scale $l_s$ and therefore $g \ll 1$. We point out that the lowest order correction to the NG action can be viewed as a bimetric action for the string. We derive the classical equations of motion for this area-corrected NG action. In \autoref{sec:inverse-metric-Polyakov} we construct the Polyakov version of the modified NG action, show that the equations of motion it yields are the same as those coming from the modified NG action. Further we solve the string equations of motion to find the fundamental modes and perform the quantization of the Bosonic string in the usual manner. In \autoref{sec:string-geometry} we describe how to construct an action for the Bosonic string which encodes the geometry of the bulk spacetime in the language of the connection formulation of general relativity. We discuss possible avenues for quantizing this action and further studying its implications for the relationship between string theory and LQG.

In \autoref{app:bimetric-string} we discuss further the bimetric extension of the string action in both the Nambu-Goto and Polyakov version.

\section{Strings in Background Fields and Emergent Gravity}\label{sec:strings-gravity}

Let us begin by recalling the Polyakov action for the bosonic string.

\begin{equation}\label{eqn:polyakov-action}
	S_{P} = -\frac{T}{2} \int \diff \tau \wedge \diff \sigma \sqrt{-g} g^{ab} \partial_a X^\mu \partial_b X^\nu \eta_{\mu\nu}.
\end{equation}

Here $\mu, \nu \in \{0,1,\ldots,D-1\}$ are co-ordinates of the $D$ dimensional worldvolume (the geometry in which the string propagates), $a,b \in \{0,1\}$ are co-ordinates on the string worldsheet, $X^\mu$ are the embedding co-ordinates which specify the location of a point on the string worldsheet in the bulk worldvolume, $\eta_{\mu\nu}$ is the flat Minkowski metric on the worldvolume, $g_{ab}$ is the metric on the string worldsheet, $T$ is the string tension and $\tau, \sigma $ are the co-ordinates on the string worldsheet.

Now, one proceeds in the usual way by determining the symmetries of the action \eqref{eqn:polyakov-flat}, varying the action to find the equations of motion, fixing the gauge using the Weyl freedom of the worldsheet and then solving the classical equations of motion. Imposition of (bosonic) commutation relations on the operator versions of the embedding fields $\hat X^\mu$ then leads us to description of the quantum state of the bosonic string in terms of an infinite ladder of harmonic oscillators which obey the Virasoro algebra.

The obvious drawback of this approach is that the background metric is non-dynamical and is fixed to be the flat Minkowski metric $\eta_{\mu\nu}$. Clearly, one would like to be able to understand the physics of a string propagating on an arbitrary curved background. It wouldn't make much sense to refer to string theory as a theory of ``quantum gravity'' if strings can only be described on flat backgrounds. The way this is accomplished is by treating the metric of the worldvolume as a ``background field'' $G_{\mu\nu}$, in terms of which the Polyakov action becomes:

\begin{equation}\label{eqn:polyakov-gravity}
	S'_{P} = -\frac{T}{2} \int \diff \tau \wedge \diff \sigma \sqrt{-g} g^{ab} \partial_a X^\mu \partial_b X^\nu G_{\mu\nu}(X),
\end{equation}

where the bulk metric is now a function of the bulk co-ordinates $G_{\mu\nu}(X)$. Now this metric is still non-dynamical because the action \eqref{eqn:polyakov-gravity} does not contain any terms with time derivatives of $G_{\mu\nu}$. However, if we view the action purely as a two-dimensional theory of $D$ scalar fields, then the bulk metric can be viewed as a collection of \emph{coupling constants}, rather than a dynamical entity which exists independent of the string. One can now proceed in the usual manner for any field theory and calculate the beta function of the coupling constants of the theory as a function of the energy scale.

Then, as shown long ago by Friedan \cite{Friedan1985Nonlinear} (see also \cite{Callan1989Sigma}, \cite{Callan1985Strings} or \cite[Sec 3.7]{Polchinski1998aString}, \cite[Sec 7.2]{Tong2010Lectures} for a more pedagogical explanation), that the garviton beta function is proportional to $R_{\mu\nu}$, the Ricci curvature of the bulk geometry. The requirement of Weyl invariance of the string worldsheet implies that this beta function should vanish:
\begin{equation}\label{eqn:ricci-flat}
	\beta(G_{\mu\nu}) \propto R_{\mu\nu} = 0.
\end{equation}
Therefore we find that conformal invariance of the string worldsheet \emph{implies} that the background geometry in which the string propagates satisfies the \emph{vacuum} Einstein equations. It is this result which is often cited as evidence for the claim that string theory is a \emph{background independent} theory of quantum gravity.

In our opinion, however, this result is insufficient evidence for asserting that string theory is background independent. Our reasons are the following:

\begin{enumerate}[itemsep=0.5em]
	\item While the conformal invariance of the string worldsheet implies that the background is Ricci flat, this, in itself, is not very surprising. After all conformal invariance of worldsheet is equivalent to the statement of diffeomorphism invariance of the worldsheet, and in order for this to be consistent with the bulk geometry, the bulk geometry should also have a description in terms of diffeomorphism invariant physics. The simplest form of this requirement is that the background should satisfy the vacuum Einstein equations \todo{Does this make sense? Check!}.
	\item One can ask what is the quantum mechanical description of the bulk geometry? It is often suggested that the bulk geometry corresponds to a coherent state or a graviton condensate, where the graviton are themselves excitations of the fundamental string. However, there does not appear to be an actual mathematical realization of this statement in terms of non-perturbative physics.
	\item Finally, as mentioned in the introduction, in any \emph{quantum} theory of gravity one would expect to be able to describe the ``atoms of geometry''. There is no sense in which this is realised in this picture of gravity emerging from stringy physics. Even though one can calculate higher order corrections to the graviton beta function and this provides us with the quantum corrected effective action for the bulk geometry, all the physical quantities are still defined in terms of continuous fields such as the metric. For \eg, to the best of our knowledge, there is no clear sense in which one construct a \emph{non-perturbative} state which corresponds to a superposition of geometries using the quantum theory of a bosonic string alone.
\end{enumerate}

Despite all the concerns string theory has emerged as the primary candidate for a theory of quantum gravity. There are good reasons for this. AdS/CFT - the most concrete version of a \emph{non-perturbative} theory of quantum gravity - arose out of string theory. AdS/CFT is also far from being a complete quantum theory of quantum gravity. Our universe, for instance, is described by a positive cosmological constant $\Lambda > 0$ rather than $\Lambda < 0$ as is implied by the ``Anti-deSitter'' part of the correspondence. However, the correspondence has passed numerous checks and is now considered part of the established canon. Moreoever, it has led to numerous theoretical insights. We have gained a deep understanding of the properties of real world systems referred to as ``strange metals'' \todo{insert refs} by studying their dual gravitational theories. The significance of quantum error correction and quantum information for a theory of quantum gravity was also first most clearly illuminated via AdS/CFT \todo{insert refs}. For these, are other reasons too numerous to mention here, most theorists still regard string theory as the primary (or even the sole) candidate for a theory of quantum gravity.

On the other hand LQG is also built upon a rigorous mathematical framework which follows the well-established rules of quantization\footnote{Though there are those who disagree with this assertion \todo[inline]{insert refs}}. Moreoever it allows us to construct a non-perturbative framework for quantum geometry which makes it possible to probe regimes which do not appear to be accessible via string theory.

Thus, we are now at an impasse. We have two theories, both built upon rigorous mathematical foundations, but which do not appear to be concordant with each other. There are two possibilities. The first is that one of the two approaches is fundamentally flawed in some way and must be rejected. The second is that we are missing some crucial ingredient which would allow us to find a missing link between the two theories. We feel that it worthwhile to explore the second option in greater detail, if for no other reason then to rule out the possibility of such a link between these two theories.

Before proceeding let us state our ``central dogma''. We take as physically well-motivated the central result of LQG that at the Planck scale well-defined geometric operators - such as those which measure the area and volume of a given region of space - exist, have discrete spectra and are gapped, \ie, the smallest eigenvalue is greater than zero. We now wish to understand how to construct a worldsheet action which will reflect this central dogma.

\section{Modified Nambu Goto Action}\label{sec:modified-ng}

In a very real sense string theory is a quantum theory of geometry. After all, the Nambu-Goto action:
\begin{equation}\label{eqn:ng-action}
	S_{NG} = -T \int \diff ^2 x \sqrt{- \det h}
\end{equation}
is nothing more than the \emph{area} of the string worldsheet. So when we are constructing the quantum theory of the string we are constructing a theory in which the fundamental excitations are geometric in nature. What we would like to do is to modify the string action in a way which incorporates the insight from LQG that at the Planck scale there is a minimum quantum of area. One simple way in which to do this is to modify the integrand in \eqref{eqn:ng-action} as follows:
\begin{equation*}
	\sqrt{- h} \rightarrow \sqrt{- (h + g \Delta)}.
\end{equation*}
Here, $g = l_{pl}/l_s$, where $l_{pl}$ is the Planck scale, the scale at which quantum geometric effects become important; $l_s$ is the string scale where quantum geometric effects due to area quantization become small; $\Delta$ is a constant which corresponds to the square of the minimal quantum of area. In the limit that $l_s \gg l_{pl}$, $g \ll 1$ and the string action reduces to the conventional Nambu-Goto action with small corrections coming from quantum geometry. In this limit we can expand the integrand as follows:
\begin{equation} \label{eqn:inverse-metric-correction}
	\sqrt{- (h + g \Delta)} \sim \sqrt{- h}\left(1 - \frac{1}{2} \frac{g \Delta}{(- h)} + \mc{O}(g^2/h^2)\right).
\end{equation}
Presumably the limit $g \ll 1$ corresponds to when the area of the string worldsheet is large compared to $\sqrt{h} \gg \Delta$. Thus in this limit $ g^2 \ll 1$ and $ h^2 \gg 1$ because of which we can drop the $\mc{O}(g^2/h^2)$ corrections. The final area corrected Nambu-Goto action becomes:
\begin{equation} \label{eqn:modified-NG-action}
	S'_{NG} = -T \int \diff ^2 x \left( \sqrt{- h} - \frac{1}{2} \frac{g \Delta}{\sqrt{- h}} \right).
\end{equation}
This action has, to the best of our knowledge, not been explored in the string theory literature. However, such an inverse area term arises very naturally in the LQG approach and goes by the name of ``inverse triad'' corrections, in the context of the 3D volume of spatial hypersurfaces.

The action \eqref{eqn:modified-NG-action} has some interesting properties. First, on the face of it it does not appear to satisfy diffeomorphism invariance of the string worldsheet because of the second term. Given that we are assuming the existence of a minimum quantum of area, it is obvious that diffeomorphism invariance, at least in the classical sense, will no longer be an exact symmetry of the theory. However, if we look at that expression we see that it can be written as:
\begin{equation} \label{eqn:modified-NG-action-v2}
	S'_{NG}[h_{ab}] = -T \int \diff ^2 x \left( \sqrt{- h} - \frac{1}{2} {g \Delta}{\sqrt{- (h^{-1})}} \right),
\end{equation}
where $h^{-1}$ is now the determinant of the \emph{inverse} metric $h^{ab}$. So, while the second term is not diffeomorphism invariant \wrt the original metric $h_{ab}$ it is clearly diffeomorphism invariant \wrt the inverse metric $h^{ab}$.

Second, in the expression \eqref{eqn:modified-NG-action} we can see a duality between the first and second terms. If we replace the metric $h_{ab}$ by its inverse $h'_{ab} \equiv h^{ab}$, then the action in terms of $h'_{ab}$ is the same as \eqref{eqn:modified-NG-action}, but with the difference that the factor of $g \Delta/2$ is mapped to $2/(g\Delta)$:
\begin{align} \label{eqn:modified-NG-action-v3}
	S'_{NG}[h'_{ab}] & = -T' \int \diff ^2 x \left( \sqrt{- h'} - \frac{2}{g \Delta} \sqrt{- (h')^{-1}} \right) \nonumber \\
	                 & = -T' \int \diff ^2 x \left( \sqrt{- h'} - \onehalf {g' \Delta'}\sqrt{- (h')^{-1}} \right),
\end{align}
which has the same form as the original action \eqref{eqn:modified-NG-action}, as long as make the replacements $T \rightarrow T' = \frac{1}{2} {g \Delta} T $ and $ \frac{1}{2} {g \Delta} \rarrow \frac{1}{2} {g' \Delta'} = \frac{2} {g \Delta}$
This is a duality between the physics at the Planck scale and the physics at the string scale, \ie between strong coupling ($g \gg 1$) and weak coupling ($g \ll 1$), small quantum of area ($\Delta$) and large quantum of area ($\Delta'$) and between large string tension $T$ and small string tension $T'$.

Now, one might object and say that having an action which is the sum of Nambu-Goto actions for two different metrics seems strange. However, such actions have been extensively studied under the heading of ``bimetric gravity'', with the earliest work dating as far back as 1940 in two papers \cite{Rosen1940General,Rosen1940aGeneral} by Einstein's future collaborator Nathan Rosen (the `R' in `EPR' and `ER'). In 2010, bimetric gravity gained great popularity due to the seminal paper \cite{de-Rham2011Resummation} which showed that by introducing a second \emph{reference} metric into a model for massive gravity one can get rid of the Boulware-Deser ghost \cite{Boulware1972Can-Gravitation} which otherwise plagues theories with massive gravitons. The \emph{reference} metric introduced in the work \cite{de-Rham2011Resummation} was taken to be a flat metric. However later work \cite{Hassan2012Bimetric,Hassan2012Ghost-Free} showed that the second metric need not be flat and that an action of the following form:
\begin{align}\label{eqn:hassan-bimetric}
	S & = M_g^2 \int \diff^4 x \sqrt{-g} R^{(g)} + M_f^2 \int \diff^4 x \sqrt{-f} R^{(f)} \nonumber \\
	  & + 2 m^2 M^2_{eff} \int \diff^4 x \sqrt{-g}\sum_{n=0}^{4}\beta_n e_n (\sqrt{g^{-1} f}),
\end{align}
where $g_{\mu\nu}$ and $f_{\mu\nu}$ are \emph{arbitrary} metrics, $M_g^2, M_f^2$ are the Planck masses for the two sectors respectively. $R^{(g)}, R^{(f)}$ are the respective Ricci scalars for each sector. The third term is an interaction term which is responsible for making one of the gravitons massive while the other remains massless. $M^2_{eff}$ is an \emph{effective} Planck mass given by:
\begin{equation}\label{eqn:hassan-effective-mpl}
	M^2_{eff} = \left( \frac{1}{M^2_g} + \frac{1}{M^2_f} \right)^{-1}
\end{equation}

\subsection{Modified Equations of Motion}\label{sec:modified-eom}

Following the procedure done in \cite{Zwiebach2009A-First}, the correction(\ref{eqn:modified-NG-action}) does not change the form of the equations of motion and neither of the boundary conditions, i.e they are still given by
\begin{equation} \label{eqn:modified-NG-eom}
	\mathrm{EoM:} \partial _{\tau} \PS ^{\tau} _{\mu} + \partial _{\sigma} \PS ^{\sigma} _{\mu} = 0
\end{equation}
and
\begin{equation} \label{eqn:modified-NG-bc}
	\mathrm{B.C.:} \PS ^{\sigma} _{\mu} \delta X^{\mu} \big| _{\sigma = 0} ^{\sigma = \sigma _1} = 0 ,
\end{equation}
where
\begin{align} \label{eqn:modified-NG-tau-currents}
	\PS ^{\tau} _{\mu} & = \frac{\partial \Lagr ' _{N G}}{\partial \Dot{X}^{\mu}} \nonumber                                                                 \\
	                   & = - T \frac{(\Dot{X} \cdot X') X' _{\mu} - (X')^2 \Dot{X}_{\mu}}{\sqrt{- h}} \left( 1 + \frac{g \Delta}{2 (- h)} \right) \nonumber \\
	                   & = \PS ^{\tau} _{\mu (N G)} \left( 1 + \frac{g \Delta}{2 (- h)} \right)
\end{align}
\begin{align} \label{eqn:modified-NG-sigma-currents}
	\PS ^{\sigma} _{\mu} & = \frac{\partial \Lagr ' _{N G}}{\partial X^{\prime \mu}} \nonumber                                                                     \\
	                     & = - T \frac{(\Dot{X} \cdot X') \Dot{X}_{\mu} - (\Dot{X})^2 X' _{\mu}}{\sqrt{- h}} \left( 1 + \frac{g \Delta}{2 (- h)} \right) \nonumber \\
	                     & = \PS ^{\sigma} _{\mu (N G)} \left( 1 + \frac{g \Delta}{2 (- h)} \right)
\end{align}
and $\PS ^{\tau} _{\mu (N G)}$, $\PS ^{\sigma} _{\mu (N G)}$ are the regular Nambu-Goto world-sheet currents.

Using the static gauge $\tau = t$ and the transverse gauge with $\frac{\partial X}{\partial s} \cdot \frac{\partial X}{\partial t} = 0$, where $s$ is the length along the string parameter, we find that the new form for the conserved energy per infinitesimal piece of the string is the regular energy of the Nambu-Goto string times the correction factor
\begin{equation} \label{eqn:string-energy}
	E \diff \sigma = T \gamma _{v_{\perp}} \left( 1 + \frac{g \Delta}{2 (- h)} \right) \diff s ,
\end{equation}
and the form of the wave equation maintains its shape as in
\begin{equation} \label{eqn:corrected-NG-wave-equation}
	\mu _{eff} \frac{\partial ^2 \Vec{X}}{\partial t^2} - \frac{\partial}{\partial s} \left[ T_{eff} \frac{\partial \Vec{X}}{\partial s} \right] = 0 ,
\end{equation}
where the effective mass density and effective tension are given by
\begin{align}
	\mu _{eff} & = \mu \gamma _{v_{\perp}} \left( 1 + \frac{g \Delta}{2 (- h)} \right) \label{eqn:effective-mass}            \\
	T_{eff}    & = \frac{T}{\gamma _{v_{\perp}}} \left( 1 + \frac{g \Delta}{2 (- h)} \right) . \label{eqn:effective-tension}
\end{align}

Rewriting eq(\ref{eqn:corrected-NG-wave-equation}) in terms of $\sigma$ derivatives and parameterizing $\sigma$ as (let $F^+ = 1 + g \Delta / 2 (- h)$)
\begin{equation} \label{eqn:sigma-parameterization}
	\sigma (q) = \frac{1}{T} \int _{0} ^{q} \diff E \frac{1}{(F^+)^2} = \frac{E (q)}{T (F^+)^2} ,
\end{equation}
we find the wave equation
\begin{equation} \label{eqn:simplified-modified-wave-eqn}
	(F^+)^2 \frac{\partial ^2 \Vec{X}}{\partial t^2} - \frac{\partial ^2 \Vec{X}}{\partial \sigma ^2} = 0 ,
\end{equation}
from which we can see that the correction factor modifies the speed of propagation of the wave in the string as
\begin{equation} \label{eqn:wave-speed-corrected}
	v = \frac{c}{F^+}.
\end{equation}

\section{Inverse Metric Polyakov} \label{sec:inverse-metric-Polyakov}

In the same way we considered the quantum geometry inverse metric correction to the Nambu-Goto action, it is also natural to consider such correction to the Polyakov string. we start with the action(\ref{eqn:polyakov-gravity}) and proceed with the inverse area correction(\ref{eqn:inverse-metric-correction}), which yields our corrected Polyakov action
\begin{multline} \label{eqn:inverse-metric-Polyakov}
	S_{I A P} = - \frac{T}{2} \int \diff ^2 x \left( \sqrt{- g} - \frac{k \Delta}{2 \sqrt{- g}} \right) \times \\ \times g^{a b} \partial _{a} X^{\mu} \partial _{b} X^{\nu} G_{\mu \nu} .
\end{multline}
The equations of motion for the world-sheet metric are completely unchanged by this correction, so they are
\begin{equation} \label{eqn:vary-wrt-g}
	\frac{\delta S_{I A P}}{\delta g^{c d}} \overset{!}{=} 0
\end{equation}
from which we obtain the stress-energy tensor for the worldsheet:
\begin{multline} \label{eqn:inverse-area-em-tensor}
	T_{c d} := \partial _{c} X^{\mu} \partial _{d} X^{\nu} G_{\mu \nu} - \\ - \frac{1}{2} g_{c d} g^{a b} \partial _{a} X^{\mu} \partial _{b} X^{\nu} G_{\mu \nu} = 0 ,
\end{multline}
from which we can extract the metric $g_{c d}$ as
\begin{equation} \label{eqn:ws-metric}
	g_{c d} = 2 f \partial _{c} X^{\mu} \partial _{d} X^{\nu} G_{\mu \nu} ,
\end{equation}
where the function $f$ is given by
\begin{equation} \label{eqn:inverse-area-f}
	\frac{1}{f} = g^{a b} \partial _{a} X^{\mu} \partial _{b} X^{\nu} G_{\mu \nu} .
\end{equation}

As for variation w.r.t the embbeding fields $X^{\lambda}$, we have
\begin{equation} \label{eqn:vary-wrt-X}
	\frac{\delta S_{I A P}}{\delta X^{\lambda}} \overset{!}{=} 0
\end{equation}

\begin{multline} \label{eqn:X-eom-inverse-area}
	F^{-} \sqrt{- g} g^{a b} \partial _{a} \partial _{b} X^{\nu} G_{\lambda \nu} + \partial _{a} \left( F^{-} \sqrt{- g} g^{a b} \right) \partial _{b} X^{\nu} G_{\lambda \nu} = \\ = \frac{1}{2} F^{-} \sqrt{- g} g^{a b} \partial _{a} X^{\mu} \partial _{b} X^{\nu} \partial _{\lambda} G_{\mu \nu} ,
\end{multline}
where $F^{-} = 1 - k \Delta / 2 (- g)$. Using conformal symmetry $g^{a b} = (\phi (x))^{-1} \eta ^{a b}$ and assuming flat background $G_{\mu \nu} = \eta _{\mu \nu}$, eq(\ref{eqn:X-eom-inverse-area}) simplify to
\begin{equation} \label{eqn:X-eom-simplified}
	\eta ^{a b} \partial _{a} \partial _{b} X^{\mu} + \eta ^{a b} \partial _{a} \ln (F^{-}) \partial _{b} X^{\mu} = 0 .
\end{equation}

\subsection{Simplifying The Equations of Motion} \label{sec:solving-eoms-inverse-area}

Equation(\ref{eqn:X-eom-simplified}) seems a bit odd, but we can polish it by choosing a plane-wave ansatz for the embedding fields $X^{\mu} = X^{\mu} _{0} \e ^{-i (E \tau - p \sigma)}$ and using it to fix a form for the conformal function $\phi (x)$ appearing in $F^{-}$. By using the ansatz in eq(\ref{eqn:X-eom-simplified}), it becomes
\begin{equation}
	(E^{2} - p^{2}) X^{\mu} + i (E \partial _{\tau} \ln (F^{-}) - p \partial _{\sigma} \ln (F^{-})) X^{\mu} = 0 ,
\end{equation}
from which we can use the freedom in the conformal function to fix the second term as $- m^2 X^{\mu}$, with $- m^2 = p^2 - E^2$ on-shell, and this in turn implies that the derivatives of $\ln (F^{-})$ are given by
\begin{equation} \label{eqn:derivatives-F-minus}
	\partial _{\tau} \ln (F^{-}) = i E , \ \partial _{\sigma} \ln (F^{-}) = i p ,
\end{equation}
thus we have
\begin{equation} \label{eqn:ln-F}
	\ln (F^{-}) = i (E \tau + p \sigma) + a , \ a \in \mathbb{C}
\end{equation}

\begin{equation} \label{eqn:F-minus-exp}
	F^{-} = A \e ^{i (E \tau + p \sigma + \varphi)} , \ A > 0
\end{equation}
\begin{equation} \label{eqn:phi}
	\phi (\tau , \sigma) = \sqrt{\frac{k \Delta}{2 \left( 1 - A \e ^{i (E \tau + p \sigma + \varphi)} \right)}} .
\end{equation}
At first glance it seems problematic that we have a complex conformal function $\phi (x)$, but the complex structure comes from our ansatz for the embedding fields, and when we choose either the real or imaginary part of the latter it should also pick the respective component of $\phi$. With this choice, we thus turned eq(\ref{eqn:X-eom-simplified}) into a Klein-Gordon equation
\begin{equation} \label{eqn:X-Klein-Gordon-eqn}
	\left( \partial _{\sigma} ^2 - \partial _{\tau} ^2 \right) X^{\mu} - \mu ^2 X^{\mu} = 0 .
\end{equation}

\subsection{Relation With The Nambu-Goto Analysis} \label{sec:relation-with-NG-case}

At first glance, our recent analysis of the Polyakov string with inverse metric correction seems to not be related to what we found in the analysis of the Nambu-Goto string. However, as pointed out in \cite{Grimshaw2010Homogenization}, wave equations with a variable speed (the result of the Nambu-Goto analysis) can be turned into Klein-Gordon equations (what we just found). While they're work deals only with space-varying wave speed, it is straight forward to generalise the procedure to also include time dependence, but ironically enough the calculations lead to the conclusion that the wave speed $v = 1 / F^{+}$ is actually time independent.

Following the procedure found in \cite{Grimshaw2010Homogenization}, we start with eq(\ref{eqn:simplified-modified-wave-eqn}) obtained in sec(\ref{sec:modified-eom}), renaming $\sigma$ as $x$ and dividing by $(F^{+})^2$ such that we have $v^2$ on the second term like the mentioned work, and perform a change of variables $(t , x) \rightarrow (\tau (t , x) , \sigma (t , x))$, where without loss of generality (one needs just to rescale and/or rotate the coordinates) we set
\begin{multline} \label{eqn:new-coordinates-constraints-quadratic}
	\ \ \ \ \ \ \ \ \ \ (\partial _{t} \sigma)^2 - v^2 (\partial _{x} \sigma)^2 = v^2 (\partial _{x} \tau)^2 - (\partial _{t} \tau)^2 \\
	\partial _{t} \tau \partial _{t} \sigma = v^2 \partial _{x} \tau \partial _{x} \sigma , \ \ \ \ \ \ \ \ \ \ \ \
\end{multline}
or more compactly,
\begin{multline} \label{eqn:new-coordinates-constraints-compact}
	\ \ \ \ \ \ \ \ \ \ \ \ \ \ \ \ \ \ \ \ \ \ \ \ \ \ \ \partial _{t} \tau = v \partial _{x} \sigma \\
	\partial _{t} \sigma = v \partial _{x} \tau \ \ \ \ \ \ \ \ \ \ \ \ \ \ \ \ \ \ \ \ ,
\end{multline}
which is equivalent to
\begin{multline} \label{eqn:new-coordinates-constraints-second-order}
	\ \ \ \ \ \ \ \ \ \ \ \partial _{t} ^2 \tau - v^2 \partial _{x} ^2 \tau = \partial _{t} v \partial _{x} \sigma + v \partial _{x} v \partial _{x} \tau \\
	\partial _{t} ^2 \sigma - v^2 \partial _{x} ^2 \sigma = \partial _{t} v \partial _{x} \tau + v \partial _{x} v \partial _{x} \sigma . \ \ \ \ \ \ \
\end{multline}
This turns eq(\ref{eqn:simplified-modified-wave-eqn}) into
\begin{multline} \label{eqn:modified-wave-eqn-new}
	\partial _{\tau} ^2 X^{\mu} - \partial _{\sigma} ^2 X^{\mu} + \\
	+ \frac{\partial _{t} v \partial _{x} \sigma + v \partial _{x} v \partial _{x} \tau}{((\partial _{t} \tau)^2 - v^2 (\partial _{x} \tau)^2)} \partial _{\tau} X^{\mu} + \\
	+ \frac{\partial _{t} v \partial _{x} \tau + v \partial _{x} v \partial _{x} \sigma}{((\partial _{t} \tau)^2 - v^2 (\partial _{x} \tau)^2)} \partial _{\sigma} X^{\mu} = 0 .
\end{multline}
Next, we introduce $X^{\mu} = \kappa (\tau , \sigma) W^{\mu} (\tau , \sigma)$ such that terms proportional to first derivatives of $W^{\mu}$ cancel out. This is obtained by choosing $\kappa = v^{1/2}$ and concluding that $v$ only depends on $x$. With this, eq(\ref{eqn:modified-wave-eqn-new}) turns into a Klein-Gordon equation
\begin{equation}
	\partial _{\tau} ^2 W^{\mu} - \partial _{\sigma} ^2 W^{\mu} + \mu ^2 W^{\mu} = 0 ,
\end{equation}
with \textquotedblleft mass" squared given by
\begin{equation}
	\mu ^2 = \frac{(\partial _{x} v)^2 - 2 v \partial _{x} ^2 v}{4 \left( (\partial _{t} \tau)^2 - v^2 (\partial _{x} \tau)^2 \right)}
\end{equation}
and $\tau (t , x)$ satisfying constraints(\ref{eqn:new-coordinates-constraints-compact}). This shows that our action(\ref{eqn:inverse-metric-Polyakov}) is indeed in agreement with the inverse area corrected Nambu-Goto action(\ref{eqn:modified-NG-action}).

\subsection{Solving The String Klein-Gordon Equation}

From our analysis of the inverse-area corrected Polyakov action, we got $D$ finite-space one-dimensional Klein-Gordon equations (\ref{eqn:X-Klein-Gordon-eqn}), with constraints given by eq\eqref{eqn:inverse-area-em-tensor}. The constraints are nothing new, they're precisely the same as the ones of regular String Theory. What's new in our analysis is the $\mu ^2$ term, which makes it so the solution is given by Fourier transform
\begin{multline} \label{eqn:KG-Polyakov-solution-initial}
	X^{\mu} (\tau , \sigma) = \frac{1}{2 \pi \alpha '} \sum _{p \in \mathbb{Z}} \frac{1}{2 E_{p}} \Big( \alpha ^{\mu} _{p} \e ^{i (- E_{p} \tau + p \sigma)} + \\
	+ \beta ^{\mu} _{p} \e ^{i (E_{p} \tau + p \sigma)} \Big) ,
\end{multline}
with $E_{p} = \sqrt{p^2 + \mu ^2}$. From here on, following an analogous procedure to the one found in \cite{Tong2009Lectures}, reality of $X^{\mu}$ implies $(X^{\mu})^{*} = X^{\mu}$, from which we get
\begin{equation} \label{eqn:reality-consequence}
	\beta ^{\mu} _{n} = (\alpha ^{\mu} _{- n})^{*} .
\end{equation}

When looking at the constraints, we see that we can't decouple the exponentials in $\tau$ from any of the sums, thus we get two-index \textquotedblleft Virasoro modes"
\begin{align}
	L_{n , p}              & := \frac{n - p}{E_{n - p}} \alpha ^{\mu} _{p} \alpha ^{\nu} _{n - p} \eta _{\mu \nu} = 0 \label{eqn:"Virasoro"-modes-1}         \\
	\widetilde{L} _{n , p} & := \frac{n - p}{E_{n - p}} \alpha ^{\mu} _{p} (\alpha ^{\nu} _{p - n})^{*} \eta _{\mu \nu} = 0 \label{eqn:"Virasoro"-modes-2} .
\end{align}

Before proceeding to quantization, we have one last remark: the 0-th term of the solution reads
\begin{equation} \label{eqn:X-sol-0-term}
	\frac{1}{2 \pi \alpha '} \frac{1}{2 \mu} \left( \alpha ^{\mu} _{0} \e ^{- i \mu \tau} + (\alpha ^{\mu} _{0})^{*} \e ^{i \mu \tau} \right) ,
\end{equation}
which is clearly divergent in the $\mu \rightarrow 0$ limit unless $\mathfrak{Re}(\alpha ^{\mu} _{0}) = 0$. Moreover, if we want to recover the classic String Theory expression in this regime, it is not difficult to see that we require
\begin{equation} \label{eqn:a-p-relation}
	\alpha ^{\mu} _{0} = 4 \pi i (\alpha ')^2 p^{\mu} .
\end{equation}
Since our Virasoro-like \textquotedblleft modes" are double-indexed without summation, the string rest mass $M^2$ is given only in terms of the 0-th oscillator mode
\begin{equation} \label{eqn:string-rest-mass}
	M^2 = - p^{\mu} p_{\mu} = \frac{1}{16 \pi^2 (\alpha ')^{4}} \alpha ^{\mu} _{0} \alpha ^{\nu} _{0} \eta _{\mu \nu} .
\end{equation}

Collecting our results, we have that the solution for our string is given by
\begin{multline} \label{eqn:KG-Polyakov-solution-final}
	X^{\mu} (\tau , \sigma) = \frac{1}{2 \pi \alpha '} \sum _{p \in \mathbb{Z}} \frac{1}{2 E_{p}} \Big( \alpha ^{\mu} _{p} \e ^{i (- E_{p} \tau + p \sigma)} + \\
	+ (\alpha ^{\mu} _{- p})^{*} \e ^{i (E_{p} \tau + p \sigma)} \Big)
\end{multline}
subject to the infinite family of constraints
\begin{equation}
	L_{n , p} = \widetilde{L}_{n , p} = 0
\end{equation} \label{eqn:"Virasoro"-constraints}
and 0-th oscillation mode related to center of mass momentum
\begin{equation*}
	\alpha ^{\mu} _{0} = 4 \pi i (\alpha ')^2 p^{\mu} .
\end{equation*}

\section{The Quantum Klein-Gordon String} \label{sec:quantum-KG-string}

In the following, we proceed to quantize the Klein-Gordon string via canonical covariant quantization, in a manner much similar to those found in \cite{Tong2009Lectures} and \cite{Green2012Superstring}. To start, we calculate the momentum canonical to the embbeding fields $X^{\mu}$, which reads
\begin{multline} \label{eqn:canonical-momentum}
	\Pi _{\mu} (\tau , \sigma) = F^{-} \frac{i}{(4 \pi \alpha ')^2} \sum _{p} \big( - \alpha ^{\nu} _{p} \e ^{i (- E_{p} \tau + p \sigma)} + \\
	+ (\alpha ^{\nu} _{- p})^{*} \e ^{i (E_{p} \tau + p \sigma)} \big) \eta _{\nu \mu} ,
\end{multline}
and then we proceed to turn the equal-time Poisson bracket relations
\begin{equation} \label{eqn:Poisson-brackets-1}
	\left\{ X^{\mu} (\tau , \sigma) , \Pi ^{\nu} (\tau , \sigma ') \right\} = \delta (\sigma - \sigma ') \eta ^{\mu \nu}
\end{equation}
\begin{equation} \label{eqn:Poisson-brackets-2}
	\left\{ X^{\mu} (\tau , \sigma) , X^{\nu} (\tau , \sigma ') \right\} = 0 = \left\{ \Pi ^{\mu} (\tau , \sigma) , \Pi ^{\nu} (\tau , \sigma ') \right\}
\end{equation}
into equal-time commutation relations between the now operators $\Hat{X}^{\mu}$ and $\Hat{\Pi}^{\nu}$:
\begin{equation} \label{eqn:canonical-commutator-1}
	\Big[ \Hat{X}^{\mu} (\tau , \sigma) , \Hat{\Pi}^{\nu} (\tau , \sigma ') \Big] \equiv i \delta (\sigma - \sigma ') \eta ^{\mu \nu}
\end{equation}
\begin{equation} \label{eqn:canonical-commutator-2}
	\Big[ \Hat{X}^{\mu} (\tau , \sigma) , \Hat{X}^{\nu} (\tau , \sigma ') \Big] = 0 = \Big[ \Hat{\Pi}^{\mu} (\tau , \sigma) , \Hat{\Pi}^{\nu} (\tau , \sigma ') \Big] .
\end{equation}
From these relations, we get commutation relations for the vibrational modes $\Hat{\alpha}^{\mu} _{p}$ and $(\Hat{\alpha}^{\nu} _{- k})^{\dagger}$:
\begin{equation} \label{eqn:non-vanishing-commutator}
	\Big[ \Hat{\alpha}^{\mu} _{p} , (\Hat{\alpha}^{\nu} _{- k})^{\dagger} \Big] = \eta ^{\mu \nu} \delta _{p , - k}
\end{equation}
\begin{equation} \label{eqn:vanishing-commutators}
	\Big[ \Hat{\alpha}^{\mu} _{p} , \Hat{\alpha}^{\nu} _{k} \Big] = 0 = \Big[ (\Hat{\alpha}^{\mu} _{- p})^{\dagger} , (\Hat{\alpha}^{\nu} _{- k})^{\dagger} \Big] ,
\end{equation}
which are just harmonic oscillator relations with
\begin{align}
	\Hat{a}^{\mu} _{p}             & = \Hat{\alpha}^{\mu} _{p} \label{eqn:annihilation-operator}                                       \\
	(\Hat{a}^{\mu} _{p})^{\dagger} & = (\Hat{\alpha}^{\mu} _{p})^{\dagger} , \ \mathrm{for} \ p \neq 0 . \label{eqn:creation-operator}
\end{align}
For $p = 0$ we have the center of mass momentum operator $\Hat{p}^{\mu}$. As usual, we define a vacuum state for the string to obey
\begin{equation} \label{eqn:annihilation-vacuum}
	\Hat{a}^{\mu} _{p} \ket{0} = 0 , \ \mathrm{for} \ p \neq 0 ,
\end{equation}
and for $p = 0$
\begin{equation} \label{eqn:momentum-vacuum}
	\Hat{p}^{\mu} \ket{0} = p^{\mu} \ket{0} .
\end{equation}
Our Fock space is built from the vacuum state $\ket{0}$ by operating with a sequence of creation operators
\begin{multline} \label{eqn:general-state}
	\ket{\psi} = ((a^{\mu _{1}} _{1})^{\dagger})^{n_{\mu _1}} ((a^{\mu _2} _{2})^{\dagger})^{n_{\mu _2}} ... \times \\
	\times ((a^{\nu _1} _{1})^{\dagger})^{n_{\nu _1}} ((a^{\mu _2} _{2})^{\dagger})^{n_{\nu _2}} ... \ket{0} .
\end{multline}

As usual, by the appearance of the Minkowski metric in the non-zero commutator
\begin{equation} \label{eqn:ghosts}
	\Big[ \Hat{a}^{\mu} _{p} , (\Hat{a}^{\nu} _{k})^{\dagger} \Big] = \eta ^{\mu \nu} \delta _{p k} ,
\end{equation}
we have ghosts in the theory which must be dealt with. When translating the constraints as quantum operators, we require that they have vanishing matrix elements when sandwiched by physical states $\psi$ and $\phi$:
\begin{equation} \label{eqn:L-constraint-quantum-1}
	\bra{\phi} \Hat{L}_{p , k} \ket{\psi} = 0 \, \,
\end{equation}
\begin{equation} \label{eqn:L-constraint-quantum-2}
	\bra{\phi} \Hat{\widetilde{L}}_{p , k} \ket{\psi} = 0 .
\end{equation}
$\Hat{L}_{p , k}$ has no ambiguities since it's composed of only annihilation operators, however, we have an ambiguity for $\Hat{\widetilde{L}}_{p , 0}$. To resolve this issue, we pick
\begin{equation} \label{eqn:L-normal-ordering}
	\Hat{\widetilde{L}}_{p , 0} = \frac{- p}{E_{- p}} (\Hat{\alpha}^{\mu} _{p})^{\dagger} \Hat{\alpha}^{\nu} _{p} \eta _{\mu \nu}
\end{equation}
with the annihilation operator to the right, which manifests in the imposition of the constraint as
\begin{equation} \label{eqn:shifted-L-constraint}
	\bra{\phi} \Hat{\widetilde{L}}_{p , 0} - c_{p} \ket{\psi} = 0 ,
\end{equation}
for some sequence $c_{p}$. The reason for it to be a sequence of $p$ and not a constant like in the regular String Theory literature is that in the usual theory, we would have a sum over $p$ and thus it would end up being a constant, but in our analysis we have this extra freedom, which we'll now see ends up being important. Classically, we had eq(\ref{eqn:string-rest-mass}) for the string rest mass, which we can rewrite as
\begin{equation} \label{eqn:alternative-classical-rest-mass}
	M^2 = - \frac{1}{(4 \pi (\alpha ')^2)^2} \lim _{n \rightarrow 0} \left( - \frac{E_{p}}{p} \widetilde{L}_{p , 0} \right) ,
\end{equation}
from where we can now see that the string mass spectrum will get shifted by
\begin{equation} \label{eqn:shifted-mass-spectrum}
	\Hat{M}^2 = - \frac{1}{(4 \pi (\alpha ')^2)^2} \left( (\Hat{\alpha} ^{\mu} _{0})^{\dagger} \Hat{\alpha}^{\nu} _{0} \eta _{\mu \nu} + \lim _{p \rightarrow 0} \frac{E_{p}}{p} c_{p} \right) .
\end{equation}
Since $E_{0} = \mu$, it seems like the natural candidate for $c_{p}$ would be some function $f (\mu , p)$ such that
\begin{equation} \label{eqn:cp-limit}
	\lim _{p \rightarrow 0} \frac{c_{p}}{p} = \mu
\end{equation}
like $c_{p} = \mu p$, $c_{p} = \sin (\mu p)$, $c_{p} = \tan (\mu p)$, etc, making it so the mass spectrum of the string gets shifted by the \textquotedblleft mass" term of the Klein-Gordon equation (\ref{eqn:X-Klein-Gordon-eqn}), $\mu ^2$.

The \textquotedblleft Virasoro-Klein-Gordon" algebra our constraint modes satisfy is given by
\begin{equation} \label{eqn:L-commutators-1}
	\Big[ \Hat{L}_{n , m} , \Hat{L}_{p , k} \Big] = 0
\end{equation}
\begin{multline} \label{eqn:L-commutators-2}
	\Big[ \Hat{L}_{n , m} , \Hat{\widetilde{L}}_{p , k} \Big] = \frac{m - n}{E_{m - n}} \Big( \Hat{L}_{n + k - p , n} \delta _{n - m , k} + \\
	+ \Hat{L}_{n - m + k - p , n - m} \delta _{n , k} \Big)
\end{multline}
\begin{multline} \label{eqn:L-commutators-3}
	\Big[ \Hat{\widetilde{L}}_{n , m} , \Hat{\widetilde{L}}_{p , k} \Big] = \frac{(m - n)}{E_{m - n}} \frac{(k - p)}{E_{k - p}} \times \\
	\times \Bigg( \frac{E_{- n}}{(- n)} \Hat{\widetilde{L}}_{k - p - n , k - p} \delta _{n - m , k} + \\
	+ \frac{E_{- k}}{(- k)} \Hat{\widetilde{L}}_{n - m - k , n - m} \delta _{n , k - p} \Bigg) .
\end{multline}
The commutator (\ref{eqn:L-commutators-2}) ends up with only annihilation operators in the R.H.S, so it has no ordering ambiguities, meanwhile the commutator (\ref{eqn:L-commutators-3}) seem like it would have ordering ambiguities for $k - p = 0$ and $n - m = 0$. However it has a $(m - n)(k - p)$ factor multiplying the whole expression, so the commutator is $0$, which can also be seen by setting $n = m$ or $k = p$ on the L.H.S, in which case we get $\Hat{\widetilde{L}}_{n , n}$, which is identically equal to $0$ by the $n - m$ factor in front, thus we would be commuting something with $0$, which has to yield 0 on the R.H.S. As a result, our \textquotedblleft Virasoro-Klein-Gordon" algebra has no anomalous terms.

\section{Strings and Quantum Geometry}\label{sec:string-geometry}

Uptil now we have studied a phenomenological modification of the string action. However, ultimately all phenomenological ideas must have a grounding in some exact, microscopic description of the system. Further, we have neglected one very important aspect of this whole endeavor, which is the description of the background geometry in the connection formalism and how to couple that to the Bosonic string. In this section we present a possible action in which the bulk geometry is encoded in the string action in terms of a connection living on the worldsheet.

First let us recall the form of the Polyakov action for the bosonic string \eqref{eqn:polyakov-action}:

\begin{equation*}
	S_{P} = -\frac{T}{2} \int \diff \tau \wedge \diff \sigma \sqrt{-g} g^{ab} \partial_a X^\mu \partial_b X^\nu \eta_{\mu\nu}.
\end{equation*}

If one was to ask a beginning graduate student as to how one can incorporate a gauge field in the above expression, the natural answer would be that we can promote the $X^\mu$ to vectors which live in the representation of some gauge group $\mc G$ and promote the ordinary derivative to a gauge covariant derivative \wrt a connection valued in $\mc G$. The resulting action would now have the form:

\begin{equation}\label{eqn:polyakov-gauged}
	S_{P} = -\frac{T}{2} \int \diff \tau \wedge \diff \sigma \sqrt{-g} g^{ab} \mc{D}_a X^\mu \mc{D}_b X^\nu \eta_{\mu\nu},
\end{equation}

where the covariant derivative $\mc{D}_a$ is of the form:
\begin{equation}\label{eqn:covariant-deriv-v1}
	\mc{D}_a X^\mu(\tau , \sigma) = \partial_a X^\mu + g \mc{A}_{a}^\mu{}_\nu X^\nu,
\end{equation}
where $\mc{A}_{a}^\mu{}_\nu$ is the gauge field and $g$ is the gauge coupling. A more experienced researcher would object saying that we know that the $X^\mu$ are spacetime co-ordinates and not vectors transforming 

Let us begin by writing down the action of a string propagating in a background geometry. The background geometry is specified not in terms of a metric, but in terms of a connection $A_\mu^{IJ}$ and vielbein fields $E_\mu^I$. Further, we will require that the bulk connection and the bulk vielbein can be pulled back onto a connection $\mc{A}_{a}^{ij}$ and a dyad $e^a_i$, respectively, living on the string worldsheet $\Sigma$. $X^\mu(\tau , \sigma)$ are the space-time coordinates of a point $(\tau , \sigma)$ on the worldsheet. Now if one takes the sigma model viewpoint then the $X$'s can be viewed as $d+1$ scalar fields living on a $1+1$ worldsheet. But these fields are not really scalars. They are components of a $d+1$ dim vector since they transform with the fundamental representation of $\mathrm{SO}(d,1)$ under global Lorentz transformations acting on the ambient space-time, which we take to be flat for the time being.

The key point here is that the Lorentz transformations acting on the $X$'s are \textbf{global}. If the background geometry is taken to be Minkowski then the $X$'s become points on this manifold and $\mathrm{SO}(d,1)$ just takes one point in Minkowski to another point in Minkowski. The overall geometry is invariant under $\mathrm{SO}(d,1)$. Now, consider the case when the background geometry is not flat. Immediately you can see that the $X$'s can no longer be treated as elements of a $\mathrm{SO}(d,1)$ vector under \textbf{global} Lorentz transformations.

\subsection{Gauging Lorentz Transformations}\label{sec:gauging-lt}

Instead of viewing the $X$'s as elements of a Minkowski space, we think of them as elements of the tangent space $T_p (M)$ at a given point $p$ of the background manifold $M$.
When we travel along some path $\mc{C} \subset M$ from point $p$ to point $p'$, the local Lorentz frames at the two points are, in general, related by an element of $g(p,p') \in \mathrm{SO}(d,1)$. This group element arises from taking the holonomy of a $\mf{so}(d,1)$ Lie algebra valued connection living on $M$, along a the path $\mc{C}$.

Let us remind ourselves what we're trying to do here. We want a gauge field, which lives on the background geometry and which encodes the background geometry, to somehow make an appearance in the action of a Bosonic string. There is an obvious connection $\mc{A}_a^{IJ}$ living on the worldsheet which is the pullback from $M$ to $\Sigma$ of the connection $A_{\mu}^{IJ}$.

The end result is that we now have with a gauge field $\mc{A}_a^{IJ}(\tau , \sigma)$ living on the worldsheet $\Sigma$. Here the lower index $a$ is the worldsheet index and $I , J$ labels the generators of the Lie algebra $\mf{so}(d,1)$. Remember this is a gauge field which is now living on the worldsheet. The consequence for the string action is that in order for the action to be invariant under local gauge transformations of $\mc{A}$, we have to replace the ordinary derivatives of $X^\mu$ with the gauge covariant derivative $\partial_a X^\mu \rightarrow \mc{D}_a X^\mu$, which is defined as:

\begin{equation}\label{eqn:covariant-deriv-v2}
	\mc{D}_a X^I(\tau , \sigma) = \partial_a X^I + k \mc{A}_{a J}^{I} X^J,
\end{equation}
where $\mc{A}_a^{I J}$ is a field living on $\Sigma$ which takes values in the Lie algebra $\mf{so}(d,1)$. We should therefore think of the embedding fields as $d+1$ dimensional vectors living in the fundamental representation of $\mathrm{SO}(d,1)$.

This viewpoint on the embedding fields, as elements of the tangent space $T_p(M)$, might appear to be at odds with their prescribed usage as co-ordinate functions on $M$. However, some reflection shows that this is not the case. For instance, one can take the embedding manifold to be a two-dimensional sphere $S^2$. Then the co-ordinates of a point on the string worldsheet $\Sigma$ can be expressed in terms of the polar and azimuthal angles $\theta$ and $\phi$. The action for this system would be:
\begin{equation}\label{eqn:s2-string}
	S_{S^2} = -\frac{T}{2} \int \diff^2 x \sqrt{-g} g^{ab} \partial_a \theta^\mu \partial_b \theta^\nu G_{\mu\nu},
\end{equation}
where $\theta^\mu(\sigma,\tau) \in \{\theta, \phi\}$ are the embedding fields, and $G_{\mu\nu}$ is \emph{a} metric living on $S^2$.

\subsection{Two Expressions for Area}\label{eqn:area-formulae}

There is another explanation for why it makes sense to promote the $X$'s to local fields. This can be found in terms of the expression for the area of a 2-surface. We have a manifold $\mc M$, within which is embedded a two dimensional submanifold $\mc N$. Let $(\tau,\sigma)$ be the location of a point on the worldsheet and $X^\mu(\tau, \sigma)$ be the location of that same point in terms of co-ordinates on $\mc M$. Then we can define the induced metric on $\mc N$, $h_{ab}$ in terms of the embedding fields and their derivatives:

\begin{equation}\label{eqn:ws-metric}
	h_{ab} = \frac{\partial X^\mu}{\partial x^a} \frac{\partial X^\nu}{\partial x^b} g_{\mu\nu},
\end{equation}
where $g_{\mu\nu}$ is the metric in the ambient space $\mc M$. Now, we can calculate the area of a small patch of $\mc P \in \mc N$ by taking the square root of the determinant of the induced metric $\sqrt{\det(h)}$ and integrating over that patch:
\begin{equation}\label{eqn:patch-area-v1}
	A^{(1)}_{\mc P} = \int_{\mc P} d^2 x \, \sqrt{\det(h)}.
\end{equation}
This is, of course, the expression for the Nambu-Goto action - modulo some constants - which we all know and love. In LQG we don't have a notion of embedding fields. Since we are dealing with gauge invariant observables constructed from holonomies of a connection along a 1d-curve and the flux of a tetrad across a 2-surface, the embedding co-ordinates of the curves and surface turn to not be needed. When promoting classical observables into quantum operators acting on lines and surfaces the embedding co-ordinates of the lines and surfaces are not relevant as long as we work with gauge invariant quantities.

How then does one define the area of a surface in LQG? It is defined as the magnitude of the area two form. Recall that given a tetradic basis $e_\mu^I$ for a local Lorentz frame whose metric is given by $g_{\mu\nu}$, we can write the metric in terms of the tetrad in the usual manner:
\begin{equation}\label{eqn:metric-from-tetrad}
	g_{\mu\nu} = e_\mu^I e_\nu^J \eta_{IJ},
\end{equation}
where $\eta_{IJ}$ is the flat Minkowski metric on the ``internal'' flat space. If we consider any two dimensional surface in the local Lorentz frame, then its area can be written in terms of the vielbein as:
\begin{equation}
	A^{(2)}_{\mc P} = \int_{\mc P} \Tr (e_\mu \wedge e_\nu) dx^\mu \wedge dx^\nu,
\end{equation}
the trace is taken over the ``internal'' Lorentz indices.

Now, if we compare the two expressions \eqref{eqn:ws-metric} and \eqref{eqn:metric-from-tetrad} we notice an obvious parallel between the derivatives $\partial X^\mu/\partial x_a$ of the embedding co-ordinates and the vielbein $e_\mu^I$. This similarity becomes more apparent once we write down the expression for the area of a 2-surface by taking the determinant in \eqref{eqn:patch-area-v1} and comparing that to the expression obtained using frame fields.

\textbf{From Embedding Fields:}

We take the bulk metric $g_{\mu\nu}$ to be flat. Let $\{\tau,\sigma\} := \{x^0, x^1\}$ be the spacetime co-ordinates on the worldsheet. Derivatives of the $X$'s in terms of these co-ordinates can be written as:
\begin{equation}
	\dot X := \left\{\frac{\partial X^\mu}{\partial \tau}\right\}; \quad X' := \left\{\frac{\partial X^\mu}{\partial \sigma}\right\}
\end{equation}
 we obtain the following form for the determinant of the worldsheet metric (\eqref{eqn:patch-area-v1}) in terms of the embedding fields:
\begin{align}\label{eqn:area-embedding}
	\det h & = \begin{vmatrix}
		h_{00} && h_{01} \\
		h_{10} && h_{11} 
	\end{vmatrix} = \begin{vmatrix}
		\dot X^2 && \dot X \cdot X' \\
		\dot X \cdot X' && X'^2
	\end{vmatrix} \nonumber \\
	& = \dot X^2 X'^2 - (\dot X \cdot X')^2
\end{align}

\textbf{From Vielbein:}

We can proceed to obtain the area of a 2-surface from the bulk vielbein as follows. Starting with the bulk vielbein $e_\mu^I$, we perform a local Lorentz transformation such that two ``legs'' of the vielbein are tangent to the worldsheet surface. We then project down the vielbein $e_\mu^I$ to a dyad $e_a^i$ living in the tangent space of the worldsheet, in terms of which the worldsheet metric can be written as:
\begin{equation}
	h_{ab} = e_a^i e_b^j \eta_{ij},
\end{equation}
where $\eta_{ij}$ is the flat $(1+1)$ Minkowski metric. The determinant then becomes:
\begin{align}\label{eqn:area-vielbein}
	\det h & = \begin{vmatrix}
		h_{00} && h_{01} \\
		h_{10} && h_{11} 
	\end{vmatrix} = \begin{vmatrix}
		e_x{}^i e_x{}^j & e_x{}^i e_y{}^j \\
		e_y{}^i e_x{}^j & e_y{}^i e_y{}^j
	\end{vmatrix} \eta_{ij} \nonumber \\
	& = \vec e_x{}^2 \vec e_y{}^2 - (\vec e_x \cdot \vec e_y)^2,
\end{align}
where $\vec e_x \cdot \vec e_y = e_x{}^i e_y{}^j \eta_{ij}$. It is clear by comparing the two expressions \eqref{eqn:area-embedding} and \eqref{eqn:area-vielbein} that we have the following correspondence between the vielbein the $X$ derivatives:
\begin{equation}
	\vec e_\tau \equiv \frac{\partial \vec X}{\partial \tau}; \quad \vec e_\sigma \equiv \frac{\partial \vec X}{\partial \sigma}.
\end{equation}
In other words we can identify the vielbein with the derivatives of the embedding fields. Of course, this is not altogether surprising. If we consider any manifold on which we have some set of fields $\{X^I(x^\mu)\}$, where $I$ is valued in the Lie algebra ${\mc g}$ of some group $\mc G$, as a function of some  ``fiducial'' co-ordinates $\{x^\mu\}$, then we can always define a set of ${\mc g}$ valued vector fields: 
$$
	e^I_\mu := \frac{\partial X^I}{\partial x^\mu}.
$$
In the usual formulation of the string action the embedding fields are \emph{not} Lie-algebra valued but are co-ordinate fields on what is generally taken to be a flat background. In order for this correspondence between vielbein and $X$ derivatives to work we have to view the $X$ fields as taking values in a Lie algebra. But these same $X$ fields \emph{also} serve as local co-ordinates for the bulk. In order for the co-ordinates to change from point to point in the bulk geometry, we must therefore have that the connection $\mc A$ in \eqref{eqn:covariant-deriv-v2} can never vanish completely! If $\mc A$ vanishes then that will imply that the fields $X$ don't change from one point to the next. Thus the amplitude $|\mc A|$ can be arbitrarily small but it can never be zero.

\section{Physical Implications}

\section{Discussion}

\begin{acknowledgments}

\end{acknowledgments}

\appendix

\section{Worldsheet Action in Terms of Transverse Velocity}\label{app:transverse-action}

\section{The Bimetric String}\label{app:bimetric-string}

It would be natural to suspect that the action \eqref{eqn:modified-NG-action-v2} might be related to bimetric gravity in the target space of the string. We can see that this is indeed the case as follows.

Consider a string propagating in a target space whose geometry is described by two bulk metrics. There are two ways that a string with two metrics might arise. One possibility is that there are two sets of embedding fields $X^\mu$ and $Y^\mu$. As a result, the induced metrics on the string worldsheet would also be different for both sets of embedding fields. However, in this case, we would not be talking about \emph{one} string but two \emph{different} strings described by the respective embedding functions. One could then proceed along the lines of \eqref{eqn:hassan-bimetric} and introduce a third term for the worldsheet action which describes a coupling between the two sets of embedding fields, of the form:
\begin{equation}\label{eqn:ws-bimetric-interaction}
	S_{int} = - T_{eff} \int \diff^2 x\, \Phi(g,h) \Psi(\mb{X}, \mb{Y}).
\end{equation}
Here $\Phi(g,h)$ is some function of the \emph{induced} metrics $g_{ab}$ and $h_{ab}$ on the two worldsheets and $\Psi(\mb{X}, \mb{Y})$ is some function of the embedding fields. The resulting action would be of the form:
\begin{equation}\label{eqn:ws-bimetric-action-v1}
	S_{BNG} = -T \int \diff^2 x \, \sqrt{-h} - T' \int \diff^2 x \, \sqrt{-g} + S_{int}.
\end{equation}
Here the subscript $BNG$ stands for ``Bimetric Nambu-Goto''.

The second possibility is the following. Rather than considering two different sets of embedding fields, and consequently two different worldsheets, we can work with one set of embeddings $X^\mu(\tau,\sigma)$ and \emph{two different} bulk metrics $G_{\mu\nu}$ and $H_{\mu\nu}$. As a result, we would now get two different induced metrics on the \emph{same} string worldsheet, given by:
\begin{subequations}\label{eqn:ws-induced-metrics}
	\begin{align}
		g_{ab} & = \frac{\partial X^\mu}{\partial x^a} \frac{\partial X^\nu}{\partial x^b} G_{\mu\nu}  \\
		h_{ab} & = \frac{\partial X^\mu}{\partial x^a} \frac{\partial X^\nu}{\partial x^b} H_{\mu\nu}.
	\end{align}
\end{subequations}
We can then write down the combined action for the string worldsheet as:
\begin{equation}\label{eqn:ws-bimetric-action-v2}
	S'_{BNG} = -T \int \diff^2 x\, \sqrt{-h} - T' \int \diff^2 x \, \sqrt{-g} + S'_{int},
\end{equation}
where now $S'_{int}[g,h,\mb{X}]$ is a term describing the interaction between the two metrics living on the \emph{same} worldsheet.

\subsection{Bimetric Polyakov Action}\label{sec:bimetric-polyakov}

The expression \eqref{eqn:ws-bimetric-action-v2} is not ideal because it does not retain any memory of the form of the bulk metrics unless these are included in the interaction term $S'_{int}$. One way to manifestly exhibit the dependence of the respective worldsheet actions on the two bulk metrics is to work with the Polaykov action.

Following the general procedure, given in \cite{Callan1985Strings}, we can couple the worldsheet to a bulk metric as follows:
\begin{equation}\label{eqn:polyakov-curved}
	S_{P} = -\frac{T}{2} \int \diff^2 x \sqrt{-g} g^{ab} \partial_a X^\mu \partial_b X^\nu G_{\mu\nu},
\end{equation}
where $G_{\mu\nu}$ is now the bulk or target space metric. Using this form we can now write down the worldsheet Polyakov bimetric action for two bulk metrics:
\begin{align}\label{eqn:polyakov-flat}
	S_{PB} = & -\frac{T}{2} \int \diff^2 x\sqrt{-g} g^{ab} \partial_a X^\mu \partial_b X^\nu G_{\mu\nu} \nonumber \\
	         & -\frac{T'}{2} \int \diff^2 x \sqrt{-h} h^{ab} \partial_a X^\mu \partial_b X^\nu H_{\mu\nu}.
\end{align}

\subsection{The Bimetric Polyakov Equations} \label{sec:bimetric-polyakov-eoms}

Following the usual procedure such as the one found in \cite{Tong2009Lectures} with the bulk metric fixed, we find two copies of the string energy-momentum tensor being set to zero, which can be used to find an expression for the induced metrics:
\begin{equation} \label{eqn:variation-wrt-g}
	\frac{\delta S_{P B}}{\delta g^{c d}} \overset{!}{=} 0
\end{equation}
\begin{multline} \label{eqn:g-em-tensor}
	T^{(G)} _{c d} := G_{\mu \nu} \bigg( \partial _{c} X^{\mu} \partial _{d} X^{\nu} - \\
	- \frac{1}{2} g_{c d} g^{a b} \partial _{a} X^{\mu} \partial _{b} X^{\nu} \bigg) = 0
\end{multline}

\begin{equation} \label{eqn:g-expression}
	g_{c d} = 2 f^{(G)} \partial _{c} X^{\mu} \partial _{d} X^{\nu} G_{\mu \nu} ,
\end{equation}
where the function $f^{(G)}$ is given by
\begin{equation} \label{eqn:g's-f-function}
	\frac{1}{f^{(G)}} = g^{a b} \partial _{a} X^{\mu} \partial _{b} X^{\nu} G_{\mu \nu},
\end{equation}
and completely analogously for the $H_{\mu \nu}$ metric,
\begin{equation} \label{eqn:variation-wrt-h}
	\frac{\delta S_{P B}}{\delta h^{c d}} \overset{!}{=} 0
\end{equation}
\begin{multline} \label{eqn:h-em-tensor}
	T^{(H)} _{c d} := H_{\mu \nu} \bigg( \partial _{c} X^{\mu} \partial _{d} X^{\mu} - \\
	- \frac{1}{2} h_{c d} h^{a b} \partial _{a} X^{\mu} \partial _{b} X^{\nu} \bigg) = 0
\end{multline}
\begin{equation} \label{eqn:h-expression}
	h_{c d} = 2 f^{(H)} \partial _{c} X^{\mu} \partial _{d} X^{\nu} H_{\mu \nu} ,
\end{equation}
with
\begin{equation} \label{eqn:h's-f-function}
	\frac{1}{f^{(H)}} = h^{a b} \partial _{a} X^{\mu} \partial _{b} X^{\nu} H_{\mu \nu} .
\end{equation}

As for the embbeding fields equation, we get a non-homogeneous wave equation for the coupled metric
\begin{equation} \label{eqn:variation-wrt-X}
	\frac{\delta S_{P B}}{\delta X^{\lambda}} \overset{!}{=} 0
\end{equation}

\bibliographystyle{JHEP3}

\bibliography{lqg-strings.bib}

\end{document}