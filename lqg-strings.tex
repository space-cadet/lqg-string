\RequirePackage{snapshot}
\RequirePackage[final]{graphicx}
\documentclass[twocolumn,final,aps,nofootinbib,nobibnotes]{revtex4-2}

% revtex4 imports the natbib package which conflicts with biblatex. To remedy the resulting name conflicts the following recipe is taken from https://tex.stackexchange.com/a/37077
% Start of 'ignore natbib' hack
%\let\bibhang\relax
%\let\citename\relax
%\let\bibfont\relax
%\let\Citeauthor\relax
%\let\textcite\relax
%\makeatletter
%\DeclareRobustCommand{\MakeUppercase}[1]{{%
%      \def\i{I}\def\j{J}%
%      \def\reserved@a##1##2{\let##1##2\reserved@a}%
%      \expandafter\reserved@a\@uclclist\reserved@b{\reserved@b\@gobble}%
%      \protected@edef\reserved@a{\uppercase{#1}}%
%      \reserved@a
%   }}
%\DeclareRobustCommand{\MakeLowercase}[1]{{%
%      \def\reserved@a##1##2{\let##2##1\reserved@a}%
%      \expandafter\reserved@a\@uclclist\reserved@b{\reserved@b\@gobble}%
%      \protected@edef\reserved@a{\lowercase{#1}}%
%      \reserved@a
%   }}
%\makeatother
%\expandafter\let\csname ver@natbib.sty\endcsname\relax
% End of 'ignore natbib' hack

%\usepackage{graphicx}

\usepackage{amsmath,amssymb,amsfonts,amsthm}
%\usepackage[sort&compress,numbers]{natbib}
%\usepackage[english]{babel}

\usepackage{enumerate}
\usepackage{enumitem}
\usepackage{epsfig}
\usepackage{latexsym}
\usepackage{xcolor}
\usepackage[colorinlistoftodos, shadow, textsize=small, obeyDraft]{todonotes}

\usepackage[colorlinks=true,citecolor=blue,linkcolor=blue,final]{hyperref}
%\usepackage{hyperref}

%\usepackage{hypernat}

\usepackage{comment}

% Biblatex setup
% The default is the 'numerical' style.
%\usepackage[backend=biber,url=true,eprint=true,doi=true,sorting=none,backref=true,style=phys]{biblatex}
% We use the following bibliography databases:
%\usepackage{bibtex}
%\addbibresource{../bib_library.bib}
% End biblatex


%%% BEGIN Custom commands %%%
\newcommand{\bite}{\begin{itemize}}
	\newcommand{\eat}{\end{itemize}}
\newcommand{\beq}{\begin{equation}}
	\newcommand{\eeq}{\end{equation}}
\newcommand{\rarrow}{\rightarrow}
\newcommand{\beqa}{\begin{align}}
	\newcommand{\eeqa}{\end{align}}
\newcommand{\barr}{\begin{array}}
	\newcommand{\earr}{\end{array}}
\newcommand{\del}{\partial}
\newcommand{\de}{\mathrm{d}}
%\newcommand{\mu\nu}{{\mu\nu}}
\renewcommand{\th}{\mathrm{th}}
\newcommand{\com}[1]{\begin{itemize}\color{RED}{{#1}}\end{itemize}}
\newcommand{\C}{\mathbb{C}}
\newcommand{\R}{\mathbb{R}}
\newcommand{\btw}[1]{\color{PURPLE}{{#1}}\color{BLACK}}
\newcommand{\cut}[1]{\color{RED}{{#1}}\color{BLACK}}

%text
\newcommand{\ie}{\textit{i.e.}~}
\newcommand{\eg}{\textit{e.g.}~}
\newcommand{\wrt}{\textit{w.r.t.}~}
\newcommand{\etc}{\textit{etc.}~}


\newcommand{\M}{\mathcal{M}}
\newcommand{\N}{\mathcal{N}}
% \newcommand{\H}{\mathcal{H}}

\newcommand{\bz}{\mathbf{z}}

\newcommand{\mb}[1]{\mathbf{#1}}
\newcommand{\mc}[1]{\mathcal{#1}}
\newcommand{\mbb}[1]{\mathbb{#1}}
\newcommand{\mf}[1]{\mathfrak{#1}}

\newcommand{\unit}[1]{\mathbf{\hat{#1}}}

%% Package bbold for bold identity symbol
\usepackage{bbold}

\newcommand{\id}{\mathbb{1}}

\newcommand{\utilde}[1]{\underaccent{\tilde}{#1}}

\newcommand{\vect}[1]{\boldsymbol{#1}}
\newcommand{\bvec}[1]{\boldsymbol{\vec #1}}
\newcommand{\expect}[1]{\langle #1\rangle}
\newcommand{\innerp}[2]{\langle #1 \vert #2 \rangle}
\newcommand{\expectop}[3]{\langle #1 \vert #2 \vert #3 \rangle}
\newcommand{\bra}[1]{\langle #1 \vert}
\newcommand{\ket}[1]{\vert #1 \rangle}
\newcommand{\supersc}[1]{$^{\textrm{#1}}$}
\newcommand{\subsc}[1]{$_{\textrm{#1}}$}
\newcommand{\sltwoc}{\mathfrak{sl}(2,\mathbb{C})}

\newcommand{\norm}[1]{\lVert #1 \rVert}

\newcommand{\rket}[1]{\vert #1 ]}
\newcommand{\rbra}[1]{[ #1 \vert}

\newcommand{\rinnerp}[2]{[ #1 \vert #2 ]}

\newcommand{\bket}[1]{\vert #1 )}
\newcommand{\bbra}[1]{( #1 \vert}

\newcommand{\binnerp}[2]{( #1 \vert #2 )}

\newcommand{\innerpA}[2]{\langle #1 \vert #2 ]}
\newcommand{\innerpB}[2]{[ #1 \vert #2 \rangle}

\newcommand{\onehalf}{\frac{1}{2}}

\newcommand{\Tr}{\mathrm{Tr}}

\newcommand{\Cyl}{\mathrm{Cyl}}
%%% END Custom commands %%%

\graphicspath{{figures/}}

%opening
\begin{document}

\preprint{This line only printed with preprint option}

\title{Quantizing the Bosonic String on a Loop Quantum Gravity Background}

\author{Deepak Vaid}
\email{dvaid79@gmail.com}
\affiliation{National Institute of Technology, Karnataka, India}
\author{Luigi Teixeira de Sousa}
\email{luigi.tiraque@gmail.com}
\affiliation{Universidade Federal de São Carlos, Brasil}

\date{\today}

\begin{abstract}
	We write down an action for a bosonic string propagating in a bulk background whose geometry is specified in terms of connection and tetrad (or ``vierbein'') variables
\end{abstract}

\maketitle

\tableofcontents

\listoftodos

\section{Introduction}

``Quantum gravity'' refers to the broad enterprise dedicated to finding a complete, consistent theory in which quantum mechanics coexists peacefully with general relativity. There are two major approaches to quantum gravity which are widely recognized for their success in describing various aspects of physics at the Planck scale. The first of these is String Theory\footnote{In this work we will capitalize the words ``string theory'' since the second approach in question, loop quantum gravity is usually abbreviated as LQG.}, the quantum theory of a one dimensional object which is said to contain within it a complete description of all particles, forces and their interactions, including gravity. However, where precisely in the vast space of effective field theories which can possibly arise as in the low energy limit of a conjectured ``M-Theory'', our Universe with its collection of particles, forces and coupling constants, exists, is still a matter of vigorous debate.

One of the primary criticisms of String Theory is its apparent lack of ``background independence''. This is more general than the principle of \emph{general covariance} which states that physics should be independent of the choice of co-ordinates. Background independence is the statement that any \emph{quantum} theory of gravity - which, presumably, should provide a description not only of classical spacetimes, but also of fluctuating, semiclassical geometries and those in the deep quantum regime which do not have any sensible description in terms of any Riemannian geometry - should provide a description of physics which is independent, not only of the choice of co-ordinates, but also of the choice of background manifold on which those co-ordinates live. On the face of it String Theory, at least in its conventional form, does not satisfy this principle.

It is sometimes argued that String Theory is, in fact, background independent since the equations of motion which arise when the (Bosnic) string is coupled to the metric of the ``worldvolume'' (the background manifold in which the string is propagating) turn out to include Einstein's field equations for gravity. In  other words, Weyl invariance - one of the fundamental symmetries of the string worldsheet, \emph{requires} \todo{insert refs} that the background geometry must satisfy Einstein's equations. While technically correct, it begs the question, if general relativity ``emerges'' from String Theory then why is it not possible to quantize the string on an arbitrary background to begin with? As anybody who has studied the basics of String Theory knows, the standard quantization procedure \emph{assumes} that the worldvolume has a flat metric. It is, thus far, not technically feasible to do away with this assumption.

There is another sense in which String Theory falls short of being a theory of quantum gravity. This, however, rests upon a philosophical perspective, about the nature of quantum gravity, which one may or may not choose to subscribe to. This is the viewpoint that in any quantum theory of gravity, geometry itself must be quantized. Again, one might say that gravitons - the quantum perturbations of the metric - are ``quanta of geometry''. However, gravitons \emph{live} upon some background manifold. Moreover, as mentioned, they are \emph{perturbative} by construction. Gravitons cannot provide a quantum description of geometry, anymore than quantizing water waves can provide a picture of the molecular structure of water\footnote{This viewpoint has been most clearly articulated by Jacobson \todo{insert ref}}. These considerations bring us to the second major\footnote{the characterization of LQG as a ``major'' approach is the our choice and not necessarily reflective of the consensus in the broader quantum gravity community.} approach towards a quantum theory of gravity, known as Loop Quantum Gravity or LQG for short.

LQG arose almost by accident, something it has in common with String Theory. Traditionally, there are two ways in which a classical theory can be quantized. These are the Hamiltonian and Lagrangian approaches. The Lagrangian or path integral approach follows the prescription first suggested by Dirac and then made concrete by Feynman. There one views the classical action associated to a given evolution as corresponding to a \emph{phase angle} which determines the complex weight of the associated evolution.

Prior to the advent of String Theory a great deal of effort was put into attempting to quantize gravity in exactly the same way as theories such as electrodynamics and the weak interaction had been successfuly quantized previously, leading to the triumphant prediction of the existence of the positron by Dirac

\section{Modified Polyakov Action}

\section{Equations of Motion}

\section{Physical Implications}

\section{Discussion}

\begin{acknowledgments}

\end{acknowledgments}

\appendix

\section{}

\bibliographystyle{JHEP3}

\bibliography{lqg-strings.bib}


\end{document}