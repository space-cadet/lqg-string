\RequirePackage{snapshot}
\RequirePackage[final]{graphicx}
\documentclass[twocolumn,final,aps,nofootinbib,nobibnotes]{revtex4-2}

% revtex4 imports the natbib package which conflicts with biblatex. To remedy the resulting name conflicts the following recipe is taken from https://tex.stackexchange.com/a/37077
% Start of 'ignore natbib' hack
%\let\bibhang\relax
%\let\citename\relax
%\let\bibfont\relax
%\let\Citeauthor\relax
%\let\textcite\relax
%\makeatletter
%\DeclareRobustCommand{\MakeUppercase}[1]{{%
%      \def\i{I}\def\j{J}%
%      \def\reserved@a##1##2{\let##1##2\reserved@a}%
%      \expandafter\reserved@a\@uclclist\reserved@b{\reserved@b\@gobble}%
%      \protected@edef\reserved@a{\uppercase{#1}}%
%      \reserved@a
%   }}
%\DeclareRobustCommand{\MakeLowercase}[1]{{%
%      \def\reserved@a##1##2{\let##2##1\reserved@a}%
%      \expandafter\reserved@a\@uclclist\reserved@b{\reserved@b\@gobble}%
%      \protected@edef\reserved@a{\lowercase{#1}}%
%      \reserved@a
%   }}
%\makeatother
%\expandafter\let\csname ver@natbib.sty\endcsname\relax
% End of 'ignore natbib' hack

%\usepackage{graphicx}

\usepackage{amsmath,amssymb,amsfonts,amsthm}
%\usepackage[sort&compress,numbers]{natbib}
%\usepackage[english]{babel}

\usepackage{enumerate}
\usepackage{enumitem}
\usepackage{epsfig}
\usepackage{latexsym}
\usepackage{xcolor}
\usepackage[colorinlistoftodos, shadow, textsize=small, obeyDraft]{todonotes}

\usepackage[colorlinks=true,citecolor=blue,linkcolor=blue,final]{hyperref}
%\usepackage{hyperref}

%\usepackage{hypernat}

\usepackage{comment}

% Biblatex setup
% The default is the 'numerical' style.
%\usepackage[backend=biber,url=true,eprint=true,doi=true,sorting=none,backref=true,style=phys]{biblatex}
% We use the following bibliography databases:
%\usepackage{bibtex}
%\addbibresource{../bib_library.bib}
% End biblatex


%%% BEGIN Custom commands %%%
\newcommand{\bite}{\begin{itemize}}
	\newcommand{\eat}{\end{itemize}}
\newcommand{\beq}{\begin{equation}}
	\newcommand{\eeq}{\end{equation}}
\newcommand{\rarrow}{\rightarrow}
\newcommand{\beqa}{\begin{align}}
	\newcommand{\eeqa}{\end{align}}
\newcommand{\barr}{\begin{array}}
	\newcommand{\earr}{\end{array}}
\newcommand{\del}{\partial}
\newcommand{\de}{\mathrm{d}}
%\newcommand{\mu\nu}{{\mu\nu}}
\renewcommand{\th}{\mathrm{th}}
\newcommand{\com}[1]{\begin{itemize}\color{RED}{{#1}}\end{itemize}}
\newcommand{\C}{\mathbb{C}}
\newcommand{\R}{\mathbb{R}}
\newcommand{\btw}[1]{\color{PURPLE}{{#1}}\color{BLACK}}
\newcommand{\cut}[1]{\color{RED}{{#1}}\color{BLACK}}

%text
\newcommand{\ie}{\textit{i.e.}~}
\newcommand{\eg}{\textit{e.g.}~}
\newcommand{\wrt}{\textit{w.r.t.}~}
\newcommand{\etc}{\textit{etc.}~}


\newcommand{\M}{\mathcal{M}}
\newcommand{\N}{\mathcal{N}}
% \newcommand{\H}{\mathcal{H}}

\newcommand{\bz}{\mathbf{z}}

\newcommand{\mb}[1]{\mathbf{#1}}
\newcommand{\mc}[1]{\mathcal{#1}}
\newcommand{\mbb}[1]{\mathbb{#1}}
\newcommand{\mf}[1]{\mathfrak{#1}}

\newcommand{\unit}[1]{\mathbf{\hat{#1}}}

%% Package bbold for bold identity symbol
\usepackage{bbold}

\newcommand{\id}{\mathbb{1}}

\newcommand{\utilde}[1]{\underaccent{\tilde}{#1}}

\newcommand{\vect}[1]{\boldsymbol{#1}}
\newcommand{\bvec}[1]{\boldsymbol{\vec #1}}
\newcommand{\expect}[1]{\langle #1\rangle}
\newcommand{\innerp}[2]{\langle #1 \vert #2 \rangle}
\newcommand{\expectop}[3]{\langle #1 \vert #2 \vert #3 \rangle}
\newcommand{\bra}[1]{\langle #1 \vert}
\newcommand{\ket}[1]{\vert #1 \rangle}
\newcommand{\supersc}[1]{$^{\textrm{#1}}$}
\newcommand{\subsc}[1]{$_{\textrm{#1}}$}
\newcommand{\sltwoc}{\mathfrak{sl}(2,\mathbb{C})}

\newcommand{\norm}[1]{\lVert #1 \rVert}

\newcommand{\rket}[1]{\vert #1 ]}
\newcommand{\rbra}[1]{[ #1 \vert}

\newcommand{\rinnerp}[2]{[ #1 \vert #2 ]}

\newcommand{\bket}[1]{\vert #1 )}
\newcommand{\bbra}[1]{( #1 \vert}

\newcommand{\binnerp}[2]{( #1 \vert #2 )}

\newcommand{\innerpA}[2]{\langle #1 \vert #2 ]}
\newcommand{\innerpB}[2]{[ #1 \vert #2 \rangle}

\newcommand{\onehalf}{\frac{1}{2}}

\newcommand{\Tr}{\mathrm{Tr}}

\newcommand{\Cyl}{\mathrm{Cyl}}
%%% END Custom commands %%%

\graphicspath{{figures/}}

%opening
\begin{document}

\preprint{This line only printed with preprint option}

\title{Quantizing the Bosonic String on a Loop Quantum Gravity Background}

\author{Deepak Vaid}
\email{dvaid79@gmail.com}
\affiliation{National Institute of Technology, Karnataka, India}
\author{Luigi Teixeira de Sousa}
\email{luigi.tiraque@gmail.com}
\affiliation{Universidade Federal de São Carlos, Brasil}

\date{\today}

\begin{abstract}
	We write down an action for a bosonic string propagating in a bulk background whose geometry is specified in terms of connection and tetrad (or ``vierbein'') variables
\end{abstract}

\maketitle

\tableofcontents

\listoftodos

\section{Introduction}

``Quantum gravity'' refers to the broad enterprise dedicated to finding a complete, consistent theory in which quantum mechanics coexists peacefully with general relativity. There are two major approaches to quantum gravity which are widely recognized for their success in describing various aspects of physics at the Planck scale. The first of these is String Theory\footnote{In this work we will capitalize the words ``string theory'' since the second approach in question, loop quantum gravity is usually abbreviated as LQG.}, the quantum theory of a one dimensional object which is said to contain within it a complete description of all particles, forces and their interactions, including gravity. However, where precisely in the vast space of effective field theories which can possibly arise as in the low energy limit of a conjectured ``M-Theory'', our Universe with its collection of particles, forces and coupling constants, exists, is still a matter of vigorous debate.

One of the primary criticisms of String Theory is its apparent lack of ``background independence''. This is more general than the principle of \emph{general covariance} which states that physics should be independent of the choice of co-ordinates. Background independence is the statement that any \emph{quantum} theory of gravity - which, presumably, should provide a description not only of classical spacetimes, but also of fluctuating, semiclassical geometries and those in the deep quantum regime which do not have any sensible description in terms of any Riemannian geometry - should provide a description of physics which is independent, not only of the choice of co-ordinates, but also of the choice of background manifold on which those co-ordinates live. On the face of it String Theory, at least in its conventional form, does not satisfy this principle.

It is sometimes argued that String Theory is, in fact, background independent since the equations of motion which arise when the (Bosnic) string is coupled to the metric of the ``worldvolume'' (the background manifold in which the string is propagating) turn out to include Einstein's field equations for gravity. In  other words, Weyl invariance - one of the fundamental symmetries of the string worldsheet, \emph{requires} \todo{insert refs} that the background geometry must satisfy Einstein's equations. While technically correct, it begs the question, if general relativity ``emerges'' from String Theory then why is it not possible to quantize the string on an arbitrary background to begin with? As anybody who has studied the basics of String Theory knows, the standard quantization procedure \emph{assumes} that the worldvolume has a flat metric. It is, thus far, not technically feasible to do away with this assumption.

There is another sense in which String Theory falls short of being a theory of quantum gravity. This, however, rests upon a philosophical perspective, about the nature of quantum gravity, which one may or may not choose to subscribe to. This is the viewpoint that in any quantum theory of gravity, geometry itself must be quantized. Again, one might say that gravitons - the quantum perturbations of the metric - are ``quanta of geometry''. However, gravitons \emph{live} upon some background manifold. Moreover, as mentioned, they are \emph{perturbative} by construction. Gravitons cannot provide a quantum description of geometry, anymore than quantizing water waves can provide a picture of the molecular structure of water\footnote{This viewpoint has been most clearly articulated by Jacobson \todo{insert ref}}. These considerations bring us to the second major\footnote{the characterization of LQG as a ``major'' approach is the our choice and not necessarily reflective of the consensus in the broader quantum gravity community.} approach towards a quantum theory of gravity, known as Loop Quantum Gravity or LQG for short.

LQG arose almost by accident, something it has in common with String Theory. Traditionally, there are two ways in which a classical theory can be quantized. These are the Hamiltonian and Lagrangian approaches. The Lagrangian or path integral approach follows the prescription first suggested by Dirac and then made concrete by Feynman. There one views the classical action associated to a given evolution as corresponding to a \emph{phase angle} which determines the complex weight of the associated evolution. This approach respects spacetime covariance since here the central object is the action which is invariant under spacetime transformations by construction. The Hamiltonian approach, on the other hand, involves making a choice of a spacelike surface $\Sigma_t$ and a corresponding timelike vector $t^\mu$, normal to $\Sigma_t$ (here $t$ is a continuous parameter which labels the family of surfaces). This approach, therefore, necessarily does not respect spacetime covariance since it involves making a specific choice of the foliation of the background into spacelike surfaces.

In the decades prior to the advent of String Theory a great deal of effort was put into attempting to quantize gravity via both the Lagrangian and Hamiltonian approaches. The Hamiltonian approach involves, as stated earlier, a choice of the foliation of the background geometry into a collection of spacelike surfaces $\Sigma_t$. Central to this approach was the ADM (Arnowitt-Deser-Misner) formalism which allows one to construct a Hamiltonian $H_{GR}$ for general relativity starting from the Einstein-Hilbert action $S_{EH}$. The resulting phase is co-ordinatized by a configuration variable which is the 3-metric $h_{ab}$ of the ``leaves'' $\Sigma_t$ of the foliation and a momentum variable which is a function of a the extrinsic curvature $k_{ab}$ of the leaves\footnote{Our notational convention is the following. Lowercase Greek letters from the middle of the alphabet $\mu, \nu, \ldots$ are spacetime indices which run from $(0 \ldots 3)$, whereas lowercase Roman letters $a,b,\ldots$ are spatial indices for quantities which live solely on the surfaces $\Sigma_t$ and take values in $(1,2,3)$}. The gravity Hamiltonian turns out to be a sum of two constraints known as the diffeomorphism (or ``momentum'') constraint $H_{diff}$ and the Hamiltonian constraint $H_{ham}$:

\begin{equation}
	H_{GR} = H_{diff} + H_{ham},
\end{equation}

and these are, in turn, functionals on the phase space of general relativity written in terms of $h_{ab}$ and $k_{ab}$. The idea then is that physical states of the theory $\ket{\Psi_{phys}} $, which can be written as functionals $\Psi[h_{ab}]$ of the 3-metric $h_{ab}$ , must be annihilated by these constraints:

\begin{equation}
	H_{GR} \ket{\Psi_{phys}} \equiv 0
\end{equation}

While this procedure is straightforward in principle, in practice is was impossible to implement in the quantum theory due to highly complicated non-polynomial dependence of the diffeomorphism and Hamiltonian constraints on the configuration and momentum variables. This unfortunate state of affairs persisted until the 1980s when Abhay Ashtekar recast general relativity as a theory of a Minkowski tetrad $E^I_\mu$ and a self-dual $\sltwoc$ connection $A_\mu^{IJ}$. Ashtekar realised that the constraints of general relativity, when expressed in terms of these ``new variables'', simplified drastically and became polynomial functions of the tetrad and connection variables. It was quickly realised by various researchers that the resulting theory could be quantized using the same methods used to quantize Yang-Mills gauge theory and that the diffeomorphism constraint could be solved exactly in terms of so-called \emph{spin-network} states.

Following this, work by Rovelli and Smolin \todo{insert ref} and by Ashtekar, Rovelli and Smolin demonstrated the most remarkable feature of this theory, which came to be known as ``Loop Quantum Gravity'', was that one could construct quantum operators for geometric quantities such as areas of two dimensional surfaces and volumes of three dimensional regions. Moreover, these operators could be diagonlized exactly in the spin-network basis and a lower bound on the smallest possible quantum of area and quantum of volume could be derived. This was the first time that physicists had discovered the ``atoms of space'' or the ``quanta of geometry'' in the true sense of the expression.

\section{Strings in Background Fields and Emergent Gravity}

Let us begin by recalling the Polyakov action for the bosonic string.

\begin{equation}\label{eqn:polyakov-flat}
	S_{P} = -\frac{T}{2} \int d\tau d\sigma \sqrt{-g} g^{ab} \partial_a X^\mu \partial_b X^\nu \eta_{\mu\nu}.
\end{equation}
Here $\mu, \nu \in \{0,1,\ldots,D-1\}$ are co-ordinates of the $D$ dimensional worldvolume (the geometry in which the string propagates), $a,b \in \{0,1\}$ are co-ordinates on the string worldsheet, $X^\mu$ are the embedding co-ordinates which specify the location of a point on the string worldsheet in the bulk worldvolume, $\eta_{\mu\nu}$ is the flat Minkowski metric on the worldvolume, $g_{ab}$ is the metric on the string worldsheet, $T$ is the string tension and $\tau, \sigma $ are the co-ordinates on the string worldsheet.

Now, one proceeds in the usual way by determining the symmetries of the action \eqref{eqn:polyakov-flat}, varying the action to find the equations of motion, fixing the gauge using the Weyl freedom of the worldsheet and then solving the classical equations of motion. Imposition of (bosonic) commutation relations on the operator versions of the embedding fields $\hat X^\mu$ then leads us to description of the quantum state of the bosonic string in terms of an infinite ladder of harmonic oscillators which obey the Virasoro algebra.

The obvious drawback of this approach is that the background metric is non-dynamical and is fixed to be the flat Minkowski metric $\eta_{\mu\nu}$. Clearly, one would like to be able to understand the physics of a string propagating on an arbitrary curved background. It wouldn't make much sense to refer to string theory as a theory of ``quantum gravity'' if strings can only be described on flat backgrounds. The way this is accomplished is by treating the metric of the worldvolume as a ``background field'' $G_{\mu\nu}$, in terms of which the Polyakov action becomes:

\begin{equation}\label{eqn:polyakov-gravity}
	S'_{P} = -\frac{T}{2} \int d\tau d\sigma \sqrt{-g} g^{ab} \partial_a X^\mu \partial_b X^\nu G_{\mu\nu}(X),
\end{equation}

where the bulk metric is now a function of the bulk co-ordinates $G_{\mu\nu}(X)$. Now this metric is still non-dynamical because the action \eqref{eqn:polyakov-gravity} does not contain any terms with time derivatives of $G_{\mu\nu}$. However, if we view the action purely as a two-dimensional theory of $D$ scalar fields, then the bulk metric can be viewed as a collection of \emph{coupling constants}, rather than a dynamical entity which exists independent of the string. One can now proceed in the usual manner for any field theory and calculate the beta function of the coupling constants of the theory as a function of the energy scale.

Then, as shown long ago by Friedan \cite{Friedan1985Nonlinear} (see also \cite{Callan1989Sigma}, \cite{Callan1985Strings} or \cite[Sec 3.7]{Polchinski1998aString}, \cite[Sec 7.2]{Tong2010Lectures} for a more pedagogical explanation), that the garviton beta function is proportional to $R_{\mu\nu}$, the Ricci curvature of the bulk geometry. The requirement of Weyl invariance of the string worldsheet implies that this beta function should vanish:
\begin{equation}\label{eqn:ricci-flat}
	\beta(G_{\mu\nu}) \propto R_{\mu\nu} = 0.
\end{equation}
Therefore we find that conformal invariance of the string worldsheet \emph{implies} that the background geometry in which the string propagates satisfies the \emph{vacuum} Einstein equations. It is this result which is often cited as evidence for the claim that string theory is a \emph{background independent} theory of quantum gravity.

\section{Equations of Motion}

\section{Physical Implications}

\section{Discussion}

\begin{acknowledgments}

\end{acknowledgments}

\appendix

\section{}

\bibliographystyle{JHEP3}

\bibliography{lqg-strings.bib}


\end{document}