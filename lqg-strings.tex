\RequirePackage{snapshot}
\RequirePackage[final]{graphicx}
\documentclass[twocolumn,draft,aps,nofootinbib,nobibnotes]{revtex4-2}

% revtex4 imports the natbib package which conflicts with biblatex. To remedy the resulting name conflicts the following recipe is taken from https://tex.stackexchange.com/a/37077
% Start of 'ignore natbib' hack
%\let\bibhang\relax
%\let\citename\relax
%\let\bibfont\relax
%\let\Citeauthor\relax
%\let\textcite\relax
%\makeatletter
%\DeclareRobustCommand{\MakeUppercase}[1]{{%
%      \def\i{I}\def\j{J}%
%      \def\reserved@a##1##2{\let##1##2\reserved@a}%
%      \expandafter\reserved@a\@uclclist\reserved@b{\reserved@b\@gobble}%
%      \protected@edef\reserved@a{\uppercase{#1}}%
%      \reserved@a
%   }}
%\DeclareRobustCommand{\MakeLowercase}[1]{{%
%      \def\reserved@a##1##2{\let##2##1\reserved@a}%
%      \expandafter\reserved@a\@uclclist\reserved@b{\reserved@b\@gobble}%
%      \protected@edef\reserved@a{\lowercase{#1}}%
%      \reserved@a
%   }}
%\makeatother
%\expandafter\let\csname ver@natbib.sty\endcsname\relax
% End of 'ignore natbib' hack

%\usepackage{graphicx}

\usepackage{amsmath,amssymb,amsfonts,amsthm}
%\usepackage[sort&compress,numbers]{natbib}
%\usepackage[english]{babel}

\usepackage{enumerate}
\usepackage{enumitem}
\usepackage{epsfig}
\usepackage{latexsym}
\usepackage{xcolor}
\usepackage{mathrsfs}
\usepackage[colorinlistoftodos, shadow, textsize=small, obeyDraft,loadshadowlibrary]{todonotes}

\setlength{\emergencystretch}{2em}

\usepackage[colorlinks=true,citecolor=blue,linkcolor=blue,final]{hyperref}
%\usepackage{hyperref}

%\usepackage{hypernat}

\usepackage{comment}

% Biblatex setup
% The default is the 'numerical' style.
%\usepackage[backend=biber,url=true,eprint=true,doi=true,sorting=none,backref=true,style=phys]{biblatex}
% We use the following bibliography databases:
%\usepackage{bibtex}
%\addbibresource{../bib_library.bib}
% End biblatex


%%% BEGIN Custom commands %%%
\newcommand{\bite}{\begin{itemize}}
	\newcommand{\eat}{\end{itemize}}
\newcommand{\beq}{\begin{equation}}
	\newcommand{\eeq}{\end{equation}}
\newcommand{\rarrow}{\rightarrow}
\newcommand{\beqa}{\begin{align}}
	\newcommand{\eeqa}{\end{align}}
\newcommand{\barr}{\begin{array}}
	\newcommand{\earr}{\end{array}}
\newcommand{\del}{\partial}
\newcommand{\de}{\mathrm{d}}
%\newcommand{\mu\nu}{{\mu\nu}}
\renewcommand{\th}{\mathrm{th}}
\newcommand{\com}[1]{\begin{itemize}\color{RED}{{#1}}\end{itemize}}
\newcommand{\C}{\mathbb{C}}
\newcommand{\R}{\mathbb{R}}
\newcommand{\btw}[1]{\color{PURPLE}{{#1}}\color{BLACK}}
\newcommand{\cut}[1]{\color{RED}{{#1}}\color{BLACK}}

%text
\newcommand{\ie}{\textit{i.e.}~}
\newcommand{\eg}{\textit{e.g.}~}
\newcommand{\wrt}{\textit{w.r.t.}~}
\newcommand{\etc}{\textit{etc.}~}


\newcommand{\M}{\mathcal{M}}
\newcommand{\N}{\mathcal{N}}
% \newcommand{\H}{\mathcal{H}}

\newcommand{\bz}{\mathbf{z}}

\newcommand{\mb}[1]{\mathbf{#1}}
\newcommand{\mc}[1]{\mathcal{#1}}
\newcommand{\mbb}[1]{\mathbb{#1}}
\newcommand{\mf}[1]{\mathfrak{#1}}

\newcommand{\unit}[1]{\mathbf{\hat{#1}}}

%% Package bbold for bold identity symbol
\usepackage{bbold}

\newcommand{\id}{\mathbb{1}}

\newcommand{\utilde}[1]{\underaccent{\tilde}{#1}}

\newcommand{\vect}[1]{\boldsymbol{#1}}
\newcommand{\bvec}[1]{\boldsymbol{\vec #1}}
\newcommand{\expect}[1]{\langle #1\rangle}
\newcommand{\innerp}[2]{\langle #1 \vert #2 \rangle}
\newcommand{\expectop}[3]{\langle #1 \vert #2 \vert #3 \rangle}
\newcommand{\bra}[1]{\langle #1 \vert}
\newcommand{\ket}[1]{\vert #1 \rangle}
\newcommand{\supersc}[1]{$^{\textrm{#1}}$}
\newcommand{\subsc}[1]{$_{\textrm{#1}}$}
\newcommand{\sltwoc}{\mathfrak{sl}(2,\mathbb{C})}

\newcommand{\norm}[1]{\lVert #1 \rVert}

\newcommand{\rket}[1]{\vert #1 ]}
\newcommand{\rbra}[1]{[ #1 \vert}

\newcommand{\rinnerp}[2]{[ #1 \vert #2 ]}

\newcommand{\bket}[1]{\vert #1 )}
\newcommand{\bbra}[1]{( #1 \vert}

\newcommand{\binnerp}[2]{( #1 \vert #2 )}

\newcommand{\innerpA}[2]{\langle #1 \vert #2 ]}
\newcommand{\innerpB}[2]{[ #1 \vert #2 \rangle}

\newcommand{\onehalf}{\frac{1}{2}}

\newcommand{\Tr}{\mathrm{Tr}}

\newcommand{\Cyl}{\mathrm{Cyl}}

%Luigi's custom commands
\newcommand{\R}{\mathbb{R}}
\newcommand{\diff}{\mathrm{d}}
\newcommand{\inp}{\mathrm{i}}
\newcommand{\e}{\mathrm{e}}
\newcommand{\Lagr}{\mathscr{L}}
\newcommand{\PS}{\mathcal{P}}
%%% END Custom commands %%%

\graphicspath{{figures/}}

%opening
\begin{document}

\preprint{This line only printed with preprint option}

\title{Quantizing the Bosonic String on a Loop Quantum Gravity Background}

\author{Deepak Vaid}
\email{dvaid79@gmail.com}
\affiliation{National Institute of Technology, Karnataka, India}
\author{Luigi Teixeira de Sousa}
\email{luigi.tiraque@gmail.com}
\affiliation{Universidade Federal de São Carlos, Brasil}

\date{\today}

\begin{abstract}
	We write down an action for a bosonic string propagating in a bulk background whose geometry is specified in terms of connection and tetrad (or ``vierbein'') variables
\end{abstract}

\maketitle

\tableofcontents

\listoftodos

\section{Introduction}\label{sec:intro}

``Quantum gravity'' refers to the broad enterprise dedicated to finding a complete, consistent theory in which quantum mechanics coexists peacefully with general relativity. There are two major approaches to quantum gravity which are widely recognized for their success in describing various aspects of physics at the Planck scale. The first of these is string theory, the quantum theory of a one dimensional object which is said to contain within it a complete description of all particles, forces and their interactions, including gravity. However, where precisely in the vast space of effective field theories which can possibly arise as in the low energy limit of a conjectured ``M-Theory'', our Universe with its collection of particles, forces and coupling constants, exists, is still a matter of vigorous debate.

One of the primary criticisms of string theory is its apparent lack of ``background independence''. This is more general than the principle of \emph{general covariance} which states that physics should be independent of the choice of co-ordinates. Background independence is the statement that any \emph{quantum} theory of gravity - which, presumably, should provide a description not only of classical spacetimes, but also of fluctuating, semiclassical geometries and those in the deep quantum regime which do not have any sensible description in terms of any Riemannian geometry - should provide a description of physics which is independent, not only of the choice of co-ordinates, but also of the choice of background manifold on which those co-ordinates live. On the face of it string theory, at least in its conventional form, does not satisfy this principle.

It is sometimes argued that string theory is, in fact, background independent since the equations of motion which arise when the (Bosnic) string is coupled to the metric of the ``worldvolume'' (the background manifold in which the string is propagating) turn out to include Einstein's field equations for gravity. In  other words, Weyl invariance - one of the fundamental symmetries of the string worldsheet, \emph{requires} \todo{insert refs} that the background geometry must satisfy Einstein's equations. While technically correct, it begs the question, if general relativity ``emerges'' from string theory then why is it not possible to quantize the string on an arbitrary background to begin with? As anybody who has studied the basics of string theory knows, the standard quantization procedure \emph{assumes} that the worldvolume has a flat metric. It is, thus far, not technically feasible to do away with this assumption.

There is another sense in which string theory falls short of being a theory of quantum gravity. This, however, rests upon a philosophical perspective, about the nature of quantum gravity, which one may or may not choose to subscribe to. This is the viewpoint that in any quantum theory of gravity, geometry itself must be quantized. Again, one might say that gravitons - the quantum perturbations of the metric - are ``quanta of geometry''. However, gravitons \emph{live} upon some background manifold. Moreover, as mentioned, they are \emph{perturbative} by construction. Gravitons cannot provide a quantum description of geometry, anymore than quantizing water waves can provide a picture of the molecular structure of water\footnote{This viewpoint has been most clearly articulated by Jacobson \cite{Jacobson1995Thermodynamics}}. These considerations bring us to the second major\footnote{the characterization of LQG as a ``major'' approach is the our choice and not necessarily reflective of the consensus in the broader quantum gravity community.} approach towards a quantum theory of gravity, known as Loop Quantum Gravity or LQG for short.

LQG arose almost by accident, something it has in common with string theory. Traditionally, there are two ways in which a classical theory can be quantized. These are the Hamiltonian and Lagrangian approaches. The Lagrangian or path integral approach follows the prescription first suggested by Dirac and then made concrete by Feynman. There one views the classical action associated to a given evolution as corresponding to a \emph{phase angle} which determines the complex weight of the associated evolution. This approach respects spacetime covariance since here the central object is the action which is invariant under spacetime transformations by construction. The Hamiltonian approach, on the other hand, involves making a choice of a spacelike surface $\Sigma_t$ and a corresponding timelike vector $t^\mu$, normal to $\Sigma_t$ (here $t$ is a continuous parameter which labels the family of surfaces). This approach, therefore, necessarily does not respect spacetime covariance since it involves making a specific choice of the foliation of the background into spacelike surfaces.

In the decades prior to the advent of string theory a great deal of effort was put into attempting to quantize gravity via both the Lagrangian and Hamiltonian approaches. The Hamiltonian approach involves, as stated earlier, a choice of the foliation of the background geometry into a collection of spacelike surfaces $\Sigma_t$. Central to this approach was the ADM (Arnowitt-Deser-Misner) formalism which allows one to construct a Hamiltonian $H_{GR}$ for general relativity starting from the Einstein-Hilbert action $S_{EH}$. The resulting phase is co-ordinatized by a configuration variable which is the 3-metric $h_{ab}$ of the ``leaves'' $\Sigma_t$ of the foliation and a momentum variable which is a function of a the extrinsic curvature $k_{ab}$ of the leaves\footnote{Our notational convention is the following. Lowercase Greek letters from the middle of the alphabet $\mu, \nu, \ldots$ are spacetime indices which run from $(0 \ldots 3)$, whereas lowercase Roman letters $a,b,\ldots$ are spatial indices for quantities which live solely on the surfaces $\Sigma_t$ and take values in $(1,2,3)$}. The gravity Hamiltonian turns out to be a sum of two constraints known as the diffeomorphism (or ``momentum'') constraint $H_{diff}$ and the Hamiltonian constraint $H_{ham}$:

\begin{equation}
	H_{GR} = H_{diff} + H_{ham},
\end{equation}

and these are, in turn, functionals on the phase space of general relativity written in terms of $h_{ab}$ and $k_{ab}$. The idea then is that physical states of the theory $\ket{\Psi_{phys}} $, which can be written as functionals $\Psi[h_{ab}]$ of the 3-metric $h_{ab}$ , must be annihilated by these constraints:

\begin{equation}
	H_{GR} \ket{\Psi_{phys}} \equiv 0
\end{equation}

While this procedure is straightforward in principle, in practice is was impossible to implement in the quantum theory due to highly complicated non-polynomial dependence of the diffeomorphism and Hamiltonian constraints on the configuration and momentum variables. This unfortunate state of affairs persisted until the 1980s when Abhay Ashtekar recast general relativity as a theory of a Minkowski tetrad $E^I_\mu$ and a self-dual $\sltwoc$ connection $A_\mu^{IJ}$. Ashtekar realised that the constraints of general relativity, when expressed in terms of these ``new variables'', simplified drastically and became polynomial functions of the tetrad and connection variables. It was quickly realised by various researchers that the resulting theory could be quantized using the same methods used to quantize Yang-Mills gauge theory and that the diffeomorphism constraint could be solved exactly in terms of so-called \emph{spin-network} states.

Following this, work by Rovelli and Smolin \todo{insert ref} and by Ashtekar, Rovelli and Smolin demonstrated the most remarkable feature of this theory, which came to be known as ``Loop Quantum Gravity'', was that one could construct quantum operators for geometric quantities such as areas of two dimensional surfaces and volumes of three dimensional regions. Moreover, these operators could be diagonlized exactly in the spin-network basis and a lower bound on the smallest possible quantum of area and quantum of volume could be derived. This was the first time that physicists had discovered the ``atoms of space'' or the ``quanta of geometry'' in the true sense of the expression.

\todo[inline]{Insert brief intro to spin nets}

\section{Strings in Background Fields and Emergent Gravity}\label{sec:strings-gravity}

Let us begin by recalling the Polyakov action for the bosonic string.

\begin{equation}\label{eqn:polyakov-flat}
	S_{P} = -\frac{T}{2} \int \diff \tau \wedge \diff \sigma \sqrt{-g} g^{ab} \partial_a X^\mu \partial_b X^\nu \eta_{\mu\nu}.
\end{equation}
Here $\mu, \nu \in \{0,1,\ldots,D-1\}$ are co-ordinates of the $D$ dimensional worldvolume (the geometry in which the string propagates), $a,b \in \{0,1\}$ are co-ordinates on the string worldsheet, $X^\mu$ are the embedding co-ordinates which specify the location of a point on the string worldsheet in the bulk worldvolume, $\eta_{\mu\nu}$ is the flat Minkowski metric on the worldvolume, $g_{ab}$ is the metric on the string worldsheet, $T$ is the string tension and $\tau, \sigma $ are the co-ordinates on the string worldsheet.

Now, one proceeds in the usual way by determining the symmetries of the action \eqref{eqn:polyakov-flat}, varying the action to find the equations of motion, fixing the gauge using the Weyl freedom of the worldsheet and then solving the classical equations of motion. Imposition of (bosonic) commutation relations on the operator versions of the embedding fields $\hat X^\mu$ then leads us to description of the quantum state of the bosonic string in terms of an infinite ladder of harmonic oscillators which obey the Virasoro algebra.

The obvious drawback of this approach is that the background metric is non-dynamical and is fixed to be the flat Minkowski metric $\eta_{\mu\nu}$. Clearly, one would like to be able to understand the physics of a string propagating on an arbitrary curved background. It wouldn't make much sense to refer to string theory as a theory of ``quantum gravity'' if strings can only be described on flat backgrounds. The way this is accomplished is by treating the metric of the worldvolume as a ``background field'' $G_{\mu\nu}$, in terms of which the Polyakov action becomes:

\begin{equation}\label{eqn:polyakov-gravity}
	S'_{P} = -\frac{T}{2} \int \diff \tau \wedge \diff \sigma \sqrt{-g} g^{ab} \partial_a X^\mu \partial_b X^\nu G_{\mu\nu}(X),
\end{equation}

where the bulk metric is now a function of the bulk co-ordinates $G_{\mu\nu}(X)$. Now this metric is still non-dynamical because the action \eqref{eqn:polyakov-gravity} does not contain any terms with time derivatives of $G_{\mu\nu}$. However, if we view the action purely as a two-dimensional theory of $D$ scalar fields, then the bulk metric can be viewed as a collection of \emph{coupling constants}, rather than a dynamical entity which exists independent of the string. One can now proceed in the usual manner for any field theory and calculate the beta function of the coupling constants of the theory as a function of the energy scale.

Then, as shown long ago by Friedan \cite{Friedan1985Nonlinear} (see also \cite{Callan1989Sigma}, \cite{Callan1985Strings} or \cite[Sec 3.7]{Polchinski1998aString}, \cite[Sec 7.2]{Tong2010Lectures} for a more pedagogical explanation), that the garviton beta function is proportional to $R_{\mu\nu}$, the Ricci curvature of the bulk geometry. The requirement of Weyl invariance of the string worldsheet implies that this beta function should vanish:
\begin{equation}\label{eqn:ricci-flat}
	\beta(G_{\mu\nu}) \propto R_{\mu\nu} = 0.
\end{equation}
Therefore we find that conformal invariance of the string worldsheet \emph{implies} that the background geometry in which the string propagates satisfies the \emph{vacuum} Einstein equations. It is this result which is often cited as evidence for the claim that string theory is a \emph{background independent} theory of quantum gravity.

In our opinion, however, this result is insufficient evidence for asserting that string theory is background independent. Our reasons are the following:

\begin{enumerate}[itemsep=0.5em]
	\item While the conformal invariance of the string worldsheet implies that the background is Ricci flat, this, in itself, is not very surprising. After all conformal invariance of worldsheet is equivalent to the statement of diffeomorphism invariance of the worldsheet, and in order for this to be consistent with the bulk geometry, the bulk geometry should also have a description in terms of diffeomorphism invariant physics. The simplest form of this requirement is that the background should satisfy the vacuum Einstein equations \todo{Does this make sense? Check!}.
	\item One can ask what is the quantum mechanical description of the bulk geometry? It is often suggested that the bulk geometry corresponds to a coherent state or a graviton condensate, where the graviton are themselves excitations of the fundamental string. However, there does not appear to be an actual mathematical realization of this statement in terms of non-perturbative physics.
	\item Finally, as mentioned in the introduction, in any \emph{quantum} theory of gravity one would expect to be able to describe the ``atoms of geometry''. There is no sense in which this is realised in this picture of gravity emerging from stringy physics. Even though one can calculate higher order corrections to the graviton beta function and this provides us with the quantum corrected effective action for the bulk geometry, all the physical quantities are still defined in terms of continuous fields such as the metric. For \eg, to the best of our knowledge, there is no clear sense in which one construct a \emph{non-perturbative} state which corresponds to a superposition of geometries using the quantum theory of a bosonic string alone.
\end{enumerate}

Despite all the concerns string theory has emerged as the primary candidate for a theory of quantum gravity. There are good reasons for this. AdS/CFT - the most concrete version of a \emph{non-perturbative} theory of quantum gravity - arose out of string theory. AdS/CFT is also far from being a complete quantum theory of quantum gravity. Our universe, for instance, is described by a positive cosmological constant $\Lambda > 0$ rather than $\Lambda < 0$ as is implied by the ``Anti-deSitter'' part of the correspondence. However, the correspondence has passed numerous checks and is now considered part of the established canon. Moreoever, it has led to numerous theoretical insights. We have gained a deep understanding of the properties of real world systems referred to as ``strange metals'' \todo{insert refs} by studying their dual gravitational theories. The significance of quantum error correction and quantum information for a theory of quantum gravity was also first most clearly illuminated via AdS/CFT \todo{insert refs}. For these, are other reasons too numerous to mention here, most theorists still regard string theory as the primary (or even the sole) candidate for a theory of quantum gravity.

On the other hand LQG is also built upon a rigorous mathematical framework which follows the well-established rules of quantization\footnote{Though there are those who disagree with this assertion \todo[inline]{insert refs}}. Moreoever it allows us to construct a non-perturbative framework for quantum geometry which makes it possible to probe regimes which do not appear to be accessible via string theory.

Thus, we are now at an impasse. We have two theories, both built upon rigorous mathematical foundations, but which do not appear to be concordant with each other. There are two possibilities. The first is that one of the two approaches is fundamentally flawed in some way and must be rejected. The second is that we are missing some crucial ingredient which would allow us to find a missing link between the two theories. We feel that it worthwhile to explore the second option in greater detail, if for no other reason then to rule out the possibility of such a link between these two theories.

Before proceeding let us state our ``central dogma''. We take as physically well-motivated the central result of LQG that at the Planck scale well-defined geometric operators - such as those which measure the area and volume of a given region of space - exist, have discrete spectra and are gapped, \ie, the smallest eigenvalue is greater than zero. We now wish to understand how to construct a worldsheet action which will reflect this central dogma.

\section{Modified Nambu Goto Action}\label{sec:modified-ng}

In a very real sense string theory is a quantum theory of geometry. After all, the Nambu-Goto action:
\begin{equation}\label{eqn:ng-action}
	S_{NG} = -T \int \diff ^2 x \sqrt{- \det h}
\end{equation}
is nothing more than the \emph{area} of the string worldsheet. So when we are constructing the quantum theory of the string we are constructing a theory in which the fundamental excitations are geometric in nature. What we would like to do is to modify the string action in a way which incorporates the insight from LQG that at the Planck scale there is a minimum quantum of area. One simple way in which to do this is to modify the integrand in \eqref{eqn:ng-action} as follows:
\begin{equation*}
	\sqrt{- h} \rightarrow \sqrt{- h + g \Delta}.
\end{equation*}
Here, $g = l_{pl}/l_s$, where $l_{pl}$ is the Planck scale, the scale at which quantum geometric effects become important; $l_s$ is the string scale where quantum geometric effects due to area quantization become small; $\Delta$ is a constant which corresponds to the square of the minimal quantum of area. In the limit that $l_s \gg l_{pl}$, $g \ll 1$ and the string action reduces to the conventional Nambu-Goto action with small corrections coming from quantum geometry. In this limit we can expand the integrand as follows:
\begin{equation}
	\sqrt{- h + g \Delta} \sim \sqrt{- h}\left(1 + \frac{1}{2} \frac{g \Delta}{- h} + \mc{O}(g^2/h^2)\right).
\end{equation}
Presumably the limit $g \ll 1$ corresponds to when the area of the string worldsheet is large compared to $\sqrt{h} \gg \Delta$. Thus in this limit $ g^2 \ll 1$ and $ h^2 \gg 1$ because of which we can drop the $\mc{O}(g^2/h^2)$ corrections. The final area corrected Nambu-Goto action becomes:
\begin{equation} \label{eqn:modified-NG-action}
	S'_{NG} = -T \int \diff ^2 x \left( \sqrt{- h} +  \frac{1}{2} \frac{g \Delta}{\sqrt{- h}} \right).
\end{equation}
This action has, to the best of our knowledge, not been explored in the string theory literature. However, such an inverse area term arises very naturally in the LQG approach and goes by the name of ``inverse triad'' corrections, in the context of the 3D volume of spatial hypersurfaces.

This correction does not change the form of the equations of motion and neither of the boundary conditions, i.e they are still given by
\begin{equation} \label{eqn:modified-NG-eom}
	\mathrm{EoM:} \partial _{\tau} \PS ^{\tau} _{\mu} + \partial _{\sigma} \PS ^{\sigma} _{\mu} = 0
\end{equation}
and
\begin{equation} \label{eqn:modified-NG-bc}
	\mathrm{B.C.:} \PS ^{\sigma} _{\mu} \delta X^{\mu} \big| _{\sigma = 0} ^{\sigma = \sigma _1} = 0 ,
\end{equation}
where
\begin{multline} \label{eqn:modified-NG-tau-currents}
	\PS ^{\tau} _{\mu} = \frac{\partial \Lagr ' _{N G}}{\partial \Dot{X}^{\mu}} = \\
	= - T \frac{(\Dot{X} \cdot X') X' _{\mu} - (X')^2 \Dot{X}_{\mu}}{\sqrt{- h}} \left( 1 - \frac{g \Delta}{2 (- h)} \right) = \\
	= \PS ^{\tau} _{\mu (N G)} \left( 1 - \frac{g \Delta}{2 (- h)} \right)
\end{multline}
\begin{multline} \label{eqn:modified-NG-sigma-currents}
	\PS ^{\sigma} _{\mu} = \frac{\partial \Lagr ' _{N G}}{\partial X^{\prime \mu}} = \\
	= - T \frac{(\Dot{X} \cdot X') \Dot{X}_{\mu} - (\Dot{X})^2 X' _{\mu}}{\sqrt{- h}} \left( 1 - \frac{g \Delta}{2 (- h)} \right) = \\
	= \PS ^{\sigma} _{\mu (N G)} \left( 1 - \frac{g \Delta}{2 (- h)} \right)
\end{multline}
and $\PS ^{\tau} _{\mu (N G)}$, $\PS ^{\sigma} _{\mu (N G)}$ are the regular Nambu-Goto world-sheet currents.

\section{Strings and Quantum Geometry}\label{sec:string-geometry}

Let us begin by writing down the action of a string propagating in a background geometry which is specified not in terms of a metric, but in terms of a connection $A_\mu^{IJ}$ and vielbein fields $E_\mu^I$. Further we will require that the bulk connection and the bulk vielbein can be pulled back onto a connection $\omega _{a} ^{i j}$ and a dyad $e^a_i$, respectively, living on the string worldsheet $\Sigma$. In order for the bulk connection $A_\mu^{IJ}$, and its pullback $\omega _a ^{i j}$ on $\Sigma$, to have an effect on the string dynamics we will also promote the embedding fields $X^\mu$ to Lie algebra valued vector fields $X^{\mu} _{I}$. However, due to the fact that parallel transport of the embedding fields on the worldsheet should only involve the worldsheet connection $\omega _a ^{i j}$, we can replace $X^{\mu} _{I}$ with $X^{\mu} _{i}$, and to not lose the physical meaning of embedding fields as string coordinates, we include a new internal world-sheet field $v^{i}$, and define the string coordinates to be $\Tilde{X}^{\mu} \equiv X^{\mu} _{i} v^{i}$.
\begin{multline} \label{eqn:lqg-string-action}
	S_{PA} = - \frac{T}{2} \int \diff \tau \wedge \diff \sigma e e^{a} _{i} e^{b i} D_{a} X^{\mu} _{j} v^{j} \times \\
	\times D_{b} X^{\nu} _{k} v^{k} E_{\mu} ^{I} E_{\nu I} ,
\end{multline}
where $e = \det (e^{i} _{a})$ and $D_a$ is the covariant derivative with respect to the worldsheet connection, and is given by:
\begin{equation} \label{eqn:worldsheet-deriv}
	D_a X^{\mu} _{i} & = \partial_a X^{\mu} _{i} - \omega_{a\,i}^{j} X^{\mu} _{j} ,
\end{equation}
where $\omega_{a\,j}^i$ is the pullback of the bulk connection to the worldsheet.

\section{Equations of Motion}\label{sec:eom}

\subsection{The Dyad Equation}\label{subsec:dyad-eom}

Variation of the modified Polyakov action(\ref{eqn:lqg-string-action}) w.r.t the zweibein $e^{c} _{l}$ yields the vanishing of the Lie algebra-valued energy-momentum tensor, which can be contracted with the dyad to yield the regular stress-energy tensor:
\begin{equation} \label{eqn:variation-wrt-e}
	\frac{\delta S_{PA}}{\delta e^{c} _{l}} \overset{!}{=} 0
\end{equation}
\begin{equation*}
	\downarrow
\end{equation*}
\begin{multline} \label{eqn:vanishing-em-tensor}
	T^{l} _{c} := E^{I} _{\mu} E_{\nu I} (e^{a l} D_{a} X^{\mu} _{j} v^{j} D_{c} X^{\nu} _{k} v^{k} - \\
	- e^{l} _{c} e^{a} _{i} e^{b i} D_{a} X^{\mu} _{j} v^{j} D_{b} X^{\nu} _{k} v^{k}) = 0 .
\end{multline}
Equation(\ref{eqn:vanishing-em-tensor}) can be used to obtain an expression for the dyads in terms of the embbeding fields $X^{\mu} _{i}$ as in
\begin{equation} \label{eqn:dyad-string-coordinates}
	e^{l} _{c} = F e^{a l} D_{a} X^{\mu} _{j} v^{j} D_{c} X^{\nu} _{k} v^{k} E^{I} _{\mu} E_{\nu I} ,
\end{equation}
where the function $F$ is given by
\begin{equation} \label{eqn:F-function-dyad}
	\frac{1}{F} = e^{a} _{i} e^{b i} D_{a} X^{\mu} _{j} v^{j} D_{b} X^{\nu} _{k} v^{k} E^{I} _{\mu} E_{\nu I} .
\end{equation}

\subsection{The Connection Equation} \label{sec:omega-eom}

Varying our action w.r.t connection $\omega ^{m n} _{c}$ gives the vanishing of world-sheet torsion $\mathcal{T}^{i} _{a b}$ as
\begin{equation} \label{eqn:variation-wrt-omega}
	\frac{\delta S_{P A}}{\delta \omega ^{m n} _{c}} \overset{!}{=} 0
\end{equation}
\begin{equation*}
	\downarrow
\end{equation*}
\begin{equation} \label{eqn:vanishing-torsion}
	\mathcal{T}^{i} _{a b} := e^{c i} D_{c} X^{\mu} _{k} v^{k} X^{\nu} _{[m} v_{n]} e^{m} _{a} e^{n} _{b} E^{I} _{\mu} E_{\nu I} = 0 .
\end{equation}

\subsection{The New Field Equation} \label{sec:v-eom}

The equation of motion for the new field $v^{l}$ gives a constraint on said field and the embbeding fields $X^{\mu} _{i}$
\begin{equation} \label{eqn:variation-wrt-v}
	\frac{\delta S_{P A}}{\delta v^{l}} \overset{!}{=} 0
\end{equation}
\begin{equation*}
	\downarrow
\end{equation*}
\begin{equation} \label{eqn:v-and-X-constraint}
	e^{a} _{i} e^{b i} D_{a} X^{\mu} _{l} D_{b} X^{\nu} _{j} v^{j} E^{I} _{\mu} E_{\nu I} = 0 .
\end{equation}

\subsection{The Embbeding Field Equation}

Finally, varying the action w.r.t the embbeding fields $X^{\lambda} _{l}$ yields a slightly modified version of the classical equation of motion for strings propagating in curved space-time:
\begin{equation} \label{eqn:variation-wrt-X}
	\frac{\delta S_{P A}}{\delta X^{\lambda} _{l}} \overset{!}{=} 0
\end{equation}
\begin{equation*}
	\downarrow
\end{equation*}
\begin{multline} \label{eqn:string-eom}
	e^{a} _{i} e^{b i} D_{b} (e v^{l} D_{a} X^{\mu} _{j} v^{j} E^{I} _{\mu} E_{\lambda I}) = \\
	= e e^{a} _{i} e^{b i} D_{a} X^{\mu} _{j} v^{j} D_{b} X^{\nu} _{k} v^{k} \frac{\partial E^{I} _{\mu}}{\partial X^{\lambda} _{l}} E_{\nu I} .
\end{multline}

\section{Physical Implications}

\section{Discussion}

\begin{acknowledgments}

\end{acknowledgments}

\appendix

\section{}

\bibliographystyle{JHEP3}

\bibliography{lqg-strings.bib}


\end{document}