\subsection{Strings = Matter \& Loops = Gravity}\label{eqn:strings-loops}

We would also like to inform the discussion by pointing out some complementary aspects of the historical development of both string theory and loop quantum gravity which should already have been a hint many years ago that the two would eventually turn out to be intimately related.

\textbf{The String Theory Perspective}

String theory arose from attempts to understand the Regge slope behavior of high energy scattering. Tullio Regge \cite{Regge1959Introduction,Chew1962S-Matrix,Chew1961Principle} had predicted in 1959 that when considering high energy scattering the mass squared $M^2$ and angular momentum $J$ of mesonic and hadronic resonances satisfy a linear relationship\footnote{Though a more complete analysis of recent experimental data \cite{Tang2000Properties,Chen2018Concavity} shows that the trajectories actually turn out to be nonlinear and intersecting.} of the form:
\begin{equation}\label{eqn:regge-slope}
	J = \alpha(0) + \alpha' M^2,
\end{equation}
where $\alpha'$ is generally referred to as the ``Regge slope''. One such trajectory is shown in \autoref{fig:chew-frautschi-plot-p2}.

\begin{figure}[htbp]
	\centering
	\includegraphics[scale=0.35]{chew-frautschi-plot-p2.png}
	\caption{A Chew-Frautchi plot of $M^2$ vs $\alpha(M^2) \sim J$  for several high energy resonances. Figure courtesy \cite{Desgrolard2001Exchange-Degenerate} \todo{Pending permission from authors.}}.
	\label{fig:chew-frautschi-plot-p2}
\end{figure}
In 1968, Gabriele Veneziano discovered \cite{Veneziano1968Construction} an explicit analytic expression for the scattering amplitude which exhibited Regge behavior and satisfied certain consistency conditions known as ``crossing symmetry''. Soon after works by Nambu \cite{Nambu1969Quark}, Nielsen \cite{Nielsen1970An-Almost} and Susskind \cite{Susskind1970Dual} proposed a string like model for Regge behavior.

What this very brief history of the early years of string theory tells us is that the theory arose from trying to understand particle scattering. This, in part, explains why string theorists didn't pay much attention to the question of background independence of string theory during its early decades. For particle physicists the flat Minkowski background was a given - the entire machinery of Feynman diagrams and perturbation theory was built with the implicit assumption of a flat background manifold. The ``string'' in the first conception was solely meant to be a description of bound states of mesons and hadrons. From the abstract of Susskind's 1970 paper \cite{Susskind1970Dual} we quote

\begin{quote}
	\emph{A theory of mesons is based on the experimentally observed linear level spacing of hadron excitations. The theory uses a model for the internal structure of mesons consisting of a system of harmonic vibrations. Roughly speaking, a meson is described by the degrees of freedom of a four-dimensional rubber band with a quark pair embedded in it.}
\end{quote}

It was realized only later \cite{Scherk1974bDual} that excitations of the string included a spin-2 object which could be identified with a graviton.

\textbf{The Loop Quantum Gravity Perspective}

It is clear from the above discussion that the discovery of a spin-2 excitation in the spectrum of the closed bosonic string was something unexpected since the theory was always formulated on a flat background. Gravity was never intended to be a part of the framework. The resulting spin-2 excitation is therefore, correctly interpreted as the quantization of metric perturbations around a flat background.

In stark contrast, loop quantum gravity arose from the desire of relativists to quantize gravity in a manifestly background independent manner. Einstein's great insight was that co-ordinates were ``unphysical''. They are arbitrary labels which we assign to describe physical systems around us.