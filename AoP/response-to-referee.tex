\documentclass{article}
\def\lettertitle{Author Response to Reviews of}
\def\papertitle{A Loop Quantum Gravity Inspired Action for the Bosonic String and Emergent Dimensions at Large Scales}
\def\authors{Deepak Vaid and Luigi Teixeira de Sousa}
\def\journal{Annals of Physics}
\def\doi{AOP 81029}

\input{AR_to_RC.tex}

We thank the referee for their thorough and constructive review of our manuscript. We have carefully considered all the points raised and provide detailed responses below.

\section{Reviewer \#1}

\subsection{Introduction}

\RC The introduction contrasts string theory and LQG, highlighting their respective strengths and limitations. The discussion of background independence in string theory could benefit from references to more recent critiques or developments such as recent advances in non-perturbative string theory.

\AR We have added the following two footnotes with references to recent critiques of string theory's background independence \cite{Hohm2017Background,Hohm2018Background} and recent advances in non-perturbative string theory \cite{Maccaferri2024String,Sen1994Quantum}:

\emph{Recent work has highlighted fundamental challenges to achieving manifest background independence in string theory. Hohm has shown that there exists a fundamental conflict between manifest background independence and duality invariance when higher-derivative corrections are included \cite{Hohm2017Background,Hohm2018Background}.}

\emph{Developments in string field theory continue to address these foundational issues \cite{Maccaferri2024String,Sen1994Quantum} primarily in the framework of string field theory. For an overview of the general concept of background independence in quantum gravity see \cite{Read2023Background}.}

\RC The "central dogma" (minimal area $\Delta$) is well-motivated from LQG but could be better contextualized within existing string theory literature, where minimal length scales are discussed. The paper's goals are clearly stated, but the transition to the modified Nambu-Goto action feels abrupt, a smoother connection between LQG's area quantization and the string action would improve this.

\AR 

\subsection{Section 2 - Part (a): Action Justification}

\RC The proposed action $S_{MNG}=-T\int dx^2\sqrt{-(h+k \Delta)}$ needs a clearer justification: the author states in a footnote that $\Delta$ is as a scalar density later identified with the Kalb-Ramond field should be, at least, physically justified because the remaining of this section is based in this argument, otherwise it should be rigorously derived.

\AR 

\subsection{Section 2 - Part (b): Expansion Parameter}

\RC The expansion in $k=l_{pl}/l_s$ seems sound, but the assumption $k\ll 1$ (Planck scale much smaller than string scale) should be discussed further—does this hold in all regimes?

\AR 

\subsection{Section 2 - Part (c): Duality Interpretation}

\RC The duality between $h_{ab}$ and $h^{ab}$ (Eq. 2.6a-2.6c) is intriguing but lacks a deeper physical interpretation. How does this duality manifest in observable string dynamics?

\AR 

\subsection{Section 3 - Part (a): Bimetric String and Ghost Avoidance}

\RC The author introduces the connection to bimetric gravity, but the analogy should be strengthened by discussing how the bimetric string avoids issues like the Boulware-Deser ghost.

\AR 

\subsection{Section 3 - Part (b): Kalb-Ramond Field Connection}

\RC The link between $\Delta_{ab}$ and the Kalb-Ramond field $B_{\mu\nu}$ (Sec. 3.2) is a highlight, but the derivation of Eq. (3.21) should be detailed for clarity.

\AR 

\subsection{Section 3 - Part (c): Bimetric Gravity Background}

\RC The section assumes familiarity with bimetric gravity, a brief review of key concepts would aid readers outside this niche.

\AR 

\subsection{Section 4 - Part (a): Physical Meaning of Coupling}

\RC In equations (4.7-4.10) the physical meaning of the coupling between $\phi(x)$, $\Delta$ and $X^\mu$ (akin to a dilaton) needs a more thorough explanation.

\AR 

\subsection{Section 4 - Part (b): Born-Infeld Analogy}

\RC The Born-Infeld analogy (Sec. 4.2) should address why the Kalb-Ramond field must be exact.

\AR 

\subsection{Section 5 - Part (a): Anomaly-Free Claims}

\RC The covariant quantization procedure seems correct, but the absence of anomalies (Sec. 5.12) is a significant claim that deserves more discussion. How does this compare to standard string theory?

\AR 

\subsection{Section 5 - Part (b): Mode Truncation}

\RC The truncation of modes $\alpha_\mu^\nu =0$ for $|n|>d$ due to discreteness needs validation—does this align with other approaches to quantum gravity?

\AR 

\subsection{Section 5 - Part (c): Table 1 Context}

\RC Table 1 (values of $a$ for different $D$ and $\mu$) lacks context. A brief explanation of how these values affect the string spectrum would help.

\AR 

\subsection{Section 6 - Part (a): Afshordi Model Connection}

\RC The link to Afshordi's model is interesting, but the constraints on $m_\mu$ (Eq. 6.3) feel ad hoc, a better and deeper connection to LQG's discreteness would strengthen this section.

\AR 

\subsection{Section 6 - Part (b): Fundamental Spacetime Dimensionality}

\RC The suggestion of (2+1)-dimensional fundamental spacetime (Sec. 6) is speculative and should be better addressed, more evidence or references to support this claim are needed.

\AR 

\subsection{Section 7 - Part (a): Holographic Dimension Interpretation}

\RC The physical interpretation of the emergent holographic dimension $X^r$ as a "scaling dimension" needs clarification.

\AR 

\subsection{Section 7 - Part (b): AdS/CFT Connection}

\RC The connection to AdS/CFT (Sec. 7.2) lacks concrete evidence, how does this classical construction relate to the original Maldacena's duality?

\AR 

\subsection{Discussion - Part (a): Open Questions}

\RC The discussion synthesizes the paper's key results but could better highlight open questions (e.g., experimental signatures, relation to black hole physics), considering the ideas established in the manuscript.

\AR 

\subsection{Discussion - Part (b): Anomaly-Free and Ghost-Free Claims}

\RC The claim that the model is anomaly-free and ghost-free (Sec. 5) should be tempered with caveats (e.g., limitations of the truncation scheme).

\AR 

\subsection{General: Writing and Typos}

\RC A full review should be done to the writing to correct typos.

\AR 

\section{Summary of Major Changes}

% This section will be filled once revisions are complete
\begin{enumerate}
    \item 
    \item 
    \item 
\end{enumerate}

\section{Submission Requirements}

We confirm that we will submit the following files with our revision:
\begin{itemize}
    \item Manuscript source files (LaTeX format)
    \item Figure source files in acceptable formats (TIFF, EPS)
    \item Highlights file (3-5 bullet points, maximum 85 characters each)
\end{itemize}

\bibliographystyle{JHEP3}

\bibliography{../lqg-strings.bib}

\end{document}
