\documentclass{article}
\usepackage[square,comma]{natbib}
\def\lettertitle{Author Response to Reviews of}
\def\papertitle{A Loop Quantum Gravity Inspired Action for the Bosonic String and Emergent Dimensions at Large Scales}
\def\authors{Deepak Vaid and Luigi Teixeira de Sousa}
\def\journal{Annals of Physics}
\def\doi{AOP 81029}

\input{AR_to_RC.tex}

We thank the referee for their thorough and constructive review of our manuscript. We have carefully considered all the points raised and provide detailed responses below.

\section{Reviewer \#1}

\subsection{Introduction}

\RC The introduction contrasts string theory and LQG, highlighting their respective strengths and limitations. The discussion of background independence in string theory could benefit from references to more recent critiques or developments such as recent advances in non-perturbative string theory.

\AR We have added the following two footnotes with references to recent critiques of string theory's background independence \citep{Hohm2017Background,Hohm2018Background} and recent advances in non-perturbative string theory \citep{Maccaferri2024String,Sen1994Quantum}:

\emph{Recent work has highlighted fundamental challenges to achieving manifest background independence in string theory. Hohm has shown that there exists a fundamental conflict between manifest background independence and duality invariance when higher-derivative corrections are included \citep{Hohm2017Background,Hohm2018Background}.}

\emph{Developments in string field theory continue to address these foundational issues \citep{Maccaferri2024String,Sen1994Quantum} primarily in the framework of string field theory. For an overview of the general concept of background independence in quantum gravity see \citep{Read2023Background}.}

\RC The "central dogma" (minimal area $\Delta$) is well-motivated from LQG but could be better contextualized within existing string theory literature, where minimal length scales are discussed. The paper's goals are clearly stated, but the transition to the modified Nambu-Goto action feels abrupt, a smoother connection between LQG's area quantization and the string action would improve this.

\AR We have added the following paragraphs, including the new references \citep{Amati1989Canb,Chang2011Minimal,Gross1987Highenergy,Gross1988String,Yoneya1989Interpretation,Yoneya2000String}, after the central dogma statement, in order to make the transition between LQG's area quantization and the modified Nambu-Goto action smoother and to include a comparison with previous minimal length proposals in string theory:

\emph{The key insight is to translate this fundamental discreteness of area from spacetime geometry to the worldsheet of the string itself. In LQG, the eigenvalues of the area operator are discrete, meaning physical areas can only take values from a quantized spectrum with minimum eigenvalue $\Delta$. For a string worldsheet, this suggests that the total worldsheet area $\mathcal{A} = \int d^2\sigma \sqrt{-h}$ cannot be arbitrarily small but must satisfy $\mathcal{A} \geq \Delta$.}

\emph{Rather than imposing this constraint externally, we incorporate it directly into the action by modifying the area element itself. The natural way to ensure the worldsheet area respects LQG's minimal area is to replace the standard area element $\sqrt{-h}$ with $\sqrt{-(h+k\Delta)}$ where $k = l_{pl}/l_s$ is the ratio of Planck to string scales. This modification has several appealing features: it reduces to the standard Nambu-Goto action when $k \to 0$, it ensures the worldsheet area can never become arbitrarily small, and it provides a concrete mechanism to incorporate LQG's geometric quantization into string theory.}

\emph{This approach differs from previous minimal length proposals in string theory \citep{Amati1989Canb,Chang2011Minimal,Gross1987Highenergy,Gross1988String,Yoneya1989Interpretation,Yoneya2000String} in that it specifically targets area rather than length, reflecting LQG's emphasis on area as the fundamental geometric observable.}

\subsection{Section 2 - Part (a): Action Justification}

\RC The proposed action $S_{MNG}=-T\int dx^2\sqrt{-(h+k \Delta)}$ needs a clearer justification: the author states in a footnote that $\Delta$ is as a scalar density later identified with the Kalb-Ramond field should be, at least, physically justified because the remaining of this section is based in this argument, otherwise it should be rigorously derived.

\AR We must be honest and state that at present the area modified action is not rigorously derived. We have added a paragraph right after the action area-modified action (eqn 2.2. in the text) to explain this point:

\emph{At present we do not have a rigorous derivation of this expression. It is the simplest possible ansatz one can think of to encode the insight from LQG that at the Planck scale there is a minimum quantum of area. What makes us more confident in this ansatz is that it is consistent with the Kalb-Ramond field identification and the minimal area interpretation of the scalar density $\Delta$. However, this was not known prior to coming up with the ansatz and only discovered after the fact.}

\subsection{Section 2 - Part (b): Expansion Parameter}

\RC The expansion in $k=l_{pl}/l_s$ seems sound, but the assumption $k\ll 1$ (Planck scale much smaller than string scale) should be discussed further—does this hold in all regimes?

\AR We've added the following paragraph right after the expansion parameter $k$ definition to explain our choice of $k \ll 1$:

\emph{If we take $k$ to be an expansion parameter, then it follows that it should also be small. The physical justification for this is the following picture. The microscopic picture of geometry at the Planck scale, which we adopt as the ``correct'' picture, is that given by loop quantum gravity in terms of spin networks and spin foams. The worldsheet picture emerges when one considers a large enough section of the spin network with a sufficiently large number of punctures. In this limit, the spin network can be approximated by a smooth geometry, and the worldsheet picture emerges as a low-energy effective description of the spin network. When $k \sim 1$, the discrete structure of the underlying spin network becomes manifest and the smooth worldsheet approximation is no longer valid, indicating that a full quantum gravitational treatment - i.e. where geometry itself is fully quantum - is required.}

\subsection{Section 2 - Part (c): Duality Interpretation}

\RC The duality between $h_{ab}$ and $h^{ab}$ (Eq. 2.6a-2.6c) is intriguing but lacks a deeper physical interpretation. How does this duality manifest in observable string dynamics?

\AR We have added the following paragraph which hopefully addresses this point:

\emph{This duality is reminiscent of, but also distinct from, the T and S dualities in string theory. In T-duality the behavior of a string compactified on a circle of radius $R$ is related to the behavior of a string compactified on a circle of radius $\alpha'/R$. In S-duality the physics at strong string coupling $g_s \gg 1$ can be mapped to the physics at weak string coupling $g_s \ll 1$. In our case, the duality is between a description of physics at large string tension $T$ and small worldsheet areas $h_{ab}$ and a description of physics at small string tension $T'$ and large worldsheet areas $h'_{ab} = (h_{ab})^{-1}$. This appears to have features of both T-duality and S-duality, in that the area inversion is reminiscent of T-duality and the string tension inversion is reminiscent of S-duality's strong-weak coupling correspondence.}

\subsection{Section 3 - Part (a): Bimetric String and Ghost Avoidance}

\RC The author introduces the connection to bimetric gravity, but the analogy should be strengthened by discussing how the bimetric string avoids issues like the Boulware-Deser ghost.

\AR We had already mentioned the Boulware-Deser ghost avoidance in Section 2, right after the section when we talk about duality between the area metric and its inverse. As for how this would affect string theory, we feel that a proper discussion of this issue would be beyond the scope of this paper and would require a separate paper in its own right. We have added the following paragraph in Section 3 addressing this point, right after "Here the subscript $BNG$ stands for ``Bimetric Nambu-Goto''":

\emph{As mentioned at the end of the previous section, the appeal of bimetric gravity lies in the fact that it eliminates the Boulware-Deser ghost in theories of massive gravity. However, what effect this would have on stringy dynamics, beyond the ansatz for a bimetric string action we have given above, and which to our knowledge is the first time such an expression has been written down, is beyond the scope of the present work and would require a separate paper in its own right.}

\subsection{Section 3 - Part (b): Kalb-Ramond Field Connection}

\RC The link between $\Delta_{ab}$ and the Kalb-Ramond field $B_{\mu\nu}$ (Sec. 3.2) is a highlight, but the derivation of Eq. (3.21) should be detailed for clarity.

\AR \AR The derivation of the relation between $\Delta_{ab}$ and the Kalb-Ramond field $B_{\mu\nu}$ is detailed in Section 4a, specifically in equations (4.12) and (4.13). Equation (4.12) shows the area-corrected string action coupled to the Kalb-Ramond field, while equation (4.13) demonstrates that by varying this action with respect to $\Delta^{cd}$, we obtain $\Delta_{cd} = \widetilde{f} \partial_c X^\mu \partial_d X^\nu B_{\mu\nu}$, establishing the explicit connection between the worldsheet quantum of area tensor and the pullback of the bulk Kalb-Ramond field.

\subsection{Section 3 - Part (c): Bimetric Gravity Background}

\RC The section assumes familiarity with bimetric gravity, a brief review of key concepts would aid readers outside this niche.

\AR We agree that a comprehensive review of bimetric gravity would aid readers unfamiliar with this area. However, we feel that including such a review would take the paper too far afield from its main focus on the LQG-inspired string action. We have provided the key reference to seminal works \citep{de-Rham2011Resummation, Hassan2012Bimetric} for readers who wish to explore bimetric gravity in greater detail.

\subsection{Section 4 - Part (a): Physical Meaning of Coupling}

\RC In equations (4.7-4.10) the physical meaning of the coupling between $\phi(x)$, $\Delta$ and $X^\mu$ (akin to a dilaton) needs a more thorough explanation.

\AR We have expanded the paragraph following eqn 4.10, from the following:

\emph{revealing a wave equation sourced by a coupling between the WS conformal factor, quantum of area field $\Delta$ and the embedding fields much akin to the dilaton field \citep{Balthazar-t-backgrounds}.}

to:

\emph{revealing a wave equation sourced by a coupling between the WS conformal factor, quantum of area field $\Delta$ and the embedding fields. The coupling term $\frac{(2\partial_a \phi - \partial_a \ln(\Delta))}{\frac{2e^{2\phi}}{k\Delta} - 1}$ shows how the area quantization correction modifies the propagation of embedding coordinates - the numerator combines gradients from both conformal and quantum geometric effects, while the denominator represents the modification to the wave operator due to LQG's minimal area constraint. This behavior is much akin to the dilaton field \citep{Balthazar-t-backgrounds}.}

\subsection{Section 4 - Part (b): Born-Infeld Analogy}

\RC The Born-Infeld analogy (Sec. 4.2) should address why the Kalb-Ramond field must be exact.

\AR The mention of exactness in the Born-Infeld analogy section is an additional observation rather than a requirement for the analogy itself. The core point is the structural similarity between our area-corrected action and the Born-Infeld form, where $\Delta _{ab}$ plays the role of a field strength. The exactness condition and its implications for background torsion, while potentially interesting, are not essential to this analogy and can be omitted without affecting the main argument. We have accordingly edited the paragraph following eqn 4.10 to omit the reference to exactness and torsion, from:

\emph{thus the action (2.2) can be recast as a Born-Infeld model for the string with $k$ playing the role of inverse tension squared and $\Delta _{ab}$ the role of the pullback of a field strength. Since we explored the relation between $\Delta _{ab}$ and the Kalb-Ramond field $B_{\mu \nu}$ in Section 4A, this could mean that the Kalb-Ramond 2-form is exact, $B = d A$, and thus its exterior derivative, which gives the torsion of the background spacetime \cite{Tong2009Lectures}, vanishes $d B = d (d A) = 0$. Also, we have a direct relation between string tension and scale ...}

to:

\emph{thus the action (2.2) can be recast as a Born-Infeld model for the string with $k$ playing the role of inverse tension squared and $\Delta _{ab}$ the role of the pullback of a field strength. Also, we have a direct relation between string tension and scale ...}

\subsection{Section 5 - Part (a): Anomaly-Free Claims}

\RC The covariant quantization procedure seems correct, but the absence of anomalies (Sec. 5.12) is a significant claim that deserves more discussion. How does this compare to standard string theory?

\AR We have added the following two paragraphs at the end of Section 5, after the sentence ``... and no negative-norm states in physical processes.'':

\emph{It is important to clarify what we mean by ``no anomalies'' in this context. This refers specifically to the absence of central charge anomalies in our modified Virasoro-like algebra, which arises due to the finite mode truncation imposed by LQG's discreteness. This discreteness is encoded in the paramter $\mu$ in (4.27), which is a cutoff parameter that controls the maximum number of modes we consider in the action.}

\emph{This differs fundamentally from standard string theory, where anomaly cancellation requires critical dimensions ($D = 26$ for bosonic strings) to ensure vanishing of the central charge from an infinite sum over modes. Our finite sum naturally avoids this constraint, permitting unconstrained background dimensionality $D$. However, this result applies specifically to our truncated model in the regime where $k \ll 1$, where the smooth worldsheet approximation remains valid. In the full quantum gravitational regime where $k \sim 1$, the discrete structure of the underlying spin network becomes manifest and a more comprehensive analysis would be required.}

\subsection{Section 5 - Part (b): Mode Truncation}

\RC The truncation of modes $\alpha_\mu^\nu =0$ for $|n|>d$ due to discreteness needs validation—does this align with other approaches to quantum gravity?

\AR We have added the following two paragraphs to clarify the issues regarding mode truncation and comparison with other quantum gravity approaches:

\emph{The mode truncation is a key assumption that requires validation. While we do not have a rigorous derivation from first principles, the physical motivation comes from LQG's discrete structure: high-frequency modes corresponding to sub-Planckian physics should be suppressed when the spin network discreteness becomes relevant. The parameter $\mu$ in (4.27) provides an effective energy cutoff, with modes $|n| > d \approx \mu$ being suppressed.}

\emph{This approach shares similarities with other discrete quantum gravity approaches like causal dynamical triangulation and spin foam models, which also impose natural cutoffs. However, a complete justification would require deriving this truncation directly from LQG's canonical quantization procedure, which remains an open question for future work.}

\subsection{Section 5 - Part (c): Table 1 Context}

\RC Table 1 (values of $a$ for different $D$ and $\mu$) lacks context. A brief explanation of how these values affect the string spectrum would help.

\AR We have added the following clarifying paragraph right after the table in the text:

\emph{This table shows that $a$ becomes more negative as either the spacetime dimension $D$ or the mass parameter $\mu$ increases. This means:}

\begin{itemize}
	\item \emph{Higher dimensions lead to more massive ground states}
	\item \emph{Larger $\mu$ (stronger discreteness effects) also increases ground state masses}
	\item \emph{The spectrum becomes increasingly separated from the massless limit}
\end{itemize}

\emph{For example, in $D=2$ with $\mu=1$, $a=-4.24$ gives a relatively light spectrum, while $D=3$ with $\mu=5$ yields $a=-34.27$, producing much heavier states. This demonstrates how LQG's discreteness parameter $\mu$ and the background dimensionality jointly control the departure from standard string theory's massless spectrum, with stronger quantum geometric effects pushing all states away from the massless limit expected in conventional string theory.}

\subsection{Section 6 - Part (a): Afshordi Model Connection}

\RC The link to Afshordi's model is interesting, but the constraints on $m_\mu$ (Eq. 6.3) feel ad hoc, a better and deeper connection to LQG's discreteness would strengthen this section.

\AR We acknowledge the reviewer's concern that the constraints on $m_\mu$ (Eq. 6.3) appear ad hoc. Upon reflection, we must be honest that a rigorous connection between Afshordi's mass parameters and LQG's fundamental discreteness has not yet been established. We have added the following two paragraphs to the text right after ``... making it possible to predict in which scales the other dimensions should start to open up.''

\emph{Our constraint equation (6.3) represents an attempt to bridge Afshordi's phenomenological emergent dimension model with LQG's general prediction of modified physics at the Planck scale. However, the specific form of these constraints and their derivation from first-principles LQG remains an open problem. While both frameworks involve discreteness - Afshordi's through hierarchical mass parameters and LQG through quantized geometric operators - establishing a concrete mathematical connection requires significantly more work.}

\emph{This represents an important limitation of our current approach and highlights the need for future research to either derive such constraints from canonical LQG or develop alternative frameworks that more naturally incorporate both emergent dimensions and quantum geometric discreteness. Similar caveats also apply to our suggestion that (6.3) should imply that spacetime is $(2 + 1)$-dimensional.}

\subsection{Section 6 - Part (b): Fundamental Spacetime Dimensionality}

\RC The suggestion of (2+1)-dimensional fundamental spacetime (Sec. 6) is speculative and should be better addressed, more evidence or references to support this claim are needed.

\AR We hope that the last sentence of the second paragraph in the previous point address this point.

\subsection{Section 7 - Part (a): Holographic Dimension Interpretation}

\RC The physical interpretation of the emergent holographic dimension $X^r$ as a "scaling dimension" needs clarification.

\AR We acknowledge that the interpretation of $X^r$ as a "scaling dimension" requires clearer explanation. We have added the following clarifying paragraph at the end of section 7B ``Gauging Translations'':

\emph{The interpretation of $X^r$ as a "scaling dimension" follows directly from the transformation properties of the enlarged coordinate system. In equations (7.5) and (7.6), we introduce $X^r$ as an auxiliary coordinate needed to close the Poincaré algebra, initially fixing $X^r = 1$. However, when $X^r$ is allowed to vary from this reference value, the transformation properties show that the other coordinates $X^\mu$ undergo scale transformations. Specifically, in the enlarged coordinate system, physical coordinates can be interpreted as $x^\mu = X^\mu / X^r$, so variations in $X^r$ directly correspond to scaling transformations of the physical coordinates. This provides a concrete mathematical justification for interpreting $X^r$ as a scaling dimension - it is precisely the coordinate that controls the scale of the embedding. This geometric scaling property is what allows our construction to naturally accommodate the holographic principle, where the extra dimension encodes information about scales in the lower-dimensional theory.}

\subsection{Section 7 - Part (b): AdS/CFT Connection}

\RC The AdS/CFT connection (Sec. 7.2) lacks concrete evidence, how does this classical construction relate to the original Maldacena's duality?

\AR We have clarified this point by adding the following text at the end of Section 7C, right after `` ... for describing the background geometry and its coupling to the string worldsheet.''

\emph{The key observation which enables this is that $X^r$ behaves as a scaling or ``holographic'' dimension as explained at the end of Sec 7B.}

\emph{Our classical construction provides a geometric framework that naturally incorporates key features of the AdS/CFT correspondence. The scaling dimension $X^r$ introduced through the mathematical necessity of closing the Poincaré algebra plays a role analogous to the radial coordinate in AdS space. Just as in AdS/CFT where the radial direction encodes the energy scale of the boundary theory, variations in our $X^r$ coordinate directly control the scale transformations of the bulk coordinates $X^\mu$. This creates a natural holographic setup where information about different energy scales in the ``boundary'' ($X^r = \text{constant}$ surfaces) is encoded in the ``bulk'' (the extended geometry with the extra $X^r$ dimension).}

\emph{However, our construction differs from standard AdS/CFT in several important ways: (1) it emerges at the classical level from gauge theory requirements rather than from quantum/non-perturbative effects, (2) the holographic direction arises from the string action itself rather than being an external bulk geometry, and (3) the correspondence is between worldsheet dynamics and the extended embedding space rather than between a boundary CFT and bulk gravity. While this provides a concrete classical realization of holographic principles, establishing a precise correspondence with Maldacena's duality would require demonstrating how our construction reproduces the specific correlation functions and scaling behaviors characteristic of AdS/CFT, which remains an open challenge for future work.}

\subsection{Discussion - Part (a): Open Questions}

\RC The discussion synthesizes the paper's key results but could better highlight open questions (e.g., experimental signatures, relation to black hole physics), considering the ideas established in the manuscript.

\AR 

\subsection{Discussion - Part (b): Anomaly-Free and Ghost-Free Claims}

\RC The claim that the model is anomaly-free and ghost-free (Sec. 5) should be tempered with caveats (e.g., limitations of the truncation scheme).

\AR 

\subsection{General: Writing and Typos}

\RC A full review should be done to the writing to correct typos.

\AR 

\section{Summary of Major Changes}

% This section will be filled once revisions are complete
\begin{enumerate}
    \item 
    \item 
    \item 
\end{enumerate}

\section{Submission Requirements}

We confirm that we will submit the following files with our revision:
\begin{itemize}
    \item Manuscript source files (LaTeX format)
    \item Figure source files in acceptable formats (TIFF, EPS)
    \item Highlights file (3-5 bullet points, maximum 85 characters each)
\end{itemize}

\bibliographystyle{apalike}

\bibliography{../lqg-strings.bib}

\end{document}
