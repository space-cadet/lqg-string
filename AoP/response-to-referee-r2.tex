\documentclass{article}
\usepackage[square,comma]{natbib}
\def\lettertitle{Author Response to Reviews of}
\def\papertitle{A Loop Quantum Gravity Inspired Action for the Bosonic String and Emergent Dimensions at Large Scales}
\def\authors{Deepak Vaid and Lin Teixeira de Sousa}
\def\journal{Annals of Physics}
\def\doi{AOP 81029}
\def\date{\today}

\input{AR_to_RC.tex}

\date{}

We thank the editor for his patience through the whole process and the referee for taking the time to review the revised manuscript.

\section{Reviewer \#1}

\subsection{Clarifications and Enhancements}

\RC The discussion on the duality between $h_{ab}$ and $h^{ab}$ (Section 2) is now clearer, but a brief mention of how this duality might manifest in observable string dynamics (e.g., through measurable spectra or scattering amplitudes) would provide an wider physical interpretation.

\AR We have added the following paragraph right after ``\emph{and the string tension inversion is reminiscent of S-duality’s strong-weak coupling correspondence.}'' to explain this point:

\emph{Physically, one way in which this duality could manifest in the real world is in the mass spectrum of string excitations which when computed in the strong-coupling regime should exhibit a precise correspondence with spectra in the weak-coupling limit under the transformation rules (2.7a)-(2.7c), while world-sheet correlation functions should satisfy similar duality relations, potentially providing calculable predictions for string phenomenology at both energy scales.}

\RC The connection to the Kalb-Ramond field (Section 3.2) is well-justified, but a brief comment on whether this implies any constraints on the Kalb-Ramond field's dynamics (e.g., torsionless backgrounds) would be insightful.

\AR We have added the following para at the end of Sec 4A, right after ``\emph{since the co- efficient in eq (4.16) is a true scalar and the Levi-Civita symbol is itself a density.}''

\emph{This pullback relationship would suggest a restriction on the allowed bulk configurations. Since the worldsheet area spectrum is discrete due to the underlying LQG quantization, the Kalb-Ramond field strength must be correspondingly quantized. This is analogous to flux quantization in the quantum Hall effect, where the magnetic flux through the 2D electron system is constrained to be integer multiples of $h/e$. In our case, the integrated field strength $\int H_{\mu \nu \rho}$ over appropriate 3-cycles in the bulk should be quantized in units related to the fundamental area quantum $\Delta$, restricting the allowed background field configurations to those compatible with the discrete worldsheet area spectrum.}

\subsection{Open Questions and Future Directions (Section 8.C)}

\RC Open Questions and Future Directions (Section 8.C):
The discussion on emergent dimensions and their cosmological implications (Section 6) is somewhat speculative. A brief acknowledgment of the phenomenological challenges (e.g., observational constraints on extra dimensions) would provide more support.

\AR We have added the following para at the end of Section 6 (Emergent Dimensions and LQG):

\emph{Of course the phenomenological realization of emergent dimensions faces significant observational constraints. Current precision tests of general relativity, including lunar laser ranging experiments and binary pulsar observations, constrain deviations from Einstein's theory at scales larger than $\sim 10^{-4}$ m, while accelerator experiments at the LHC probe physics down to $\sim 10^{-19}$ m without detecting signatures of extra dimensions. The emergent dimension $X^r$ proposed here would need to manifest in a way that preserves the apparent four-dimensional nature of spacetime at these intermediate scales, possibly through a mechanism analogous to compactification or dimensional reduction, where the emergent dimension becomes effectively decoupled from standard model physics except at very large cosmological scales or very high energies approaching the string scale.}

\RC The suggestion of a (2+1)-dimensional fundamental spacetime (Section 6) requires more evidence or references to support this claim. You might provide references to recent work on lower-dimensional gravity or holography that could bolster this argument.

\AR We have added the following footnote right after ``\emph{leaves $m_{0}, m_{1}$ and $m_{2}$ unconstrained, the related dimensions might be regarded as being \emph{fundamental}}'' in Sec 6:

\emph{Of course, the notion of what is ``fundamental'' is relative. After all, this is a phenomenological model and a very crude one at that. Moreover if holography is correct then it stands to reason that our $3+1$ dimensional bulk geometry has an equivalent description in terms of a theory living on a $2 + 1$ dimensional  ``boundary''. And if one takes the argument to its logical limit then there is no reason \emph{not} to reduce further from $2+1$ $\rightarrow$ $1+1$ $\rightarrow$ $0+1$ dimensions. In fact, this is what proponents of matrix models would argue that \emph{is} the ``correct'' picture.}

\subsection{Technical Details}

\RC The derivation of the relation between $\Delta_{ab}$ and the Kalb-Ramond field (Eq. 3.21) is now clearer, but a step-by-step derivation in an appendix or supplementary material would provide reproducibility.

\AR 

\subsection{Cosmological Implications (Section 7)}

\RC The interpretation of the emergent dimension $X^r$ as a "scaling dimension" is compelling. However, the link to the cosmological horizon scale (Page 30) is tentative. A brief discussion of how this might align or conflict with current cosmological data would be valuable.

\AR In the text we have used the accepted value of the cosmological constant as measured by the Planck collaboration. In this sense our argument already relies on current cosmological data. Moreover we have already mentioned in the text that this is a speculative idea based on a ``back of the envelope calculation''. Doing a deeper analysis of the relation of this idea to current cosmological data would take us too far afield.

\bibliographystyle{apalike}

\bibliography{../lqg-strings.bib}

\end{document}
